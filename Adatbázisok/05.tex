

rownum  sorszám

select rownum, date from dataz;

medusa.inf.elte.hu/oradoc11/index.html

zhn használható


left join using:
a usingos oszlop az elejére rakja, mergelve
az on os változat belerakja mindkét oszlopot ugyanolyan néven

descartes szorzatnál is doplázódnak a culomnok!
 
group by  having

csak having csak oraclebe megy, de nem jó, mert az egész táblát egy groupnak veszi

natural join: ha nincs közös oszlop, akkor descarter szorzat lesz

natural join egy táblát magával -> minden sor amibe nincs null



Select:

Select - 1 -- from ---2-- where --3-- group by --4-- having --5-- order by --6--;

1: distinct, unique, *, kifejezéslista
2: tábla [új név]m tábla[join], allekérdezés
3: összesítő függvény nem szerepelhet, de lehet like, between, logikai fv, in, all, any, exists


A = full

From FUllból select kisebb rész lesz, Group by kisebb rész csoportosítva, having még kisebb adag, order by rendezve

all : unió,metszet,kivonés all változat -> nem hajd végre distinctet


rowid oracleben, máshol lehet h oid

char(10) 5 nél szóközökkel van kitöltve
varcharnál változó
numeric/decimal, lehet fixed tizedespont


create table tnev (
    mezőnév típus [default tbl][sor megszorítás]
    ...

), táblamegszorítás, tblmegcsortás;



sor megszorítások:
null, not null, unique, primary key, references, check()

null az alap

oszop megszorítások:
-, -, unique(), primary key(), foreign key(), references(), check()

insert

insert into táblanév[mezőlista] values(kifejezéslista)
vagy
insert into táblanév[mezőlista] allekérdezés