% Compile twice!
% With the current MiKTeX, you need to install the beamer, and the translator packages directly form the package manager!

\documentclass{beamer}
\usepackage{tikz}

\usepackage[T1]{fontenc}
\usepackage{amsfonts}
\usepackage{amsmath}
\usepackage[utf8]{inputenc}

\usetheme{boxes}


\begin{document}

\begin{frame}[plain]
\begin{tikzpicture}[overlay, remember picture]
\node[anchor=center] at (current page.center) {
\begin{beamercolorbox}[center]{title}
    {\Huge A Számítástudomány Alapjai I}\\
    {\Large Vizsgatételek}
\end{beamercolorbox}};
\end{tikzpicture}
\end{frame}

\begin{frame}[plain]
\begin{tikzpicture}[overlay, remember picture]
\node[anchor=center] at (current page.center) {
\begin{beamercolorbox}[center]{title}
    {\Huge Logika}
\end{beamercolorbox}};
\end{tikzpicture}
\end{frame}


\begin{frame}

\begin{block}{Tétel: Minden formula egyértelműen olvasható}
F formulára a következő állítások közül pontosan egy teljesül:

\begin{enumerate}
\item F egy változó.
\item Pontosan egy G formulára $F = \neg G$
\item Pontosan egy G és pontosan egy H formuláta $F = (G \land H)$
\item Ponsotan egy G és pontosan egy H formulára $F = (G \lor H)$
\end{enumerate}

\end{block}

\end{frame}


\begin{frame}

\begin{block}{Tétel: Az ítéletkalkulus kompaktsági tétele}
Egy formulahalmaz akkor és csak akkor elégíthető ki, ha minden véges részhalmaza kielégíthető.

\end{block}

\begin{block}{Tétel: Adekvát halmazok}
$\{\neg, \lor, \land\}, \{\neg, \lor\}, \{\neg, \land\}$ adekvát (azaz bármilyen formula leírható ezekkel), $\{\lor, \land\}$ nem adekvát.

\end{block}

\end{frame}


\begin{frame}

\begin{block}{tétel: Equivalens állítások formulákra}
Legyenek $F, F_1, ... , F_n$ tetszőleges formulák, ekkor a következő állítások equivalensek:

\begin{enumerate}
\item $\{F_1, ... , F_n\} \models F$
\item $F_1 \land ... \land F_n \implies F$ tautológia
\item $F_1 \land ... \land F_n \land \neg F$ kielégíthetetlen.
\end{enumerate}

\end{block}

\end{frame}

\begin{frame}

\begin{block}{Lemma: Helyettesítési Lemma}
Legyenek $F, G, H$ formulák úgy, hogy $F \equiv G$ és $F$ a $G$ részformulája.\\
Ha $H[F/G]$ azt a formulát jelöli, amelyben $F$ valamely előfordulását helyettesítettük $G$-vel, akkor
$$H \equiv H[F/G]$$

\end{block}

\end{frame}


\begin{frame}

\begin{block}{Tétel: Konjunktív és diszjunktív normálforma létezése}
Minden $F$ Formulához létezik vele logikailag ekvivalens konjunktív és diszjunktív normálforma.

\end{block}

\begin{block}{Bizonyítás}
Konjunktív:
\begin{enumerate}
	\item (Negáció bevitele.) Amíg lehetséges, helyettesítsük $F$-ben a
	\begin{itemize}
		\item $\neg \neg G$ alakú részformulákat $G$-vel,
		\item $\neg (G \land H)$ alakú részformulákat $\neg G \lor \neg H$-val,
		\item $\neg (G \lor H)$ alakú részformulákat $\neg G \land \neg H$-val.
	\end{itemize}
	\item Amíg lehetséges, helyettesítsük $F$-ben a
	\begin{itemize}
		\item $F \lor (G \land H)$ alakú részformulákat $(F \lor G) \land (F \lor H)$-val,
		\item $(F \land G) \lor H$ alakú részformulákat $(F \lor H) \land (G \lor H)$-val.
	\end{itemize}
\end{enumerate}

Diszjunktív:
\begin{enumerate}
	\item Ugyanaz mint a konjunktív normálforma esetén.
	\item Amíg lehetséges, helyettesítsük $F$-ben a
	\begin{itemize}
		\item $F \land (G \lor H)$ alakú részformulákat $(F \land G) \lor (F \land H)$-val,
		\item $(F \lor G) \land H$ alakú részformulákat $(F \land H) \lor (G \land H)$-val.
	\end{itemize}
\end{enumerate}
\end{block}

\end{frame}

\end{document}