% Compile twice!

\documentclass{beamer}
\usepackage{tikz}

\usepackage[T1]{fontenc}
\usepackage{amsfonts}
\usepackage{amsmath}
\usepackage[utf8]{inputenc}
\usepackage[makeroom]{cancel}

\usetheme{boxes}

\geometry{paperwidth=160mm,paperheight=160mm}

\begin{document}

\begin{frame}[plain]
\begin{tikzpicture}[overlay, remember picture]
\node[anchor=center] at (current page.center) {
\begin{beamercolorbox}[center]{title}
    {\Huge Diszkrét Matematika}\\
    {\Large Vizsgatételek}
\end{beamercolorbox}};
\end{tikzpicture}
\end{frame}

\begin{frame}[plain]
\begin{tikzpicture}[overlay, remember picture]
\node[anchor=center] at (current page.center) {
\begin{beamercolorbox}[center]{title}
    {\Huge Halmazok, Relációk}
\end{beamercolorbox}};
\end{tikzpicture}
\end{frame}


\begin{frame}

\begin{block}{Tétel: Minden dolog halmaza}
Nincs olyan halmaz, amelynek minden dolog eleme.
\end{block}

\begin{block}{Biz}
asasdad
\end{block}

\end{frame}

\begin{frame}

\begin{block}{Definíció: Unió}
Ha A és B halmazok, akkor A és B unióján a következő halmazt értjük:\\
$$A \cup B = \{x | x \in A \vee x \in B\}$$
\end{block}

\begin{block}{Tétel: Az unió tulajdonságai}
Legyenek A, B, C tetszőleges halmazok. Ekkor:

\begin{enumerate}
\item $A \cup \emptyset = A$
\item $A \cup B = B \cup A$ (Kommutativitás)
\item $A \cup (B \cup C) = (A \cup B) \cup )$ (Asszociativitás)
\item $A \cup A = A$ (Idempotencia)
\item $A \subseteq B$ akkor, és csak akkor, ha $A \cup B = B$
\end{enumerate}

\end{block}

\end{frame}

\begin{frame}

\begin{block}{Definíció: Metszet}
Ha A és B halmazok, akkor A és B metszetén a következő halmazt értjük:\\
$$A \cap B = \{x \in A \wedge x \in B\}$$
\end{block}

\begin{block}{Tétel: A metszet tulajdonságai}
Legyenek A, B, C tetszőleges halmazok. Ekkor:

\begin{enumerate}
\item $A \cap \emptyset = \emptyset$
\item $A \cap B = B \cap A$ (Kommutativitás)
\item $A \cap (B \cap C) = (A \cap B) \cap C$ (Asszociativitás)
\item $A \cap A = A$ (Idempotencia)
\item $A \subseteq B$ akkor, és csak akkor, ha $A \cap B = A$
\end{enumerate}
\end{block}

\end{frame}

\begin{frame}

\begin{block}{Tétel: Unió és metszet disztributivitása}
Legyenek A, B, C tetszőleges halmazok. Ekkor:

\begin{enumerate}
\item $A \cap (B \cup C) = (A \cap B) \cup (B \cap C)$ (A metszet disztributivitása az unióra nézve)

\item $A \cup (B \cap C) = (A \cup B) \cap (B \cup C)$ (Az unió disztributivitása a metszetre nézve)
\end{enumerate}
\end{block}

\end{frame}

\begin{frame}

\begin{block}{Definíció: Komplementer}
Ha X halmaz, A $\wedge$ X, akkor A halmaz X-re vonatkoztatott komplementere:\\
$$A' = X \setminus A$$
\end{block}

\begin{block}{Tétel: A komplementer tulajdonságai}
Legyenek A, B $\wedge$ X halmazok. Ekkor:

\begin{enumerate}
\item $(A')' = A$
\item $\emptyset' = X$
\item $A \cap A' = \emptyset$
\item $A \cup A' = X$
\item $A \subseteq B$ akkor, és csak akkor, ha $B' \subseteq A'$
\item $(A \cap B)' = A' \cup B'$
\item $(A \cup B)' = A' \cap B'$
\end{enumerate}
\end{block}

\end{frame}

\begin{frame}

\begin{block}{Definíció: Halmaz osztályfelbontása}
A tetszőleges X halmazt \textbf{osztályozzuk (osztályokra bontjuk)}, ha páronként diszjunkt, nemüres részhalmazainak uniójaként állítjuk elő.
\end{block}

\begin{block}{Az X $\in$ X elem \textbf{ekvivalencia osztálya}:}
$$\overline{x} = \{y \in X : y \sim x\}$$
\end{block}

\begin{block}{Tétel: Ekvivalenciareláció és osztályfelbontás kapcsolata}
Valamely X halmazon értelmezett $\sim$ ekvivalenciareláció X-nek egy osztályfelbontását adja. Megfordítva, az X halmaz minden osztályfelbontása egy $\sim$ ekvivalenciarelációt hoz létre.
\end{block}

\begin{block}{Biz}
asasdad
\end{block}

\end{frame}


\begin{frame}[plain]
\begin{tikzpicture}[overlay, remember picture]
\node[anchor=center] at (current page.center) {
\begin{beamercolorbox}[center]{title}
    {\Huge Algebrai struktúrák, számhalmazok}
\end{beamercolorbox}};
\end{tikzpicture}
\end{frame}

\begin{frame}

\begin{block}{Tétel: Egységelem és inverz félcsoportban}
Félcsoportban legfeljebb egy egységelem létezik, és minden elemnek legfeljebb egy, az egységelemre vonatkozó inverze létezik.
\end{block}

\begin{block}{Bizonyítás}
Legyen $(G, *)$ félcsoport, $e_b$ bal oldali, $e_j$ pedig jobb oldali egységelem $G$-ben.\\
Ekkor $e_b = e_j$, hiszen:\\
$e_be_j = e_j$ és $e_be_j = e_b$, (nyíl éshez $\rightarrow$ a függvény egyértelmű!)\\
mert $e_b$ bal, $e_j$ jobb oldali egységelem.\\
Ha az $a \in G$ elemnek $a_b$ balinverze, $a_j$ pedig jobbinverze, akkor $a_b = a_j$.
$a_baa_j = a_b(aa_j) = a_be = a_b$ és $a_baa_j = (a_ba)a_j = ea_j = a_j$. (Asszociatív tulajdonság) (nyíl éshez ide is $\rightarrow$ a függvény egyértelmű!).d
\end{block}

\end{frame}

\begin{frame}

\begin{block}{Lemma: Észrevételek gyűrűkben}
\begin{enumerate}
\item \textbf{Szorzás nullelemmel:} Legyen 0 az R gyűrű nulleleme. Ekkor $a0 = 0a = 0$, minden $a \in R$ esetén.
\item \textbf{Előjelszabály:} Legyen R gyűrű, és $a, b \in R$. Az $a$ elem additív inverzés jelöljük $-a$-val. Ekkor $-(ab) = (-a)b = a(-b)$, tobábbá $(-a)(-b) = ab$.
\item \textbf{Véges integritási tartomány test.}
\item \textbf{Testben nincs nullosztó.}
\end{enumerate}
\end{block}

\end{frame}

\begin{frame}

\begin{block}{Lemma: Nullosztó és regularitás}
R gyűrűben a multiplikatív művelet akkor, és csak akkor reguláris, ha R zérusosztómentes.
\end{block}

\begin{block}{Bizonyítás}
\textbf{1. Rész}\\
Tfh $a \neq 0$, a nem bal oldali nullosztó és $ab = ac$.\\
$ab = ac  / -(ac)$ (+ additív inverz)\\
$ab + (-(ac)) = 0$. Előjel szabály + disztri.\\
$ab + (a(-c)) = a(b+(-c)) = 0$ (Kiemeljük, csak akkor lehet, ha $(b + -1 = 0) \implies (b = c)$)\\
A feltételből ($a$ nem baloldali nullosztó) következik, hogy $b + (-c) = 0)$ $\implies$\\
$\implies$ b = c.\\
\bigskip
\textbf{2. Rész}\\
Tfh $a$ bal oldali nullosztó, tehát $a \neq 0$ és létezik $b \neq 0\: ab = 0$.\\
tetszőleges $c \in R$-re: $ac = ac$.\\
$ac = ac / +0 (0 = ab)$\\
$ac = ac + ab$ /(Disztributivitás)\\
$ac = a(c + b)$ Ellentmondás!\\
Mivel $(b \neq 0) \implies (c \neq (c + b))$ (A b nem additív egységelem).

\end{block}

\end{frame}

\begin{frame}

\begin{block}{Tétel: Természetes számok}
A $(N, +, *)$ struktúrában mindkét művelet asszociatív, kommutatív, reguláris.\\
Nullelem (additív egységelem): 0.\\
Multiplikatív egységelem: 1.\\
A szorzat mindkét oldalról disztributív az összeadásra.\\
${\forall}m \in N : 0 * m = m * 0 = 0$.

\end{block}

\end{frame}

\begin{frame}

\begin{block}{Tétel: N rendezése}
A természetes számok halmaza a $\leq$ relációval jólrendezett.
\end{block}

\end{frame}

\begin{frame}

\begin{block}{Tétel: Felső határ és arkhimédészi tulajdonság}
$T$ felső határ tulajdonságú test, $\implies$ $T$ arkhimédészi tulajdonságú.
\end{block}

\begin{block}{Bizonyítás (Indirekt)}
Tfh $T$ felső határ tulajdonságú rendezett test, de nem arkhimédészi tulajdonságú.\\
$\implies : {\nexists}n \in \mathbb{N} : nx \geq y$.\\
Azaz y felső korlátja az $A = \{ nx | n \in \mathbb{N} \}$ halmaznak.\\
Ekkor viszont létezik $z = sup A$ $\implies$ $z - x < z$ nem felső korlát. $\implies$\\
$implies$ ${\exists}n : nx > z - x \implies (n + 1)x > z$. ($(n + 1)x \in A$).\\
Ellentmondás, mivel ha $n \in \mathbb{N}$, akkor $n^+ \in \mathbb{N}$ $\rightarrow$ Peano axióma!
\end{block}

\end{frame}

\begin{frame}

\begin{block}{Tétel: Q nem felső határ tulajdonságú}
$\mathbb{Q}$ arkhimédészi tulajdonságú, de nem felső határ tulajdonságú.
\end{block}

\end{frame}


\begin{frame}

\begin{block}{Tétel: $\sqrt{2}$ nem racionális}
Nincs $\mathbb{Q}$-ban olyan szám, amelynek négyzete 2.
\end{block}

\begin{block}{Bizonyítás (Indirekt)}
Tfh van, és ez $x$.\\
$x = \frac{m}{n}, m,n \in \mathbb{N}^+$, és az $m$ minimális.\\
$2 = x^2 = \frac{m^2}{n^2} \implies m^2 = 2n^2$\\
Ebből következik, hogy $m$ páros. $\implies$ $m = 2k, k \in \mathbb{N}^+$\\
Ebből következik, hogy $n$ is páros: $n = 2j, j \in \mathbb{N}^+$\\
Ekkor viszont $\frac{m}{n} = \frac{2k}{2j} = \frac{k}{j}$.\\
Viszont ebből koövetkezik, hogy m nem minimális $\rightarrow$ Ellentmondás!

\end{block}
\end{frame}

\begin{frame}

\begin{block}{Tétel: Az algebra alaptétele}
Ha $n \in \mathbb{N}^+$, valamint $c_0, c_1, ... c_n$ komplex számok, $c_n \neq 0$, akkor van olyan $u$ komplex szám, amelyre:\\
$$\sum_{k = 0}^n c_kz^k = 0$$
\end{block}

\end{frame}

\begin{frame}[plain]
\begin{tikzpicture}[overlay, remember picture]
\node[anchor=center] at (current page.center) {
\begin{beamercolorbox}[center]{title}
    {\Huge Számelmélet}
\end{beamercolorbox}};
\end{tikzpicture}
\end{frame}

\begin{frame}

\begin{block}{Tétel: Az oszthatóság tulajdonságai EIT-ban}
\begin{enumerate}
\item Ha $b|a$ és $b'|a'$, akkor $bb'|aa'$.
\item A nullának minden elem osztója.
\item A nulla csak saját magának osztója.
\item Az 1 egységelem minden elemnek osztója.
\item Ha $b|a$, akkor $bc|ac$ minden $c \in R$-re.
\item Ha $bc|ac$ és $c \neq 0$, akkor $b|a$.
\item Ha $b|a_i$ és $c_i \in R, (i = 1, 2, ..., j)$, akkor $b|\sum^j_{i=1} c_ia_i$.
\item Az $|$ reláció reflexív, és tranzitív.
\end{enumerate}
\end{block}

\end{frame}

\begin{frame}

\begin{block}{Tétel: Prím és irreducibilis elem EIT-ban}
Tetszőleges $R$ egységelemes integritási tartományban minden $p$ elemre:\\
Ha $p$ prím $\implies$ $p$ felbonthatatlan.
\end{block}

\begin{block}{Bizonyítás}
Tfh $p$ prím, és, $p = bc$\\
Ekkor vagy $p|b$, vagy $p|c$\\
$b = pq = b(cq) \implies cq = 1$ $\implies$ $c, q$ egység $p, b$ asszociáltak.

\end{block}

\end{frame}

\begin{frame}

\begin{block}{Tétel: Maradékos osztás $\mathbb{Z}$-ben}
${\exists}a, b({\neq}0) \in \mathbb{Z}$ számhoz egyértelműen létezik olyan $q, r \in \mathbb{Z}$, hogy\\
$a = qb + r \land 0 \leq r < |b|$.
\end{block}

\end{frame}

\begin{frame}

\begin{block}{Tétel: Prím és irreducibilis elem $\mathbb{Z}$-ben}
Az egész számok körében $p$ prím $\iff$ $p$ felbonthatatlan.
\end{block}

\begin{block}{Bizonyítás}
Már láttuk, hogy prím felbonthatatlan!\\
Tfh p felbonthatatlan\\
Legyen $p|bc$, ekkor vagy $p | b$-nek, ekkor ksz vagyunk.
Vagy $p \nmid b$ ekkor $(p,b) = 1$.\\
$c = pcx +bcx \implies 0 mod p \implies p | c$.\\
(Észrevétel: $(a, b) = 1 \land a | bc \implies a | c$

\end{block}

\end{frame}

\begin{frame}

\begin{block}{Tétel: A számelmélet alaptétele}
Minden $m$ nemnulla, nemegység, egész szám sorrendre és asszociáltásgra való tekintet nélkül egyértelműen bontható fel felbonthatatlanok szorzatára.
\end{block}

\begin{block}{Bizonyítás (Pozitívakra)}
\textbf{(egzisztencia)}\\
Tfh $n > 1$\\
Teljes indukció: $n = 2$ kész, tfk $n - 1$-ig kész.\\
Ha $n$ felbonthatatlan $\rightarrow$ kész.\\
Ha $n$ nem felbonthatatlan $\rightarrow$ $n = ab \land a, b$ (a, b nem egység!), $a, b < n$ $\implies$ igaz rájuk az ind. feltétel.\\
$n$ felbontása $=$ $a$ felbontása szor $b$ felbontása.\\
\bigskip
\textbf{(unicitás) (Indirekt)}\\
Tfh $n$ a legkisebb olyan szám, amely felbontása nem egyértelmű.\\
$n = p_1 ... p_k = q_1 ... q_r$ $\implies$\\
$p_j|n \implies p_1|q_1 ... q_r$\\
$p_1|q_1$, $p_1|q_2 ... q_r$\\
           $p_1|q_2   p_1|q_3 ... q_r$\\
                      $p_1|q_i  \implies p_1 = q_i \implies$\\
$\implies$ $n_1 = \frac{n}{p_1} = p_2 ... p_k = q_1 ... q_{i-1}q_{i+1} ... q_r$\\
$n_1 < n$ és van két lényegesen különböző felbontása!
\end{block}

\end{frame}

\begin{frame}

\begin{block}{Tétel: Eukleidész tétele}
Végetlen sok prímszám van.
\end{block}

\begin{block}{Bizonyítás (Indirekt)}
Tfh véges sok van:\\
$p_1, p_2, ... ,p_k$.\\
Legyen $n = p_1p_2...p_k$.\\
Számelmélet alaptételéből következik hogy létezik $p_j : p_j | n + 1$\\
$p_j : p_j | n + 1 \implies p_j | 1$ Ellentmondás!
\end{block}

\end{frame}

\begin{frame}

\begin{block}{Tétel: Kongruencia tulajdonságai}
\begin{enumerate}
\item Ekvivalencia reláció
\item $a \equiv b \pmod{m} \land c \equiv d \pmod{m} \implies$ \textbf{$a + c \equiv b + d \pmod{m}$}
\item $a \equiv b \pmod{m} \land c \equiv d \pmod{m} \implies$ \textbf{$ac \equiv bd \pmod{m}$}
\item $a \equiv b \pmod{m} \land f(x) \in z[x] \implies$ \textbf{$f(a) \equiv f(b) \pmod{m}$}
\item Ha $(c, m) = d$, $ac \equiv bc \pmod{m} \iff a \equiv b \pmod{\frac{m}{d}}$
\end{enumerate}
\end{block}

\end{frame}

\begin{frame}

\begin{block}{Tétel: Omnibusz tétel}
Legyen: $m > 1$ egész, $\{a_1, ..., a_m\}$ TMR modulo $m$, $\{b_1, ..., b_{{\phi}(m)}\}$ RMR modulo $m$, $c, d \in \mathbb{Z}$, és $(c,m) = 1$.\\
\smallskip
Ekkor:\\
\smallskip
$\{ ca_1 + d, ..., ca_m + d \}$ TMR modulo $m$\\
$\{ cb_1, ..., cb{{\phi}(m)}\}$ RMR modulo $m$
\end{block}

\begin{block}{Bizonyítás (Indirekt)}
Tfh van két nem inkongruens elem\\
$ca_i + d = ca_i + d$\\
${\cancel{c}}a_i + {\cancel{d}} = {\cancel{c}}a_i + {\cancel{d}}$ $(c, m) = 1$, és pontosan $m$ db elem!\\
$(c, m) = 1$ és $(b_j,m) = 1$ $\implies$ $(cb_j, m) = 1$

\end{block}

\end{frame}

\begin{frame}

\begin{block}{Tétel: Euler-Fermat tétel}
Legyen $m > 1$ egész és $a$ relatív prím $m$-hez. Ekkor $a^{{\phi}(m)} \equiv 1 \pmod{m}$
\end{block}

\begin{block}{Bizonyítás}
Legyen $\{ r_1, ..., r_{{\phi}(m)}\}$ RMR modulo $m$, $(a, m) = 1$.\\
Az omnibusz tétel miatt, ekkor $\{ ar_1, ..., ar_{{\phi}(m)}\}$ is RMR modulo $m$.\\
Megfelelő párosítás $\implies$ $r_i \equiv ar_j \pmod{m}$.\\
Összehozva: $(r_i, m) = 1$\\
\smallskip
$$a^{{\phi}(m)} \prod^{{\phi}(m)}_{i=1} r_i \equiv \prod^{{\phi}(m)}_{i=1} r_i \pmod{m}$$

\end{block}

\end{frame}

\begin{frame}

\begin{block}{Tétel: (Kis) Fermat tétel}
Legyen $p$ prím és $a \in \mathbb{Z}$. Ekkor\\
(első alak) ha $p \nmid a$, akkor $a^{p-1} \equiv 1 \pmod{p}$.\\
(második alak) ha $a$ tetszőleges, akkor $a^p \equiv a \pmod{p}$.

\end{block}

\begin{block}{Bizonyítás}
Első alak: ${\phi}(p) \equiv p - 1$ $\rightarrow$ előző tétel miatt kész.\\
\bigskip
Második alak:\\
Ha $p|a$ $\rightarrow$ $0 \equiv 0$ $\rightarrow$ kész.\\
Ha $p{\nmid}a$ $\rightarrow$ ekkor ez az első alak $\rightarrow$ kész.

\end{block}

\end{frame}

\begin{frame}

\begin{block}{Tétel: A diofantikus egyenlet megoldása}
Rögzített $a, b, c$ egész számok esetén az \textbf{$ax + by = c$} diofantikus egyenletnek akkor, és csak akkor van megoldása, ha $(a, b)|c$.
\end{block}

\begin{block}{Bizonyítás}
Tfh $ax + by = c$ egyenletnek van megoldása.\\
\textbf{1. Rész ($\implies$)}\\
\smallskip
Tfh $x_0, y_0$ megoldás. $\implies$ $(a, b)|a \land (a, b)|b$ $\implies$ lin. kombinációs tul. $\implies$\\
$\implies$ $(a, b)|ax_0 + by_0 = c$ (Igaz, mert az a, b osztója az $ax_0 + by_0$-nak.)\\
\bigskip
\textbf{2.Rész ($\Longleftarrow$)}\\
\smallskip
Tfh (a, b)|c. Ekkor:\\
$c = (a, b)q$\\
$c = (au + bv)q$\\
$c = a(uq) + b(vq)$\\
$c = a(uq) + b(vq)$ $\implies$ egy megoldás: $x = uq, y = vq$.\\

\end{block}

\end{frame}

\begin{frame}

\begin{block}{Tétel: Kínai maradéktétel}
Legyen $n \in \mathbb{N}^+, m_1, m_2, ..., m_n \in \mathbb{N}^+, a_i, b_i \in \mathbb{Z} (1 <leq i \leq n)$, ahol
\begin{enumerate}
\item $m_i, m_j$ páronként relatív prímek.
\item $(m_i, a_i) = 1$, minden $1 \leq i \leq n$ esetén.
\end{enumerate}
Ekkor az\\
\bigskip
	$a_1x \equiv b_1 \pmod{m_1}$\\
	$a_2x \equiv b_2 \pmod{m_2}$\\
	...\\
	$a_nx \equiv b_n \pmod{m_n}$\\
\bigskip
Kongruenciarendszer megoldható és bármely két megoldása kongruens modulo $m_1m_2...m_n$.

\end{block}

\end{frame}

\begin{frame}

\begin{block}{Tétel: Számelméleti függvények}
Legyen $n \in \mathbb{N}^+$ kanonikus alakja $p_1^{{\alpha}_1}...p_k^{{\alpha}_k}$. Ekkor:\\
\begin{enumerate}
\item Ha $f$ additív számelméleti függvény, akkor $$f(n) = f(p_1^{{\alpha}_1}) + ... + f(p_k^{{\alpha}_k})$$
\item Ha $f$ multiplikatív számelméleti függvény, akkor $$f(n) = f(p_1^{{\alpha}_1})...f(p_k^{{\alpha}_k})$$
\item Ha $f$ teljesen additív számelméleti függvény, akkor $$f(n) = {\alpha}_1f(p_1) + ... + {\alpha}_kf(p_k)$$
\item Ha $f$ teljesen multiplikatív számelméleti függvény, akkor $$f(n) = f(p_1)^{{\alpha}_1}...f(p_k)^{{\alpha}_k}$$
\end{enumerate}

\end{block}

\end{frame}

\begin{frame}

\begin{block}{Tétel: $\phi$ multiplikativitása}
$\phi$ multiplikatív.
\end{block}

\begin{block}{Bizonyítás}
\smallskip
\begin{tabular}{c c c c}
1 & 2 & ... & a \\
a + 1 & a + 2 & ... & 2a\\
 &  & ... &  \\
(b - 1)a + 1 & (b - 1)a + 2 & ... & ba
\end{tabular}
\smallskip
Számoljuk meg, hogy a táblázatban hány relatív prím van $ab$-hez: ennyi lesz ${\phi}(ab)$ értéke.\\
(Ha $a$ is $b$ is relatív prím $c$-hez, akkor $ab$ is. $\implies$ azokat kell számolni, amelyek $a$-hoz és $b$-hez is rel. prímek)\\
\smallskip
AZ Omnibusz tételből következik hogy minden oszlop TMR mod $b$, ha $(a, b) = 1$ $\implies$\\
$\implies$ minden oszlopban ${\phi}(b)$ rel. prím $b$-hez.\\
\smallskip
| Minden oszlom kongruens elemeket tart mod $a$.\\
| Minden sor egy TMR mod $a$ $\implies$ minden sorban ${\phi}(a)$ db elem relatív prím $a$-hoz.\\
$\implies$ ${\phi}(a)$ db oszlopnak rel prímek az elemei $a$-hoz. $\implies$ összesen ${\phi}(a){\phi}(b)$ rel. prím van $ab$-hez.

\end{block}

\end{frame}

\begin{frame}

\begin{block}{Tétel: ${\phi}$(n) kiszámolása}
Ha $n \in \mathbb{N}^+$ kanonikus alakja $p_1^{{\alpha}_1}...p_k^{{\alpha}_k}$, akkor\\
$${\phi}(n) = \prod^k_{j=1} (p_j^{{\alpha}_j} - p_j^{{\alpha}_j - 1}) = n \prod^k_{j=1} (1 - \frac{1}{p_j}).$$
\end{block}

\begin{block}{Bizonyítás}
$\phi$ multiplikatív\\
Kiszámoljuk az értékeket prímhatványhelyeken, majd összeszorozzuk az értékeket.\\
${\phi}(p^{\alpha}) = ?$\\
$1, 2, ..., p, ..., 2p, ..., 3p, ..., (p-1)p, ..., p^2, ..., (p+1)p, ..., (p-1)p^{{\alpha}-1}, ..., p^{\alpha}$\\
Melyek nem relatív prímek $p$-hez?\\
\smallskip
$p^2$-ig $p - 1$ db van + maga $p^2$, azaz ${\phi}(p^2) = p^2 - p^1$.\\
Tovább számolva:\\
${\phi}(p^{\alpha}) = p^{\alpha} - p^{\phi - 1}$

\end{block}

\end{frame}

\begin{frame}[plain]
\begin{tikzpicture}[overlay, remember picture]
\node[anchor=center] at (current page.center) {
\begin{beamercolorbox}[center]{title}
    {\Huge Kombinatorika}
\end{beamercolorbox}};
\end{tikzpicture}
\end{frame}

\begin{frame}

\begin{block}{Tétel: Véges halmaz valódi részhalmaza}
Ha $n$ természetes szám, akkor nem létezik ekvivalencia $\{1, 2, ..., n\}$ és egy valódi részhalmaza között.	

\end{block}

\end{frame}

\begin{frame}

\begin{block}{Tétel: Skatulya-elv}
\end{block}

\begin{block}{Bizonyítás}
\end{block}

\end{frame}

\begin{frame}

\begin{block}{Tétel: Permutációk száma}
\end{block}

\begin{block}{Bizonyítás}
\end{block}

\end{frame}

\begin{frame}

\begin{block}{Tétel: Variációk száma}
\end{block}

\begin{block}{Bizonyítás}
\end{block}

\end{frame}

\begin{frame}

\begin{block}{Tétel: Ismétléses variációk száma}
\end{block}

\begin{block}{Bizonyítás}
\end{block}

\end{frame}

\begin{frame}

\begin{block}{Tétel: Kombinációk száma}
\end{block}

\begin{block}{Bizonyítás}
\end{block}

\end{frame}

\begin{frame}

\begin{block}{Tétel: Ismétléses kombinációk száma}
\end{block}

\begin{block}{Bizonyítás}
\end{block}

\end{frame}

\begin{frame}

\begin{block}{Tétel: Ismétléses permutációk száma}
\end{block}

\begin{block}{Bizonyítás}
\end{block}

\end{frame}

\begin{frame}

\begin{block}{Tétel: Binomiális tétel}
\end{block}

\begin{block}{Bizonyítás}
\end{block}

\end{frame}

\begin{frame}

\begin{block}{Tétel: Logikai szita formula}
\end{block}

\begin{block}{Bizonyítás}
\end{block}

\end{frame}








\begin{frame}

\begin{block}{Tétel: Egységelem és inverz félcsoportban}
\end{block}

\begin{block}{Bizonyítás}
\end{block}

\end{frame}


\end{document}