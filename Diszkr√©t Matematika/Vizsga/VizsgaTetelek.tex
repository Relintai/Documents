% Compile twice!
% With the current MiKTeX, you need to install the beamer, and the translator packages directly form the package manager!

% Uncomment these to get the presentation form
%\documentclass{beamer}
%\geometry{paperwidth=200mm,paperheight=200mm, top=0in, bottom=0.2in, left=0.2in, right=0.2in}

% Uncomment these, and comment the 2 lines above, to get a paper-type article
%\documentclass[10pt]{article}
%\usepackage{geometry}
%\geometry{top=0.2in, bottom=0.2in, left=0.2in, right=0.2in}
%\usepackage{beamerarticle}
%\renewcommand{\\}{\par\noindent}
%\setbeamertemplate{note page}[plain]

% Half A4 geometry
%\geometry{paperwidth=105mm,paperheight=297mm,top=0.2in, bottom=0.2in, left=0.2in, right=0.2in}

% "1/3" A4 geometry
%\geometry{paperwidth=105mm,paperheight=455mm,top=0.1in, bottom=0.1in, left=0.1in, right=0.1in}

% "1/6" A4 geometry
%\geometry{paperwidth=105mm,paperheight=891mm,top=0.1in, bottom=0.1in, left=0.1in, right=0.1in}

% "1/5" A4 geometry
%\geometry{paperwidth=105mm,paperheight=740mm,top=0.1in, bottom=0.1in, left=0.1in, right=0.1in}

% "1/4" A4 geometry
%\geometry{paperwidth=105mm,paperheight=594mm,top=0.1in, bottom=0.1in, left=0.1in, right=0.1in}

% Uncomment these, to put more than one slide / page into a generated page.
%\usepackage{pgfpages}
% Choose one
%\pgfpagesuselayout{2 on 1}[a4paper]
%\pgfpagesuselayout{4 on 1}[a4paper]
%\pgfpagesuselayout{8 on 1}[a4paper]

% Includes
\usepackage{tikz}
\usepackage{tkz-graph}
\usetikzlibrary{shapes,arrows,automata}
\usepackage[T1]{fontenc}
\usepackage{amsfonts}
\usepackage{amsmath}
\usepackage[utf8]{inputenc}
\usepackage{booktabs}
\usepackage{array}
\usepackage{arydshln}
\usepackage{enumerate}
\usepackage[many, poster]{tcolorbox}
\usepackage{pgf}
\usepackage[makeroom]{cancel}

% Colors
\definecolor{myred}{rgb}{0.87,0.18,0}
\definecolor{myorange}{rgb}{1,0.4,0}
\definecolor{myyellowdarker}{rgb}{1,0.69,0}
\definecolor{myyellowlighter}{rgb}{0.91,0.73,0}
\definecolor{myyellow}{rgb}{0.97,0.78,0.36}
\definecolor{myblue}{rgb}{0,0.38,0.47}
\definecolor{mygreen}{rgb}{0,0.52,0.37}
\colorlet{mybg}{myyellow!5!white}
\colorlet{mybluebg}{myyellowlighter!3!white}
\colorlet{mygreenbg}{myyellowlighter!3!white}

\setbeamertemplate{itemize item}{\color{black}$-$}
\setbeamertemplate{itemize subitem}{\color{black}$-$}
\setbeamercolor*{enumerate item}{fg=black}
\setbeamercolor*{enumerate subitem}{fg=black}
\setbeamercolor*{enumerate subsubitem}{fg=black}

\renewcommand{\tiny}{\footnotesize}
\renewcommand{\small}{\footnotesize}

% These are different themes, only uncomment one at a time
\tcbset{enhanced,fonttitle=\bfseries,boxsep=7pt,arc=0pt,colframe={myyellowlighter},colbacktitle={myyellow},colback={mybg},coltitle={black}, coltext={black},attach boxed title to top left={xshift=-2mm,yshift=-2mm},boxed title style={size=small,arc=0mm}}

%\tcbset{colback=yellow!5!white,colframe=yellow!84!black}
%\tcbset{enhanced,colback=red!10!white,colframe=red!75!black,colbacktitle=red!50!yellow,fonttitle=
%\tcbset{enhanced,attach boxed title to top left}
%\tcbset{enhanced,fonttitle=\bfseries,boxsep=5pt,arc=8pt,borderline={0.5pt}{0pt}{red},borderline={0.5pt}{5pt}{blue,dotted},borderline={0.5pt}{-5pt}{green}}

% Beamer theme
\usetheme{boxes}

% tikz settings for the flowchart(s)
\tikzstyle{decision} = [diamond, minimum width=3cm, minimum height=1cm, text centered, draw=black, fill=green!15]
\tikzstyle{tcolorbox} = [rectangle, draw, fill=blue!15, text width=20em, text centered, minimum height=1em]

\tikzstyle{line} = [draw, -latex']
\tikzstyle{cloud} = [draw, ellipse,fill=red!20, node distance=3cm,
    minimum height=2em]
\tikzstyle{arrow} = [thick,->,>=stealth]

\newcolumntype{C}[1]{>{\centering\let\newline\\\arraybackslash\hspace{0pt}}m{#1}}
\renewcommand{\arraystretch}{1.2}

\setlength\dashlinedash{0.2pt}
\setlength\dashlinegap{1.5pt}
\setlength\arrayrulewidth{0.3pt}

\newcommand{\mtinyskip}{\vspace{0.2em}}
\newcommand{\msmallskip}{\vspace{0.3em}}
\newcommand{\mmedskip}{\vspace{0.5em}}
\newcommand{\mbigskip}{\vspace{1em}}


\begin{document}

\begin{frame}[plain]
\begin{tcolorbox}[center, colback={myyellow}, coltext={black}, colframe={myyellow}]
    {\Huge Diszkrét Matematika I}\\
\mbigskip
\\
A kisbetűs szövegek (LaTeX-ben tiny), (Ha nincs előttük (S) jelzés, akkor lemaradt)\\
a saját értelmezést jelentik, és egyáltalán nem garantált hogy jók!
\end{tcolorbox}
\end{frame}


%\begin{tcolorbox}[title={Def.: }]
%\end{tcolorbox}

% --------------------  HALMAZOK, RELÁCIÓK --------------------

\begin{frame}[plain]
\begin{tcolorbox}[center, colback={myyellow}, coltext={black}, colframe={myyellow}]
    {\Huge Halmazok, Relációk}
    \mmedskip
\end{tcolorbox}
\end{frame}

\begin{frame}
\begin{tcolorbox}[title={Def.: A halmazelmélet "Definiálatlan alapfogalmai"}]
\end{tcolorbox}

\begin{tcolorbox}[title={Def.:  Meghatározottsági Axióma (Halmazok egyenlősége)}]
\end{tcolorbox}

\begin{tcolorbox}[title={Def.: Az üres halmz axiómája}]
\end{tcolorbox}

\begin{tcolorbox}[title={Def.: Részhalmaz-axióma}]
\end{tcolorbox}
\end{frame}

\begin{frame}
\begin{tcolorbox}[title={Tétel: Minden dolog halmaza}]
Nincs olyan halmaz, amelynek minden dolog eleme.
\end{tcolorbox}

\begin{tcolorbox}[title={Biz}]
\end{tcolorbox}
\end{frame}

\begin{frame}
\begin{tcolorbox}[title={Definíció: Unió}]
Ha A és B halmazok, akkor A és B unióján a következő halmazt értjük:\\
$$A \cup B = \{x | x \in A \vee x \in B\}$$
\end{tcolorbox}

\begin{tcolorbox}[title={Tétel: Az unió tulajdonságai}]
Legyenek A, B, C tetszőleges halmazok. Ekkor:

\begin{enumerate}
\item $A \cup \emptyset = A$
\item $A \cup B = B \cup A$ (Kommutativitás)
\item $A \cup (B \cup C) = (A \cup B) \cup )$ (Asszociativitás)
\item $A \cup A = A$ (Idempotencia)
\item $A \subseteq B$ akkor, és csak akkor, ha $A \cup B = B$
\end{enumerate}
\end{tcolorbox}
\end{frame}

\begin{frame}
\begin{tcolorbox}[title={Definíció: Metszet}]
Ha A és B halmazok, akkor A és B metszetén a következő halmazt értjük:\\
$$A \cap B = \{x \in A \wedge x \in B\}$$
\end{tcolorbox}

\begin{tcolorbox}[title={Tétel: A metszet tulajdonságai}]
Legyenek A, B, C tetszőleges halmazok. Ekkor:

\begin{enumerate}
\item $A \cap \emptyset = \emptyset$
\item $A \cap B = B \cap A$ (Kommutativitás)
\item $A \cap (B \cap C) = (A \cap B) \cap C$ (Asszociativitás)
\item $A \cap A = A$ (Idempotencia)
\item $A \subseteq B$ akkor, és csak akkor, ha $A \cap B = A$
\end{enumerate}
\end{tcolorbox}
\end{frame}

\begin{frame}
\begin{tcolorbox}[title={Tétel: Unió és metszet disztributivitása}]
Legyenek A, B, C tetszőleges halmazok. Ekkor:

\begin{enumerate}
\item $A \cap (B \cup C) = (A \cap B) \cup (B \cap C)$ (A metszet disztributivitása az unióra nézve)

\item $A \cup (B \cap C) = (A \cup B) \cap (B \cup C)$ (Az unió disztributivitása a metszetre nézve)
\end{enumerate}
\end{tcolorbox}
\end{frame}

\begin{frame}
\begin{tcolorbox}[title={Def.: Diszjunkt, Páronként diszjunkt halmazok.}]
\end{tcolorbox}

\begin{tcolorbox}[title={Def.: Halmazok Különbsége}]
\end{tcolorbox}

\begin{tcolorbox}[title={Def.: Halmazok Szimmetrikus Differenciája}]
\end{tcolorbox}
\end{frame}

\begin{frame}
\begin{tcolorbox}[title={Definíció: Komplementer}]
Ha X halmaz, A $\wedge$ X, akkor A halmaz X-re vonatkoztatott komplementere:\\
$$A' = X \setminus A$$
\end{tcolorbox}

\begin{tcolorbox}[title={Tétel: A komplementer tulajdonságai}]
Legyenek A, B $\wedge$ X halmazok. Ekkor:

\begin{enumerate}
\item $(A')' = A$
\item $\emptyset' = X$
\item $A \cap A' = \emptyset$
\item $A \cup A' = X$
\item $A \subseteq B$ akkor, és csak akkor, ha $B' \subseteq A'$
\item $(A \cap B)' = A' \cup B'$
\item $(A \cup B)' = A' \cap B'$
\end{enumerate}
\end{tcolorbox}
\end{frame}

\begin{frame}
\begin{tcolorbox}[title={Definíció: Halmaz osztályfelbontása}]
A tetszőleges X halmazt \textbf{osztályozzuk (osztályokra bontjuk)}, ha páronként diszjunkt, nemüres részhalmazainak uniójaként állítjuk elő.
\end{tcolorbox}

\begin{tcolorbox}[title={Az X $\in$ X elem \textbf{ekvivalencia osztálya}:}]
$$\overline{x} = \{y \in X : y \sim x\}$$
\end{tcolorbox}

\begin{tcolorbox}[title={Tétel: Ekvivalenciareláció és osztályfelbontás kapcsolata}]
Valamely X halmazon értelmezett $\sim$ ekvivalenciareláció X-nek egy osztályfelbontását adja. Megfordítva, az X halmaz minden osztályfelbontása egy $\sim$ ekvivalenciarelációt hoz létre.
\end{tcolorbox}

\begin{tcolorbox}[title={Biz}]
\end{tcolorbox}
\end{frame}

\begin{frame}
\begin{tcolorbox}[title={Def.: Rendezett pár}]
\end{tcolorbox}

\begin{tcolorbox}[title={Def.: Rendezett $n$-es}]
\end{tcolorbox}

\begin{tcolorbox}[title={Def.: Descartes (Direkt) szorzat}]
\end{tcolorbox}

\begin{tcolorbox}[title={Def.: $n$ változós reláció}]

(1 változós = unér, 2 változós = binér)
\end{tcolorbox}

\begin{tcolorbox}[title={Def.: Homogén reláció}]
\end{tcolorbox}

\begin{tcolorbox}[title={Def.: Identikus leképzés}]
\end{tcolorbox}

\begin{tcolorbox}[title={Def.: Relávió értelmezési tartománya}]
\end{tcolorbox}

\begin{tcolorbox}[title={Def.: Reláció értékkészlete}]
\end{tcolorbox}
\end{frame}

\begin{frame}
\begin{tcolorbox}[title={Def.: Leszűkítés, kierjesztés}]
\end{tcolorbox}

\begin{tcolorbox}[title={Def.: Az $R$ reláció $X$ halmazra való Leszűkítése}]
\end{tcolorbox}

\begin{tcolorbox}[title={Def.: Ar $r \subset X x Y$ reláció inverze}]
\end{tcolorbox}

\begin{tcolorbox}[title={Ész}]
\end{tcolorbox}
\end{frame}

\begin{frame}
\begin{tcolorbox}[title={Def.: Az $A$ halmaz képe, (ős)képe / inverz képe}]
\end{tcolorbox}

\begin{tcolorbox}[title={Ész}]
\end{tcolorbox}

\begin{tcolorbox}[title={Def.: Az $S$ és $R$ relációk kompozíciója}]
\end{tcolorbox}

\begin{tcolorbox}[title={Ész}]
\end{tcolorbox}
\end{frame}

\begin{frame}
\begin{tcolorbox}[title={Def.: Homogén binér relációk tulajdonságai}]
\end{tcolorbox}
\end{frame}

\begin{frame}
\begin{tcolorbox}[title={Def.: Ekvivalenciareláció}]
\end{tcolorbox}

\begin{tcolorbox}[title={Def.: Halmaz osztályfelbontása}]
\end{tcolorbox}

\begin{tcolorbox}[title={Def.: Az $c \in X$ elem ekvivalencia osstálya}]
\end{tcolorbox}

%\begin{tcolorbox}[title={Tétel.: Ekvivalenciarelácó, és osztályfelbontás}]
%\end{tcolorbox}
\end{frame}

\begin{frame}

\begin{tcolorbox}[title={Def.: Részbenrendezés, Szigorú részbenrendezés}]
\end{tcolorbox}

\begin{tcolorbox}[title={Def.: Teljes rendezés}]
\end{tcolorbox}

\begin{tcolorbox}[title={Def.: Részbenrendezett, vagy rendezett struktúra}]
\end{tcolorbox}

\begin{tcolorbox}[title={Def.: Diagonális reláció}]
\end{tcolorbox}

\begin{tcolorbox}[title={Def.: Szigorú-gyenge reláció}]
\end{tcolorbox}
\end{frame}

\begin{frame}

\begin{tcolorbox}[title={Def.: Zárt, nyílt intervallum}]
\end{tcolorbox}

\begin{tcolorbox}[title={Def.: Minimális Legkisebb, Maximális, Legnagyobb elem}]
\end{tcolorbox}

\begin{tcolorbox}[title={Ész}]
\end{tcolorbox}
\end{frame}


\begin{frame}
\begin{tcolorbox}[title={Def.: Alsó korlát, Felős korlát}]
\end{tcolorbox}

\begin{tcolorbox}[title={Def.: Infinum, Supremum}]
\end{tcolorbox}

\begin{tcolorbox}[title={Def.: Jólrendezett halmaz}]
\end{tcolorbox}

\begin{tcolorbox}[title={Ész}]
\end{tcolorbox}
\end{frame}

% ----------------   FÜGGVÉNYEK ------------------

\begin{frame}[plain]
\begin{tikzpicture}[overlay, remember picture]
\node[anchor=center] at (current page.center) {
\begin{beamercolorbox}[center]{title}
    {\Huge Függvények}
\end{beamercolorbox}};
\end{tikzpicture}
\end{frame}

\begin{frame}
\begin{tcolorbox}[title={Def.: Függvény, Parciális függvény}]
\tcblower
%Kapcsolódó jelölések, fogalmak
\end{tcolorbox}

%\begin{tcolorbox}[title={Mikor egyenlő két függvény?}]
%\end{tcolorbox}
\end{frame}

\begin{frame}

\begin{tcolorbox}[title={Def.: Függvények típusai}]
Szürjektív, inj...
\end{tcolorbox}

\begin{tcolorbox}[title={Ész}]
\end{tcolorbox}

\begin{tcolorbox}[title={Def.: Kanonikus leképzés}]
\end{tcolorbox}
\end{frame}

\begin{frame}

\begin{tcolorbox}[title={Def.: Monoton, Szigorúan monoton függvények}]
\end{tcolorbox}

\begin{tcolorbox}[title={Ész}]
\end{tcolorbox}
\end{frame}


\begin{frame}
\begin{tcolorbox}[title={Def.: Család, (Indexhalmaz, Indexelt halmaz, Indexelt család)}]
\end{tcolorbox}
\end{frame}

\begin{frame}

\begin{tcolorbox}[title={Def.: Kiválasztási függvény}]
\end{tcolorbox}

\begin{tcolorbox}[title={Def.: Halmazcsalád Descartes-szorzata}]
\end{tcolorbox}

\begin{tcolorbox}[title={Ész}]
\end{tcolorbox}

\begin{tcolorbox}[title={Def.: Leképzás $j$-edik projekciója}]
\end{tcolorbox}
\end{frame}

\begin{frame}

\begin{tcolorbox}[title={Def.: $n$-változós művelet}]
\end{tcolorbox}

\begin{tcolorbox}[title={Def.: Műveleti tábla, Operandus}]
\end{tcolorbox}
\end{frame}

% ---------------------- ALGEBRAI STRUKTÚRÁK, SZÁMHALMAZOK ---------------------

\begin{frame}[plain]
\begin{tikzpicture}[overlay, remember picture]
\node[anchor=center] at (current page.center) {
\begin{beamercolorbox}[center]{title}
    {\Huge Algebrai struktúrák, számhalmazok}
\end{beamercolorbox}};
\end{tikzpicture}
\end{frame}

\begin{frame}

\begin{tcolorbox}[title={Def.:Algebrai struktúrák, izomorfiájuk}]
\end{tcolorbox}

\begin{tcolorbox}[title={Def.: Műveleti zártság}]

\end{tcolorbox}

\begin{tcolorbox}[title={Def.: Grupoid}]
\tcblower
\end{tcolorbox}

\begin{tcolorbox}[title={Def.: Morfizmusok}]
\end{tcolorbox}

\begin{tcolorbox}[title={Def.: Félcsoport}]
\end{tcolorbox}

\begin{tcolorbox}[title={Def.: Baloldali, Jobboldali egységelem, Egységelem}]
\end{tcolorbox}
\end{frame}

\begin{frame}
\begin{tcolorbox}[title={Def.: Balinverz, Jobbinverz, Inverz (Félcsoport)}]
\end{tcolorbox}
\end{frame}


\begin{frame}
\begin{tcolorbox}[title={Tétel: Egységelem és inverz félcsoportban}]
Félcsoportban legfeljebb egy egységelem létezik, és minden elemnek legfeljebb egy, az egységelemre vonatkozó inverze létezik.
\tcblower
Legyen $(G, *)$ félcsoport, $e_b$ bal oldali, $e_j$ pedig jobb oldali egységelem $G$-ben.\\
Ekkor $e_b = e_j$, hiszen:\\
$e_be_j = e_j$ és $e_be_j = e_b$, (nyíl éshez $\rightarrow$ a függvény egyértelmű!)\\
mert $e_b$ bal, $e_j$ jobb oldali egységelem.\\
Ha az $a \in G$ elemnek $a_b$ balinverze, $a_j$ pedig jobbinverze, akkor $a_b = a_j$.
$a_baa_j = a_b(aa_j) = a_be = a_b$ és $a_baa_j = (a_ba)a_j = ea_j = a_j$. (Asszociatív tulajdonság) (nyíl éshez ide is $\rightarrow$ a függvény egyértelmű!).d
\end{tcolorbox}
\end{frame}

\begin{frame}
\begin{tcolorbox}[title={Def.: Abel-csoport}]
\end{tcolorbox}

\begin{tcolorbox}[title={Def.: $n$ tényezős szorzat / Hatványozás egész kitevővel}]
\end{tcolorbox}
\end{frame}

\begin{frame}

\begin{tcolorbox}[title={Def.: Gyűrűk}]

\end{tcolorbox}
\end{frame}

\begin{frame}
\begin{tcolorbox}[title={Algebrai struktúrák kpacsolata (Kép)}]
\end{tcolorbox}
\end{frame}

\begin{frame}
\begin{tcolorbox}[title={Lemma: Észrevételek gyűrűkben}]
\begin{enumerate}
\item \textbf{Szorzás nullelemmel:} Legyen 0 az R gyűrű nulleleme. Ekkor $a0 = 0a = 0$, minden $a \in R$ esetén.
\item \textbf{Előjelszabály:} Legyen R gyűrű, és $a, b \in R$. Az $a$ elem additív inverzés jelöljük $-a$-val. Ekkor $-(ab) = (-a)b = a(-b)$, tobábbá $(-a)(-b) = ab$.
\item \textbf{Véges integritási tartomány test.}
\item \textbf{Testben nincs nullosztó.}
\end{enumerate}
\end{tcolorbox}
\end{frame}

\begin{frame}
\begin{tcolorbox}[title={Lemma: Nullosztó és regularitás}]
R gyűrűben a multiplikatív művelet akkor, és csak akkor reguláris, ha R zérusosztómentes.
\end{tcolorbox}

\begin{tcolorbox}[title={Bizonyítás}]
\textbf{1. Rész}\\
Tfh $a \neq 0$, a nem bal oldali nullosztó és $ab = ac$.\\
$ab = ac  / -(ac)$ (+ additív inverz)\\
$ab + (-(ac)) = 0$. Előjel szabály + disztri.\\
$ab + (a(-c)) = a(b+(-c)) = 0$ (Kiemeljük, csak akkor lehet, ha $(b + -1 = 0) \implies (b = c)$)\\
A feltételből ($a$ nem baloldali nullosztó) következik, hogy $b + (-c) = 0)$ $\implies$\\
$\implies$ b = c.\\
\bigskip
\textbf{2. Rész}\\
Tfh $a$ bal oldali nullosztó, tehát $a \neq 0$ és létezik $b \neq 0\: ab = 0$.\\
tetszőleges $c \in R$-re: $ac = ac$.\\
$ac = ac / +0 (0 = ab)$\\
$ac = ac + ab$ /(Disztributivitás)\\
$ac = a(c + b)$ Ellentmondás!\\
Mivel $(b \neq 0) \implies (c \neq (c + b))$ (A b nem additív egységelem).
\end{tcolorbox}
\end{frame}

\begin{frame}

\begin{tcolorbox}[title={Def.: Rendezett Integritási Tartomány}]
\end{tcolorbox}

\begin{tcolorbox}[title={Def.: Egységelemes Integritási Tartomány}]
\end{tcolorbox}
\end{frame}


% ---------------------- SZÁMHALMAZOK ---------------------

\begin{frame}[plain]
\begin{tikzpicture}[overlay, remember picture]
\node[anchor=center] at (current page.center) {
\begin{beamercolorbox}[center]{title}
    {\Huge Számhalmazok}
\end{beamercolorbox}};
\end{tikzpicture}
\end{frame}


\begin{frame}

\begin{tcolorbox}
{\Huge Természetes számok}
\end{tcolorbox}
\end{frame}

\begin{frame}
\begin{tcolorbox}[title={Def.: Peano-axiómák}]
\end{tcolorbox}

\begin{tcolorbox}[title={Ész}]
\end{tcolorbox}

\begin{tcolorbox}[title={Def.: Természetes számok halmaza}]
\end{tcolorbox}
\end{frame}

\begin{frame}
\begin{tcolorbox}[title={Def.:}]
TODO Műveletek
\end{tcolorbox}
\end{frame}

\begin{frame}
\begin{tcolorbox}[title={Tétel: Természetes számok}]
A $(N, +, *)$ struktúrában mindkét művelet asszociatív, kommutatív, reguláris.\\
Nullelem (additív egységelem): 0.\\
Multiplikatív egységelem: 1.\\
A szorzat mindkét oldalról disztributív az összeadásra.\\
${\forall}m \in N : 0 * m = m * 0 = 0$.
\end{tcolorbox}
\end{frame}

\begin{frame}
\begin{tcolorbox}[title={Def.: $\mathbb{N}$ rendezése}]
\end{tcolorbox}

\begin{tcolorbox}[title={Tétel: N rendezése}]
A természetes számok halmaza a $\leq$ relációval jólrendezett.
\end{tcolorbox}

\begin{tcolorbox}[title={Def.: Végtelen sorozatok}]
\end{tcolorbox}

%\begin{tcolorbox}[title={Def.: Fibonacci számok}]
%\end{tcolorbox}

\begin{tcolorbox}[title={Ész}]
\end{tcolorbox}
\end{frame}

\begin{frame}
\begin{tcolorbox}[title={Def.: Egész számok}]
\end{tcolorbox}

\begin{tcolorbox}[title={Ész}]
\end{tcolorbox}

\begin{tcolorbox}[title={Def.: Racionális számok}]
\end{tcolorbox}

\begin{tcolorbox}[title={Ész}]
\end{tcolorbox}
\end{frame}

\begin{frame}
\begin{tcolorbox}
{\Huge Valós Számok}
\end{tcolorbox}
\end{frame}

\begin{frame}
\begin{tcolorbox}[title={Def.: Rendezett test}]
\end{tcolorbox}

\begin{tcolorbox}[title={Def.: Arkhimédészi tulajdonság}]
\end{tcolorbox}

\begin{tcolorbox}[title={Def.: Felső határ tulajdonság}]
\end{tcolorbox}
\end{frame}

\begin{frame}
\begin{tcolorbox}[title={Tétel: Felső határ és arkhimédészi tulajdonság}]
$T$ felső határ tulajdonságú test, $\implies$ $T$ arkhimédészi tulajdonságú.
\end{tcolorbox}

\begin{tcolorbox}[title={Bizonyítás (Indirekt)}]
Tfh $T$ felső határ tulajdonságú rendezett test, de nem arkhimédészi tulajdonságú.\\
$\implies : {\nexists}n \in \mathbb{N} : nx \geq y$.\\
Azaz y felső korlátja az $A = \{ nx | n \in \mathbb{N} \}$ halmaznak.\\
Ekkor viszont létezik $z = sup A$ $\implies$ $z - x < z$ nem felső korlát. $\implies$\\
$implies$ ${\exists}n : nx > z - x \implies (n + 1)x > z$. ($(n + 1)x \in A$).\\
Ellentmondás, mivel ha $n \in \mathbb{N}$, akkor $n^+ \in \mathbb{N}$ $\rightarrow$ Peano axióma!
\end{tcolorbox}
\end{frame}

\begin{frame}
\begin{tcolorbox}[title={Tétel: Q nem felső határ tulajdonságú}]
$\mathbb{Q}$ arkhimédészi tulajdonságú, de nem felső határ tulajdonságú.
\end{tcolorbox}
\end{frame}


\begin{frame}
\begin{tcolorbox}[title={Tétel: $\sqrt{2}$ nem racionális}]
Nincs $\mathbb{Q}$-ban olyan szám, amelynek négyzete 2.
\end{tcolorbox}

\begin{tcolorbox}[title={Bizonyítás (Indirekt)}]
Tfh van, és ez $x$.\\
$x = \frac{m}{n}, m,n \in \mathbb{N}^+$, és az $m$ minimális.\\
$2 = x^2 = \frac{m^2}{n^2} \implies m^2 = 2n^2$\\
Ebből következik, hogy $m$ páros. $\implies$ $m = 2k, k \in \mathbb{N}^+$\\
Ebből következik, hogy $n$ is páros: $n = 2j, j \in \mathbb{N}^+$\\
Ekkor viszont $\frac{m}{n} = \frac{2k}{2j} = \frac{k}{j}$.\\
Viszont ebből koövetkezik, hogy m nem minimális $\rightarrow$ Ellentmondás!
\end{tcolorbox}
\end{frame}

\begin{frame}
\begin{tcolorbox}[title={Def.: Valós számok halmaza}]
\end{tcolorbox}

\begin{tcolorbox}[title={Def.: néhány Függvény (?)}]
\end{tcolorbox}
\end{frame}

\begin{frame}
\begin{tcolorbox}[title={Def.: Bővített valós számok}]
\end{tcolorbox}
\end{frame}

\begin{frame}
\begin{tcolorbox}
{\Huge Komplex Számok}
\end{tcolorbox}
\end{frame}

\begin{frame}
\begin{tcolorbox}[title={Def.: Komplex számok}]
\end{tcolorbox}

\begin{tcolorbox}[title={Ész}]
\end{tcolorbox}

\begin{tcolorbox}[title={Alakok}]
\end{tcolorbox}

\begin{tcolorbox}[title={Ész}]
\end{tcolorbox}
\end{frame}

\begin{frame}
\begin{tcolorbox}[title={Def.: Moivre azonosságok}]
\end{tcolorbox}

\begin{tcolorbox}[title={Def.: Gyökvonás komplex számokból}]
\end{tcolorbox}

\begin{tcolorbox}[title={Def.: $n$-edik primitív egységgyökök}]
\end{tcolorbox}
\end{frame}

\begin{frame}
\begin{tcolorbox}[title={Tétel: Az algebra alaptétele}]
Ha $n \in \mathbb{N}^+$, valamint $c_0, c_1, ... c_n$ komplex számok, $c_n \neq 0$, akkor van olyan $u$ komplex szám, amelyre:\\
$$\sum_{k = 0}^n c_kz^k = 0$$
\end{tcolorbox}
\end{frame}

\begin{frame}[plain]
\begin{tikzpicture}[overlay, remember picture]
\node[anchor=center] at (current page.center) {
\begin{beamercolorbox}[center]{title}
    {\Huge Számelmélet}
\end{beamercolorbox}};
\end{tikzpicture}
\end{frame}

\begin{frame}
\begin{tcolorbox}[title={Oszthatóság egységelemes integritási tarományban (Emlékeztető)}]
\end{tcolorbox}
\end{frame}

\begin{frame}
\begin{tcolorbox}[title={Tétel: Az oszthatóság tulajdonságai EIT-ban}]
\begin{enumerate}
\item Ha $b|a$ és $b'|a'$, akkor $bb'|aa'$.
\item A nullának minden elem osztója.
\item A nulla csak saját magának osztója.
\item Az 1 egységelem minden elemnek osztója.
\item Ha $b|a$, akkor $bc|ac$ minden $c \in R$-re.
\item Ha $bc|ac$ és $c \neq 0$, akkor $b|a$.
\item Ha $b|a_i$ és $c_i \in R, (i = 1, 2, ..., j)$, akkor $b|\sum^j_{i=1} c_ia_i$.
\item Az $|$ reláció reflexív, és tranzitív.
\end{enumerate}
\end{tcolorbox}
\end{frame}

\begin{frame}
\begin{tcolorbox}[title={Tétel: Prím és irreducibilis elem EIT-ban}]
Tetszőleges $R$ egységelemes integritási tartományban minden $p$ elemre:\\
Ha $p$ prím $\implies$ $p$ felbonthatatlan.
\end{tcolorbox}

\begin{tcolorbox}[title={Bizonyítás}]
Tfh $p$ prím, és, $p = bc$\\
Ekkor vagy $p|b$, vagy $p|c$\\
$b = pq = b(cq) \implies cq = 1$ $\implies$ $c, q$ egység $p, b$ asszociáltak.
\end{tcolorbox}
\end{frame}

\begin{frame}
\begin{tcolorbox}[title={Def.: Legnagyobb közös osztó}]
\end{tcolorbox}
\end{frame}

\begin{frame}
\begin{tcolorbox}[title={Tétel: Maradékos osztás $\mathbb{Z}$-ben}]
${\exists}a, b({\neq}0) \in \mathbb{Z}$ számhoz egyértelműen létezik olyan $q, r \in \mathbb{Z}$, hogy\\
$a = qb + r \land 0 \leq r < |b|$.
\end{tcolorbox}
\end{frame}

\begin{frame}
\begin{tcolorbox}[title={Tétel: Prím és irreducibilis elem $\mathbb{Z}$-ben}]
Az egész számok körében $p$ prím $\iff$ $p$ felbonthatatlan.
\end{tcolorbox}

\begin{tcolorbox}[title={Bizonyítás}]
Már láttuk, hogy prím felbonthatatlan!\\
Tfh p felbonthatatlan\\
Legyen $p|bc$, ekkor vagy $p | b$-nek, ekkor ksz vagyunk.
Vagy $p \nmid b$ ekkor $(p,b) = 1$.\\
$c = pcx +bcx \implies 0 mod p \implies p | c$.\\
(Észrevétel: $(a, b) = 1 \land a | bc \implies a | c$

\end{tcolorbox}

\end{frame}

\begin{frame}

\begin{tcolorbox}[title={Tétel: A számelmélet alaptétele}]
Minden $m$ nemnulla, nemegység, egész szám sorrendre és asszociáltásgra való tekintet nélkül egyértelműen bontható fel felbonthatatlanok szorzatára.
\end{tcolorbox}

\begin{tcolorbox}[title={Bizonyítás (Pozitívakra)}]
\textbf{(egzisztencia)}\\
Tfh $n > 1$\\
Teljes indukció: $n = 2$ kész, tfk $n - 1$-ig kész.\\
Ha $n$ felbonthatatlan $\rightarrow$ kész.\\
Ha $n$ nem felbonthatatlan $\rightarrow$ $n = ab \land a, b$ (a, b nem egység!), $a, b < n$ $\implies$ igaz rájuk az ind. feltétel.\\
$n$ felbontása $=$ $a$ felbontása szor $b$ felbontása.\\
\bigskip
\textbf{(unicitás) (Indirekt)}\\
Tfh $n$ a legkisebb olyan szám, amely felbontása nem egyértelmű.\\
$n = p_1 ... p_k = q_1 ... q_r$ $\implies$\\
$p_j|n \implies p_1|q_1 ... q_r$\\
$p_1|q_1$, $p_1|q_2 ... q_r$\\
           $p_1|q_2   p_1|q_3 ... q_r$\\
                      $p_1|q_i  \implies p_1 = q_i \implies$\\
$\implies$ $n_1 = \frac{n}{p_1} = p_2 ... p_k = q_1 ... q_{i-1}q_{i+1} ... q_r$\\
$n_1 < n$ és van két lényegesen különböző felbontása!
\end{tcolorbox}

\end{frame}

\begin{frame}

\begin{tcolorbox}[title={Tétel: Eukleidész tétele}]
Végetlen sok prímszám van.
\end{tcolorbox}

\begin{tcolorbox}[title={Bizonyítás (Indirekt)}]
Tfh véges sok van:\\
$p_1, p_2, ... ,p_k$.\\
Legyen $n = p_1p_2...p_k$.\\
Számelmélet alaptételéből következik hogy létezik $p_j : p_j | n + 1$\\
$p_j : p_j | n + 1 \implies p_j | 1$ Ellentmondás!
\end{tcolorbox}
\end{frame}

\begin{frame}
\begin{tcolorbox}[title={Def.: Kanonikus alak, Módosított kanonikus alak}]
\end{tcolorbox}

\begin{tcolorbox}[title={Def.: Erathosztenész SZitája}]
\end{tcolorbox}
\end{frame}

\begin{frame}

\begin{tcolorbox}[title={Def.: Lineáris Kongruencia}]
$a \equiv b \pmod{m}$, ha $m | a - b$.
\end{tcolorbox}

\begin{tcolorbox}[title={Tétel: Kongruencia tulajdonságai}]
\begin{enumerate}
\item Ekvivalencia reláció
\item $a \equiv b \pmod{m} \land c \equiv d \pmod{m} \implies$ \textbf{$a + c \equiv b + d \pmod{m}$}
\item $a \equiv b \pmod{m} \land c \equiv d \pmod{m} \implies$ \textbf{$ac \equiv bd \pmod{m}$}
\item $a \equiv b \pmod{m} \land f(x) \in z[x] \implies$ \textbf{$f(a) \equiv f(b) \pmod{m}$}
\item Ha $(c, m) = d$, $ac \equiv bc \pmod{m} \iff a \equiv b \pmod{\frac{m}{d}}$
\end{enumerate}
\end{tcolorbox}

\begin{tcolorbox}[title={Ész}]
\end{tcolorbox}
\end{frame}

\begin{frame}
\begin{tcolorbox}[title={Def.: Az Euler-féle $\phi$ függvény}]
\end{tcolorbox}

\begin{tcolorbox}[title={Def.: A $\tau$ függvény}]
\end{tcolorbox}
\end{frame}

\begin{frame}
\begin{tcolorbox}[title={Def.: TMR, RMR}]
\end{tcolorbox}
\end{frame}

\begin{frame}
\begin{tcolorbox}[title={Tétel: Omnibusz tétel}]
Legyen: $m > 1$ egész, $\{a_1, ..., a_m\}$ TMR modulo $m$, $\{b_1, ..., b_{{\phi}(m)}\}$ RMR modulo $m$, $c, d \in \mathbb{Z}$, és $(c,m) = 1$.\\
\smallskip
Ekkor:\\
\smallskip
$\{ ca_1 + d, ..., ca_m + d \}$ TMR modulo $m$\\
$\{ cb_1, ..., cb{{\phi}(m)}\}$ RMR modulo $m$
\end{tcolorbox}

\begin{tcolorbox}[title={Bizonyítás (Indirekt)}]
Tfh van két nem inkongruens elem\\
$ca_i + d = ca_i + d$\\
${\cancel{c}}a_i + {\cancel{d}} = {\cancel{c}}a_i + {\cancel{d}}$ $(c, m) = 1$, és pontosan $m$ db elem!\\
$(c, m) = 1$ és $(b_j,m) = 1$ $\implies$ $(cb_j, m) = 1$
\end{tcolorbox}
\end{frame}

\begin{frame}
\begin{tcolorbox}[title={Tétel: Euler-Fermat tétel}]
Legyen $m > 1$ egész és $a$ relatív prím $m$-hez. Ekkor $a^{{\phi}(m)} \equiv 1 \pmod{m}$
\end{tcolorbox}

\begin{tcolorbox}[title={Bizonyítás}]
Legyen $\{ r_1, ..., r_{{\phi}(m)}\}$ RMR modulo $m$, $(a, m) = 1$.\\
Az omnibusz tétel miatt, ekkor $\{ ar_1, ..., ar_{{\phi}(m)}\}$ is RMR modulo $m$.\\
Megfelelő párosítás $\implies$ $r_i \equiv ar_j \pmod{m}$.\\
Összehozva: $(r_i, m) = 1$\\
\smallskip
$$a^{{\phi}(m)} \prod^{{\phi}(m)}_{i=1} r_i \equiv \prod^{{\phi}(m)}_{i=1} r_i \pmod{m}$$

\end{tcolorbox}

\end{frame}

\begin{frame}
\begin{tcolorbox}[title={Tétel: (Kis) Fermat tétel}]
Legyen $p$ prím és $a \in \mathbb{Z}$. Ekkor\\
(első alak) ha $p \nmid a$, akkor $a^{p-1} \equiv 1 \pmod{p}$.\\
(második alak) ha $a$ tetszőleges, akkor $a^p \equiv a \pmod{p}$.

\end{tcolorbox}

\begin{tcolorbox}[title={Bizonyítás}]
Első alak: ${\phi}(p) \equiv p - 1$ $\rightarrow$ előző tétel miatt kész.\\
\bigskip
Második alak:\\
Ha $p|a$ $\rightarrow$ $0 \equiv 0$ $\rightarrow$ kész.\\
Ha $p{\nmid}a$ $\rightarrow$ ekkor ez az első alak $\rightarrow$ kész.

\end{tcolorbox}
\end{frame}

\begin{frame}
\begin{tcolorbox}[title={Tétel: A diofantikus egyenlet megoldása}]
Rögzített $a, b, c$ egész számok esetén az \textbf{$ax + by = c$} diofantikus egyenletnek akkor, és csak akkor van megoldása, ha $(a, b)|c$.
\end{tcolorbox}

\begin{tcolorbox}[title={Bizonyítás}]
Tfh $ax + by = c$ egyenletnek van megoldása.\\
\textbf{1. Rész ($\implies$)}\\
\smallskip
Tfh $x_0, y_0$ megoldás. $\implies$ $(a, b)|a \land (a, b)|b$ $\implies$ lin. kombinációs tul. $\implies$\\
$\implies$ $(a, b)|ax_0 + by_0 = c$ (Igaz, mert az a, b osztója az $ax_0 + by_0$-nak.)\\
\bigskip
\textbf{2.Rész ($\Longleftarrow$)}\\
\smallskip
Tfh (a, b)|c. Ekkor:\\
$c = (a, b)q$\\
$c = (au + bv)q$\\
$c = a(uq) + b(vq)$\\
$c = a(uq) + b(vq)$ $\implies$ egy megoldás: $x = uq, y = vq$.\\
\end{tcolorbox}
\end{frame}

\begin{frame}
\begin{tcolorbox}[title={Tétel: Kínai maradéktétel}]
Legyen $n \in \mathbb{N}^+, m_1, m_2, ..., m_n \in \mathbb{N}^+, a_i, b_i \in \mathbb{Z} (1 <leq i \leq n)$, ahol
\begin{enumerate}
\item $m_i, m_j$ páronként relatív prímek.
\item $(m_i, a_i) = 1$, minden $1 \leq i \leq n$ esetén.
\end{enumerate}
Ekkor az\\
\bigskip
	$a_1x \equiv b_1 \pmod{m_1}$\\
	$a_2x \equiv b_2 \pmod{m_2}$\\
	...\\
	$a_nx \equiv b_n \pmod{m_n}$\\
\bigskip
Kongruenciarendszer megoldható és bármely két megoldása kongruens modulo $m_1m_2...m_n$.
\end{tcolorbox}

\begin{tcolorbox}[title={Def.: A kínai maradéktétel megoldása}]
\end{tcolorbox}
\end{frame}

\begin{frame}

\begin{tcolorbox}[title={Def.: A számelméleti függvények}]
\end{tcolorbox}
\begin{tcolorbox}[title={Tétel: Számelméleti függvények}]
Legyen $n \in \mathbb{N}^+$ kanonikus alakja $p_1^{{\alpha}_1}...p_k^{{\alpha}_k}$. Ekkor:\\
\begin{enumerate}
\item Ha $f$ additív számelméleti függvény, akkor $$f(n) = f(p_1^{{\alpha}_1}) + ... + f(p_k^{{\alpha}_k})$$
\item Ha $f$ multiplikatív számelméleti függvény, akkor $$f(n) = f(p_1^{{\alpha}_1})...f(p_k^{{\alpha}_k})$$
\item Ha $f$ teljesen additív számelméleti függvény, akkor $$f(n) = {\alpha}_1f(p_1) + ... + {\alpha}_kf(p_k)$$
\item Ha $f$ teljesen multiplikatív számelméleti függvény, akkor $$f(n) = f(p_1)^{{\alpha}_1}...f(p_k)^{{\alpha}_k}$$
\end{enumerate}
\end{tcolorbox}
\end{frame}

\begin{frame}
\begin{tcolorbox}[title={Tétel: $\phi$ multiplikativitása}]
$\phi$ multiplikatív.
\end{tcolorbox}

\begin{tcolorbox}[title={Bizonyítás}]
\smallskip
\begin{tabular}{c c c c}
1 & 2 & ... & a \\
a + 1 & a + 2 & ... & 2a\\
 &  & ... &  \\
(b - 1)a + 1 & (b - 1)a + 2 & ... & ba
\end{tabular}
\smallskip
Számoljuk meg, hogy a táblázatban hány relatív prím van $ab$-hez: ennyi lesz ${\phi}(ab)$ értéke.\\
(Ha $a$ is $b$ is relatív prím $c$-hez, akkor $ab$ is. $\implies$ azokat kell számolni, amelyek $a$-hoz és $b$-hez is rel. prímek)\\
\smallskip
AZ Omnibusz tételből következik hogy minden oszlop TMR mod $b$, ha $(a, b) = 1$ $\implies$\\
$\implies$ minden oszlopban ${\phi}(b)$ rel. prím $b$-hez.\\
\smallskip
| Minden oszlom kongruens elemeket tart mod $a$.\\
| Minden sor egy TMR mod $a$ $\implies$ minden sorban ${\phi}(a)$ db elem relatív prím $a$-hoz.\\
$\implies$ ${\phi}(a)$ db oszlopnak rel prímek az elemei $a$-hoz. $\implies$ összesen ${\phi}(a){\phi}(b)$ rel. prím van $ab$-hez.
\end{tcolorbox}
\end{frame}

\begin{frame}
\begin{tcolorbox}[title={Tétel: ${\phi}$(n) kiszámolása}]
Ha $n \in \mathbb{N}^+$ kanonikus alakja $p_1^{{\alpha}_1}...p_k^{{\alpha}_k}$, akkor\\
$${\phi}(n) = \prod^k_{j=1} (p_j^{{\alpha}_j} - p_j^{{\alpha}_j - 1}) = n \prod^k_{j=1} (1 - \frac{1}{p_j}).$$
\end{tcolorbox}

\begin{tcolorbox}[title={Bizonyítás}]
$\phi$ multiplikatív\\
Kiszámoljuk az értékeket prímhatványhelyeken, majd összeszorozzuk az értékeket.\\
${\phi}(p^{\alpha}) = ?$\\
$1, 2, ..., p, ..., 2p, ..., 3p, ..., (p-1)p, ..., p^2, ..., (p+1)p, ..., (p-1)p^{{\alpha}-1}, ..., p^{\alpha}$\\
Melyek nem relatív prímek $p$-hez?\\
\smallskip
$p^2$-ig $p - 1$ db van + maga $p^2$, azaz ${\phi}(p^2) = p^2 - p^1$.\\
Tovább számolva:\\
${\phi}(p^{\alpha}) = p^{\alpha} - p^{\phi - 1}$
\end{tcolorbox}
\end{frame}

\begin{frame}[plain]
\begin{tikzpicture}[overlay, remember picture]
\node[anchor=center] at (current page.center) {
\begin{beamercolorbox}[center]{title}
    {\Huge Kombinatorika}
\end{beamercolorbox}};
\end{tikzpicture}
\end{frame}

\begin{frame}

\begin{tcolorbox}[title={Def.: Halmazok ekvivalenciája}]
\end{tcolorbox}

\begin{tcolorbox}[title={Tétel: Véges halmaz valódi részhalmaza}]
Ha $n$ természetes szám, akkor nem létezik ekvivalencia $\{1, 2, ..., n\}$ és egy valódi részhalmaza között.
\end{tcolorbox}
\end{frame}

\begin{frame}
\begin{tcolorbox}[title={Tétel: Skatulya-elv}]
Ha $X, Y$ véges halmazok, és $|X| > |Y|$, akkor nem létezik $f: X \rightarrow Y$ bijekció.
\end{tcolorbox}

\begin{tcolorbox}[title={Bizonyítás (Indirekt)}]
Tfh $f$ bijektív.\\
$Y ~ \{1, 2, ..., m\}$ és $X ~ \{1, 2, ..., m\}$, ahol $m < n$ $\implies$\\
$\implies$ $\{1, 2, ..., m\}$ bármely részhalmaza $\{1, 2, ..., n\}$-nek is részhalmaza,\\
$f$ bijektív $\implies$ $\{1, 2, ..., n\}$ $~$ saját valódi részhalmazával. $\rightarrow$ Ellentmondás!\\
\bigskip
\textbf{Más megfogalmazás:} Ha $n$ db tárgyat $m$ db skatulyába teszünk, akkor van olyan skatulya, amelyik legalább\\
$\lfloor (n - 1) / m \rfloor + 1$ tárgyat tartalmaz.
\end{tcolorbox}
\end{frame}

\begin{frame}
\begin{tcolorbox}[title={Def.: Permutáció}]
\end{tcolorbox}

\begin{tcolorbox}[title={Tétel: Permutációk száma}]
$$P_n = n!$$
\end{tcolorbox}

\begin{tcolorbox}[title={Bizonyítás}]
Teljes indukció $n$ szerint\\
1. lépés: $P_0 = P_1 = 1$ Igaz. (Megegyezés szerint $0! = 1$)\\
2. lépés: Tfh $n > 1$ és $n - 1$-ig már beláttuk.\\
ekvivalencia reláció:\\
amely sorozatok 1. eleme megegyezik $\implies$ $n$ db osztály.\\
Ind. feltétel $\implies$ $\forall$ osztályban $P_{n - 1}$ elem.\\
$P_n = nP_{n - 1} = n(n - 1)! = n!$
\end{tcolorbox}

\begin{tcolorbox}[title={Def.: Ciklikus permutáció}]
\end{tcolorbox}
\end{frame}

\begin{frame}
\begin{tcolorbox}[title={Def.:Ismétlés nélküli variáció}]
\end{tcolorbox}

\begin{tcolorbox}[title={Tétel: Variációk száma}]
$$V_n^k = \frac{n!}{(n - k)!} = n * (n - 1) * (n - 2) * ... * (n - k + 1)$$, ha $k \leq n$, kölünben 0.
\end{tcolorbox}

\begin{tcolorbox}[title={Bizonyítás}]
Legyen két sorozat egy ekv. osztályban, ha első $k$ elemük megegyezik.\\
Ekkor: $P_n = $ (osztályok száma ($V_n^k = \frac{P_n}{P_n - k}$))*(ahány elem egy osztályban ($P_{n - k}$)
\end{tcolorbox}
\end{frame}

\begin{frame}
\begin{tcolorbox}[title={Def.: Ismétléses Variáció}]
\end{tcolorbox}

\begin{tcolorbox}[title={Tétel: Ismétléses variációk száma}]
$$V_n^{k, i} = n^k$$
\end{tcolorbox}

\begin{tcolorbox}[title={Bizonyítás}]
Teljes indukció $k$ szerint, $n$ rögzített\\
1. lépés: $k = 1$-re igaz: $V_n^{1, i} = n \rightarrow n^1$\\
2. lépés: Tfh $k > 1$ és $k - 1$-ig már beláttuk, ekkor\\
$(k - 1)$-es osztályú variációból $k$-ad osztályú:\\
$n$ db választás $\implies$ $V_n^{k, i}$ (n választás) $= V_n^{k - 1, j} * n$ (n - 1 választás).
\end{tcolorbox}
\end{frame}

\begin{frame}
\begin{tcolorbox}[title={Def.: Ismétlés nélküli Kombináció}]
\end{tcolorbox}

\begin{tcolorbox}[title={Tétel: Kombinációk száma}]
$$C_n^k = {{n}\choose{k}} = \frac{n!}{k!(n - k)!} $$, ha $k \neq n$, különben 0.
\end{tcolorbox}

\begin{tcolorbox}[title={Bizonyítás}]
$V_n^k$ db különböző $k$-tagú sorozat, sorrend nem számít $\implies$\\
$\implies$ minden $P_k$ sb sorozat ugyanaz $\implies$ számoljuk egyszer.\\
\end{tcolorbox}
\end{frame}

\begin{frame}
\begin{tcolorbox}[title={Def.: Ismétléses Kombináció}]
\end{tcolorbox}

\begin{tcolorbox}[title={Tétel: Ismétléses kombinációk száma}]
$$C_n^{k, i} = C_{n + k -1}^k = {{n + k - 1}\choose{k}} = \frac{(n + k - 1)!}{k!((n + k - 1) - k)!} $$
\end{tcolorbox}

\begin{tcolorbox}[title={Bizonyítás}]
Legyen $A = \{a_1, a_2, ..., a_n\}$.\\
MInden egyes választási lehetőségnek feleltessünk meg egy bitsorozatot:\\
$1 1 ... 1 0 1 1 ... 1 0 ... 0 1 1 1 ... 1$\\
$k_1$ db,    $k_2$ db,         $k_3$ db 1es.\\
Az $a_i$ elemet $k_i$-szer választottuk, tehát $k_1 + ... + k_n = k$ az összes $1$-es száma (ennyi elemet választottunk összesen).\\
Továbbá az elválasztó $0$k száma $n - 1$, tehát a sorozatban $n - 1 + k$ pozíció lesz\\
\mmedskip

Ekkor a $k$ db $1$-es beírása $k$ különböző pozícióba nem más, mint $k$ db választás egy $n - 1 + k$ elemű halmazból ismétlés nélkül:\\
$$C_{n + k - 1}^k = {n + k - 1 \choose k}$$
\end{tcolorbox}
\end{frame}

\begin{frame}

\begin{tcolorbox}[title={Def.: Ismátláses Permutáció}]
\end{tcolorbox}

\begin{tcolorbox}[title={Tétel: Ismétléses permutációk száma}]
\end{tcolorbox}

\begin{tcolorbox}[title={Bizonyítás}]
\end{tcolorbox}
\end{frame}

\begin{frame}
\begin{tcolorbox}[title={Tétel: Binomiális tétel}]
\end{tcolorbox}

\begin{tcolorbox}[title={Bizonyítás}]
\end{tcolorbox}
\end{frame}

\begin{frame}
\begin{tcolorbox}[title={Tétel: Logikai szita formula}]
\end{tcolorbox}

\begin{tcolorbox}[title={Bizonyítás}]
\end{tcolorbox}
\end{frame}

\end{document}
