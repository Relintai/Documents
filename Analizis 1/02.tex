
(f o g)(x) = f(g(x))

pl: 
g(x) = 5x
f(y) = \sqrt{3}{y}

y -(g)-> 5x -> \sqrt{3}{5x}

konstans sorozat (minden eleme ugyan az)

rekurzív sorozat

x_0 \in \mathbb{R} adott, f fv adott

x_{n + 1} = f(x_n)

f(x) = \frac{x}{2} + \frac{1}{x}

x_{n + 1} = \frac{x_n}{2} + \frac{1}{x_n}

Egy
-Sorozat növő, ha x_{n + 1} >qeq x_n
-Korlátos x_n, ha \exists K \in \mathbb{R} : |x_n| \leq K
-alulról korlátos, ha - || -                : |x_n| \geq K
\forall n \in \mathbb{N}

Tétel:
\forall sorozatnak van monoton részsorozata.

Biz:
Csúcs:
\forakk k \geq n : x_k \leq x_n

Két eset:

1.
Végtelen sok csúcs van

1. első: x_{\nu_0}
\forall k \geq \nu_0 x_k \leq \nu_0

spec: k = \nu_1 := következő csúcs
x_\nu_1 \leq x_\nu_0, és így tovább

... \leq X_\mu_2 \leq x_\mu_1 \leq x_\mu_0 monoton fogyó sorozat


2. eset
Véges sok csúcs van

!N_0 \in \mathbb{N} az első olyan index, ami már nem csúcs

\nu_0 := N_0
x_\nu_0 := x_\nu_0 (\nu_1 := ez a k)
x:\nu_1 := x_kx_\nu_2 :=..

ekkor

\exists k \geq N_0, x_k > x_N_0 


pl.:

x_n := (-1)^n 

a sorozat:

-1, 1, -1, 1, -1, 1, -1, 1 ...

attól lessz valami csúcs, hogy az utána jövő bármelyik elem legalább akkora mint ők.

! (i)N \subset \mathbb{N} tetszőleges:

pontosan (i)N elemei csúcsok

x_n = I  1, ha n \in (i)N
      I  a - \frac{1}{n + 1}, egyébként

1 - \frac{1}{n + 1} < 1

max-ok csúcsok; pl.: monoton fogyó sorozatok

Def konvergencia:

Amh an (x_n) sorozat konvergens és a határértéke \alpha , ha
\exists \epsilon > 0 : \forall N_0 \in \mathbb{N} köszübindex, hogy \forall n \geq N_0 esetén
| x_n - \alpha | < \epsilon
Jel.: lim x_n = \alpha, lim x_n = \alpha, x_n - (n -> + \infty) -> \alpha

Pl.:

x_n := \frac{1}{n}
x_n -(n->+\infty) 0

Kellene: |x_n - 0|

Tétel:
Ha x_n konvergens, akkor lim x_n egyértelmű.

Biz.: (ind)
tfh a ef nem egyértelmű, és 2 külön számra is működik

\exists sorozat, úgy hogy
\forall \epsilon > 0 \exists N_0 \in \mathbb{N} |x_n - \alpha | < \epsilon
\forall n \geq N_0

\forall \epsilon > 0 \exists M_0 \in \mathbb{N} |x_n - \beta | < \Äpsilon

Legyen
(felultilde) N_0 := max \{ N_0, M_0 \}

\epsilon := \frac{|\Beta - \Alpha |}{2}

prec.:

|\alpha - \beta | = | \alpha - x_n - (\beta - x_n)| \leq |\alpha - x_n| + |\Beta - x_n| <
< \frac{}

kész, mert azt kaptuk, hogy egy szám nagyobb önmagánál

Áll.:
Ha egy sorozat konvergens \Roghtarrow \forall részsorozata is konvergens (t n.a. a limesz)

Biz.:
\forall \epsilon > 0 \exists N_0 \in \mathb{N} \forall n \geq N_0

|x_n - \alpha | \epsilon

ha vesszük ennek \forall \nu_n indexsorozatta indexelt részsorozatát, bármelyikre igaz, hogy \nu_n \geq n
\Rightarrow |x_\nu_k - \Alpha | < \epsilon is igaz (kész)



Tétel:
\forall monoton és korlátos sorozat konvergens.
növő + felülről korlátos \RIghtarrow lim x_n = sup \{ x_n : n \in \mathbb{N} \} = sup R_x
fogyó + alulról korlátos \Rightarrow lim x_n = inf \{ x_n : n \in \mathbb{N} \} = inf R_x

Biz.:

\Exists \alpha : \forall \epsilon > 0
\exists N_0 \in \mathbb{N} \foralln \geq N_0 | x_n - \alpha| < \epsilon
(kellene) \alpha := sup \{ x_n : n \in \mathbb{N} \} = sup H, ahol H := \{ x_n : n \in \mathbb{N} \}

\forall \epsilon > 0: \exists m: \alpha - \epsilon < x_m < \alpha



Def.:
Amh az \alpha \in \mathbb{R} r > 0 - sugarú környezetén a K_r(\alpha) := \{ x \in \mathbb{R} : |x - \alpha | < r \}

K_r(\alpha) = (\alpha - r, \alpha + r)
K_\epsilon(\alpha) = (\alpha - \epsilon, \alpha + \epsilon) = 
= \{ x \in \mathbb{R} : | x - \alpha | < \epsilon \}

Amh x_n sorozat az \alpha számhoz konvergál, ha \forall > 0, \exists N_0 \in \mathb{N}
\forall m \geq N_0   x_n \in Kr(\alpha)

Tétel \Forall konergens sorozat korlátos is.

Biz.:
|x_n . \alpha| < \epsilon  "elég nagy m-re" /\esists N_0 \in \mathbb{N} n \geq N_0/

spec.: \epsilon = 1
|x_n - \alpha | < 1

Kell.:
|x_n \leq K  |x_n| = |x_m - \alpha + \alpha| \leq |x_n - \alpha| + |\alpha| < 1 + |\alpha| (elég nagy m-re)

K := max \{ 1 + |\alpha|, |x_0|, |x_1|, ..., |x_N_0| \} (kész)

Tétel.: (Bolzano- Weierstrass)
\forall korlátos sorozatnak \exists konvergens részorozta.

Biz.:
\forall sorozatnak van monoton részsorozata.
\forall korlátos részsorozata is korlátoss. \Rightarrow \exists monoton és korlátos részsorozat \Rightarrow ez konvergens







