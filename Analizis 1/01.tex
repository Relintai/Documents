
1. Valós számok axiómái:
R \neq \emptyset  (sima R, mert halmaz)

A.
\exists \theta \epsilon \in R, \forall x, y \in R
x + y \in R
xy \in R
tetszőleges a, b, c \in R esetén:
1. a + b = b + a
2. (a + b) + c = a + (b + c), (ab)c = a(bc) etc (6 db)

B.
Adott R halmazon \leq teljes rendezés (reflexív, tranzitív, antiszimmetrikus) reláció

(teljes rendezés: a \leq b és b \leq a relációk közül az egyik legalább teljesül.)

C.
a \leq b

(i) a + c \leq b + c
(ii) c \geq 1 \Rightarrow ac \ leq bc
a (c \geq 1 nél az 1 az előbbi \theta, egységelem)

D.
Minden felülről korlátos halmaznak \exists legkisebb felső korlátja

(vesszük a felső korlátok halmazát)
(írott) K := {(sima) K \in R: x \leq (sima) K (x \in A)}

(legyen) !A \neq \emptyset, felölről korlátos, ha {\exists}M,

a legkisebb felső korlát = sup A (az A halmaz supremuma).

Ehhez hasonlóan:

Minden aluról korlátos halmaznak \exists legnagyobb alsó korlátja.
Ez az inf A (az A halmaz infinuma).


A - D, \RIghtarrow számtest, (\mathbb) Z, R, Q, C, R^n

!A \neq \emptyset felölről korlátos, \forall x \in A: x \leq sup A := \alpha \iff
\iff \forall \epsilon > 0: "\aplha - \epsilon már nem supremum" azaz \exists y \in A: \alpha > y \geq \alpha - \epsilon

példa:

A:= { -\frac{1}{n}: n \in \mathbb{N}, n \geq 1}

azaz: -1, -1/2, -1/3, -1/4, ....   \leq 0 = sup A

Valüs számok alaptételei:

1. Tétel (1.3.1):

Legyen x, y \in \mathbb{R} x > 0:

\forall y \in \mathbb{R} esetén \exists n \in \mathbb{N}, hogy nx > y

Biz.: (Indirekt)

Tfh.: \exists x, y \in \mathbb{R}, x > 0 és \forall n \in \mathbb{N} nx \leq y

Ekkor létezik a halmaznak legkisebb felső korlátja, legyen ez az \alpha :
A := { nx : n \in \mathbb{N}}
\alpha := sup A

ekkor:
\exists \epsilon > 0 : \exists x \in A, hogy nx = y > \alpha - \epsilon

\epsilon := x \Rightarrow nx > \aplha - x =
(n + 1)x > \aplha \Rightarrow ELLENTMONDÁS

2. Tétel (Dedekind) (1.3.2)

A, B \neq \emptyset halmazok, \forall a \in A és \forall b \in B esetén:
a \leq b \Rightarrow \exists \gamma \in \mathbb{R}
a \leq \gamma \leq b tetszőleges a \in A és b \in B esetén.

Biz.:

(írott) K := { (sima)K \in \mathbb{R}: x \leq (sima) K (x \in A)} (azaz az A halmaz felölről korlátos halmaz.)

min (írott) K = sup A : (legyen sup A = {\gamma}): a \leq \gamma
viszont a \gamma \leq b, mivel a B (részhalmaza) (írott) K-nak.

3. Tétel (Cantor):

"Bármely egymásba skatulyázott zárt intervallumsorozatnak \exists közös pontja."

Legyenek [a_n, b:n] intervallumok adottak, az [a_{n+1}, b_{n+1}] (részhalmaz =) [a_n, b_n] \Rightarrow

\bigcap_{n = 0}^{+ \infty} [a_n, b_n] \neq \emptyset

Biz.:
A := {a_n : n \in \mathbb{N}}
B := {b_n : n \in \mathbb{N}}

Egymásba skatulyázott intervallumok, azaz a_n \leq b_n
a_l \leq b_m triviális

\Rightarrow Dedekind \Rightarrow \exists \gamma \in \mathbb{R}, hogy \gamma \in \bigcap_{n = 0}^{+ \infty} [a_n, b_n] \neq \emptyset

1.3.6 Tétel (négyzetgyök):

{\exists}! \aphy \in \mathbb{R}, \alpha > 0: \alpha^2 = 2

Biz:

A := { x > 0: x^2 \leq 2 } (A felölről korlátos.)

\forall x \leq 2
(x \in A)


felülről korlátos: (Ha \exists x \in A: x > 2, x \cdot x > 4, x^2 \leq 2 kellene.)

legyen \alphy := sup A.

áll.: {\alpha}^2 = 2 (\alpha \in A)


tfh (indirekt):
{\alpha}^2 \neq 2, {\alpha}^2 < 2, {\alpha}^2 > 2

\exists \epsilon > 0: {\alpha} + {\epsilon} nem a legkisebb felső korlát.

(\aplha + \epsilon)^2 = \alpha^2 + 2\alpha \epsilon + \epsilon^2 < 2 \rightarrow ELLENTMONDÁS


Áll.:
\forall a, b > 0 (a, b) tartalmaz racionális számot:

Biz.:
vegyük az a számot, hogy b - a.
b - a biztosan > 0.

Arkhim.:
\exists n \in \mathbb{N}: n(b - a) > 1

b - a > 1 / n := x   *


\exists M \in \mathbb{N}: mx > a
mx = m / n

!A := { m \in \mathbb{N}: m/n > a }


\exists p \in A, p legkisebb: p/n > a
p-1/n < leq a

p/n = p-1/n + 1/n < p-1/n \leq (= lehúzva) a + *(b - a)
(* szerint)
nem teljesülhet az egyenlőség.


3. Relációk:

Def (Descarted szorzat):
(legyen)!A, B \neq \emptyset : A x B

c = { (a, b): a \in A, b \in B }
(a, b) := { {a}, {a, b}}

Def.:

Az A x B tetszőleges részhalmazát relációnak nevezzük.
(relációk a függvények általánosításai)

Def.:

!r (részh) A x B reláció

Ért. Tartomány: D_r := {x \in A : \exists y, (x, y) \in r }
Értékkészlet: R_r := {y \in B : \exists x \in A, (x, y) \in r}

Def.:

Amh. az r (részh) A x B reláció homogén, ha A = B.

Ekkor amh a reláció:

(i) Szimmetrikus: (a, b) \in r \rightarrow (b, a) \in r
(ii) reflexív: (a, a) \in r
(iii) tranzitív: (a, b) \in r, és (b, c) \in r \Rightarrow (a, c) \in r
(iv) Antiszimmetrikus: (a, b) \in r és (b, a) \in r \Rightarrow a = b

(ii) + (i) + (iii) => ekvivalencia reláció.
(ii)+(iii)+ (iv) => rendezési reláció, teljes, ha (a, b) \in r, ls (a, b) \in r közül legalább az egyik fennáll.

Függvények:

!f (részh) A x B reláció, x \in A

f_x := {y \in B : (x, y) \in f}

x -hez tartozó képtér

def.:
amh az f reláció függvény, ha

f_x halmaz lgefeljebb egy elemű.

jel.: f \in A \rightarrow B

Def. (Jelölés):

Ha D_f = A, akkor mondhatjuk, hogy f: A \rightarrow B,


Def:
Amh f = g (fv-ek) \iff D_f = D_g (f(x) = g(x))

f: A \rightarrow B -> f (részh) A x B -> (a, b) \in f

(x, y) \in f \iff (x, y) \in g

Def.:
Amh f injektív, ha x \neq y \in D_f \Rightarrow f(x) \neq f(y)
relációk nyelvén: (f_x \neq f_y)

(ekvivalens: ha f(x) = f(y) \Rightarrow x = y)

Def.:
szürjektív

Amh. f szürjektív ("f képtere teljes"), ha R_f = B.

Def.:
Amh f bijektív, ha f injektív, és szürjektív is.

def.:
Halmaz képe:

A 'V' halmaz f függvény által létesített képe:
jel: f(V).  (V (részh) D_f)

f(V) := {y \in R_f : \exists x \in V : f(x) = y }


Őskép:
A 'W' halmaz ősképe f szerint:

f^{-1}(W) = {x \in D_f : \exists y \in W, hogy f(x) = y}

ha egy függvény injektív, akkor lehet az inverzéről beszélni.


1. Számtani, mértani közép:

\sqrt{a_1a_2} \leq (a_1 + a_2) / 2

Biz.:

/ *2, ^2

4a_1a_2 \leq (a_1 + a_2)^2 = a_1^2 + a_2^2 + 2a_1a_2

0 \leq a_1^2 + a_2^2 - 2a_1a_2 = (a_1 - a_2)^2

Megj. =-ség csak akko van, ha a_1, és a_2 megegyezik.

Áll.:

!n \in \mathbb{N} adott, ill a_1, a_2, ... a_n \geq 0 \Rightarrow \sqrt{n}{a_1 \cdot a_2 \cdot ,,, \cdot a_n } \leq \frac{a_1 + ... + a_n}{n}

Biz.: (teljes ind)
n = 2 done
tfh n-re igaz, belátjuk, hogy 2n-re is igaz.

{a_1 + a_2 + .. + a_n + a_{n + 1} + ... + a_{2n}}/2n \geq \sqrt{2n}{a_1 \cdot ... \cdot a_n \cdot a_{n + 1} \cdot ... \cdot a_{2n}}

a_1 + a_2 + ... + a_n / n + a_{n + 1} + ... + a_{2n} / n \geq 1/2 (\sqrt{n}{a_1 \cdot ... \cdot a_n} (= A) + \sqrt{b}{a_{n + 1} + ... + a_{2n}} (= B))  (aésó \geq felső -> trivi)

A + B / 2 \geq \sqrt{AB}

\Rightarrow  2 hatványokra beláttuk

2^{k - 1}  n < 2^k

A = \frac{a_1 + ... + a_n}{n}

a_1 + ... + a_n = nA



\frac{a_1 + ... + a_n + A + A + ... + A}{2k}

A = nA + (2^k - n)A /  2^k = \frac{a_1 + ... + a_n + (A + A + ... + A (=2^k - n))}{2k} \geq \sqrt{2^k}{a_1 \cdot ... \cdot a_n \cdot A^{2^k - n}}

A \geq \sqrt{2^k}{a_1 \cdot ... \cdot a_n \cdot A^{2^k - n}}


A^{2^k} \geq a_1 \cdot ... \a_n \cdot A^{2^k - n} (egyszerüsít)

(a_1 + ... + a_n  /n )^n = A^n \geq a_1 \cdot ... \cdot a_n

2. Bernoulli egyenlőtlenség:
(1 + h)^n \geq 1 + nh

Biz: (teljes ind)

h \geq -1
1. n = 1 re 1 + h \geq 1 + h (igaz)
n \in \mathbb{N}, n \geq 1

Indukciós feltevés:
(1 + h)^n \geq 1 + nh

Kellene: (n + 1)-re:

(1 + h)^{n + 1} \geq 1 + (n + 1)h

(1 + h)^{n + 1} = (1 + h)^n \cdot (1 + h) \geq (1 + n^h)(1 + h) =
1 + nh + h + nh^2 = 1 + (n + 1)h + nh^2 \geq 1 + (n + 1)h

3.

n! \leq (\frac{n + 1}{2})^n

Biz.:

4.

\sum_{k = 1}^n \frac{1}{\sqrt{k}} > 2 \sqrt{n + 1} - 2

Biz.:

5.

Ha a_1, ..., q_n > 0 és \sqrt_{i = 1}^n a_i = 1 \Rightarrow (1 + a_1) \cdot ... \cdot (1 + a_n) \geq 2^n

Biz.:

6.

2 \leq (1 + 1/n)^n < 4

Biz.:

7.

8abc \geq (a + b)(b + c)(a + c) \geq 8/27 (a + b + c)^3


(részh = részh = nélkül)
