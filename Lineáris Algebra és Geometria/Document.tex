% Compile twice!
% With the current MiKTeX, you need to install the beamer, and the translator packages directly form the package manager!

% !TEX root = ./Headers/PrezA4Page.tex

% Uncomment these to get the presentation form
%\documentclass{beamer}
%\geometry{paperwidth=200mm,paperheight=200mm, top=0in, bottom=0.2in, left=0.2in, right=0.2in}

% Uncomment these, and comment the 2 lines above, to get a paper-type article
%\documentclass[10pt]{article}
%\usepackage{geometry}
%\geometry{top=0.2in, bottom=0.2in, left=0.2in, right=0.2in}
%\usepackage{beamerarticle}
%\renewcommand{\\}{\par\noindent}
%\setbeamertemplate{note page}[plain]

% Half A4 geometry
%\geometry{paperwidth=105mm,paperheight=297mm,top=0.2in, bottom=0.2in, left=0.2in, right=0.2in}

% "1/3" A4 geometry
%\geometry{paperwidth=105mm,paperheight=455mm,top=0.1in, bottom=0.1in, left=0.1in, right=0.1in}

% "1/6" A4 geometry
%\geometry{paperwidth=105mm,paperheight=891mm,top=0.1in, bottom=0.1in, left=0.1in, right=0.1in}

% "1/5" A4 geometry
%\geometry{paperwidth=105mm,paperheight=740mm,top=0.1in, bottom=0.1in, left=0.1in, right=0.1in}

% "1/4" A4 geometry
%\geometry{paperwidth=105mm,paperheight=594mm,top=0.1in, bottom=0.1in, left=0.1in, right=0.1in}

% Uncomment these, to put more than one slide / page into a generated page.
%\usepackage{pgfpages}
% Choose one
%\pgfpagesuselayout{2 on 1}[a4paper]
%\pgfpagesuselayout{4 on 1}[a4paper]
%\pgfpagesuselayout{8 on 1}[a4paper]

% Includes
\usepackage{tikz}
\usepackage{tkz-graph}
\usetikzlibrary{shapes,arrows,automata}
\usepackage[T1]{fontenc}
\usepackage{amsfonts}
\usepackage{amsmath}
\usepackage[utf8]{inputenc}
\usepackage{booktabs}
\usepackage{array}
\usepackage{arydshln}
\usepackage{enumerate}
\usepackage[many, poster]{tcolorbox}
\usepackage{pgf}
\usepackage[makeroom]{cancel}
\usepackage{verbatim}

\providecommand{\includecolors}{
% Colors
\definecolor{myred}{rgb}{0.87,0.18,0}
\definecolor{myorange}{rgb}{1,0.4,0}
\definecolor{myyellowdarker}{rgb}{1,0.69,0}
\definecolor{myyellowlighter}{rgb}{0.91,0.73,0}
\definecolor{myyellow}{rgb}{0.97,0.78,0.36}
\definecolor{myblue}{rgb}{0,0.38,0.47}
\definecolor{mygreen}{rgb}{0,0.52,0.37}
\colorlet{mybg}{myyellow!5!white}
\colorlet{mybluebg}{myyellowlighter!3!white}
\colorlet{mygreenbg}{myyellowlighter!3!white}

\setbeamertemplate{itemize item}{\color{black}$-$}
\setbeamertemplate{itemize subitem}{\color{black}$-$}
\setbeamercolor*{enumerate item}{fg=black}
\setbeamercolor*{enumerate subitem}{fg=black}
\setbeamercolor*{enumerate subsubitem}{fg=black}

% These are different themes, only uncomment one at a time
\tcbset{enhanced,fonttitle=\mdseries,boxsep=7pt,arc=0pt,colframe={myyellowlighter},colbacktitle={myyellow},colback={mybg},coltitle={black}, coltext={black},attach boxed title to top left={xshift=-2mm,yshift=-2mm},boxed title style={size=small,arc=0mm}}

%\tcbset{colback=yellow!5!white,colframe=yellow!84!black}
%\tcbset{enhanced,colback=red!10!white,colframe=red!75!black,colbacktitle=red!50!yellow,fonttitle=
%\tcbset{enhanced,attach boxed title to top left}
%\tcbset{enhanced,fonttitle=\bfseries,boxsep=5pt,arc=8pt,borderline={0.5pt}{0pt}{red},borderline={0.5pt}{5pt}{blue,dotted},borderline={0.5pt}{-5pt}{green}}
}% fallback definition
\includecolors

\setbeamertemplate{itemize item}{\color{black}$-$}
\setbeamertemplate{itemize subitem}{\color{black}$-$}
\setbeamercolor*{enumerate item}{fg=black}
\setbeamercolor*{enumerate subitem}{fg=black}
\setbeamercolor*{enumerate subsubitem}{fg=black}

 \renewcommand{\familydefault}{\sfdefault}
%\renewcommand{\familydefault}{\rmdefault}

\renewcommand{\footnotesize}{\fontsize{1.2em}{0.2em}}
\renewcommand{\normalsize}{\fontsize{1.2em}{0.2em}}
\renewcommand{\large}{\footnotesize}
\renewcommand{\Large}{\footnotesize}


\renewcommand{\scriptsize}{\footnotesize}
\renewcommand{\LARGE}{\footnotesize}
\renewcommand{\Huge}{\footnotesize}

\renewcommand{\tiny}{\footnotesize}
\renewcommand{\small}{\footnotesize}

\fontsize{1.2em}{0.2em}
\selectfont

\newcommand{\RHuge}{\fontsize{1.8em}{0.3em}\selectfont}

\newsavebox\CBox 
%\newcommand<>*\textBF[1]{\sbox\CBox{#1}\resizebox{\wd\CBox}{\ht\CBox}{\textbf#2{#1}}}
\newcommand<>*\textBF[1]{\only#2{\sbox\CBox{#1}\resizebox{\wd\CBox}{\ht\CBox}{\textbf{#1}}}}

% Beamer theme
\usetheme{boxes}

% tikz settings for the flowchart(s)
\tikzstyle{decision} = [diamond, minimum width=3cm, minimum height=1cm, text centered, draw=black, fill=green!15]
\tikzstyle{tcolorbox} = [rectangle, draw, fill=blue!15, text width=20em, text centered, minimum height=1em]

\tikzstyle{line} = [draw, -latex']
\tikzstyle{cloud} = [draw, ellipse,fill=red!20, node distance=3cm,
    minimum height=2em]
\tikzstyle{arrow} = [thick,->,>=stealth]

\newcolumntype{C}[1]{>{\centering\let\newline\\\arraybackslash\hspace{0pt}}m{#1}}
\renewcommand{\arraystretch}{1.2}

\setlength\dashlinedash{0.2pt}
\setlength\dashlinegap{1.5pt}
\setlength\arrayrulewidth{0.3pt}

\newcommand{\mtinyskip}{\vspace{0.2em}}
\newcommand{\msmallskip}{\vspace{0.3em}}
\newcommand{\mmedskip}{\vspace{0.5em}}
\newcommand{\mbigskip}{\vspace{1em}}
\renewcommand{\u}[1]{\underline{#1}}

\begin{document}

\begin{frame}[plain]
\begin{tcolorbox}[center, colback={myyellow}, coltext={black}, colframe={myyellow}]
    {\RHuge Lineáris Algebra és Geometria}\\
\end{tcolorbox}
\end{frame}


%\begin{tcolorbox}[title={Def.: }]
%\end{tcolorbox}

% --------------------  HALMAZOK, RELÁCIÓK --------------------

\begin{frame}[plain]
\begin{tcolorbox}[center, colback={myyellow}, coltext={black}, colframe={myyellow}]
    {\RHuge Vektorterek, Leképzések}
    \mmedskip
\end{tcolorbox}
\end{frame}

\begin{frame}
  \begin{tcolorbox}[title={Def.: Linearitás}]
    $f$ leképzés lineáris, ha:\\
    \begin{itemize}
      \item $f(a + b) = f(a) + f(b)$
      \item ${\lambda}f(a) = f({\lambda}b)$
    \end{itemize}
  \end{tcolorbox}


  \begin{tcolorbox}[title={Def.: Vektorok}]
  $\u{a} = (a_1, a_2), \u{b} = (b_1, b_2)$\\
  \mmedskip
  
    Összeadás: $\u{a + b} = (a_1 + b_1, a_2 + b_2)$\\
    Nyújtás: ${\lambda}a = (a_1, a_2) \lor {\lambda}\u{a} = ({\lambda}a_1, {\lambda}a_2)$
  \end{tcolorbox}
  
  \end{frame}
  
  \begin{frame}  
  
	\begin{tcolorbox}[title={Def.: Összeadás}]	
		\begin{align}
			\u{a} + \u{b} &= \begin{bmatrix}
				a_1 + b_1 \\
				a_2 + b_2 \\
				... \\
				a_n + b_n
			\end{bmatrix}
		\end{align}
				
		\tcblower
		
		\textBF{Tulajdonságok} \\
		
		\msmallskip
		
		\begin{enumerate}
			\item Van értelme
			\item Kommutativitás - $\u{a} + \u{b} = \u{b} + \u{a}$
			\item Asszociativitás - $(\u{a} + \u{b}) + \u{c} = \u{a} + (\u{b} + \u{c})$
			\item Van nullelem - ${\exists}0 \rightarrow \u{0}$
			\item Minden elemre létezik additív inverz - ${\forall}\u{a} \in \mathbb{R}^n : {\exists}\u{-a}$, ahol $\u{a} + \u{-a} = \u{0}$ \\
			$\u{-a} = -1 \cdot \u{a} = \u{-a}$, $\u{a} + \u{-a} = \u{0}$
		\end{enumerate}				
	\end{tcolorbox}

  \end{frame}
  
  \begin{frame}
  
    \begin{tcolorbox}[title={Def.: Szorzás számmal}]	
		\begin{align}
			\u{a} + \u{b} &= \begin{bmatrix}
				{\lambda}a_1 \\
				{\lambda}a_2 \\
				... \\
				{\lambda}a_n
			\end{bmatrix}
		\end{align}
		
		\tcblower
		
		\textBF{Tulajdonságok} \\		
		
		 \msmallskip
		 
		 \begin{enumerate}
		 	\item Van értelme
		 	\item Asszociativitás ${\lambda}, {\mu} \in \mathbb{R}$, $({\lambda}{\mu})\u{a} = {\lambda}({\mu}\u{a})$
		 	\item Disztributivitás ${\lambda}, {\mu} \in \mathbb{R}$, $({\lambda} + {\mu})\u{a} = {\lambda}\u{a} + {\mu}\u{b}$
		 	\item Disztributivitás $\u{a}, \u{b} \in \mathbb{R}^n, {\lambda} \in \mathbb{R}$, ${\lambda}\u{a} + \u{b}) = {\lambda}\u{a} + {\lambda}\u{b}$
		 	\item Létezik egységelem. $1 \cdot \u{a} = \u{a}$
		 \end{enumerate}
		
	\end{tcolorbox}
  	
    \end{frame}
  
	\begin{frame}
		 \begin{tcolorbox}[title={Def.: Vektortér}]
			$\mathbb{R}^n$vektortér $\mathbb{R}$ felett, ha igazak rá az összeadás, és a szorzás tulajdonságai.\\

			\mmedskip
			
			Azaz, ha egy $V \neq \emptyset$ tudja ezeket a tulajdonságokat, akkor $V$ vektortér $\mathbb{R}$ felett.			
		\end{tcolorbox}
		
		\begin{tcolorbox}[title={Def.: Altér}]
			Azt mondjuk, hogy $W \leq \mathbb{R}^n$ altere $\mathbb{R}^n$-nek, ha
			\begin{enumerate}
				\item $W \neq \emptyset$
				\item Ha zárt az összeadásra ($\u{a}, \u{b} \in W \Rightarrow \u{a} + \u{b} \in W$)
				\item Ha zárt a számmal való szorzásra ($\u{a} \in W, {\lambda}\u{a} \in W$)
			\end{enumerate}
		\end{tcolorbox}
		
		\begin{tcolorbox}[title={Megj}]
			$\mathbb{R}^2$ $\rightarrow$ alterei: $x, y$ tengely\\
			$\mathbb{R}^3$ $\rightarrow$ alteret: A síkok is.
		\end{tcolorbox}
		
		\begin{tcolorbox}[title={Def.: Vektorrendszer, Lineáris kombináció}]
			\textBF{Vektorrendszer}:\\
			Legyen $k \geq 1$ egész. és legyenek $\u{v_1}, \u{v_2}, ..., \u{v_k} \in \mathbb{R}^n$.\\
			Ezeket a vektorokat együtt \textBF{vektorrendszernek} hívjuk.\\			
			\msmallskip
			
			\textBF{Lineáris kombináció}:\\
			Legyenek ${\lambda}_1, {\lambda}_2, ..., {\lambda}_k \in \mathbb{R}$ adottak,\\
			ekkor a ${\lambda}_1\u{v_1} + {\lambda}_2\u{v_2} + ... + {\lambda}_k\u{v_k}$ kifejezést a\\
			$\u{v_1}, \u{v_2}, ..., \u{v_k}$ vektorrendszer \u{lineáris kombinációjának} nevezzük.\\
			\msmallskip			
			
			\textBF{triviális lineáris kombináció}:\\
			Ha ${\lambda}_1 = {\lambda}_2 = ... = {\lambda}_k = 0$, akkor a lineáris kombináció triviális.
		\end{tcolorbox}
	\end{frame}

	\begin{frame}
	\begin{tcolorbox}[title={Def.: Lineáris összefüggőség}]
			Legyen $k \geq 1$ egész. és legyen $\u{v_1}, \u{v_2}, ..., \u{v_k} \in \mathbb{R}^n$. vektorrendszer.\\
			Ekkor azt mondjuk, hogy a $\u{v_1}, \u{v_2}, ..., \u{v_k}$ vektorrendszerünk \textBF{lineárisan összefüggő}, ha létezik nemtriviális lineáris kombinációja, melyre:\\
			${\lambda}_1\u{v_1} + {\lambda}_2\u{v_2} + ... + {\lambda}_k\u{v_k} = \u{0}$
		\end{tcolorbox}
		
		\begin{tcolorbox}[title={Def.: Lineáris függetlenség}]
			Legyen $k \geq 1$ egész. és legyen $\u{v_1}, \u{v_2}, ..., \u{v_k} \in \mathbb{R}^n$. vektorrendszer.\\
			Ekkor azt mondjuk, hogy a $\u{v_1}, \u{v_2}, ..., \u{v_k}$ vektorrendszerünk \textBF{lineárisan független}, ha csak a triviális lineáris kombinációjára igaz, hogy:\\
			${\lambda}_1\u{v_1} + {\lambda}_2\u{v_2} + ... + {\lambda}_k\u{v_k} = \u{0}$
		\end{tcolorbox}
		
		\begin{tcolorbox}[title={Def.: Bázis}]
			Legyen $ V \leq \mathbb{R}^k$ altér, és legyen adott $\u{v_1}, \u{v_2}, ..., \u{v_k}$ vektorrendszer.\\
			Azt mondjuk, hogy a $\u{v_1}, \u{v_2}, ..., \u{v_k}$ vektorrendszer \textBF{bázis} $V$-ben, ha:\\
			\begin{itemize}
				\item Lineárisan függetlenek
				\item Tetszőleges eleme $V$-nek előáll belőlük lineáris kobinációként.
			\end{itemize}
			\mmedskip
			
			 (Megj:   $n$ dimenzóban $n$ elemű egy bázis)
		\end{tcolorbox}
		
		\begin{tcolorbox}[title={Tétel: Lineáris kombináció, és bázisok}]
			$\u{b_1}, \u{b_2}, ..., \u{b_k}$ bázis $V$-ben, akkor $\forall \u{v} \in V$ elem \textBF{egyértelműen} előáll belőle lineáris kombinációjaként.
		\end{tcolorbox}
		
		\begin{tcolorbox}[title={Tétel: Bázisok, és Lineáris kombináció}]
			Ha a $\u{v_1}, \u{v_2}, ..., \u{v_k}$ vektorrendszer olyan V-ben, hogy ha $\forall a \in V$ egyértelműen létezik ${\alpha}_1, ..., {\alpha}_k \in \mathbb{R}$, hogy $\u{a} = {\alpha}_1\u{b_1} + {\alpha}_2\u{b_2} + ... + {\alpha}_k\u{b_k} \Rightarrow \u{b_1}, \u{b_2}, ..., \u{b_k}$ bázis.
		\end{tcolorbox}
		
	\end{frame}
	
	\begin{frame}
		\begin{tcolorbox}[title={Tétel.: Bázistransformáció}]
			Legyen $V \leq \mathbb{R}^n$, $\u{b_1}, u{b_2}, ..., u{b_k}$ bázis $V$-ben.\\
			Legyen $a \in V$ adott, és $\u{a} = {\alpha}_1\u{v_1} +  {\alpha}_2\u{v_2} +  ... + {\alpha}_k\u{v_k}$.\\
			Ekkor $\u{b_1}, u{b_2}, ..., u{b_k} \iff {\alpha}_i \neq 0$ bázis.\\
			\mmedskip
			
			Akkor cserélhetjük ki, ha az együtthatója nem 0 az $\u{a}$-ban.
		\end{tcolorbox}
		
		\begin{tcolorbox}[title={Képlet}]
			$x_j = x_j - \frac{x_i}{{{\alpha}_i}} {\alpha}_j$
		\end{tcolorbox}
		
		\begin{tcolorbox}[title={Öf táblázat}]
		\end{tcolorbox}
	\end{frame}
	
	\begin{frame}
		\begin{tcolorbox}[title={Def.: Lineáris függőség}]
			$A \neq \emptyset$, $A \subseteq \mathbb{R}^n$, azt mondjuk hogy $\u{v} \in \mathbb{R}^n$ \textBF{lineárisan függ} $A$-tól,\\
			ha létezik véges sok elem $A$-ban, hogy $\u{v}$ előáll az ő lineáris kombinációjaként.
		\end{tcolorbox}
		
		\begin{tcolorbox}[title={Def.: Lineáris függőség}]
			$k \geq 2$, $\u{a_1}, u{a_2}, ..., u{a_k} \in \mathbb{R}^n$,\\
			ekkor $\u{a_1}, u{a_2}, ..., u{a_k}$ összefüggő $\iff$ $\exists i \in \{ 1, ..., k \}$, hogy $a_i$ lineárisan függ a többitől.
		\end{tcolorbox}
		
		\begin{tcolorbox}[title={Áll.: Lineáris függőség}]
			Ha$\u{a_1}, u{a_2}, ..., u{a_k}$, $\u{b} \in \mathbb{R}^n$\\		
			$\u{a_1}, u{a_2}, ..., u{a_k}$ lineárisan független, de $\u{a_1}, u{a_2}, ..., u{a_k}, \u{b}$ lineárisan összefüggő, akkor\\
			$\u{b}$ lineárisan független az $\u{a_1}, u{a_2}, ..., u{a_k}$ vektorrendszertől.	
		\end{tcolorbox}
		
		\begin{tcolorbox}[title={Def.: Halmaz által generált altér / Lineáris Burok}]
			$A \neq \emptyset$, $A \leq \mathbb{R}^n$:\\
			$: W(A) = \{ \u{b} \in \mathbb{R}^n | \u{v}$ lineárisan függ $A$-tól $\}$
		\end{tcolorbox}
		
		\begin{tcolorbox}[title={Def.: Vektor koordinátái}]
			\begin{align}
				[a]_{\u{b_1}, \u{b_2}, ..., \u{b_k}} &= \begin{bmatrix}
					{\lambda}a_1 \\
					{\lambda}a_2 \\
					... \\
					{\lambda}a_n
				\end{bmatrix} \in \mathbb{R}^k
			\end{align}
		\end{tcolorbox}
		
		\begin{tcolorbox}[title={Tétel: Altér}]
			$W(A)$ altér $(A \neq \emptyset)$
		\end{tcolorbox}
	\end{frame}
	
	\begin{frame}
		\begin{tcolorbox}[title={Tétel: Alterek metszete}]
			Ha $V_1$ és $V_2$ is altér $\Rightarrow$ $V_1 \cap V_2$ is altér.
		\end{tcolorbox}
		
		\begin{tcolorbox}[title={Def.: Span}]
			Azt mondjuk, hogy az $A \subseteq \mathbb{R}^n$ halmaz által \textBF{generált / kifeszített altér} az $A$-t tartalmazó alterek / vektorterek metszete.\\
			\msmallskip
			
			Jel.: $Span(A)$
		\end{tcolorbox}
		
		\begin{tcolorbox}[title={Tétel: Span és Lineáris burok}]
			$Span(A) = W(A)$
		\end{tcolorbox}
		
		\begin{tcolorbox}[title={Def.: Generátorrendszer}]
			Azt mondjuk, hogy $G$ vektorrendszer \textBF{generátorrendszere} $V$ altérnek, ha $Span(G) = W(G)$
		\end{tcolorbox}
		
		\begin{tcolorbox}[title={Tétel.: Generátorrendszer létezése}]
			Ha $V \leq \mathbb{R}^n$-ben létezik véges méretű generátorrendszer $\Rightarrow$ belőle kiválasztható bázis.
		\end{tcolorbox}
	\end{frame}
	
	\begin{frame}
		\begin{tcolorbox}[title={Tétel: Kicserélési tétel}]
			Legyen $V \leq \mathbb{R}^n$, legyen $a_1, ..., a_k$ lineárisan független, és $b_1, ..., b_n$ generátorrendszer. Ekkor:\\
			\begin{itemize}
				\item $\exists j$, hogy tetszőleges $i$-re $v_j, a_2, ..., a_k$ is Lineárisan független.\\
				(megj.: Igazából $a_1, ..., a_k$ bármilyen eleme lecserélhető)
				\item $|LF| \leq |GR|$ ($|LF|$ = $LF$ elemszáma, $LF$ = $a_1, ..., a_k$)
			\end{itemize}
		\end{tcolorbox}
		
		\begin{tcolorbox}[title={Tétel.: Bázis}]
			Ha $V \leq \mathbb{R}^n,$ és $B_1, B_2$ bázis, akkor\\
			$|B_1| < + \infty \rightarrow |B_1| = |B_2|$
		\end{tcolorbox}
		
		\begin{tcolorbox}[title={Bázis}]
			\begin{itemize}			
				\item Minden bázis mérete $\mathbb{R}^n$-ben $n$
				\item $V \leq \mathbb{R}^n$ és van véges generátorrendszer $\Rightarrow$ Leszűkíthető bázissá.
				\item $V \leq \mathbb{R}^n$ és $v_1, ..., v_k$ vektorrendszer lineárisan független a $V$-ben. $\Rightarrow$ Leszűkíthető bázissá.
			\end{itemize}
			\mmedskip
			
			Ezekből követketik, hogy a bázis a maximális elemszámú lineárisan független vektorrendszer.\\
			\mmedskip
			
			Maximális lineárisan független vektorrendszer elemszáma = minimális generátorrendszer elemszáma = bázis elemszáma
		\end{tcolorbox}
		
		\begin{tcolorbox}[title={Def.: Dimenzió}]
			$V \leq \mathbb{R}^n$ dimenziója:\\
			\mmedskip
			
			\[
   				dim(V) = 
			\begin{cases}
   				0,				& \text{ha } V = \{ \u{0} \}\\
    				|B|,                & \text{ha } V \neq \{ \u{0} \} \text{ (B a V-nek egy bázisa.) } \\
			\end{cases}
			\]		
		\end{tcolorbox}
		
		\begin{tcolorbox}[title={Def.: Rang}]
			 $v_1, ..., v_k \in \mathbb{R}^n$ vektorrendszer rangja, az általuk generált altér dimenziója.
		\end{tcolorbox}
	\end{frame}
	
	\begin{frame}

	\begin{tcolorbox}[title={Def.: Mátrix}]	
		Legyen $e_k$ adott $a_k$ az $m$ és $n$ pozitív egész számok, továbbá minden
	
		$i \in \{1, ..., m\} és j \in \{1, ..., n\}$ esetére az $aij$ valós számok. Az
		
		táblázatot egy $\mathbb{R}$ feletti mátrixnak nevezzük, és $A$-val jelöljük, részletesebben $A = [aij]mn$, vagy
		
		Az $A$ mátrix $i$-ed $i_k$ sora $j$-edik elemén $e_k$ jelölése: $aij$ vagy $i[A]j$.
	\end{tcolorbox}
	
	\begin{tcolorbox}[title={Def.: Mátrixok egyenlősége}]	
		Az $A$ és a $B$ mátrixok egyenlők, ha alakjuk azonos (mondjuk $m x n$-es) és a megfelelő elemeik megegyeznek, azaz minden „szóbajövő" $i$, $j$ párra $(1 \leq i \leq m, 1 \leq j \leq n)$ teljesül, hogy $i[A]j = i[B]j$. Az $\mathbb{R}$ feletti $m x n$-es mátrixok halmazát $\mathbb{R}^{m x n}$-mel jelöljük ($\mathbb{R}^{m}$ = $\mathbb{R}^{m x 1}$).
	\end{tcolorbox}
	
	\begin{tcolorbox}[title={Def.: Mátrix összeadás, számmal való szorzás}]	
		Az $\mathbb{R}^{m}$-beli komponensenkénti összeadás és valós számmal való szorzás (1/2) mintájára természetes módon kínálkoznak $m x n$-es mátrixok esetén a megfelelő elemek összeadásával, illetve az összes elemnek egy valós számmal szorzásával a következő műveletek:

$+$ : $\mathbb{R}^{m x n}$ x $\mathbb{R}^{m x n} \rightarrow \mathbb{R}^{m x n}$, A, B $\in$ $\mathbb{R}^{m x n}$ esetén $A + B$ $\in$ $\mathbb{R}^{m x n}$ és minden szóbajövő
$i, j$-re $_{i} [A + B]_j$ = $_{i} [A]_j + _{i} [B]_j$.

${\lambda}$ : $\mathbb{R}$ x $\mathbb{R}^{m x n} \rightarrow \mathbb{R}^{m x n}$, ${\lambda} \in \mathbb{R}$, $A \in \mathbb{R}^{m x n}$ esetén ${\lambda}A \in \mathbb{R}^{m x n}$ és minden szóbajövő $i, j$-re $_{i} [{\lambda}A]_j$ = ${\lambda} _{i} [A]_j$.

Az $\mathbb{R}^{m x n}$-re is teljesül az 1/3 oldali 10 tulajdonság megfelelője a fenti $+$, ${\lambda}$ műveletekre nézve, így $\mathbb{R}^{m x n}$-et is $\mathbb{R}$ feletti vektortérnek mondjuk.
	\end{tcolorbox}	
	
	\begin{tcolorbox}[title={Def.: Nullmátrix}]	
		Nullmátrix: 0 := $\begin{bmatrix}
					{\lambda}a_1 \\
					{\lambda}a_2 \\
					... \\
					{\lambda}a_n
				\end{bmatrix} \in \mathbb{R}^k$
	\end{tcolorbox}		
	
	\end{frame}
	
	\begin{frame}

		\begin{tcolorbox}[title={Def.: Mátrixszorzás}]	
			$A$ $in$ $\mathbb{R}^{m x n}$, $B$ $\in$ $\mathbb{R}^{n x k}$ esetén $AB$ $\in$ $\mathbb{R}^{m x k}$ úgy, hogy minden szóbajövő $i, j$-re (most $1 \leq i \leq m, 1 \leq j \leq k$)
			\mmedskip
	
			$_{i} [AB]_j$ $=$ $_{i} [A]_1$ $_{1} [B]_j$ + $_{i} [A]_2$ $_{2} [A]_j$ + $...$ + $_{i} [A]_n$ $_{n} [B]_j$ = $\sum_{l = 1}^n$ $_{i} [A]_l$ $_{l} [B]_j$
			\mmedskip

			Ez az ún. sor-oszlop szorzás: a szorzatmátrix i-edik sora j-edik elemét úgy kapjuk, hogy a bal oldali mátrix i-edik sorának és a jobb oldali mátrix j-edik oszlopának megfelelő elemeit összeszorozzuk, s a kapott szorzatokat összeadjuk.
		\end{tcolorbox}		
		
		\begin{tcolorbox}[title={Def.: Egységmátrix}]	
			$I_n$ = $\begin{bmatrix} 
  				1 & 0 & \cdots & 0 \\ 
  				0 & 1 & \cdots & 0 \\
  				\vdots & \vdots &  & \vdots \\
  				0 & 0 & \cdots & 1
			\end{bmatrix}$ $\in \mathbb{R}$ az $n$ x $n$-es egységmátrix\\
			
			
			\[
   				{\delta}_{ij} = 
			\begin{cases}
   				1, & \text{ha } i = j\\
    			0, & \text{ha } i \neq j \\
			\end{cases}
			\]\\
			
			(A ${\delta}_{ij}$ egyik szokásos elnevezése: Kronecker-szimbólum.)
		\end{tcolorbox}	
	
		\begin{tcolorbox}[title={Tétel: Egységmátrix}]	
			$A$ $\in$ $\mathbb{R}^{m x n}$ esetén $I_mA$ = $A$ és $AI_n$ = $A$.
		\end{tcolorbox}
		
	\end{frame}
	
	\begin{frame}
		\begin{tcolorbox}[title={Def.: Tranzponált}]
			A $\mathbb{R}^{m x n}$ esetén az $A$ mátrix transzponáltja: $A^T$ $\mathbb{R}^{m x n}$, melyre minden szóbajövő $i$, $j$-re $_{i} [A^T]_j$ = $_{j} [A]_i$.
		\end{tcolorbox}
		
		\begin{tcolorbox}[title={Tétel: Tranzponálás tulajdonságai}]
			$A$, $B$ $\mathbb{R}^{m x n}$ $\Rightarrow$ $(A + B)^T$ = $A^T$ + $B^T$\\
			${\lambda}$ $\in$ $\mathbb{R}$, $A$ $\mathbb{R}^{m x n}$ $\Rightarrow$ $({\lambda}A)^T$ = ${\lambda}A^T$\\
			$A$ $\in$ $\mathbb{R}^{m x n}$, $B$ $\in$ $\mathbb{R}^{n x k}$ $\Rightarrow$ $(AB)T$ = $B^TA^T$
		\end{tcolorbox}
		
		\begin{tcolorbox}[title={Tétel: Mátrixszorzás, és asszociativitás}]
			$A$ $\in$ $\mathbb{R}^{m x n_1}$ , $B$ $\in$ $\mathbb{R}^{n_2 x k_2}$, $C$ $\in$ $\mathbb{R}^{k_3 x s}$, esetén\\
			\mmedskip
			
			$\exists$ $(AB)C)$ $\iff$ $\{n_1 = n_2$ és $k_2 = k_3\}$ $\iff$ ${\exists}A(BC)$\\
			\mmedskip
			
			($\{n_1 = n_2$ és $k_2 = k_3\}$ = $(AB)C$ = $A(BC)$)
		\end{tcolorbox}
		
		\begin{tcolorbox}[title={Tétel: Mátrixszorzás és összeadás disztributivitása}]
			$A$ $\in$ $\mathbb{R}^{m x n_1}$ , $B$ $\in$ $\mathbb{R}^{n_2 x k_2}$, $C$ $\in$ $\mathbb{R}^{n_3 x k_3}$, esetén\\
			\mmedskip
			
			$\exists$ $A(B + C)$ $\iff$ $\{n_1 = n_2$ és $k_2 = k_3\}$ $\iff$ ${\exists}$ $AB + BC$\\
			\mmedskip
			
			($\{n_1 = n_2$ és $k_2 = k_3\}$ = $A(B + C)$ = $AB + BC$)
		\end{tcolorbox}
		
		\begin{tcolorbox}[title={Tétel: Számmal való szorzás és mátrixszorzás kapcsolata}]
			$\lambda$ $\in$ $\mathbb{R}$, $A$ $\in$ $\mathbb{R}^{m x n}$, $B$ $\in$ $\mathbb{R}^{n x k}$ $\Rightarrow$\\
			\mmedskip
			
			${\lambda}(AB)$ = $({\lambda}A)B$ = $A({\lambda}B)$\\
		\end{tcolorbox}
	\end{frame}
	
	
	\begin{frame}
		\begin{tcolorbox}[title={Def.: Sorrang, Oszloprang}]
			$A$ = $[a_1, {\cdots}, a_n]$ $\in$ $\mathbb{R}^{m x n}$ \\
			\mmedskip
			
			oszloprangja: ${\varrho}_{O}(A)$ = $r(a_1, {\cdots}, a_n)$ $($ = $dim$ $Span(a_1, {\cdots}, a_n))$\\
			sorrangja: ${\varrho}_{s}(A)$ = ${\varrho}_{O}(A^T)$\\
		\end{tcolorbox}

		\begin{tcolorbox}[title={Tétel: Mátrixszorzás, dimenzió}]
			Legyenek $C$ = $[c_1, {\cdots}, c_n]$ és $D$ = $[d_1, {\cdots}, d_k]$ ebben a sorrendben összeszorozható
R feletti mátrixok. Ekkor:\\
			\mmedskip
			
	 		${\varrho}_{s}(CD)$ $\leq$ ${\varrho}_{s}(C)$
		\end{tcolorbox}
		
		\begin{tcolorbox}[title={Tétel: Mátrix, rang}]
			Tetszőleges $\mathbb{R}$ feletti $A$ mátrixra ${\varrho}_{o}(A)$ $\leq$ ${\varrho}_{s}(A)$\\
			\mmedskip
			
			(ezentúl ${\varrho}_{o}(A)$ $\leq$ ${\varrho}_{s}(A)$ = ${\varrho}(A)$ (az $\varrho$ helyett használatos a $p$, vagy $r$ is.)
		\end{tcolorbox}
		
		\begin{tcolorbox}[title={Tétel: Inverz}]
			$A$ $\in$ $\mathbb{R}^{m x n}$ esetén:\\
			\mmedskip
			
			Az $A^{(j)}$ egy jobb oldali inverze az $A$-nak, ha $A^{(j)}$ $\in$ $\mathbb{R}^{n x m}$ és $AA^{(j)}$ = $I_m$\\
			Az $A^{(b)}$ egy bal oldali inverze az $A$-nak, ha $A^{(b)}$ $\in$ $\mathbb{R}^{n x m}$ és $A^{(b)}A$ = $I_n$\\
			Az $A^{-1}$ kétoldali inverze $A$-nak, ha bal oldali inverze is és jobb oldali inverze is $A$-nak. 
		\end{tcolorbox}
		
		\begin{tcolorbox}[title={Tétel: Inverz létezése}]
			$A$ $\in$ $\mathbb{R}^{m x n}$ esetén:\\
			
			\begin{enumerate}
			\item $\exists$ $A^{(j)}$ $\iff$ ${\varrho}(A)$ = $m$
			\item $\exists$ $A^{(b)}$ $\iff$ ${\varrho}(A)$ = $n$
			\item $\exists$ $A^{-1}$ $\Rightarrow$ ${\varrho}(A)$ = $m$ = $n$ $\Rightarrow$ $\exists$ $A^{(b)}$), $\exists$ $A^{(j)}$ és egyenlők $\Rightarrow$ $\exists$ $A^{-1}$.
			\end{enumerate}
		\end{tcolorbox}
	\end{frame}
	
	\begin{frame}
		\begin{tcolorbox}[title={Def.: Adjugált}]
			 $A  \in \mathbb{C}^{m x n}$ esetén az $A$ mátrix adjungáltja: $A^{{\ast}} \in \mathbb{C}^{n x m}$, melyre minden szóbajövő $j, k$-ra $_{j} [A{\ast}]_k$ = $_{k} [A]_j$.
		\end{tcolorbox}
		
		\begin{tcolorbox}[title={Tétel.: Az adjungálás kapcsolata a mátrixműveletekkel}]
			$A, B \in \mathbb{C}^{m x n}$ $\Rightarrow$ $(A + B)^*$ = $A^* + B^*$\\
			$\lambda \in C, A \mathbb{C}^{m x n}$  $\Rightarrow$ $({\lambda}A)^* = \overline{{\lambda}}A^*$
			$A \in \mathbb{C}^{m x n}$ $B \in \mathbb{C}^{n x k}$ $\Rightarrow$ $(AB)^* = B^*A^*$
		\end{tcolorbox}
		
		\begin{tcolorbox}[title={Tétel: Rangtartó átalakítások}]
			$a_1, ..., a_k$ $\in \mathbb{C}^{n}, {\lambda} \in \mathbb{R}^n$, ${\lambda}$ $=$ $0$, $k \leq 2$ esetén
			$r(a_2, a_1, ..., a_k) = r(a_1, a_2, ..., a_k),$\\
			$r({\lambda}a_1, a_2, ..., a_k) = r(a_1, a_2, ..., a_k),$\\
			$r(a_1 + a_2, a_2, ..., a_k) = r(a_1, a_2, ..., a_k),$\\
			$r(a_1 + {\lambda}a_2, a_2, ..., a_k) = r(a_1, a_2, ..., a_k).$
		\end{tcolorbox}
		
		\begin{tcolorbox}[title={Tétel: Rangtartó átalakítások és mátrixok}]
			$A  \in \mathbb{R}^{m x n}, {\varrho}(A) = r \leq 1$ esetén $A \leadsto$ $\begin{bmatrix} 
  				I_r & 0  \\ 
  				0 & 0 \\
			\end{bmatrix}$ $\in \mathbb{R}^{m x n}$
		\end{tcolorbox}	
		
		\begin{tcolorbox}[title={Def.: Geometriai vektorok skaláris szorzata tulajdonságai}]
			$ab$ $=$ $|a| |b| cos {\gamma}(a, b)$\\
		\end{tcolorbox}	
		
		\begin{tcolorbox}[title={Def.: Skaláris szorzat, geometriai vektorokra}]
			$ab = ba$ (kommutativitás)\\
			$ab = 0$ $\iff$ $a \bot b$\\
			${\lambda}(ab) = ({\lambda}a)b = a({\lambda}b)$ (skalár kiemelhetősége)\\
			$(ab)c \neq a(bc)$\\
			$cc = |c|^2 \geq 0$\\
			$a(b + c) = ab + ac$, $(b + c)a = ba + ca$ (disztributivitás)
		\end{tcolorbox}	
	\end{frame}
	
	\begin{frame}
		\begin{tcolorbox}[title={Def.: Skaláris szorzat, geometriai vektorokra}]
			Az $a, b, c$ nem egysíkú geometriai vektorok ebben a sorrendben jobbrendszert alkotnak, ha közös kezdőponttal felrajzolva őket, a kezdőpontban az $a$ és $b$ síkjára emelt merőlegesen (mint forgástengelyen) a $c$-t tartalmazó féltérből nézve pozitív irányú, $0^{\circ}$ és $180^{\circ}$ közötti forgatással vihetjük át az $\frac{a}{|a|}$ -t a $\frac{b}{|b|}$ -be.
		\end{tcolorbox}	
		
		\begin{tcolorbox}[title={Def.: Jobbrendszer}]
			Az $a, b, c$ nem egysíkú geometriai vektorok ebben a sorrendben jobbrendszert alkotnak, ha közös kezdőponttal felrajzolva őket, a kezdőpontban az $a$ és $b$ síkjára emelt merőlegesen (mint forgástengelyen) a $c$-t tartalmazó féltérből nézve pozitív irányú, $0^{\circ}$ és $180^{\circ}$ közötti forgatással vihetjük át az $\frac{a}{|a|}$ -t a $\frac{b}{|b|}$ -be.
		\end{tcolorbox}	
		
		\begin{tcolorbox}[title={Def.: Vektoriális szorzat}]
			$a x b$ VEKTOR (,,a kereszt b,,) $:=$\\
			\mmedskip
			
			\begin{enumerate}
			\item $|a x b|$ = $|a||b| sin {\gamma}(a, b)$;
			\item $a x b \bot a, b$;
			\item $|a x b| = 0$ esetén $a, b, a x b$ jobbrendszer.
			\end{enumerate}
		\end{tcolorbox}	
		
		\begin{tcolorbox}[title={Def.: Vektoriális szorzat műveleti tulajdonságai}]
				Tetszőleges $a, b, c$ geometriai vektorokra és ${\lambda}$ skalárra: \\
				
				$a x b \iff a \parallel b$\\
				$b x a = -a x b$ (alternálás vagy antikommutativitás), így $a \nparallel b$ esetén $b x  a = a x b$ (tehát a vektoriális szorzat nem kommutatív)\\
				${\lambda}(a b) = ({\lambda}a) b = a ({\lambda}b)$ (skalár kiemelhetősége)\\
				$a x (b + c) = a x b + a x c, (b + c) x a = b x a + c x a$ (disztributivitás).\\
				\mmedskip

				A disztibutivitás bizonyítása a korábbi három segédtétel felhasználásával történhet, kiegészítve azzal, hogy $e = 1$ esetén tetszőleges $a$-ra $a_m = (e x a) x e$, ezt pedig az $e$ irányából nézve $+90^{\circ}$-os forgatással átvihetjük $e x a$-ba (ezzel kerülhető el a bizonyítandó állítás felhasználása).

		\end{tcolorbox}	
	\end{frame}
	
	\begin{frame}
		\begin{tcolorbox}[title={Tétel.: Kifejtési tétel}]
			$(a x b) x c$ $=$ $(ac)b - (bc)a$
		\end{tcolorbox}	
		
		\begin{tcolorbox}[title={Tétel.: Felcserélési tétel}]
			$(a x b)c = a(b x c)$
		\end{tcolorbox}	
		
		\begin{tcolorbox}[title={Def.: Vegyesszorzat}]
			Az $a, b, c$ geometriai vektorok vegyes szorzata: $abc = (a x b)c$.
		\end{tcolorbox}	
		
		\begin{tcolorbox}[title={Def.: Determináns}]
			Az $A$ $=$ $\begin{bmatrix} 
  				a_{11} & \cdots & a_{1n}  \\
  				\vdots &   & \vdots \\
  				a_{n1} & \cdots & a_{nn}
			\end{bmatrix}$ $\in$ $\mathbb{R}^{n x n}$ mátrix determinánsa egy alább definiált szám, melyet röviden $A$-val jelölünk, részletesebben kiírhatjuk a mátrix elemeit a szokott módon, de függőleges vonalak közé:\\
			\mmedskip
			
			$(|A| = )$ $\begin{vmatrix} 
  				a_{11} & \cdots & a_{1n} \\
  				\vdots &   & \vdots\\
  				a_{n1} & \cdots & a_{nn}
			\end{vmatrix}$ $\in$ $\mathbb{R}^{n x n}$ = $\sum_{i_1, ..., i_n\\ (1, ..., n)} (-1)^{I(i_1, i_2, ..., i_n)} a_{1i_1} \cdot a_{1i_2} \cdot a_{1i_3} \cdot ... \cdot a_{ni_n}$
		\end{tcolorbox}	
		
		\begin{tcolorbox}[title={Tétel.: Determináns elem, és sorcsere}]
			Legyen $n \geq 2$.\\
			
			\begin{enumerate}
    				\item Ha az $1, 2, ..., n$ számok $i_1, i_2, ..., i_n$ permutációjában két elemet felcserélünk, akkor az inverziószám páratlan számmal változik.
    				\item Ha az $A \in \mathbb{R}^{n x n}$ mátrix valamely két sorát felcseréljük, akkor az így nyert $B$ mátrix determinánsa: $|B| = -|A|$ , azaz két sor felcserélése esetén a determináns értéke $(- 1)$-gyel szorzódik.
			\end{enumerate}
		\end{tcolorbox}	
	\end{frame}
	
	
	\begin{frame}
		\begin{tcolorbox}[title={Tétel.: Determináns két sor egyenlősége}]
			Ha $n \geq 2$ és az $A \in \mathbb{R}^{n x n}$  mátrixnak van két megegyező sora, akkor $A$ determinánsa $0$.
		\end{tcolorbox}	
		
		\begin{tcolorbox}[title={Tétel.: Determináns két sor összeadása}]
			Ha $n \geq 2$, $\lambda \in \mathbb{R}$ esetén az $A \in \mathbb{R}^{n x n}$ mátrix egyik sorához egy másik sorának a $\lambda $-szorosát hozzáadjuk, akkor az így keletkezett mátrix determinánsa is $A$, tehát az a rangtartó átalakítás, amikor egyik sorhoz egy másik sor számszorosát adjuk, egyben determinánstartó is!
		\end{tcolorbox}	
		
		\begin{tcolorbox}[title={Tétel.: Felsőháromszög mátrix}]
			Felső háromszög mátrix determinánsa a főátlóban lévő elemek szorzata.
		\end{tcolorbox}	
		
		\begin{tcolorbox}[title={Tétel.: Inverz létezése}]
			$A \in \mathbb{R}^{n x n}$ esetén:\\
			$|A|$ $\neq$ $0$ $\iff$ ${\varrho}(A) = n \iff {\exists}A^{-1}$.
		\end{tcolorbox}	
	\end{frame}
	
	\begin{frame}
		\begin{tcolorbox}[title={Tétel.: Kifejtési tétel}]
			$n \geq 2$, $A \in \mathbb{R}^{n x n}$ esetén\\
			
			\begin{enumerate}
    				\item Tetszőleges $1 \leq i \leq n$ esetén $|A|$ $=$ $\sum_{j = 1}^n a_{ij}A_{ij}$.
    				\item Tetszőleges $1 \leq j \leq n$ esetén $|A|$ $=$ $\sum_{i = 1}^n a_{ij}A_{ij}$.
    				\end{enumerate}
		\end{tcolorbox}	
		
		\begin{tcolorbox}[title={Tétel.: Cramer-szabály}]
			$A = [a_1, ..., a_n]$ $\in \mathbb{R}^{n x n}$, $A \geq 0$, $b \in \mathbb{R}^{n}$ esetén:\\
			
			 !$\exists$ $x \in \mathbb{R}^n$, melyre: $Ax = b$, továbbá az $x$ $j$-edik komponense $(j = 1, ..., n)$\\
			$x_j =$ $\frac{det([a_1, ..., b, ..., a_n])}{det([a_1, ..., a_j, ..., a_n])}$
		\end{tcolorbox}	
		
		\begin{tcolorbox}[title={Tétel.: Vandermonde-determináns, és kifejtése}]
			$V_n(a_1, ..., a_n) =$ $\begin{vmatrix} 
  				1 & a_1 & a_1^2 & \cdots & a_1^{n-1} \\
  				\vdots & \vdots  & \vdots & & \vdots\\
  				1 & a_1 & a_1^2 & \cdots & a_1^{n-1}
			\end{vmatrix}$ $=$ $\prod_{n \geq i > j \geq 1} (a_i - a_j)$
		\end{tcolorbox}	
		
		\begin{tcolorbox}[title={Def.: Részmátrix}]
			$j x k$-as részmátrix: $j$ sor és $k$ oszlop kiválasztásával a metszéspontokba kerülő elemek alkotta $j x k$-as mátrix.
		\end{tcolorbox}	
		
		\begin{tcolorbox}[title={Tétel: Részmátrix és determináns}]
			$A \in \mathbb{R}^{n x m}$ és ${\varrho}(A) = r \geq 1$ esetén $A$-nak van olyan $r x r$-es részmátrixa, melynek determinánsa $\neq 0$, viszont minden $(r + 1) x (r + 1)$-es részmátrix determinánsa $0$.
		\end{tcolorbox}	
	\end{frame}
	
	\begin{frame}
		\begin{tcolorbox}[title={Tétel: Egyenletrendszer megoldása, determináns}]
			Legyen $A \in \mathbb{R}^{n x n}$. Az $Ax = 0$ homogén lineáris egyenletrendszernek akkor és csak akkor van nemtriviális megoldása, ha $A = 0$.
		\end{tcolorbox}	
		
		\begin{tcolorbox}[title={Tétel: Szorzástétel}]
			$A, B \in \mathbb{R}^{n x n}$ $\Rightarrow$ $|AB| = |A| |B|$
		\end{tcolorbox}	
		
		\begin{tcolorbox}[title={Tétel: Szorzástétel}]
			$A, B \in \mathbb{R}^{n x n}$ $\Rightarrow$ $|AB| = |A| |B|$
		\end{tcolorbox}	
		
		\begin{tcolorbox}[title={Def.: Hasonlóság}]
			$A, B \in \mathbb{R}^{n x n}$ esetén azt mondjuk, hogy az $A$ hasonló $\mathbb{R}$ felett a $B$-hez (jelölés: $A {\sim}_{\mathbb{R}} B$), ha létezik olyan invertálható $D \in \mathbb{R}^{n x n}$, melyre $B = D^{-1}AD$
		\end{tcolorbox}	
		
		\begin{tcolorbox}[title={Def.: Diagonizáció}]
			Az $A$ diagonalizálható $\mathbb{R}$ felett, ha $\mathbb{R}$ felett hasonló egy diagonális mátrixhoz.
		\end{tcolorbox}	
	\end{frame}
	
	\begin{frame}
		\begin{tcolorbox}[title={Def.: Jobb oldali sajátvektor, sajátérték}]
			Legyen $n$ pozitív egész, $A \in \mathbb{R}^{n x n}$\\
			Az $x \in \mathbb{R}^{n}$ jobb oldali sajátvektora $A$-nak, ha\\

			\begin{enumerate}
			\item $x \neq 0$
			\item $\exists$ ${\lambda}_0 \in \mathbb{R}$  : $Ax = {\lambda}_0x$
			\end{enumerate}						
			
			Ilyenkor a ${\lambda}_0$ az $x$ jobb oldali sajátvektorhoz tartozó jobb oldali sajátértéke az $A$-nak. 
		\end{tcolorbox}	
		
		\begin{tcolorbox}[title={Def.: Bal oldali sajátvektor, sajátérték}]
			Legyen $n$ pozitív egész, $A \in \mathbb{R}^{n x n}$\\
			Az $y \in \mathbb{R}^{1xn}$ bal oldali sajátvektora $A$-nak, ha\\

			\begin{enumerate}
			\item $y \neq 0 = [0, ..., 0]$
			\item $\exists$ ${\mu}_0 \in \mathbb{R}$  : $yA = {\mu}_0x$
			\end{enumerate}						
			
			Ilyenkor a ${\mu}_0$ az $y$ bal oldali sajátvektorhoz tartozó bal oldali sajátértéke az $A$-nak. 
		\end{tcolorbox}	
		
		\begin{tcolorbox}[title={Tétel: Diagonizálhatóság, sajátvektor}]
			Legyen $A \in \mathbb{R}^{n x n}$.\\
			$A$ diagonalizálható $\mathbb{R}$ felett $\iff$ létezik $\mathbb{R}^n$-ben az $A$ sajátvektoraiból álló bázis (röviden: SB).
		\end{tcolorbox}	
		
		\begin{tcolorbox}[title={Def.: Sajátaltér}]
			Legyen $A \in \mathbb{R}^{n x n}$, ${\lambda}_0 \in \mathbb{R}$ pedig egy (jobb oldali) sajátértéke az $A$-nak $A$ ${\lambda}_0$-hoz tartozó sajátaltér:\\
			$W_{{\lambda}_0} := \{x$ $|$ $x \in \mathbb{R}^n,$ $Ax = {\lambda}_0x\}$.
		\end{tcolorbox}	
		
		\begin{tcolorbox}[title={Def.: Karakterisztikus polinom}]
			Legyen $n$ pozitív egész, $A \in \mathbb{R}^{n x n}$.\\
			Az $A$ mátrix karakterisztikus polinomja: $k_A({\lambda}) := |A - I_n{\lambda}|$.
		\end{tcolorbox}	
		
		\begin{tcolorbox}[title={Tétel: Karakterisztikus polinom, hasonlóság}]
 			$A, B \in \mathbb{R}^{n x n}$ és $A {\sim}_\mathbb{R} B$ esetén $k_A({\lambda}) = k_B({\lambda})$.
		\end{tcolorbox}	
	\end{frame}
	
	\begin{frame}
		\begin{tcolorbox}[title={Def.: Euklideszi tér, Skaláris szorzat}]
 			Legyen $\mathbb{K} = \mathbb{R}$ vagy $\mathbb{K} = \mathbb{C}$, továbbá $V$ vektortér a $\mathbb{K}$ felett.\\
 			Azt mondjuk, hogy a $V$ (valós vagy komplex) euklideszi tér, ha adott benne egy skaláris szorzatnak nevezett ${\langle}x, y{\rangle}$ $:$ $V$ $x$ $V$ $\rightarrow$ $\mathbb{K}$ függvény, melyre a következők teljesülnek minden $x, y, z \in V$ és $\lambda \in \mathbb{K}$ esetén:

			\begin{enumerate}
			\item ${\langle}y, x{\rangle} = {\langle}x, y{\rangle}$
			\item ${\langle}{\lambda}x, y{\rangle} = {\lambda}{\langle}x, y{\rangle}$
			\item ${\langle}x, y + z{\rangle} = {\langle}x, y{\rangle} + {\langle}x, z{\rangle}$
			\item ${\langle}x, x{\rangle}$ mindíg (valós és) nemnegatív
			\item ${\langle}x, x{\rangle} = 0  \iff  x = 0$
			\end{enumerate}
		\end{tcolorbox}	
		
		\begin{tcolorbox}[title={Def.: Norma}]
 			Legyen $V$ valós vagy komplex euklideszi tér. $x \in V$ esetén az $x$ (euklideszi) normája:\\
 			$||x||$ $:=$ $\sqrt{{\langle}x, x{\rangle}}$
		\end{tcolorbox}	
		
		\begin{tcolorbox}[title={Tétel: Cauchy-egyenlőtlenség}]
 			Legyen $V$ valós vagy komplex euklideszi tér.\\
 			Ekkor tetszőleges $x, y \in V$ -re $|{\langle}x, y{\rangle}|$ $\leq$ $||x|| \cdot ||y||$.||
 			Itt egyenlőség akkor és csak akkor teljesül, ha $x, y$ lineárisan összefüggő.
		\end{tcolorbox}	
	\end{frame}
	
	\begin{frame}
		\begin{tcolorbox}[title={Def.: Ortogonált, ortonormált bázis}]
 			Legyen $V$ n-dimenziós (valós vagy komplex) euklideszi tér, $e_1, ..., e_n$ $B$ $V$-ben.\\
 			Az $e_1, ..., e_n$ ortogonális bázis (OB) $V$-ben, ha (bázis és) elemei páronként ortogonálisak.\\
 			Az $e_1, ..., e_n$ ortonormált bázis (ONB) $V$-ben, ha elemei páronként ortogonálisak és normájuk $1$.
		\end{tcolorbox}	
		
		\begin{tcolorbox}[title={Tétel: Ortonormált bázis létezése}]
 			$n > 0$ -ra tetszőleges $n$-dimenziós euklideszi térben létezik ortonormált bázis.
		\end{tcolorbox}	
		
		\begin{tcolorbox}[title={Tétel: A valós szimmetrikus mátrixok spektráltétele}]
 			$A \in \mathbb{R}^{n x n}$ esetén\\
 			
			$A$ szimmetrikus $\iff$ $\exists$ $SONB$ $\mathbb{R}^{n}$-ben és $A$ minden sajátértéke valós.
		\end{tcolorbox}	
		
		\begin{tcolorbox}[title={Def.: Quadratikus alak}]
 			$A \in \mathbb{R}^{n x n}$ és $A^T = A$ esetén az $A$-hoz tartozó $Q$ kvadratikus alak (vagy kvadratikus forma):\\
 			$Q : \mathbb{R}^n \rightarrow \mathbb{R}$, $Q(x) = x^TAx$.
		\end{tcolorbox}	
		
		\begin{tcolorbox}[title={Def.: Definitek}]
			$x \in \mathbb{R}^n \setminus {0}$-ra elnevezés:\\
			\mmedskip
			
			${\forall}{\lambda}_k > 0$: $Q(x) > 0$: Q pozitív definit\\
			${\forall}{\lambda}_k < 0$: $Q(x) < 0$: Q negatív definit\\
			${\forall}{\lambda}_k \geq 0$: $Q(x) \geq 0$: Q pozitív szemidefinit\\
			${\forall}{\lambda}_k \leq 0$: $Q(x) \leq 0$: Q negatív szemidefinit\\
			${\exists}{\lambda}_j > 0$ és ${\exists}{\lambda}_k < 0$: $Q(u_j) > 0$, $Q(u_k) < 0$: Q indefinit
		\end{tcolorbox}	
	\end{frame}
	
	\begin{frame}
		\begin{tcolorbox}[title={Tétel.: Karakterisztikus szorzat}]
 			Legyen az $A = $ $\begin{bmatrix} 
  				a_{11} & \cdots & a_{1n} \\
  				\vdots &  & \vdots \\
  				a_{n1} & \cdots & a_{nn}
			\end{bmatrix}$ $=$ $A^T \in \mathbb{R}^{n x n}$ Karakterisztikus szorzata:\\
			\mmedskip
			 
 			${\Delta}_0$ $=$ $1$; ${\Delta}_1$ $=$ $a_{11}$; ${\Delta}_2$ $=$ $\begin{vmatrix} 
  				a_{11} & a_{12} \\
  				a_{21} & a_{22}
			\end{vmatrix}$; ${\Delta}_3$ $=$ $\begin{vmatrix} 
  				a_{11} & a_{12} & a_{13} \\
  				a_{21} & a_{22}& a_{23} \\
  				a_{31} & a_{32}& a_{33}
			\end{vmatrix}$; $...$; ${\Delta}_n$ $=$ $|A|$.\\
			\mmedskip
			
			Az $A$-hoz tartozó $Q$ pozitív definit $\iff$ ${\forall}j \in \{0, 1, . . . , n\}$-re ${\Delta}_j > 0$.\\
			Az $A$-hoz tartozó $Q$ negatív definit $\iff$ ${\forall}j \in \{0, 1, . . . , [n/2]\}$-re ${\Delta}_{2j} > 0$ és ${\forall}j \in
{0, 1, . . . , [(n - 1)/2]}$-re ${\Delta}_{2j + 1} < 0$.\\
			\mmedskip

			Az első esetben azt mondják, hogy a karakterisztikus sorozat jeltartó, a másodikban pedig azt, hogy a karakterisztikus sorozat jelváltó (a ${\Delta}_0$-t ne felejtsük ki!).
		\end{tcolorbox}	
		
		\begin{tcolorbox}[title={Tétel: mátrix, tranzponáltja, skaláris szorzat}]
 			Legyen $A = A^T \in \mathbb{R}^{n x n}$. Ha $\lambda$ és $\mu$ az $A$ különböző sajátértékei, továbbá $x \in W_{\lambda}, y \in W_{\mu}$, akkor ${\langle}x, y{\rangle} = 0$.
		\end{tcolorbox}	
		
		\begin{tcolorbox}[title={Def.: Vektortérhomomorfizmus, Vektortérizomorfizmus}]
 			Legyen $\varphi :  \mathbb{R}^n \rightarrow \mathbb{R}^m$.\\
 			$\varphi$ vektortérhomomorfizmus [vagy homogén lineáris leképezés, vagy lineáris leképezés, vagy művelettartó leképezés $+, \lambda$-ra], ha\\
 			
 			\begin{enumerate}
 			\item $u, v \in \mathbb{R}^n$ $\Rightarrow$ $\varphi(u + v) = \varphi(u) + \varphi(v)$
			\item ${\lambda} \in \mathbb{R}, u \in \mathbb{R}^n$ $\Rightarrow$ ${\varphi}({\lambda}u) = {\lambda}{\varphi}(u)$
 			\end{enumerate}

			Ha egy lineáris leképezés, azaz vektortérhomomorfizmus netán bijektív, akkor vektortér-izomorfizmusnak hívjuk.
		\end{tcolorbox}	
		
		\begin{tcolorbox}[title={Tétel: Egyértelmű kiterjesztés tétel}]
 			Legyen $e_1, ..., e_n$ bázis $\in \mathbb{R}^n$-ben; $w_1, ..., w_n$ tetszőleges vektorok $\mathbb{R}^m$-ben. Ekkor ${\exists}!{\varphi} : \mathbb{R}^n \rightarrow \mathbb{R}^m$ vektortérhomomorfizmus, melyre $\varphi(e_i) = w_i (i = 1, ..., n)$.
		\end{tcolorbox}	
	\end{frame}
	
	
	
	\begin{frame}
		\begin{tcolorbox}[title={Def.: Vektortérhomomorfizmus mátrixa}]
 			Ha $e_1, ..., e_n$ bázis $\mathbb{R}^n$-ben; $f_1, ..., f_m$ bázis $\mathbb{R}^m$-ben,\\
 			$\varphi : \mathbb{R}^n \rightarrow \mathbb{R}^m$ vektortérhomomorfizmus, akkor a $\varphi$ mátrixa az $e; f$ bázispárban\\
 			\mmedskip

			$[{\varphi}]^{e;f}$ $:=$ $[[{\varphi}(e_1)]_f, ..., [{\varphi}(e_n)]_f] \in \mathbb{R}^{m x n}$
		\end{tcolorbox}	
		
		\begin{tcolorbox}[title={Def.: Vektortérhomomorfizmusok halmaza}]
 			$Hom(\mathbb{R}^n, \mathbb{R}^m)$ jelölje a $\mathbb{R}^n$-ből $\mathbb{R}^m$-be képező vektortérhomomorfizmusok halmazát.\\
 			\mmedskip
 			
			$\varphi, \psi \in Hom(\mathbb{R}^n, \mathbb{R}^m)$ esetén legyen\\
			$\varphi + \psi : \mathbb{R}^n \rightarrow \mathbb{R}^m$ úgy, hogy $u \in \mathbb{R}^n$-re\\
			$(\varphi + \psi)(u) = \varphi(u) + \psi(u)$. \\
			\mmedskip
						
			${\lambda} \in \mathbb{R}$ és $\varphi \in Hom(\mathbb{R}^n, \mathbb{R}^m)$ esetén legyen\\
			${\lambda}_{\phi} : \mathbb{R}^n \rightarrow \mathbb{R}^m$ úgy, hogy $u \in \mathbb{R}^n$-re\\
			$({\lambda}\varphi)(u) = {\lambda}(\varphi(u))$.
		\end{tcolorbox}	
		
		\begin{tcolorbox}[title={Def.: Vektortérhomomorfizmus mátrixa}]
 			Legyen $V_1 = \mathbb{R}^n$, $V_2 = \mathbb{R}^m$ és $V_3 = \mathbb{R}^s$\\
 			$\varphi \in Hom(V_1, V_2)$, $\psi \in Hom(V_2, V_3)$.\\
 			Legyen ${\psi}\varphi : V_1 \rightarrow V_3$ úgy, hogy $u \in V_1$-re $({\psi}{\varphi})(u) = \psi(\varphi(u))$.\\
			Könnyen látható, hogy a definiált ${\psi}{\varphi} \in Hom(V_1, V_3)$.
		\end{tcolorbox}	
		
		\begin{tcolorbox}[title={Def.: Képtér, magtér}]
 			 Legyen $\varphi \in Hom(\mathbb{R}^n, \mathbb{R}^m)$.\\
 			 
 			 $\varphi$ képtere: $Im \varphi$ $:=$ $\{v' | v' \in \mathbb{R}^m$ ${\exists}u \in \mathbb{R}^n$ $\varphi(u) = v'\}$\\
 			 $\varphi$ magtere: $Ker \varphi$ $:=$ $\{ x | x \in \mathbb{R}^n$ ${\varphi}(x) = 0'\}$
		\end{tcolorbox}	
	\end{frame}
	
	\begin{frame}
		\begin{tcolorbox}[title={Def.: Sajtávektor, sajátérték}]
 			 Legyen $\varphi \in Hom(\mathbb{R}^n, \mathbb{R}^m)$.\\
 			 Az $u \in \mathbb{R}^n$ sajátvektora $\varphi$-nek, ha\\
 			 
 			 \begin{enumerate}
 			 \item $u \neq 0$.
			 \item ${\exists}{\lambda}_0 \in \mathbb{R}$ $:$ $\varphi(u) = {\lambda}_0u$.
 			 \end{enumerate}

			 Ilyenkor a ${\lambda}_0$ az $u$ sajátvektorhoz tartozó sajátértéke a $\varphi$-nek.
		\end{tcolorbox}	
		
		\begin{tcolorbox}[title={Tétel: Sajátvektor, sajátérték, lineáris függetlenség}]
 			Legyen $\varphi \in Hom(\mathbb{R}^n, \mathbb{R}^n)$., továbbá $u_1, u_2, ..., u_k \in \mathbb{R}^n$ sajátvektorai $\varphi$-nek, továbbá ${\lambda}_1, {\lambda}_2, ..., {\lambda}_k \in \mathbb{R}$ a megfelelő sajátértékek, melyek páronként különbözök.\\
 			
 			Ekkor $u_1, u_2, ..., u_k$ lineárisan független sajátvektorrendszer.
		\end{tcolorbox}	
		
		\begin{tcolorbox}[title={Tétel: Sajátérték, sakátbázis}]
 			Ha a $\varphi \in Hom(\mathbb{R}^n, \mathbb{R}^n)$ lineáris transzformációnak $n$ darab páronként különbözö sajátértéke van (ahol $n = dim \mathbb{R}^n$), akkor létezik $\mathbb{R}^n$-ben SB, azaz a $\varphi$ sajátvektoraiból álló bázis.
		\end{tcolorbox}	
		
		\begin{tcolorbox}[title={Def.: Vektortér dimenziója}]
 			Az $\mathbb{R}$ feletti $V$ vektortér dimenziója:\\
 			
			$dim V$ $=$ $\begin{cases}
   				0,				& \text{ha } V = \{ \u{0} \}\\
    				|B|,                & \text{ha } V \neq \{ \u{0} \} \text{ és van véges G V-ben } \\
    				\infty  & \text{egyébként}
			\end{cases}$	
		\end{tcolorbox}	
		
		\begin{tcolorbox}[title={Def.: Vektortér dimenziója}]
 			Az $\mathbb{R}$ feletti $V$ vektortér dimenziója:\\
 			
			$dim V$ $=$ $\begin{cases}
   				0,				& \text{ha } V = \{ \u{0} \}\\
    				|B|,                & \text{ha } V \neq \{ \u{0} \} \text{ és van véges G V-ben } \\
    				\infty  & \text{egyébként}
			\end{cases}$	
		\end{tcolorbox}	
	\end{frame}
	
	\begin{frame}
		\begin{tcolorbox}[title={Def.: Vektortérhomomorfizmus, Vektortérizomorfizmus}]
 			Legyen Legyen $V_1$ és $V_2$ vektortér $\mathbb{R}$ felett, $\varphi : V_1 \rightarrow V_2$.\\
 			$\varphi$ vektortérhomomorfizmus [vagy homogén lineáris leképezés, vagy lineáris leképezés, vagy művelettartó leképezés $+, \lambda$-ra], ha\\
 			
 			\begin{enumerate}
 			\item $u, v \in V_1$ $\Rightarrow$ $\varphi(u + v) = \varphi(u) + \varphi(v)$
			\item ${\lambda} \in \mathbb{R}, u \in V_1$ $\Rightarrow$ ${\varphi}({\lambda}u) = {\lambda}{\varphi}(u)$
 			\end{enumerate}

			Ha egy lineáris leképezés, azaz vektortérhomomorfizmus netán bijektív, akkor vektortér-izomorfizmusnak hívjuk.
		\end{tcolorbox}	
		
		\begin{tcolorbox}[title={Tétel: Egyértelmű kiterjesztés tétel}]
 			Legyen $V_1$ és $V_2$ vektortér az $\mathbb{R}$ felett, $dim V_1$ $=$ $n > 0$, $e_1, ..., e_n$ bázis $V_1$-ben, $w_1, ..., w_n$ tetszőleges vektorok $V_2$-ben. Ekkor\\
 			
			${\exists}! \varphi : V_1 \rightarrow V_2$ vektortérhomomorfizmus, melyre $\varphi(e_i) = w_i (i = 1, ..., n)$.
		\end{tcolorbox}	
	
		\begin{tcolorbox}[title={Def.: Vektortérhomomorfizmus mátrixa}]
 			Ha $e_1, ..., e_n$ bázis $V_1$-ben; $f_1, ..., f_m$ bázis $V_2$-ben,\\
 			$\varphi : V_1 \rightarrow V_2$ vektortérhomomorfizmus, akkor a $\varphi$ mátrixa az $e; f$ bázispárban\\
 			\mmedskip

			$[{\varphi}]^{e;f}$ $:=$ $[[{\varphi}(e_1)]_f, ..., [{\varphi}(e_n)]_f] \in \mathbb{R}^{m x n}$
		\end{tcolorbox}	
	
		\begin{tcolorbox}[title={Def.: Lineáris tranzformáció}]
 			Azok a lineáris leképezések, amelyeknél $V_1 = V_2 = V$.\\
 			
 			Ilyenkor megállapodunk abban, hogy a mátrix definíciójában mindkét helyre ugyanazt a bázist vesszük:\\
 			Ha $e_1, ..., e_n$ bázis $V$-ben, $\varphi : V \rightarrow V$ lineáris transzformáció, akkor\\
 			$[\varphi]^e$ $:=$ $[\varphi]^{e;e}$
		\end{tcolorbox}	
	\end{frame}
	
	\begin{frame}
		\begin{tcolorbox}[title={Def.: Vektortérhomomorfizmusok halmaza}]
 			Legyen $V_1$ és $V_2$ vektortér $\mathbb{R}$ felett. $Hom(V_1, V_2)$ jelölje a $V_1$-ből $V_2$-be képező vektortérhomomorfizmusok halmazát.\\
 			\mmedskip
 			
			$\varphi, \psi \in Hom(V_1, V_2)$ esetén legyen\\
			$\varphi + \psi : V_1 \rightarrow V_2$ úgy, hogy $u \in V_1$-re\\
			$(\varphi + \psi)(u) = \varphi(u) + \psi(u)$. \\
			\mmedskip
						
			${\lambda} \in \mathbb{R}$ és $\varphi \in Hom(V_1, V_2)$ esetén legyen\\
			${\lambda}_{\phi} : V_1 \rightarrow V_2$ úgy, hogy $u \in V_1$-re\\
			$({\lambda}\varphi)(u) = {\lambda}(\varphi(u))$.
		\end{tcolorbox}	
		
		\begin{tcolorbox}[title={Def.: Képtér, magtér}]
 			 Legyen $\varphi \in V_1, V_2$, $\varphi \in Hom(V_1, V_2)$\\
 			 
 			 $\varphi$ képtere: $Im \varphi$ $:=$ $\{v' | v' \in V_2$ ${\exists}u \in V_1$ $\varphi(u) = v'\}$\\
 			 $\varphi$ magtere: $Ker \varphi$ $:=$ $\{ x | x \in V_1$ ${\varphi}(x) = 0'\}$
		\end{tcolorbox}	
	
		\begin{tcolorbox}[title={Tétel: Vektortér dimenziója}]
 			 Legyen $V_1$ és $V_2$ véges dimenziós vektortér $\mathbb{R}$ felett. Ekkor\\
 			 
 			 $V_1 \cong V_2$ $\iff$ $dim V_1 = dim V_2$
		\end{tcolorbox}	
		
		\begin{tcolorbox}[title={Tétel: Dimenzióösszefüggés}]
 			 Legyen $V_1$ és $V_2$ vektortér $\mathbb{R}$ felett, $\varphi \in Hom(V_1, V_2)$. Ha $V_1$ véges dimenziós, akkor\\
 			 
 			 $dim Im \varphi + dim Ker \varphi = dim V_1$.\\
			 (Itt $\varphi$ rangja: $r(\varphi) = dim Im \varphi$; $\varphi$ defektusa: $d(\varphi) = dim Ker \varphi$)
		\end{tcolorbox}	
		
		\begin{tcolorbox}[title={Tétel: Szorzástétel}]
 			Legyen $V_1$, $V_2$ és $V_3$ vektortér $\mathbb{R}$ felett, dimenziójuk rendre $n, m, s$ (pozitív egészek); $e_1, ..., e_n$ bázis $V_1$-ben; $f_1, ..., f_m$ bázis $V_2$-ben; $g_1, ..., g_s$ bázis $V_3$-ban; $\varphi \in Hom(V_1, V_2)$, $\psi \in Hom(V_2, V_3)$. Ekkor\\
 			
			$[{\psi}{\varphi}]^{e;g} = [{\psi}]^{f ;g}[{\varphi}]^{e;f}$.
		\end{tcolorbox}	

	\end{frame}
	
	\begin{frame}
		\begin{tcolorbox}[title={Tétel: Új bázisba való áttérés}]
 			Legyen $V$ vektortér $\mathbb{R}$ felett; $dim V = n > 0$; $e_1, ..., e_n$ bázis $V$-ben; $e'_1, ..., e'_n$ bázis $V$-ben.\\
 			Ekkor ${\exists}!{\tau} \in Hom(V, V) : {\tau}(e_i) = e'_i (i = 1, ..., n)$.\\
 			\mmedskip
 			
 			Legyen $D = [{\tau}]^e$.\\
 			Ekkor $D$ invertálható, és tetszőleges $\varphi \in Hom(V, V )$ esetén\\
 			\mmedskip
 			
			$[{\varphi}]^{e'} = D^{-1}[{\varphi}]^eD$.
		\end{tcolorbox}	
		
		\begin{tcolorbox}[title={Tétel: Skaláris szorzat}]
			Legyen $\mathbb{K} = \mathbb{R}$ vagy $\mathbb{K} = \mathbb{C}$, továbbá $V$ vektortér a $\mathbb{K}$ felett.\\
			Korábban egy skaláris szorzatnak nevezett ${\langle}x, y{\rangle}$ $:$ $V x V \rightarrow \mathbb{K}$ függvényre teljesültek a következők minden $x, y, z \in V$ és $\lambda \in \mathbb{K}$ esetén:\\

 			  \begin{enumerate}
 			 \item ${\langle}x, y{\rangle} = \overline{{\langle}x, y{\rangle}}$
 			 \item ${\langle}{\lambda}x, y{\rangle} = {\lambda}{\langle}x, y{\rangle}$, ${\langle}x, {\lambda}y{\rangle} = \overline{{\lambda}}{\langle}x, y{\rangle}$
			 \item ${\langle}x, y + z{\rangle} = {\langle}x, y{\rangle} + {\langle}x, z{\rangle}$, ${\langle}y + z, x{\rangle} = {\langle}y, x{\rangle} + {\langle}z, x{\rangle}$.
			 \item ${\langle}x, x{\rangle}$ mindíg (valós és) nemnegatív
			 \item ${\langle}x, x{\rangle}$ $=$ $0$ $\iff$ $x = 0$
 			 \end{enumerate}	
 		\end{tcolorbox}	
		
		\begin{tcolorbox}[title={Tétel: Skaláris szorzat}]
 			Legyen $\mathbb{K} = \mathbb{R}$ vagy $\mathbb{K} = \mathbb{C}$, továbbá $V$ vektortér a $\mathbb{K}$ felett.\\
 			
 			 Az $\mathcal{A}(x, y) : V x V \rightarrow \mathbb{K}$ függvényt $V$-n értelmezett ($\mathbb{K}$-tól függően valós vagy komplex) bilineáris függvénynek, bilineáris alaknak, vagy bilineáris formának hívjuk, ha teljesülnek a következők minden $x, y, z \in V$ és $\lambda \in \mathbb{K}$ esetén:
 			 
 			 \begin{enumerate}
 			 \item $\mathcal{A}({\lambda}x, y) = {\lambda} \mathcal{A}(x, y)$, $\mathcal{A}(x, {\lambda}y) = \overline{{\lambda}} \mathcal{A}(x, y)$
			\item $\mathcal{A}(x, y + z) = \mathcal{A}(x, y) + \mathcal{A}(x, z)$, $\mathcal{A}(y + z, x) = \mathcal{A}(y, x) + \mathcal{A}(z, x)$.
 			 \end{enumerate}
		\end{tcolorbox}	
	\end{frame}
	
	\begin{frame}
		\begin{tcolorbox}[title={Def.: Bilineáris alak}]
 			Legyen az $\mathcal{\mathcal{A}}(x, y) : V x V \rightarrow \mathbb{K}$ a $V$-n értelmezett bilineáris alak, továbbá $e_1, ..., e_n$ bázis $V$-ben. $[\mathcal{A}]^e \in \mathbb{K}^{nxn}$, éspedig minden szóbajövő $j, k$ esetén $(a_{jk} =)$ ${}_{j} [\mathcal{A}]_k^e =
\mathcal{A}(e_j, e_k)$.
			Így az előbbi kifejezés a következő formában is írható:\\
			\mmedskip
			
			$\mathcal{A}(\sum_{j = 1}^n x_je_j, \sum_{k = 1}^n y_ke_k) = \sum_{j = 1}^n \sum_{k = 1}^n x_j \overline{y_k} a_{jk}$\\
			\mmedskip
			
			speciálisan:\\
			$\mathcal{A}(\sum_{j = 1}^n x_je_j, \sum_{k = 1}^n x_ke_k) = \sum_{j = 1}^n \sum_{k = 1}^n x_j \overline{x_k} a_{jk}$\\
			\mmedskip
			
			ugyanez $\mathbb{K} = \mathbb{R}$ esetén:\\
			$\mathcal{A}(\sum_{j = 1}^n x_je_j, \sum_{k = 1}^n x_ke_k) = \sum_{j = 1}^n \sum_{k = 1}^n x_j x_k a_{jk}$\\
		\end{tcolorbox}		

		\begin{tcolorbox}[title={Def.: Hermite-féle bilineáris alak}]
			Legyen az $\mathcal{A}(x, y) : V x V \rightarrow \mathbb{K}$ a $V$ -n értelmezett bilineáris alak. Azt mondjuk, hogy az $\mathcal{A}$ Hermite-féle bilineáris alak ($\mathbb{K}  = \mathbb{R}$ esetén a szimmetrikus bilineáris alak kifejezés is használatos), ha $x, y \in V$ esetén teljesül:\\
			$\mathcal{A}(y, x) = \mathcal{A}(x, y)$.
		\end{tcolorbox}	
		
		\begin{tcolorbox}[title={Def.: Szimmetrikus bilineáris alak}]
			Legyen $\mathbb{K}  = \mathbb{R}$, az $\mathcal{A}(x, y) : V x V \rightarrow \mathbb{R}$ pedig $V$-n értelmezett szimmetrikus bilineáris alak (tehát most minden $x, y \in V$ esetén teljesül: $\mathcal{A}(y, x) = \mathcal{A}(x, y)$). Ekkor az $\mathcal{A}$-hoz tartozó $\mathcal{Q}$ kvadratikus alak:\\
			 $\mathcal{Q} : V \rightarrow \mathbb{R}$, $\mathcal{Q}(x) = \mathcal{A}(x, x)$.
		\end{tcolorbox}		
		
		\begin{tcolorbox}[title={Def.: Kvadratikus alak}]
			Legyen $\mathbb{K}  = \mathbb{C}$, az $\mathcal{A}(x, y) : V x V \rightarrow \mathbb{C}$ pedig $V$-n értelmezett bilineáris alak. Ekkor az $\mathcal{A}$-hoz tartozó kvadratikus alak:\\
			
			$\mathcal{Q} : V \rightarrow \mathbb{C}$, $\mathcal{Q}(x) = \mathcal{A}(x, x)$.
		\end{tcolorbox}	
		
		\begin{tcolorbox}[title={Tétel: Szimmetrikus bilineáris alak, kvadratikus alak}]
			Legyen $\mathbb{K}  = \mathbb{R}$, az $\mathcal{A}(x, y) : V x V \rightarrow \mathbb{R}$ pedig $V$-n értelmezett szimmetrikus bilineáris alak (tehát most minden $x, y \in V$ esetén teljesül: $\mathcal{A}(y, x) = \mathcal{A}(x, y)$). Ekkor az $\mathcal{A}$ minden értéke kifejezhető az $\mathcal{A}$-hoz tartozó $\mathcal{Q}$ kvadratikus alak alkalmas értékei segítségével.
		\end{tcolorbox}	
	\end{frame}
		
	\begin{frame}
		\begin{tcolorbox}[title={Tétel: Szimmetrikus bilineáris alak, kvadratikus alak}]
			Legyen $\mathbb{K}  = \mathbb{R}$, az $\mathcal{A}(x, y) : V x V \rightarrow \mathbb{R}$ pedig $V$-n értelmezett szimmetrikus bilineáris alak (tehát most minden $x, y \in V$ esetén teljesül: $\mathcal{A}(y, x) = \mathcal{A}(x, y)$). Ekkor az $\mathcal{A}$ minden értéke kifejezhető az $\mathcal{A}$-hoz tartozó $\mathcal{Q}$ kvadratikus alak alkalmas értékei segítségével.
		\end{tcolorbox}	
	\end{frame}
	
	%Kiegészítések / Számolások
	\begin{comment}
	\begin{frame}
		\begin{tcolorbox}[title={Bázistranzformáció}]
			Kérdés: Hány dimenziós?\\
			
			 \begin{center}
			\begin{tabular}{ c|c c c c } 
			 \hline
			  & a & b & c & d \\ 
			 ${e_1}$ & 3 & 9 & 1 & 5 \\ 
			 ${e_2}$ & 2 & 10 & 2 & 2 \\ 
			 ${e_3}$ & -1 & 1 & 1 & -3 \\ 
			 ${e_4}$ & 0 & -3 & -1 & 1 \\ 
			 ${e_5}$ & 1 & 2 & 0 & 2 \\ 
			 \hline
			\end{tabular}
			\end{center}
			\mmedskip
			
			asd
		\end{tcolorbox}
	\end{frame}
	\end{comment}
\end{document}
