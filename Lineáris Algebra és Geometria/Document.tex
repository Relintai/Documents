% Compile twice!
% With the current MiKTeX, you need to install the beamer, and the translator packages directly form the package manager!

% !TEX root = ./Headers/PrezA4Page.tex

% Uncomment these to get the presentation form
%\documentclass{beamer}
%\geometry{paperwidth=200mm,paperheight=200mm, top=0in, bottom=0.2in, left=0.2in, right=0.2in}

% Uncomment these, and comment the 2 lines above, to get a paper-type article
%\documentclass[10pt]{article}
%\usepackage{geometry}
%\geometry{top=0.2in, bottom=0.2in, left=0.2in, right=0.2in}
%\usepackage{beamerarticle}
%\renewcommand{\\}{\par\noindent}
%\setbeamertemplate{note page}[plain]

% Half A4 geometry
%\geometry{paperwidth=105mm,paperheight=297mm,top=0.2in, bottom=0.2in, left=0.2in, right=0.2in}

% "1/3" A4 geometry
%\geometry{paperwidth=105mm,paperheight=455mm,top=0.1in, bottom=0.1in, left=0.1in, right=0.1in}

% "1/6" A4 geometry
%\geometry{paperwidth=105mm,paperheight=891mm,top=0.1in, bottom=0.1in, left=0.1in, right=0.1in}

% "1/5" A4 geometry
%\geometry{paperwidth=105mm,paperheight=740mm,top=0.1in, bottom=0.1in, left=0.1in, right=0.1in}

% "1/4" A4 geometry
%\geometry{paperwidth=105mm,paperheight=594mm,top=0.1in, bottom=0.1in, left=0.1in, right=0.1in}

% Uncomment these, to put more than one slide / page into a generated page.
%\usepackage{pgfpages}
% Choose one
%\pgfpagesuselayout{2 on 1}[a4paper]
%\pgfpagesuselayout{4 on 1}[a4paper]
%\pgfpagesuselayout{8 on 1}[a4paper]

% Includes
\usepackage{tikz}
\usepackage{tkz-graph}
\usetikzlibrary{shapes,arrows,automata}
\usepackage[T1]{fontenc}
\usepackage{amsfonts}
\usepackage{amsmath}
\usepackage[utf8]{inputenc}
\usepackage{booktabs}
\usepackage{array}
\usepackage{arydshln}
\usepackage{enumerate}
\usepackage[many, poster]{tcolorbox}
\usepackage{pgf}
\usepackage[makeroom]{cancel}
\usepackage{verbatim}

% Colors
\definecolor{myred}{rgb}{0.87,0.18,0}
\definecolor{myorange}{rgb}{1,0.4,0}
\definecolor{myyellowdarker}{rgb}{1,0.69,0}
\definecolor{myyellowlighter}{rgb}{0.91,0.73,0}
\definecolor{myyellow}{rgb}{0.97,0.78,0.36}
\definecolor{myblue}{rgb}{0,0.38,0.47}
\definecolor{mygreen}{rgb}{0,0.52,0.37}
\colorlet{mybg}{myyellow!5!white}
\colorlet{mybluebg}{myyellowlighter!3!white}
\colorlet{mygreenbg}{myyellowlighter!3!white}

\setbeamertemplate{itemize item}{\color{black}$-$}
\setbeamertemplate{itemize subitem}{\color{black}$-$}
\setbeamercolor*{enumerate item}{fg=black}
\setbeamercolor*{enumerate subitem}{fg=black}
\setbeamercolor*{enumerate subsubitem}{fg=black}

 \renewcommand{\familydefault}{\sfdefault}
%\renewcommand{\familydefault}{\rmdefault}

\renewcommand{\footnotesize}{\fontsize{1.2em}{0.2em}}
\renewcommand{\normalsize}{\fontsize{1.2em}{0.2em}}
\renewcommand{\large}{\footnotesize}
\renewcommand{\Large}{\footnotesize}


\renewcommand{\scriptsize}{\footnotesize}
\renewcommand{\LARGE}{\footnotesize}
\renewcommand{\Huge}{\footnotesize}

\renewcommand{\tiny}{\footnotesize}
\renewcommand{\small}{\footnotesize}

\fontsize{1.2em}{0.2em}
\selectfont

\newcommand{\RHuge}{\fontsize{1.8em}{0.3em}\selectfont}

\newsavebox\CBox 
%\newcommand<>*\textBF[1]{\sbox\CBox{#1}\resizebox{\wd\CBox}{\ht\CBox}{\textbf#2{#1}}}
\newcommand<>*\textBF[1]{\only#2{\sbox\CBox{#1}\resizebox{\wd\CBox}{\ht\CBox}{\textbf{#1}}}}


% These are different themes, only uncomment one at a time
\tcbset{enhanced,fonttitle=\mdseries,boxsep=7pt,arc=0pt,colframe={myyellowlighter},colbacktitle={myyellow},colback={mybg},coltitle={black}, coltext={black},attach boxed title to top left={xshift=-2mm,yshift=-2mm},boxed title style={size=small,arc=0mm}}

%\tcbset{colback=yellow!5!white,colframe=yellow!84!black}
%\tcbset{enhanced,colback=red!10!white,colframe=red!75!black,colbacktitle=red!50!yellow,fonttitle=
%\tcbset{enhanced,attach boxed title to top left}
%\tcbset{enhanced,fonttitle=\bfseries,boxsep=5pt,arc=8pt,borderline={0.5pt}{0pt}{red},borderline={0.5pt}{5pt}{blue,dotted},borderline={0.5pt}{-5pt}{green}}

% Beamer theme
\usetheme{boxes}

% tikz settings for the flowchart(s)
\tikzstyle{decision} = [diamond, minimum width=3cm, minimum height=1cm, text centered, draw=black, fill=green!15]
\tikzstyle{tcolorbox} = [rectangle, draw, fill=blue!15, text width=20em, text centered, minimum height=1em]

\tikzstyle{line} = [draw, -latex']
\tikzstyle{cloud} = [draw, ellipse,fill=red!20, node distance=3cm,
    minimum height=2em]
\tikzstyle{arrow} = [thick,->,>=stealth]

\newcolumntype{C}[1]{>{\centering\let\newline\\\arraybackslash\hspace{0pt}}m{#1}}
\renewcommand{\arraystretch}{1.2}

\setlength\dashlinedash{0.2pt}
\setlength\dashlinegap{1.5pt}
\setlength\arrayrulewidth{0.3pt}

\newcommand{\mtinyskip}{\vspace{0.2em}}
\newcommand{\msmallskip}{\vspace{0.3em}}
\newcommand{\mmedskip}{\vspace{0.5em}}
\newcommand{\mbigskip}{\vspace{1em}}
\renewcommand{\u}[1]{\underline{#1}}

\begin{document}

\begin{frame}[plain]
\begin{tcolorbox}[center, colback={myyellow}, coltext={black}, colframe={myyellow}]
    {\RHuge Lineáris Algebra és Geometria}\\
\end{tcolorbox}
\end{frame}


%\begin{tcolorbox}[title={Def.: }]
%\end{tcolorbox}

% --------------------  HALMAZOK, RELÁCIÓK --------------------

\begin{frame}[plain]
\begin{tcolorbox}[center, colback={myyellow}, coltext={black}, colframe={myyellow}]
    {\RHuge Vektorterek, Leképzések}
    \mmedskip
\end{tcolorbox}
\end{frame}

\begin{frame}
  \begin{tcolorbox}[title={Def.: Linearitás}]
    $f$ leképzés lineáris, ha:\\
    \begin{itemize}
      \item $f(a + b) = f(a) + f(b)$
      \item ${\lambda}f(a) = f({\lambda}b)$
    \end{itemize}
  \end{tcolorbox}


  \begin{tcolorbox}[title={Def.: Vektorok}]
  $\u{a} = (a_1, a_2), \u{b} = (b_1, b_2)$\\
  \mmedskip
  
    Összeadás: $\u{a + b} = (a_1 + b_1, a_2 + b_2)$\\
    Nyújtás: ${\lambda}a = (a_1, a_2) \lor {\lambda}\u{a} = ({\lambda}a_1, {\lambda}a_2)$
  \end{tcolorbox}
  
  \end{frame}
  
  \begin{frame}  
  
	\begin{tcolorbox}[title={Def.: Összeadás}]	
		\begin{align}
			\u{a} + \u{b} &= \begin{bmatrix}
				a_1 + b_1 \\
				a_2 + b_2 \\
				... \\
				a_n + b_n
			\end{bmatrix}
		\end{align}
				
		\tcblower
		
		\textBF{Tulajdonságok} \\
		
		\msmallskip
		
		\begin{enumerate}
			\item Van értelme
			\item Kommutativitás - $\u{a} + \u{b} = \u{b} + \u{a}$
			\item Asszociativitás - $(\u{a} + \u{b}) + \u{c} = \u{a} + (\u{b} + \u{c})$
			\item Van nullelem - ${\exists}0 \rightarrow \u{0}$
			\item Minden elemre létezik additív inverz - ${\forall}\u{a} \in \mathbb{R}^n : {\exists}\u{-a}$, ahol $\u{a} + \u{-a} = \u{0}$ \\
			$\u{-a} = -1 \cdot \u{a} = \u{-a}$, $\u{a} + \u{-a} = \u{0}$
		\end{enumerate}				
	\end{tcolorbox}

  \end{frame}
  
  \begin{frame}
  
    \begin{tcolorbox}[title={Def.: Szorzás számmal}]	
		\begin{align}
			\u{a} + \u{b} &= \begin{bmatrix}
				{\lambda}a_1 \\
				{\lambda}a_2 \\
				... \\
				{\lambda}a_n
			\end{bmatrix}
		\end{align}
		
		\tcblower
		
		\textBF{Tulajdonságok} \\		
		
		 \msmallskip
		 
		 \begin{enumerate}
		 	\item Van értelme
		 	\item Asszociativitás ${\lambda}, {\mu} \in \mathbb{R}$, $({\lambda}{\mu})\u{a} = {\lambda}({\mu}\u{a})$
		 	\item Disztributivitás ${\lambda}, {\mu} \in \mathbb{R}$, $({\lambda} + {\mu})\u{a} = {\lambda}\u{a} + {\mu}\u{b}$
		 	\item Disztributivitás $\u{a}, \u{b} \in \mathbb{R}^n, {\lambda} \in \mathbb{R}$, ${\lambda}\u{a} + \u{b}) = {\lambda}\u{a} + {\lambda}\u{b}$
		 	\item Létezik egységelem. $1 \cdot \u{a} = \u{a}$
		 \end{enumerate}
		
	\end{tcolorbox}
  	
    \end{frame}
  
	\begin{frame}
		 \begin{tcolorbox}[title={Def.: Vektortér}]
			$\mathbb{R}^n$vektortér $\mathbb{R}$ felett, ha igazak rá az összeadás, és a szorzás tulajdonságai.\\

			\mmedskip
			
			Azaz, ha egy $V \neq \emptyset$ tudja ezeket a tulajdonságokat, akkor $V$ vektortér $\mathbb{R}$ felett.			
		\end{tcolorbox}
		
		\begin{tcolorbox}[title={Def.: Altér}]
			Azt mondjuk, hogy $W \leq \mathbb{R}^n$ altere $\mathbb{R}^n$-nek, ha
			\begin{enumerate}
				\item $W \neq \emptyset$
				\item Ha zárt az összeadásra ($\u{a}, \u{b} \in W \Rightarrow \u{a} + \u{b} \in W$)
				\item Ha zárt a számmal való szorzásra ($\u{a} \in W, {\lambda}\u{a} \in W$)
			\end{enumerate}
		\end{tcolorbox}
		
		\begin{tcolorbox}[title={Megj}]
			$\mathbb{R}^2$ $\rightarrow$ alterei: $x, y$ tengely\\
			$\mathbb{R}^3$ $\rightarrow$ alteret: A síkok is.
		\end{tcolorbox}
		
		\begin{tcolorbox}[title={Def.: Vektorrendszer, Lineáris kombináció}]
			\textBF{Vektorrendszer}:\\
			Legyen $k \geq 1$ egész. és legyenek $\u{v_1}, \u{v_2}, ..., \u{v_k} \in \mathbb{R}^n$.\\
			Ezeket a vektorokat együtt \textBF{vektorrendszernek} hívjuk.\\			
			\msmallskip
			
			\textBF{Lineáris kombináció}:\\
			Legyenek ${\lambda}_1, {\lambda}_2, ..., {\lambda}_k \in \mathbb{R}$ adottak,\\
			ekkor a ${\lambda}_1\u{v_1} + {\lambda}_2\u{v_2} + ... + {\lambda}_k\u{v_k}$ kifejezést a\\
			$\u{v_1}, \u{v_2}, ..., \u{v_k}$ vektorrendszer \u{lineáris kombinációjának} nevezzük.\\
			\msmallskip			
			
			\textBF{triviális lineáris kombináció}:\\
			Ha ${\lambda}_1 = {\lambda}_2 = ... = {\lambda}_k = 0$, akkor a lineáris kombináció triviális.
		\end{tcolorbox}
	\end{frame}

	\begin{frame}
	\begin{tcolorbox}[title={Def.: Lineáris összefüggőség}]
			Legyen $k \geq 1$ egész. és legyen $\u{v_1}, \u{v_2}, ..., \u{v_k} \in \mathbb{R}^n$. vektorrendszer.\\
			Ekkor azt mondjuk, hogy a $\u{v_1}, \u{v_2}, ..., \u{v_k}$ vektorrendszerünk \textBF{lineárisan összefüggő}, ha létezik nemtriviális lineáris kombinációja, melyre:\\
			${\lambda}_1\u{v_1} + {\lambda}_2\u{v_2} + ... + {\lambda}_k\u{v_k} = \u{0}$
		\end{tcolorbox}
		
		\begin{tcolorbox}[title={Def.: Lineáris függetlenség}]
			Legyen $k \geq 1$ egész. és legyen $\u{v_1}, \u{v_2}, ..., \u{v_k} \in \mathbb{R}^n$. vektorrendszer.\\
			Ekkor azt mondjuk, hogy a $\u{v_1}, \u{v_2}, ..., \u{v_k}$ vektorrendszerünk \textBF{lineárisan független}, ha csak a triviális lineáris kombinációjára igaz, hogy:\\
			${\lambda}_1\u{v_1} + {\lambda}_2\u{v_2} + ... + {\lambda}_k\u{v_k} = \u{0}$
		\end{tcolorbox}
		
		\begin{tcolorbox}[title={Def.: Bázis}]
			Legyen $ V \leq \mathbb{R}^k$ altér, és legyen adott $\u{v_1}, \u{v_2}, ..., \u{v_k}$ vektorrendszer.\\
			Azt mondjuk, hogy a $\u{v_1}, \u{v_2}, ..., \u{v_k}$ vektorrendszer \textBF{bázis} $V$-ben, ha:\\
			\begin{itemize}
				\item Lineárisan függetlenek
				\item Tetszőleges eleme $V$-nek előáll belőlük lineáris kobinációként.
			\end{itemize}
			\mmedskip
			
			 (Megj:   $n$ dimenzóban $n$ elemű egy bázis)
		\end{tcolorbox}
		
		\begin{tcolorbox}[title={Tétel: Lineáris kombináció, és bázisok}]
			$\u{b_1}, \u{b_2}, ..., \u{b_k}$ bázis $V$-ben, akkor $\forall \u{v} \in V$ elem \textBF{egyértelműen} előáll belőle lineáris kombinációjaként.
		\end{tcolorbox}
		
		\begin{tcolorbox}[title={Tétel: Bázisok, és Lineáris kombináció}]
			Ha a $\u{v_1}, \u{v_2}, ..., \u{v_k}$ vektorrendszer olyan V-ben, hogy ha $\forall a \in V$ egyértelműen létezik ${\alpha}_1, ..., {\alpha}_k \in \mathbb{R}$, hogy $\u{a} = {\alpha}_1\u{b_1} + {\alpha}_2\u{b_2} + ... + {\alpha}_k\u{b_k} \Rightarrow \u{b_1}, \u{b_2}, ..., \u{b_k}$ bázis.
		\end{tcolorbox}
		
	\end{frame}
	
	\begin{frame}
		\begin{tcolorbox}[title={Tétel.: Bázistransformáció}]
			Legyen $V \leq \mathbb{R}^n$, $\u{b_1}, u{b_2}, ..., u{b_k}$ bázis $V$-ben.\\
			Legyen $a \in V$ adott, és $\u{a} = {\alpha}_1\u{v_1} +  {\alpha}_2\u{v_2} +  ... + {\alpha}_k\u{v_k}$.\\
			Ekkor $\u{b_1}, u{b_2}, ..., u{b_k} \iff {\alpha}_i \neq 0$ bázis.\\
			\mmedskip
			
			Akkor cserélhetjük ki, ha az együtthatója nem 0 az $\u{a}$-ban.
		\end{tcolorbox}
		
		\begin{tcolorbox}[title={Képlet}]
			$x_j = x_j - \frac{x_i}{{{\alpha}_i}} {\alpha}_j$
		\end{tcolorbox}
		
		\begin{tcolorbox}[title={Öf táblázat}]
		\end{tcolorbox}
	\end{frame}
	
	\begin{frame}
		\begin{tcolorbox}[title={Def.: Lineáris függőség}]
			$A \neq \emptyset$, $A \subseteq \mathbb{R}^n$, azt mondjuk hogy $\u{v} \in \mathbb{R}^n$ \textBF{lineárisan függ} $A$-tól,\\
			ha létezik véges sok elem $A$-ban, hogy $\u{v}$ előáll az ő lineáris kombinációjaként.
		\end{tcolorbox}
		
		\begin{tcolorbox}[title={Def.: Lineáris függőség}]
			$k \geq 2$, $\u{a_1}, u{a_2}, ..., u{a_k} \in \mathbb{R}^n$,\\
			ekkor $\u{a_1}, u{a_2}, ..., u{a_k}$ összefüggő $\iff$ $\exists i \in \{ 1, ..., k \}$, hogy $a_i$ lineárisan függ a többitől.
		\end{tcolorbox}
		
		\begin{tcolorbox}[title={Áll.: Lineáris függőség}]
			Ha$\u{a_1}, u{a_2}, ..., u{a_k}$, $\u{b} \in \mathbb{R}^n$\\		
			$\u{a_1}, u{a_2}, ..., u{a_k}$ lineárisan független, de $\u{a_1}, u{a_2}, ..., u{a_k}, \u{b}$ lineárisan összefüggő, akkor\\
			$\u{b}$ lineárisan független az $\u{a_1}, u{a_2}, ..., u{a_k}$ vektorrendszertől.	
		\end{tcolorbox}
		
		\begin{tcolorbox}[title={Def.: Halmaz által generált altér / Lineáris Burok}]
			$A \neq \emptyset$, $A \leq \mathbb{R}^n$:\\
			$: W(A) = \{ \u{b} \in \mathbb{R}^n | \u{v}$ lineárisan függ $A$-tól $\}$
		\end{tcolorbox}
		
		\begin{tcolorbox}[title={Def.: Vektor koordinátái}]
			\begin{align}
				[a]_{\u{b_1}, \u{b_2}, ..., \u{b_k}} &= \begin{bmatrix}
					{\lambda}a_1 \\
					{\lambda}a_2 \\
					... \\
					{\lambda}a_n
				\end{bmatrix} \in \mathbb{R}^k
			\end{align}
		\end{tcolorbox}
		
		\begin{tcolorbox}[title={Tétel: Altér}]
			$W(A)$ altér $(A \neq \emptyset)$
		\end{tcolorbox}
	\end{frame}
	
	\begin{frame}
		\begin{tcolorbox}[title={Tétel: Alterek metszete}]
			Ha $V_1$ és $V_2$ is altér $\Rightarrow$ $V_1 \cap V_2$ is altér.
		\end{tcolorbox}
		
		\begin{tcolorbox}[title={Def.: Span}]
			Azt mondjuk, hogy az $A \subseteq \mathbb{R}^n$ halmaz által \textBF{generált / kifeszített altér} az $A$-t tartalmazó alterek / vektorterek metszete.\\
			\msmallskip
			
			Jel.: $Span(A)$
		\end{tcolorbox}
		
		\begin{tcolorbox}[title={Tétel: Span és Lineáris burok}]
			$Span(A) = W(A)$
		\end{tcolorbox}
		
		\begin{tcolorbox}[title={Def.: Generátorrendszer}]
			Azt mondjuk, hogy $G$ vektorrendszer \textBF{generátorrendszere} $V$ altérnek, ha $Span(G) = W(G)$
		\end{tcolorbox}
		
		\begin{tcolorbox}[title={Tétel.: Generátorrendszer létezése}]
			Ha $V \leq \mathbb{R}^n$-ben létezik véges méretű generátorrendszer $\Rightarrow$ belőle kiválasztható bázis.
		\end{tcolorbox}
	\end{frame}
	
	\begin{frame}
		\begin{tcolorbox}[title={Tétel: Kicserélési tétel}]
			Legyen $V \leq \mathbb{R}^n$, legyen $a_1, ..., a_k$ lineárisan független, és $b_1, ..., b_n$ generátorrendszer. Ekkor:\\
			\begin{itemize}
				\item $\exists j$, hogy tetszőleges $i$-re $v_j, a_2, ..., a_k$ is Lineárisan független.\\
				(megj.: Igazából $a_1, ..., a_k$ bármilyen eleme lecserélhető)
				\item $|LF| \leq |GR|$ ($|LF|$ = $LF$ elemszáma, $LF$ = $a_1, ..., a_k$)
			\end{itemize}
		\end{tcolorbox}
		
		\begin{tcolorbox}[title={Tétel.: Bázis}]
			Ha $V \leq \mathbb{R}^n,$ és $B_1, B_2$ bázis, akkor\\
			$|B_1| < + \infty \rightarrow |B_1| = |B_2|$
		\end{tcolorbox}
		
		\begin{tcolorbox}[title={Bázis}]
			\begin{itemize}			
				\item Minden bázis mérete $\mathbb{R}^n$-ben $n$
				\item $V \leq \mathbb{R}^n$ és van véges generátorrendszer $\Rightarrow$ Leszűkíthető bázissá.
				\item $V \leq \mathbb{R}^n$ és $v_1, ..., v_k$ vektorrendszer lineárisan független a $V$-ben. $\Rightarrow$ Leszűkíthető bázissá.
			\end{itemize}
			\mmedskip
			
			Ezekből követketik, hogy a bázis a maximális elemszámú lineárisan független vektorrendszer.\\
			\mmedskip
			
			Maximális lineárisan független vektorrendszer elemszáma = minimális generátorrendszer elemszáma = bázis elemszáma
		\end{tcolorbox}
		
		\begin{tcolorbox}[title={Def.: Dimenzió}]
			$V \leq \mathbb{R}^n$ dimenziója:\\
			\mmedskip
			
			\[
   				dim(V) = 
			\begin{cases}
   				0,				& \text{ha } V = \{ \u{0} \}\\
    				|B|,                & \text{ha } V \neq \{ \u{0} \} \text{ (B a V-nek egy bázisa.) } \\
			\end{cases}
			\]		
		\end{tcolorbox}
		
		\begin{tcolorbox}[title={Def.: Rang}]
			 $v_1, ..., v_k \in \mathbb{R}^n$ vektorrendszer rangja, az általuk generált altér dimenziója.
		\end{tcolorbox}
	\end{frame}
	
	\begin{frame}

	\begin{tcolorbox}[title={Def.: Mátrix}]	
		Legyen $e_k$ adott $a_k$ az $m$ és $n$ pozitív egész számok, továbbá minden
	
		$i \in \{1, ..., m\} és j \in \{1, ..., n\}$ esetére az $aij$ valós számok. Az
		
		táblázatot egy $\mathbb{R}$ feletti mátrixnak nevezzük, és $A$-val jelöljük, részletesebben $A = [aij]mn$, vagy
		
		Az $A$ mátrix $i$-ed $i_k$ sora $j$-edik elemén $e_k$ jelölése: $aij$ vagy $i[A]j$.
	\end{tcolorbox}
	
	\begin{tcolorbox}[title={Def.: Mátrixok egyenlősége}]	
		Az $A$ és a $B$ mátrixok egyenlők, ha alakjuk azonos (mondjuk $m x n$-es) és a megfelelő elemeik megegyeznek, azaz minden „szóbajövő" $i$, $j$ párra $(1 \leq i \leq m, 1 \leq j \leq n)$ teljesül, hogy $i[A]j = i[B]j$. Az $\mathbb{R}$ feletti $m x n$-es mátrixok halmazát $\mathbb{R}^{m x n}$-mel jelöljük ($\mathbb{R}^{m}$ = $\mathbb{R}^{m x 1}$).
	\end{tcolorbox}
	
	\begin{tcolorbox}[title={Def.: Mátrix összeadás, számmal való szorzás}]	
		Az $\mathbb{R}^{m}$-beli komponensenkénti összeadás és valós számmal való szorzás (1/2) mintájára természetes módon kínálkoznak $m x n$-es mátrixok esetén a megfelelő elemek összeadásával, illetve az összes elemnek egy valós számmal szorzásával a következő műveletek:

$+$ : $\mathbb{R}^{m x n}$ x $\mathbb{R}^{m x n} \rightarrow \mathbb{R}^{m x n}$, A, B $\in$ $\mathbb{R}^{m x n}$ esetén $A + B$ $\in$ $\mathbb{R}^{m x n}$ és minden szóbajövő
$i, j$-re $_{i} [A + B]_j$ = $_{i} [A]_j + _{i} [B]_j$.

${\lambda}$ : $\mathbb{R}$ x $\mathbb{R}^{m x n} \rightarrow \mathbb{R}^{m x n}$, ${\lambda} \in \mathbb{R}$, $A \in \mathbb{R}^{m x n}$ esetén ${\lambda}A \in \mathbb{R}^{m x n}$ és minden szóbajövő $i, j$-re $_{i} [{\lambda}A]_j$ = ${\lambda} _{i} [A]_j$.

Az $\mathbb{R}^{m x n}$-re is teljesül az 1/3 oldali 10 tulajdonság megfelelője a fenti $+$, ${\lambda}$ műveletekre nézve, így $\mathbb{R}^{m x n}$-et is $\mathbb{R}$ feletti vektortérnek mondjuk.
	\end{tcolorbox}	
	
	\begin{tcolorbox}[title={Def.: Nullmátrix}]	
		Nullmátrix:\begin{align}
				 0 &:= \begin{bmatrix}
					{\lambda}a_1 \\
					{\lambda}a_2 \\
					... \\
					{\lambda}a_n
				\end{bmatrix} \in \mathbb{R}^k
		\end{align}
	\end{tcolorbox}		
	
	\end{frame}
	
	
	
	\begin{frame}

		\begin{tcolorbox}[title={Def.: Mátrixszorzás}]	
	$A$ $in$ $\mathbb{R}^{m x n}$, $B$ $\in$ $\mathbb{R}^{n x k}$ esetén $AB$ $\in$ $\mathbb{R}^{m x k}$ úgy, hogy minden szóbajövő $i, j$-re (most $1 \leq i \leq m, 1 \leq j \leq k$)
\mmedskip
	
$_{i} [AB]_j$ $=$ $_{i} [A]_1$ $_{1} [B]_j$ + $_{i} [A]_2$ $_{2} [A]_j$ + $...$ + $_{i} [A]_n$ $_{n} [B]_j$ = $\sum_{l = 1}^n$ $_{i} [A]_l$ $_{l} [B]_j$
\mmedskip

Ez az ún. sor-oszlop szorzás: a szorzatmátrix i-edik sora j-edik elemét úgy kapjuk, hogy a bal oldali mátrix i-edik sorának és a jobb oldali mátrix j-edik oszlopának megfelelő elemeit összeszorozzuk, s a kapott szorzatokat összeadjuk.
		\end{tcolorbox}		
	
	\end{frame}
	
	
	
	
	
	
	
	
	
	
	
	
	
	
	
	
	
	
	
	
	
	
	
	
	
	
	
	
	
	
	
	
	
	
	
	
	
	
	
	
	
	
	
	
	
	
	
	
	
	
	
	
	%Kiegészítések / Számolások
	\begin{comment}
	\begin{frame}
		\begin{tcolorbox}[title={Bázistranzformáció}]
			Kérdés: Hány dimenziós?\\
			
			 \begin{center}
			\begin{tabular}{ c|c c c c } 
			 \hline
			  & a & b & c & d \\ 
			 ${e_1}$ & 3 & 9 & 1 & 5 \\ 
			 ${e_2}$ & 2 & 10 & 2 & 2 \\ 
			 ${e_3}$ & -1 & 1 & 1 & -3 \\ 
			 ${e_4}$ & 0 & -3 & -1 & 1 \\ 
			 ${e_5}$ & 1 & 2 & 0 & 2 \\ 
			 \hline
			\end{tabular}
			\end{center}
			\mmedskip
			
			asd
		\end{tcolorbox}
	\end{frame}
	\end{comment}
\end{document}
