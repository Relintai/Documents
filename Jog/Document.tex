% Compile twice!
% With the current MiKTeX, you need to install the beamer, and the translator packages directly form the package manager!

% !TEX root = ./PrezA4Page.tex

% Uncomment these to get the presentation form
%\documentclass{beamer}
%\geometry{paperwidth=200mm,paperheight=200mm, top=0in, bottom=0.2in, left=0.2in, right=0.2in}

% Uncomment these, and comment the 2 lines above, to get a paper-type article
%\documentclass[10pt]{article}
%\usepackage{geometry}
%\geometry{top=0.2in, bottom=0.2in, left=0.2in, right=0.2in}
%\usepackage{beamerarticle}
%\renewcommand{\\}{\par\noindent}
%\setbeamertemplate{note page}[plain]

% Half A4 geometry
%\geometry{paperwidth=105mm,paperheight=297mm,top=0.2in, bottom=0.2in, left=0.2in, right=0.2in}

% "1/3" A4 geometry
%\geometry{paperwidth=105mm,paperheight=455mm,top=0.1in, bottom=0.1in, left=0.1in, right=0.1in}

% "1/6" A4 geometry
%\geometry{paperwidth=105mm,paperheight=891mm,top=0.1in, bottom=0.1in, left=0.1in, right=0.1in}

% "1/5" A4 geometry
%\geometry{paperwidth=105mm,paperheight=740mm,top=0.1in, bottom=0.1in, left=0.1in, right=0.1in}

% "1/4" A4 geometry
%\geometry{paperwidth=105mm,paperheight=594mm,top=0.1in, bottom=0.1in, left=0.1in, right=0.1in}

% Uncomment these, to put more than one slide / page into a generated page.
%\usepackage{pgfpages}
% Choose one
%\pgfpagesuselayout{2 on 1}[a4paper]
%\pgfpagesuselayout{4 on 1}[a4paper]
%\pgfpagesuselayout{8 on 1}[a4paper]

% Includes
\usepackage{tikz}
\usepackage{tkz-graph}
\usetikzlibrary{shapes,arrows,automata}
\usepackage[T1]{fontenc}
\usepackage{amsfonts}
\usepackage{amsmath}
\usepackage[utf8]{inputenc}
\usepackage{booktabs}
\usepackage{array}
\usepackage{arydshln}
\usepackage{enumerate}
\usepackage[many, poster]{tcolorbox}
\usepackage{pgf}
\usepackage[makeroom]{cancel}
\usepackage{ulem}

\providecommand{\includecolors}{
% Colors
\definecolor{myred}{rgb}{0.87,0.18,0}
\definecolor{myorange}{rgb}{1,0.4,0}
\definecolor{myyellowdarker}{rgb}{1,0.69,0}
\definecolor{myyellowlighter}{rgb}{0.91,0.73,0}
\definecolor{myyellow}{rgb}{0.97,0.78,0.36}
\definecolor{myblue}{rgb}{0,0.38,0.47}
\definecolor{mygreen}{rgb}{0,0.52,0.37}
\colorlet{mybg}{myyellow!5!white}
\colorlet{mybluebg}{myyellowlighter!3!white}
\colorlet{mygreenbg}{myyellowlighter!3!white}

\setbeamertemplate{itemize item}{\color{black}$-$}
\setbeamertemplate{itemize subitem}{\color{black}$-$}
\setbeamercolor*{enumerate item}{fg=black}
\setbeamercolor*{enumerate subitem}{fg=black}
\setbeamercolor*{enumerate subsubitem}{fg=black}

% These are different themes, only uncomment one at a time
\tcbset{enhanced,fonttitle=\mdseries,boxsep=7pt,arc=0pt,colframe={myyellowlighter},colbacktitle={myyellow},colback={mybg},coltitle={black}, coltext={black},attach boxed title to top left={xshift=-2mm,yshift=-2mm},boxed title style={size=small,arc=0mm}}

%\tcbset{colback=yellow!5!white,colframe=yellow!84!black}
%\tcbset{enhanced,colback=red!10!white,colframe=red!75!black,colbacktitle=red!50!yellow,fonttitle=
%\tcbset{enhanced,attach boxed title to top left}
%\tcbset{enhanced,fonttitle=\bfseries,boxsep=5pt,arc=8pt,borderline={0.5pt}{0pt}{red},borderline={0.5pt}{5pt}{blue,dotted},borderline={0.5pt}{-5pt}{green}}
}% fallback definition
\includecolors

\setbeamertemplate{itemize item}{\color{black}$-$}
\setbeamertemplate{itemize subitem}{\color{black}$-$}
\setbeamercolor*{enumerate item}{fg=black}
\setbeamercolor*{enumerate subitem}{fg=black}
\setbeamercolor*{enumerate subsubitem}{fg=black}

 \renewcommand{\familydefault}{\sfdefault}
%\renewcommand{\familydefault}{\rmdefault}

\renewcommand{\footnotesize}{\fontsize{1.2em}{0.2em}}
\renewcommand{\normalsize}{\fontsize{1.2em}{0.2em}}
\renewcommand{\large}{\footnotesize}
\renewcommand{\Large}{\footnotesize}


\renewcommand{\scriptsize}{\footnotesize}
\renewcommand{\LARGE}{\footnotesize}
\renewcommand{\Huge}{\footnotesize}

\renewcommand{\tiny}{\footnotesize}
\renewcommand{\small}{\footnotesize}

\fontsize{1.2em}{0.2em}
\selectfont

\newcommand{\RHuge}{\fontsize{1.8em}{0.3em}\selectfont}

\newsavebox\CBox 
%\newcommand<>*\textBF[1]{\sbox\CBox{#1}\resizebox{\wd\CBox}{\ht\CBox}{\textbf#2{#1}}}
\newcommand<>*\textBF[1]{\only#2{\sbox\CBox{#1}\resizebox{\wd\CBox}{\ht\CBox}{\textbf{#1}}}}

% Beamer theme
\usetheme{boxes}

% tikz settings for the flowchart(s)
\tikzstyle{decision} = [diamond, minimum width=3cm, minimum height=1cm, text centered, draw=black, fill=green!15]
\tikzstyle{tcolorbox} = [rectangle, draw, fill=blue!15, text width=20em, text centered, minimum height=1em]

\tikzstyle{line} = [draw, -latex']
\tikzstyle{cloud} = [draw, ellipse,fill=red!20, node distance=3cm,
    minimum height=2em]
\tikzstyle{arrow} = [thick,->,>=stealth]

\newcolumntype{C}[1]{>{\centering\let\newline\\\arraybackslash\hspace{0pt}}m{#1}}
\renewcommand{\arraystretch}{1.2}

\setlength\dashlinedash{0.2pt}
\setlength\dashlinegap{1.5pt}
\setlength\arrayrulewidth{0.3pt}

\newcommand{\mtinyskip}{\vspace{0.2em}}
\newcommand{\msmallskip}{\vspace{0.3em}}
\newcommand{\mmedskip}{\vspace{0.5em}}
\newcommand{\mbigskip}{\vspace{1em}}
\renewcommand{\u}[1]{\underline{#1}}

\begin{document}

\begin{frame}[plain]
\begin{tcolorbox}[center, colback={myyellow}, coltext={black}, colframe={myyellow}]
    {\RHuge Jog}\\
\end{tcolorbox}
\end{frame}

\begin{frame}[plain]
\begin{tcolorbox}[center, colback={myyellow}, coltext={black}, colframe={myyellow}]
    {\RHuge 01 Jogi alapok}\\
\end{tcolorbox}
\end{frame}

\begin{frame}  

%1. 
\begin{tcolorbox}[title={1. Kérdés}]
Az alább felsoroltak közül melyik nem jellemzõ a társadalmi és jogi normákra?
\tcblower
a) A hipotetikus szerkezet.\\
b) A szankció.\\
\uline {c) A múltra irányultság.}\\
d) Mindegyik jellemzõ.
\end{tcolorbox}

%2.
\begin{tcolorbox}[title={2. Kérdés}]
Az alábbiak közül melyik nem tartozik a társadalmi normák lényeges mozzanatai közé?
\tcblower
a) A magatartás leírása.\\
b) A magatartás minõsítése. (tilos, kötelezõ stb.)\\
c) A következmények leírása.\\
\uline {d) Mindegyik oda tartozik.}
\end{tcolorbox}

%3. 
\begin{tcolorbox}[title={3. Kérdés}]
Melyik állítás a legjellemzõbb a jogi normákra?
\tcblower
a) A társadalom alkotja, érvényesülését az állam biztosítja.\\
\uline {b) Az állam alkotja, érvényesülését az állam biztosítja.}\\
c) Az állam alkotja, érvényesülését a társadalom biztosítja.\\
d) A társadalom alkotja, érvényesülését a társadalom biztosítja.
\end{tcolorbox}

%4. 
\begin{tcolorbox}[title={4. Kérdés}]
Az alábbi 3 állítás (a,b,c) közül melyik hamis?
\tcblower
a) A jogi normákat gyorsabban meg lehet változtatni, mint a társadalmi normákat.\\
b) Az erkölcsi normák szankciórendszerét a társadalmi nyomás jellemzi.\\
c) A jogi norma az emberek formális magatartására irányul.\\
\uline {d) Mindhárom elõzõ állítás igaz.}
\end{tcolorbox}

\end{frame}


\begin{frame}

%5. 
\begin{tcolorbox}[title={5. Kérdés}]
Az alábbi 3 állítás (a,b,c) közül melyik hamis?
\tcblower
\uline {a) A társadalmi normákat gyorsabban meg lehet változtatni, mint a jogi normákat.}\\
b) Az erkölcsi normák szankciórendszerét a társadalmi nyomás jellemzi.\\
c) A jogi norma az emberek formális magatartására irányul.\\
d) Mindhárom elõzõ állítás igaz.
\end{tcolorbox}

%6.
\begin{tcolorbox}[title={6. Kérdés}]
A jog melyik funkciójának megvalósítását biztosítja elsõsorban a bírósági szervezet?
\tcblower
a) Az integráló funkciót.\\
\uline {b) A konfliktus-feloldó funkciót.}\\
c) A társadalomszervezõ funkciót.\\
d) Egyik funkció megvalósítását sem biztosítja.
\end{tcolorbox}

%7.
\begin{tcolorbox}[title={7. Kérdés}]
Melyik állítás hamis?
\tcblower
a) Az anyagi jogforrás azt a szervet jelenti, amelytõl a jogi norma ered.\\
\uline {b) A belsõ jogforrás ugyanazt jelenti, mint az alaki jogforrás.}\\
c) Az alaki jogforrás azt a formát jelenti, amelyben a jogi norma megjelenik.\\
d) Elsõdleges jogforrás maga a jogalkotó hatalom, az állam. 
\end{tcolorbox}

%8.
\begin{tcolorbox}[title={8. Kérdés}]
Mit mond ki a "Lex superior derogat legi inferiori" alapelv?
\tcblower
a) A késõbb megalkotott jogszabály a korábbinak a hatályát lerontja.\\
b) A speciális jogszabályt kell alkalmazni az általánossal szemben.\\
\uline {c) Alacsonyabb szintû jogszabály nem ellenkezhet magasabb szintûvel.}\\
d) Jogszabállyal ellentétes másik jogszabályt nem lehet alkotni.
\end{tcolorbox}

\end{frame}


\begin{frame}

%9. 
\begin{tcolorbox}[title={9. Kérdés}]
Az alábbi jogszabályok közül melyik tartozik az anyagi jog körébe?
\tcblower
a) A büntetõeljárásról szóló törvény.\\
b) A polgári eljárásról szóló törvény.\\
c) A közigazgatási eljárásról szóló törvény.\\
\uline {d) Egyik sem.}
\end{tcolorbox}

%10.
\begin{tcolorbox}[title={10. Kérdés}]
Az alábbi jogszabályok közül melyik tartozik az anyagi jog körébe?
\tcblower
a) A büntetõeljárásról szóló törvény.\\
\uline {b) A polgári törvénykönyv.}\\
c) A közigazgatási eljárásról szóló törvény.\\
d) Egyik sem.
\end{tcolorbox}

%11. 
\begin{tcolorbox}[title={11. Kérdés}]
Az alábbi jogszabályok közül melyik tartozik az alaki jog körébe?
\tcblower
a) A büntetõ törvénykönyv.\\
\uline {b) A büntetõeljárásról szóló törvény.}\\
c) A polgári törvénykönyv.\\
d) A munka törvénykönyve.
\end{tcolorbox}

%12. 
\begin{tcolorbox}[title={12. Kérdés}]
Melyik állítás igaz?
\tcblower
a) A polgári jog a közjog területéhez tartozik.\\
b) A büntetõjog a magánjog területéhez tartozik.\\
\uline {c) A kereskedelmi jog a magánjog területéhez tartozik.}\\
d) A közigazgatási jog a magánjog területéhez tartozik.
\end{tcolorbox}

\end{frame}


\begin{frame}

%13. 
\begin{tcolorbox}[title={13. Kérdés}]
Melyik állítás igaz az angolszász jogrendszerekre?
\tcblower
a) A jogalkotás és jogalkalmazás szférája mereven elválik egymástól.\\
\uline {b) A szabályok jelentõs része nincs törvénybe foglalva.}\\
c) E jogrendszerek pillérei az írott jogforrások.\\
d) A joganyag önálló, zárt rendszert alkot.
\end{tcolorbox}

%14. 
\begin{tcolorbox}[title={14. Kérdés}]
Melyik jogrendszerre a leginkább jellemzõ, hogy politikailag erõsen kötött?
\tcblower
a) Az angolszász jogrendszerekre.\\
\uline {b) A szocialista jogrendszerekre.}\\
c) A kontinentális jogrendszerekre.\\
d) A vallási alapú jogrendszerekre.
\end{tcolorbox}

%15. 
\begin{tcolorbox}[title={15. Kérdés}]
Melyik állítás igaz a kontinentális jogrendszerekre?
\tcblower
a) A bírónak jelentõs szerepe van a jogalkotásban.\\
b) A jogszabályok kevésbé elvontak és általánosak.\\
\uline {c) A joganyag önálló, zárt rendszert alkot.}\\
d) A jogszabályok egy-egy vitás ügy eldöntésekor keletkeznek.
\end{tcolorbox}

%16. 
\begin{tcolorbox}[title={16. Kérdés}]
Melyik állítás nem igaz a szocialista jogrendszerekre vonatkozóan?
\tcblower
a) A politika közvetlenül befolyásolja a jogot.\\
b) A magánjognak kisebb a jelentõsége.\\
c) A jogi túlszabályozás jellemzõ.\\
\uline {d) Mindegyik fenti állítás igaz rá.}
\end{tcolorbox}

\end{frame}


\begin{frame}

%17. 
\begin{tcolorbox}[title={17. Kérdés}]
Melyik állítás nem igaz a szocialista jogrendszerekre vonatkozóan?
\tcblower
a) A politika közvetlenül befolyásolja a jogot.\\
\uline {b) A magánjognak nagy jelentõsége van.}\\
c) A jogi túlszabályozás jellemzõ.\\
d) Mindegyik fenti állítás igaz rá.
\end{tcolorbox}

\end{frame}

\begin{frame}[plain]
\begin{tcolorbox}[center, colback={myyellow}, coltext={black}, colframe={myyellow}]
    {\RHuge 02 Jogszabalytan}\\
\end{tcolorbox}
\end{frame}

\begin{frame}  

%18. 
\begin{tcolorbox}[title={18. Kérdés}]
A tényeknek olyan összessége, amelyhez valamilyen joghatás fûzõdik, ...
\tcblower
a) a jogi norma diszpozíciója.\\
b) a jogi norma szankciója.\\
\uline {c) a jogi norma hipotézise.}\\
d) a jogi norma jogkövetkezménye.
\end{tcolorbox}

%19. 
\begin{tcolorbox}[title={19. Kérdés}]
Az a magatartási szabály, amelyet a jogalkotó a jogalany számára elõír, ...
\tcblower
\uline {a) a jogi norma diszpozíciója.}\\
b) a jogi norma szankciója.\\
c) a jogi norma hipotézise.\\
d) a jogi norma jogkövetkezménye.
\end{tcolorbox}

%20. 
\begin{tcolorbox}[title={20. Kérdés}]
Egészítse ki a kipontozott részt.
A ... esetén a jogalkotó nem határozza meg, hogy mely tényállás esetén szükséges 
a normában meghatározott magatartást tanúsítani, csupán példaszerûen felsorolja azokat.
\tcblower
a) zárt tényállás\\
\uline {b) nyitott tényállás}\\
c) parancsoló norma\\
d) diszpozitív norma
\end{tcolorbox}

%21.
\begin{tcolorbox}[title={21. Kérdés}]
Milyenfajta rendelkezés nem fordulhat elõ a jogi norma diszpozíciójában?
\tcblower
a) tiltó\\
b) parancsoló\\
c) megengedõ\\
\uline {d) mindhárom elõfordulhat}
\end{tcolorbox}

\end{frame}


\begin{frame}

%22. 
\begin{tcolorbox}[title={22. Kérdés}]
Melyik állítás igaz?
\tcblower
a) A lopás bûncselekménye mindig vagyoni jellegû szankciót von maga után.\\
b) A jogi norma megsértése nem járhat személyi jellegû szankcióval.\\
c) A szankció egyidejûleg mindig csak egyféle joghátrányt alkalmazhat.\\
\uline {d) Az érvénytelenség a szankciók egyik típusa.}
\end{tcolorbox}

%23. 
\begin{tcolorbox}[title={23. Kérdés}]
Melyik állítás igaz?
\tcblower
a) Az apaság vélelme nem megdönthetõ.\\
b) A fikció esetén lehetõség van az ellenkezõ bizonyítására.\\
\uline {c) A kézbesítési vélelem megdönthetõ.}\\
d) A fikció legfontosabb jogi következménye a bizonyítási teher megfordítása.
\end{tcolorbox}

%24.
\begin{tcolorbox}[title={24. Kérdés}]
Melyik állítás igaz?
\tcblower
\uline {a) Az apaság vélelme megdönthetõ.}\\
b) A fikció esetén lehetõség van az ellenkezõ bizonyítására.\\
c) A kézbesítési vélelem nem megdönthetõ.\\
d) A fikció legfontosabb jogi következménye a bizonyítási teher megfordítása.
\end{tcolorbox}

\end{frame}


\begin{frame}

%25. 
\begin{tcolorbox}[title={25. Kérdés}]
Egy jogszabály érvényességének hány feltétele van? Sorolja is fel ezeket.
\tcblower
a) 3, a következõk:\\
\uline {b) 4, a következõk: megfelelõ szerv; megfelelõ eljárás; kihirdetés; hierarchiába illeszkedés}\\
c) 2, a következõk:\\
d) 5, a következõk:
\end{tcolorbox}

%26. 
\begin{tcolorbox}[title={26. Kérdés}]
Melyik állítás igaz?
\tcblower
a) A jogszabály idõbeli hatálya nem lehet visszamenõleges.\\
b) A magyar jogszabályok területi hatálya Magyarország teljes területe.\\
\uline {c) Egy jogszabály lehet egyidejûleg érvényes, de nem hatályos.}\\
d) Egy jogszabály lehet egyidejûleg hatályos és érvénytelen.
\end{tcolorbox}

%27. 
\begin{tcolorbox}[title={27. Kérdés}]
Melyik állítás igaz?
\tcblower
a) Egy magyar jogszabály minden magyar állampolgárra vonatkozik.\\
\uline {b) Egy jogszabály csak a jövõre vonatkozóan állapíthat meg kötelezettséget.}\\
c) A személyi hatály azt határozza meg, hogy ki alkotta a jogszabályt.\\
d) Egy magyar jogszabály csak magyar állampolgárokra vonatkozik.
\end{tcolorbox}

%28.
\begin{tcolorbox}[title={28. Kérdés}]
Melyik állítás igaz?
\tcblower
a) Egy jogszabály nem érvényes, ha a köztársasági elnök nem írta alá.\\
b) Egy törvény nem érvényes, ha ellenkezik egy rendelettel.\\
\uline {c) Egy jogszabály nem érvényes, ha nem hirdették ki.}\\
d) Mindhárom fenti állítás hamis.
\end{tcolorbox}

\end{frame}


\begin{frame}

%29. 
\begin{tcolorbox}[title={29. Kérdés}]
Melyik állítás igaz?
\tcblower
a) Egy jogszabály nem érvényes, ha a köztársasági elnök nem írta alá.\\
\uline {b) Egy rendelet nem érvényes, ha ellenkezik egy törvénnyel.}\\
c) Egy jogszabály akkor is érvényes, ha nem hirdették ki.\\
d) Mindhárom fenti állítás hamis.
\end{tcolorbox}

%30. 
\begin{tcolorbox}[title={30. Kérdés}]
Melyik állítás igaz?
\tcblower
a) A jogszabályok tudományos értelmezésének csak az adott ügyben van kötelezõ ereje.\\
\uline {b) A jogalkotói jogszabály értelmezés minden címzettre nézve kötelezõ.}\\
c) A jogalkalmazói jogszabály értelmezésnek nincs kötelezõ ereje.\\
d) A jogalkotói jogszabály értelmezés tilos. 
\end{tcolorbox}

%31.
\begin{tcolorbox}[title={31. Kérdés}]
Melyik állítás igaz?
\tcblower
a) A közigazgatási szervek eljárása mindig kezdeményezésre indul.\\
\uline {b) A jogalanyok önkéntes jogkövetését is jogalkalmazásnak tekintjük.}\\
c) A jogalkalmazás célja mindig valamilyen jogsérelem orvoslása.\\
d) A jogalkalmazás utolsó lépése a tényállás megállapítása.
\end{tcolorbox}

%32.
\begin{tcolorbox}[title={32. Kérdés}]
Melyik nem következhet be a jogalkalmazás eredményeképpen?
\tcblower
a) Jogviszony keletkezése.\\
b) Jogviszony megszûnése.\\
c) Jogviszony módosulása.\\
\uline {d) Mindhárom fenti bekövetkezhet.}
\end{tcolorbox}

\end{frame}


\begin{frame}

%33. 
\begin{tcolorbox}[title={33. Kérdés}]
Mely szerv biztosítja a bíróságok egységes jogalkalmazását?
\tcblower
a) Az alkotmánybíróság.\\
b) A törvényszék.\\
c) Az ítélõtábla.\\
\uline {d) A Kúria.}
\end{tcolorbox}

%34. 
\begin{tcolorbox}[title={34. Kérdés}]
Melyik nem tartozik a jogszabály értelmezésének módszerei közé?
\tcblower
a) A rendszertani értelmezés.\\
b) A történeti értelmezés.\\
\uline {c) A módszertani értelmezés.}\\
d) A logikai értelmezés.
\end{tcolorbox}

%35. 
\begin{tcolorbox}[title={35. Kérdés}]
A jogviszony szerkezeti elemei közé tartozik
\tcblower
a) a jogviszony helye.\\
b) a jogviszony ideje.\\
\uline {c) a jogviszony tartalma.}\\
d) Egyik sem.
\end{tcolorbox}

%36.
\begin{tcolorbox}[title={36. Kérdés}]
Ingatlan adásvételi szerzõdés esetén a jogviszony tárgya
\tcblower
a) a vételár kifizetése.\\
b) a tulajdonjog átruházása.\\
\uline {c) az ingatlan.}\\
d) Egyik sem.
\end{tcolorbox}

\end{frame}


\begin{frame}

%37. 
\begin{tcolorbox}[title={37. Kérdés}]
Melyik állítás hamis?
\tcblower
a) A kötelmi jogban a relatív szerkezetû jogviszonyok jellemzõk.\\
b) A tulajdonjog abszolút szerkezetû jogviszony.\\
\uline {c) Ha több, mint 2 fél szerepel a szerzõdésben, az abszolút szerkezetû jogviszonyt eredményez.}\\
d) A károkozás relatív szerkezetû jogviszonyt teremt.
\end{tcolorbox}

%38. 
\begin{tcolorbox}[title={38. Kérdés}]
Melyik nem rendelkezik jogi személyiséggel?
\tcblower
a) Az Állam.\\
b) Az egyesület.\\
\uline {c) A társasház.}\\
d) Az egyetem.
\end{tcolorbox}

%39. 
\begin{tcolorbox}[title={39. Kérdés}]
Melyik állítás igaz?
\tcblower
a) Együttes kötelezettek bármelyikétõl az egész tartozás követelhetõ.\\
b) Egyetemleges kötelezettek egyidejûleg, de arányosan kötelesek teljesíteni.\\
c) A kezes az adóssal fele-fele arányban köteles teljesíteni a tartozást.\\
\uline {d) Mindhárom fenti állítás hamis.}
\end{tcolorbox}

\end{frame}

\begin{frame}[plain]
\begin{tcolorbox}[center, colback={myyellow}, coltext={black}, colframe={myyellow}]
    {\RHuge 03 Szemelyek joga}\\
\end{tcolorbox}
\end{frame}

\begin{frame}  

%40. 
\begin{tcolorbox}[title={40. Kérdés}]
Az alábbi 3 állítás (a,b,c) közül melyik hamis?
\tcblower
\uline {a) Aki jogképes, az önállóan köthet szerzõdést.}\\
b) Aki jogképes, az jogosultságok alanya lehet.\\
c) Aki betöltötte a 14. életévét az jogképes.\\
d) Mindhárom elõzõ állítás igaz.
\end{tcolorbox}

%41. 
\begin{tcolorbox}[title={41. Kérdés}]
Melyik állítás igaz?
\tcblower
a) A jogképesség a születéssel kezdõdik.\\
b) A jogképesség mindig a születés elõtti 300-adik napon kezdõdik.\\
c) A jogképesség mindig a fogamzástól kezdõdik.\\
\uline {d) A magzat jogképessége feltételes.}
\end{tcolorbox}

%42. 
\begin{tcolorbox}[title={42. Kérdés}]
Melyik állítás igaz?
\tcblower
a) A bíróság korlátozhatja a természetes személy jogképességét.\\
b) A Ptk. nem korlátozhatja az ember cselekvõképességét.\\
\uline {c) Aki cselekvõképes, az önállóan köthet szerzõdést}.\\
d) Mindhárom fenti állítás hamis.
\end{tcolorbox}

%43. 
\begin{tcolorbox}[title={43. Kérdés}]
Az alábbi 3 állítás (a,b,c) közül melyik hamis?
\tcblower
a) 17 éves személy is lehet nagykorú.\\
b) 17 éves személy is lehet cselekvõképes.\\
c) 13 éves személy is lehet korlátozottan cselekvõképes.\\
\uline {d) Mindhárom elõzõ állítás igaz.}
\end{tcolorbox}

\end{frame}


\begin{frame}

%44.
\begin{tcolorbox}[title={44. Kérdés}]
Az alábbi 3 állítás (a,b,c) közül melyik hamis?
\tcblower
a) A kiskorú személy a házasságkötéssel nagykorúvá válik.\\
b) 17 éves személy is lehet cselekvõképes.\\
c) A 13 éves személy minden esetben cselekvõképtelennek minõsül.\\
\uline {d) Mindhárom elõzõ állítás igaz.}
\end{tcolorbox}

%45.
\begin{tcolorbox}[title={45. Kérdés}]
Az alábbi 3 állítás (a,b,c) közül melyik hamis?
\tcblower
a) Nagykorú személy is lehet cselekvõképtelen.\\
b) Kiskorú személy is lehet korlátozottan cselekvõképes.\\
\uline {c) A kiskorú személy házasságkötése érvénytelen.}\\
d) Mindhárom elõzõ állítás igaz.
\end{tcolorbox}

%46. 
\begin{tcolorbox}[title={46. Kérdés}]
Mit nem tehet meg önállóan a korlátozottan cselekvõképes kiskorú?
\tcblower
\uline {a) Tetszõleges értékû ajándékot adhat.}\\
b) Tetszõleges értékû ajándékot elfogadhat.\\
c) Rendelkezhet munkával szerzett jövedelmével.\\
d) Mindhárom fenti jognyilatkozatot megteheti önállóan.
\end{tcolorbox}

%47. 
\begin{tcolorbox}[title={47. Kérdés}]
Melyik nem tartozik a gyám feladatai közé?
\tcblower
a) A gyermek gondozása, nevelése.\\
b) A gyermek vagyonának kezelése.\\
c) A gyermek törvényes képviselete.\\
\uline {d) Mindhárom fenti a gyám feladatai közé tartozik.}
\end{tcolorbox}

\end{frame}


\begin{frame}

%48. 
\begin{tcolorbox}[title={48. Kérdés}]
Melyikkel nem feltétlenül kell rendelkeznie a jogi személynek?
\tcblower
a) Székhellyel.\\
b) Saját vagyonnal.\\
\uline {c) Telephellyel.}\\
d) Képviselõvel.
\end{tcolorbox}

%49. 
\begin{tcolorbox}[title={49. Kérdés}]
Ki felel a jogi személy tartozásaiért?
\tcblower
a) A vezetõ tisztségviselõ.\\
b) A döntéshozó szerv.\\
c) A tagok és alapítók.\\
\uline {d) Egyik sem.}
\end{tcolorbox}

%50. 
\begin{tcolorbox}[title={50. Kérdés}]
Mikor jön létre a jogi személy?
\tcblower
a) A létesítõ okirat aláírásakor.\\
b) A bejegyzési kérelem bírósághoz történõ benyújtásakor.\\
\uline {c) A bírósági nyilvántartásba vétel megtörténtekor.}\\
d) A bírósági határozat kézbesítésekor.
\end{tcolorbox}

%51. 
\begin{tcolorbox}[title={51. Kérdés}]
Ki látja el a jogi személy tulajdonosainak érdekvédelmét?
\tcblower
a) A könyvvizsgáló.\\
b) A vezetõ tisztségviselõ.\\
\uline {c) A felügyelõbizottság.}\\
d) Egyik sem.
\end{tcolorbox}

\end{frame}


\begin{frame}

%52. 
\begin{tcolorbox}[title={52. Kérdés}]
Ki látja el a jogi személy törvényes mûködésének felügyeletét?
\tcblower
a) Az ügyészség.\\
b) A felügyelõbizottság.\\
\uline {c) A nyilvántartó bíróság.}\\
d) Egyik sem.
\end{tcolorbox}

%53.
\begin{tcolorbox}[title={53. Kérdés}]
Melyik állítás hamis?
\tcblower
a) Összeolvadásnál a korábbi jogi személyek megszûnnek.\\
\uline {b) Kiválásnál a jogi személy megszûnik.}\\
c) Különválásnál a jogi személy megszûnik.\\
d) Átalakulásnál a korábbi jogi személy megszûnik.
\end{tcolorbox}

%54. 
\begin{tcolorbox}[title={54. Kérdés}]
Melyik nem tartozik a (jogi személy) jogutód nélküli megszûnés esetei közé?
\tcblower
a) A tagok kimondják a megszûnést.\\
b) A létesítõ okiratban meghatározott idõ eltelik.\\
\uline {c) A jogi személy tartozása meghaladja a vagyonát.}\\
d) A jogosult szerv megszünteti.
\end{tcolorbox}

\end{frame}


\begin{frame}

%55. 
\begin{tcolorbox}[title={55. Kérdés}]
Melyik állítás igaz?
\tcblower
a) Alapítvány létrehozásához legalább 10 alapító szükséges.\\
b) Az egyesület megszûnik, ha tagjainak száma tartósan nem éri el a 12 fõt.\\
c) Az egyesület átalakulhat alapítvánnyá.\\
\uline {d) Egyik fenti állítás sem igaz.}
\end{tcolorbox}

%56.
\begin{tcolorbox}[title={56. Kérdés}]
Melyik állítás igaz?
\tcblower
a) Alapítvány létrehozásához legalább 10 alapító szükséges.\\
\uline {b) Az egyesület megszûnik, ha tagjainak száma tartósan nem éri el a 10 fõt.}\\
c) Az alapítvány átalakulhat egyesületté.\\
d) Egyik fenti állítás sem igaz.
\end{tcolorbox}

%57.
\begin{tcolorbox}[title={57. Kérdés}]
Egyesületnél mely esetben kötelezõ a felügyelõbizottság létrehozása?
\tcblower
a) Ha a tagok több, mint negyede nem természetes személy.\\
\uline {b) Ha a tagok létszáma az 50 fõt meghaladja.}\\
c) Ha a könyvvizsgáló ezt kéri.\\
d) Egyik fenti esetben sem kötelezõ.
\end{tcolorbox}

%58.
\begin{tcolorbox}[title={58. Kérdés}]
Egyesületnél mely esetben kötelezõ a felügyelõbizottság létrehozása?
\tcblower
\uline {a) Ha a tagok több, mint fele nem természetes személy.}\\
b) Ha a tagok létszáma az 50 fõt meghaladja.\\
c) Ha a könyvvizsgáló ezt kéri.\\
d) Egyik fenti esetben sem kötelezõ.
\end{tcolorbox}

\end{frame}


\begin{frame}

%59. 
\begin{tcolorbox}[title={59. Kérdés}]
Melyik állítás igaz?
\tcblower
a) Alapítvány nem, de egyesület létrehozható gazdasági tevékenység folytatására.\\
b) Egyesület nem, de alapítvány létrehozható gazdasági tevékenység folytatására.\\
\uline {c) Sem egyesület, sem alapítvány nem hozható létre gazdasági tevékenység folytatására.}\\
d) Egyesület és alapítvány is létrehozható gazdasági tevékenység folytatására.
\end{tcolorbox}

\end{frame}

\begin{frame}[plain]
\begin{tcolorbox}[center, colback={myyellow}, coltext={black}, colframe={myyellow}]
    {\RHuge 04 Tarsasagi jog}\\
\end{tcolorbox}
\end{frame}

\begin{frame}

%60. 
\begin{tcolorbox}[title={60. Kérdés}]
Melyik jellemzõ a kontinentális (német) társasági modellre?
\tcblower
a) Erõsen megosztott a tulajdonosi szerkezet.\\
b) Csekély a társasági formaválaszték.\\
\uline {c) A társasági szervek élesen elkülönülnek egymástól.}\\
d) Egyik sem jellemzõ.
\end{tcolorbox}

%61. 
\begin{tcolorbox}[title={61. Kérdés}]
Melyik nem jellemzõ az angol-amerikai társasági modellre?
\tcblower
a) A társasági jog szorosan összefonódik a tõkepiac intézményeivel.\\
b) A társaságnak egységes vezetõ szerve van.\\
\uline {c) A könyvvizsgáló szerepe és felelõssége csekély.}\\
d) Mindegyik fenti állítás jellemzõ rá.
\end{tcolorbox}

%62.
\begin{tcolorbox}[title={62. Kérdés}]
Az alábbiak közül melyik szolgálja a közérdeket a társasági jogban?
\tcblower
a) A diszpozitív szabályozás.\\
\uline {b) A cégnyilvánosság elve.}\\
c) A választott bírósági eljárás igénybevételének lehetõsége.\\
d) Egyik sem.
\end{tcolorbox}

\end{frame}


\begin{frame}

%63. 
\begin{tcolorbox}[title={63. Kérdés}]
Melyik társasági forma nem hozható létre egyszemélyes gazdasági társaságként?
\tcblower
a) Egyszemélyes korlátolt felelõsségû társaság.\\
\uline {b) Egyszemélyes betéti társaság.}\\
c) Egyszemélyes részvénytársaság.\\
d) Mindhárom forma létrehozható.
\end{tcolorbox}

%64.
\begin{tcolorbox}[title={64. Kérdés}]
Melyik társasági forma hozható létre egyszemélyes gazdasági társaságként?
\tcblower
a) Egyszemélyes betéti társaság.\\
\uline {b) Egyszemélyes korlátolt felelõsségû társaság.}\\
c) Egyszemélyes közkereseti társaság.\\
d) Egyik sem hozható létre.
\end{tcolorbox}

%65. 
\begin{tcolorbox}[title={65. Kérdés}]
Hány tag alapíthat közkereseti társaságot?
\tcblower
a) Legalább egy beltag.\\
b) Legalább egy beltag és egy kültag.\\
\uline {c) Legalább két tag.}\\
d) Legalább két beltag.
\end{tcolorbox}

%66.
\begin{tcolorbox}[title={66. Kérdés}]
Milyen célt szolgál a társasági formakényszer?
\tcblower
a) A hitelezõk védelmét.\\
b) A kisebbségi jogok védelmét.\\
c) A társulási szabadság érvényre jutását.\\
\uline {d) A közérdeket.}
\end{tcolorbox}
 
\end{frame}


\begin{frame}

%67. 
\begin{tcolorbox}[title={67. Kérdés}]
Hányféle társasági formában alapítható gazdasági társaság? Sorolja is fel õket.
\tcblower
a) 3, a következõk:\\
\uline {b) 4, a következõk: Kkt., Bt., Kft., Zrt.}\\
c) 5, a következõk:\\
d) Egyik fenti válasz sem megfelelõ.
\end{tcolorbox}

%68. 
\begin{tcolorbox}[title={68. Kérdés}]
Melyik gazdasági társaság nem rendelkezik jogi személyiséggel?
\tcblower
a) A korlátolt felelõsségû társaság.\\
b) A betéti társaság.\\
\uline {c) A részvénytársaság.}\\
d) Mindegyik rendelkezik.
\end{tcolorbox}

%69.
\begin{tcolorbox}[title={69. Kérdés}]
Az alábbiak közül melyik nem lehet apport gazdasági társaság alapítása során?
\tcblower
a) Egy személyautó.\\
b) Egy találmány.\\
\uline {c) Szerzõdésen alapuló követelés.}\\
d) A fentiek mindegyike lehet apport.
\end{tcolorbox}

\end{frame}


\begin{frame}

%70. 
\begin{tcolorbox}[title={70. Kérdés}]
Melyik állítás igaz?
\tcblower
a) Közkereseti társaság nem lehet egy korlátolt felelõsségû társaságnak tagja.\\
\uline {b) Betéti társaság nem lehet egy közkereseti társaságnak tagja.}\\
c) Korlátolt felelõsségû társaság nem lehet egy betéti társaságnak beltagja.\\
d) Részvénytársaságnak csak természetes személyek lehetnek a tagjai.
\end{tcolorbox}

%71. 
\begin{tcolorbox}[title={71. Kérdés}]
Melyiket nem kezdeményezhetik a G.T. szavazati jogok 5%-ával rendelkezõ tagjai?
\tcblower
a) Egyedi könyvvizsgálatot.\\
b) A legfõbb szerv összehívását.\\
\uline {c) Egy tag kizárását.}\\
d) A társaság igényének érvényesítését egy taggal szemben.
\end{tcolorbox}

%72. 
\begin{tcolorbox}[title={72. Kérdés}]
Kik szavazhatnak a gazdasági társaság legfõbb szervének ülésén?
\tcblower
\uline {a) A társaság tagjai.}\\
b) A társaság tagjai és a vezetõ tisztségviselõ.\\
c) A tagok és a felügyelõ bizottság tagjai.\\
d) A tagok és a cégvezetõk.
\end{tcolorbox}

\end{frame}


\begin{frame}

%73. 
\begin{tcolorbox}[title={73. Kérdés}]
Melyik tartozik a gazdasági társaság felügyelõbizottságának feladatai közé?
\tcblower
a) A vezetõ tisztségviselõ megválasztása és visszahívása.\\
b) A könyvvizsgáló ellenõrzése.\\
\uline {c) Az ügyvezetés ellenõrzése.}\\
d) A beszámoló elkészítése.
\end{tcolorbox}

%74. 
\begin{tcolorbox}[title={74. Kérdés}]
A gazdasági társaság melyik szerve kezdeményezheti egy tag kizárását?
\tcblower
a) A felügyelõ bizottság.\\
\uline {b) A legfõbb szerv.}\\
c) A vezetõ tisztségviselõ.\\
d) Egyik sem, mert tagot nem lehet kizárni.
\end{tcolorbox}

%75.
\begin{tcolorbox}[title={75. Kérdés}]
Melyik szerv dönt a társaság jogutód nélküli megszûnésérõl?
\tcblower
\uline {a) A legfõbb szerv.}\\
b) A felügyelõ bizottság.\\
c) A vezetõ tisztségviselõ.\\
d) A könyvvizsgáló. 
\end{tcolorbox}

%76. 
\begin{tcolorbox}[title={76. Kérdés}]
Melyik állítás hamis a Kft-re vonatkozóan?
\tcblower
a) Kft. esetén a tagok törzsbetétei különbözõ mértékûek lehetnek.\\
b) Az egyes törzsbetétek mértéke legalább 100.000 Ft.\\
\uline {c) A törzstbetétek összege a törzstõke, melynek minimális mértéke 1 millió Ft.}\\
d) Minden tagnak egy törzsbetéte lehet.
\end{tcolorbox}

\end{frame}


\begin{frame}

%77. 
\begin{tcolorbox}[title={77. Kérdés}]
Melyik állítás hamis?
\tcblower
\uline {a) A Bt. alaptõkéje nem lehet kevesebb, mint 100.000 Ft.}\\
b) A Zrt. alaptõkéje nem lehet kevesebb, mint 5 millió Ft.\\
c) Az Nyrt. alaptõkéje nem lehet kevesebb, mint 20 millió Ft.\\
d) A közkereseti társaság alaptõkéjére nincs minimális elõírás.
\end{tcolorbox}

%78. 
\begin{tcolorbox}[title={78. Kérdés}]
Kinek a tulajdona a betéti társaság vagyona?
\tcblower
a) A beltagok közös tulajdona.\\
b) Valamennyi tag közös tulajdona.\\
\uline {c) A társaság tulajdona.}\\
d) Az alapító tagok közös tulajdona.
\end{tcolorbox}

\end{frame}

\begin{frame}[plain]
\begin{tcolorbox}[center, colback={myyellow}, coltext={black}, colframe={myyellow}]
    {\RHuge 05 Egyeb jogalanyok}\\
\end{tcolorbox}
\end{frame}

\begin{frame}


%79.
\begin{tcolorbox}[title={79. Kérdés}]
Melyik állítás hamis?
\tcblower
a) Az egyesülés jogi személyiséggel rendelkezik.\\
\uline {b) Az egyesülés vagyonát meghaladó tartozásaiért a tagok nem felelnek.}\\
c) Az egyesülés saját nyereségre nem törekszik.\\
d) Az egyesülés célja a tagok tevékenységének összehangolása.
\end{tcolorbox}

%80.
\begin{tcolorbox}[title={80. Kérdés}]
Melyik állítás igaz?
\tcblower
\uline {a) A szövetkezet tagjainak többsége természetes személy.}\\
b) A szövetkezet valamennyi tagjának személyes közremûködést kell vállalnia.\\
c) Szövetkezeti jogvitára választott-bírósági eljárás nem köthetõ ki.\\
d) A szövetkezet nem rendelkezik jogi személyiséggel.
\end{tcolorbox}

%81. 
\begin{tcolorbox}[title={81. Kérdés}]
Hány személy alapíthat szövetkezetet?
\tcblower
a) Legalább 2 személy.\\
\uline {b) Legalább 7 személy.}\\
c) Legalább 10 személy.\\
d) Egy személy is alapíthat.
\end{tcolorbox}

%82. 
\begin{tcolorbox}[title={82. Kérdés}]
Melyik állítás hamis?
\tcblower
a) A szövetkezet közgyûlésén minden tagnak egy szavazata van.\\
\uline {b) Szövetkezetnél kötelezõ a könyvvizsgáló kinevezése.}\\
c) Szövetkezetnél kötelezõ a felügyelõbizottság mûködése. \\
d) A szövetkezet ügyvezetését az igazgatóság látja el.
\end{tcolorbox}

\end{frame}


\begin{frame}

%83.
\begin{tcolorbox}[title={83. Kérdés}]
Melyik szervezet nem vehet részt tagként vállalatcsoportban?
\tcblower
a) Korlátolt felelõsségû társaság.\\
b) Szövetkezet.\\
c) Egyesülés.\\
\uline {d) Betéti társaság.}
\end{tcolorbox}

%84. 
\begin{tcolorbox}[title={84. Kérdés}]
Melyik állítás hamis a vállalatcsoportra vonatkozóan?
\tcblower
a) Legalább egy uralkodó tag és 3 ellenõrzött tag szükséges hozzá.\\
b) A cégbíróság a cégnyilvántartásba bejegyzi.\\
\uline {c) Az uralkodó tag nem utasíthatja az ellenõrzött tagokat.}\\
d) Önálló jogi személyiséggel nem rendelkezik.
\end{tcolorbox}

%85. 
\begin{tcolorbox}[title={85. Kérdés}]
Az alábbiak közül ki nem lehet egyéni vállalkozó?
\tcblower
a) 17 éves, házasságot kötött személy.\\
\uline {b) Közkereseti társaság tagja.}\\
c) Korlátolt felelõsségû társaság tagja.\\
d) Betéti társaság kültagja.
\end{tcolorbox}

%86. 
\begin{tcolorbox}[title={86. Kérdés}]
Melyik szervezet vezeti az egyéni vállalkozók nyilvántartását?
\tcblower
\uline {a) A Belügyminisztérium.}\\
b) A cégbíróság.\\
c) A törvényszék.\\
d) A céginformációs szolgálat.
\end{tcolorbox}

\end{frame}


\begin{frame}

%87. 
\begin{tcolorbox}[title={87. Kérdés}]
Melyik állítás igaz?
\tcblower
a) Az egyéni vállalkozónak nem lehet alkalmazottja.\\
b) A tevékenység csak egyéni vállalkozói igazolvány kiállítása után végezhetõ.\\
c) Az egyéni vállalkozó a kötelezettségeiért csak vállalkozói vagyonával felel. \\
\uline {d) Az egyéni vállalkozó köteles személyesen közremûködni a tevékenység folytatásában.}
\end{tcolorbox}

%88. 
\begin{tcolorbox}[title={88. Kérdés}]
Melyik állítás hamis az egyéni cégre vonatkozóan?
\tcblower
a) Az egyéni cég jogképes.\\
\uline {b) Az egyéni cég jogi személyiséggel rendelkezik.}\\
c) Az egyéni céget a cégbíróság tartja nyilván.\\
d) Az egyéni cégnek cégjegyzékszáma van.
\end{tcolorbox}

%89.
\begin{tcolorbox}[title={89. Kérdés}]
Ki alapíthat egyéni céget?
\tcblower
a) Minden nagykorú, cselekvõképes személy.\\
b) Minden nagykorú, cselekvõképes magyar állampolgár.\\
\uline {c) Aki az egyéni vállalkozói nyilvántartásban szerepel.}\\
d) Minden nagykorú, cselekvõképes személy, aki nem korlátlanul felelõs tagja G.T.-nak.
\end{tcolorbox}

%90.
\begin{tcolorbox}[title={90. Kérdés}]
Az alábbiak közül melyik nem lehet egy egyéni cég alapításkori jegyzett tõkéje?
\tcblower
a) 100.000 Ft plusz egy 200.000 Ft értékû autó.\\
\uline {b) 100.000 Ft plusz egy 100.000 Ft értékû autó.}\\
c) Egy 300.000 Ft értékû autó. \\
d) 100.000 Ft.
\end{tcolorbox}

\end{frame}


\begin{frame}

%91.
\begin{tcolorbox}[title={91. Kérdés}]
Melyik állítás hamis?
\tcblower
a) A köztestület közfeladatot lát el.\\
\uline {b) A köztestületet a tagjai szabad elhatározásukkal hozzák létre.}\\
c) A köztestület jogi személy.\\
d) A Magyar Orvosi Kamara is egy köztestület.
\end{tcolorbox}

%92. 
\begin{tcolorbox}[title={92. Kérdés}]
Az alábbiak közül melyik nem köztestület?
\tcblower
a) Magyar Tudományos Akadémia\\
b) Magyar Ügyvédi Kamara\\
\uline {c) Magyar Labdarúgó Szövetség}\\
d) Magyar Kereskedelmi és Iparkamara
\end{tcolorbox}

%93. 
\begin{tcolorbox}[title={93. Kérdés}]
Az alábbiak közül melyik nem lehet közhasznú státuszú?
\tcblower
\uline {a) A szövetkezet.}\\
b) Az egyesület.\\
c) Az alapítvány.\\
d) A nonprofit Kft.
\end{tcolorbox}

\end{frame}

\begin{frame}[plain]
\begin{tcolorbox}[center, colback={myyellow}, coltext={black}, colframe={myyellow}]
    {\RHuge 06 Cegeljaras}\\
\end{tcolorbox}
\end{frame}

\begin{frame}

%94. 
\begin{tcolorbox}[title={94. Kérdés}]
Az alábbi jogalanyok közül melyiknek nincs cégjegyzékszáma?
\tcblower
a) Az egyesülésnek.\\
b) A szövetkezetnek.\\
\uline {c) Az egyesületnek.}\\
d) Az egyéni cégnek.
\end{tcolorbox}

%95. 
\begin{tcolorbox}[title={95. Kérdés}]
Az alábbiak közül melyik nem a cégbíróság feladata?
\tcblower
a) Cégbejegyzési eljárás lefolytatása.\\
\uline {b) Felszámolási eljárás lefolytatása.}\\
c) Döntés a közhasznúvá minõsítésrõl.\\
d) Törvényességi felügyeleti eljárás lefolytatása cégekkel kapcsolatban.
\end{tcolorbox}

%96. 
\begin{tcolorbox}[title={96. Kérdés}]
Az alábbi esetek közül melyik fordulhat elõ egy céggel kapcsolatban?
\tcblower
a) Székhelye Vácon van, telephelye nincs, fióktelepe Vácon van.\\
b) Székhelye nincs, telephelye Vácon van, fióktelepe Vácon van.\\
\uline {c) Székhelye Vácon van, telephelye Vácon van, fióktelepe nincs.}\\
d) Székhelye Vácon van, telephelye Vácon van, fióktelepe Vácon van.
\end{tcolorbox}

%97. 
\begin{tcolorbox}[title={97. Kérdés}]
Melyik állítás hamis?
\tcblower
a) A cégbejegyzési kérelmet csak elektronikus úton lehet benyújtani.\\
b) A cégbejegyzési eljárás nem-peres eljárás. \\
c) A kérelmet a létesítõ okirat aláírásától számított 30 napon belül kell benyújtani.\\
\uline {d) A cégbejegyzési eljárás során a jogi képviselet nem kötelezõ.}
\end{tcolorbox}

\end{frame}


\begin{frame}

%98.
\begin{tcolorbox}[title={98. Kérdés}]
Ki dönt egy cég esetén a végelszámolásról?
\tcblower
a) A cégbíróság.\\
\uline {b) A cég legfõbb szerve.}\\
c) A vezetõ tisztségviselõ.\\
d) A hitelezõ(k).
\end{tcolorbox}

%99. 
\begin{tcolorbox}[title={99. Kérdés}]
Mikor van helye egyszerûsített cégbejegyzési eljárásnak?
\tcblower
a) Egyéni cég és Bt. alapításakor.\\
b) Ha a cég alaptõkéje nem haladja meg a 100.000 Ft-ot.\\
\uline {c) Ha a létesítõ okirat szerzõdésminta alapján készült.}\\
d) Ha a tagok száma egy fõ.
\end{tcolorbox}

%100. 
\begin{tcolorbox}[title={100. Kérdés}]
Mely cégeknek kell bankszámlával rendelkeznie?
\tcblower
a) Csak a Zrt.-nek és az Nyrt.-nek.\\
b) A gazdasági társaságoknak.\\
c) A Zrt.-nek, Nyrt.-nek és a Kft.-nek.\\
\uline {d) Minden cégnek bankszámlával kell rendelkeznie.}
\end{tcolorbox}

%101. 
\begin{tcolorbox}[title={101. Kérdés}]
Mennyi egy Zrt. egyszerûsített cégbejegyzési eljárásának illetéke?
\tcblower
a) 10.000 Ft\\
\uline {b) 50.000 Ft}\\
c) 100.000 Ft\\
d) Az eljárás illetékmentes.
\end{tcolorbox}

\end{frame}


\begin{frame}

%102.
\begin{tcolorbox}[title={102. Kérdés}]
Mennyi egy Kft. rendes (nem egyszerûsített) cégbejegyzési eljárásának illetéke?
\tcblower
a) 10.000 Ft\\
b) 50.000 Ft\\
c) 100.000 Ft\\
\uline {d) Az eljárás illetékmentes.}
\end{tcolorbox}

%103.
\begin{tcolorbox}[title={103. Kérdés}]
Ki képviseli a céget a végelszámolás kezdõ idõpontja után?
\tcblower
a) A vezetõ tisztségviselõ.\\
b) A vagyonfelügyelõ.\\
c) A felügyelõbizottság elnöke.\\
\uline {d) A végelszámoló.}
\end{tcolorbox}

%104. 
\begin{tcolorbox}[title={104. Kérdés}]
Melyik eljárásnak nem célja a cég jogutód nélküli megszüntetése?
\tcblower
a) A felszámolási eljárásnak.\\
\uline {b) A csõdeljárásnak.}\\
c) A végelszámolásnak.\\
d) Mindegyiknek az a célja.
\end{tcolorbox}

%105. 
\begin{tcolorbox}[title={105. Kérdés}]
Mi a csõdeljárás célja?
\tcblower
a) A fizetésképtelen adós cég jogutód nélküli megszüntetése.\\
b) A nem fizetésképtelen cég jogutód nélküli megszüntetése.\\
\uline {c) Az adós cég fizetési haladékot kap a hitelezõkkel való megegyezés céljából.}\\
d) Az adós cég vagyonának hatósági árverésen való értékesítése.
\end{tcolorbox}

\end{frame}


\begin{frame}

%106. 
\begin{tcolorbox}[title={106. Kérdés}]
Mi a végelszámolás célja?
\tcblower
a) A fizetésképtelen adós cég jogutód nélküli megszüntetése.\\
\uline {b) A nem fizetésképtelen cég jogutód nélküli megszüntetése.}\\
c) Az adós cég fizetési haladékot kap a hitelezõkkel való megegyezés céljából.\\
d) Az adós cég vagyonának hatósági árverésen való értékesítése.
\end{tcolorbox}

%107. 
\begin{tcolorbox}[title={107. Kérdés}]
Mi a felszámolási eljárás célja?
\tcblower
\uline {a) A fizetésképtelen adós cég jogutód nélküli megszüntetése.}\\
b) A nem fizetésképtelen cég jogutód nélküli megszüntetése.\\
c) Az adós cég fizetési haladékot kap a hitelezõkkel való megegyezés céljából.\\
d) Az adós cég vagyonának hatósági árverésen való értékesítése.
\end{tcolorbox}

%108.
\begin{tcolorbox}[title={108. Kérdés}]
Melyik eljárás indul meg a cégbírósághoz benyújtott változásbejegyzési kérelemmel?
\tcblower
a) Csõdeljárás.\\
\uline {b) Végelszámolás.}\\
c) Felszámolási eljárás.\\
d) A fentiek közül egyik sem.
\end{tcolorbox}

%109. 
\begin{tcolorbox}[title={109. Kérdés}]
Melyik eljárást indíthatja meg a hitelezõ az adós cég ellen?
\tcblower
a) A végelszámolást.\\
b) A csõdeljárást.\\
\uline {c) A felszámolási eljárást.}\\
d) Kényszer-törlési eljárást.
\end{tcolorbox}

\end{frame}


\begin{frame}

%110. 
\begin{tcolorbox}[title={110. Kérdés}]
Melyik eljárást milyen fórumon kell megindítani? Válassza ki a helyeset.
\tcblower
a) Csõdeljárást a cégbíróságon.\\
\uline {b) Felszámolási eljárást a rendes bíróságon (törvényszéken).}\\
c) Végelszámolást a rendes bíróságon (törvényszéken).\\
d) Cégbejegyzési eljárást a rendes bíróságon (törvényszéken).
\end{tcolorbox}

\end{frame}

\begin{frame}[plain]
\begin{tcolorbox}[center, colback={myyellow}, coltext={black}, colframe={myyellow}]
    {\RHuge 07 Dologi jog}\\
\end{tcolorbox}
\end{frame}

\begin{frame}


%111. 
\begin{tcolorbox}[title={111. Kérdés}]
Kivel szemben nem illeti meg a birtokvédelem joga a birtokost?
\tcblower
a) A tulajdonossal szemben.\\
b) A hitelezõvel szemben.\\
\uline {c) Azzal szemben, akitõl a dolgot tilos önhatalommal szerezte meg.}\\
d) Kivétel nélkül mindenkivel szemben megilleti.
\end{tcolorbox}

%112.
\begin{tcolorbox}[title={112. Kérdés}]
Melyek a birtokvédelem megengedett eszközei?
\tcblower
a) Minden esetben a bírósághoz kell fordulni.\\
b) Minden esetben a jegyzõhöz kell fordulni.\\
c) Minden esetben a rendõrséghez kell fordulni.\\
\uline {d) Bírósághoz vagy jegyzõhöz fordulás, speciális esetben önhatalommal való fellépés.}
\end{tcolorbox}

%113. 
\begin{tcolorbox}[title={113. Kérdés}]
Az alábbiak közül melyik nem tartozik a "szükséges költségek" körébe?
\tcblower
a) Az állagmegóvásra fordított költségek.\\
b) A kártól való megóvásra fordított költségek.\\
\uline {c) A dolog értékét növelõ költségek.}\\
d) Mindhárom fenti a "szükséges" költségek körébe tartozik.
\end{tcolorbox}

%114. 
\begin{tcolorbox}[title={114. Kérdés}]
Milyen költségei megtérítését követelheti a rosszhiszemû jogalap nélküli birtokos?
\tcblower
\uline {a) A dologra fordított szükséges költségei megtérítését.}\\
b) A dologra fordított hasznos költségei megtérítését.\\
c) A szükséges és hasznos költségei megtérítését is követelheti.\\
d) Semmilyen költségei megtérítését nem követelheti.
\end{tcolorbox}

\end{frame}


\begin{frame}

%115. 
\begin{tcolorbox}[title={115. Kérdés}]
Az alábbiak közül melyik nem lehet tulajdonjog tárgya?
\tcblower
a) Az elektromos energia (villamos áram).\\
\uline {b) A szellemi alkotás.}\\
c) Az állatok.\\
d) Az értékpapír.
\end{tcolorbox}

%116. 
\begin{tcolorbox}[title={116. Kérdés}]
Az alábbi 3 állítás (a,b,c) közül melyik hamis?
\tcblower
a) Az ingatlan tulajdonosa a használat jogát másnak átengedheti.\\
b) Az ingatlan tulajdonosa a tulajdonjogát másra átruházhatja.\\
\uline {c) Az ingatlan tulajdonosa a tulajdonjogával felhagyhat.}\\
d) Mindhárom elõzõ állítás igaz.
\end{tcolorbox}

%117. 
\begin{tcolorbox}[title={117. Kérdés}]
Melyik állítás igaz?
\tcblower
a) Szükséghelyzetben okozott kár esetén a (károsult) tulajdonos kártérítést követelhet.\\
b) Közérdekû használat esetén a tulajdonost kártérítés illeti meg.\\
c) Kisajátítás esetén a tulajdonost kártérítés illeti meg.\\
\uline {d) Mindhárom fenti esetben kártalanítás és nem kártérítés illeti meg a tulajdonost.}
\end{tcolorbox}

\end{frame}


\begin{frame}

%118.
\begin{tcolorbox}[title={118. Kérdés}]
Melyik állítás igaz?
\tcblower
\uline {a) Jogellenes károkozás esetén a (károsult) tulajdonos kártérítést követelhet.}\\
b) Közérdekû használat esetén a tulajdonost kártérítés illeti meg.\\
c) Kisajátítás esetén a tulajdonost kártérítés illeti meg.\\
d) Mindhárom fenti esetben kártalanítás és nem kártérítés illeti meg a tulajdonost.
\end{tcolorbox}

%119. 
\begin{tcolorbox}[title={119. Kérdés}]
Az alábbiak közül melyik tulajdonszerzési mód származékos?
\tcblower
\uline {a) Tulajdon átruházás.}\\
b) Hatósági árverés útján való tulajdonszerzés.\\
c) Elbirtoklás.\\
d) Kisajátítás.
\end{tcolorbox}

%120.
\begin{tcolorbox}[title={120. Kérdés}]
Mikor száll át az ingatlan tulajdonjoga átruházás esetén?
\tcblower
a) A szerzõdés megkötésekor.\\
b) A felek által a szerzõdésben meghatározott idõpontban.\\
c) A birtokbaadáskor.\\
\uline {d) Az új tulajdonosnak az ingatlan-nyilvántartásba való bejegyzésével.}
\end{tcolorbox}

\end{frame}


\begin{frame}

%121. 
\begin{tcolorbox}[title={121. Kérdés}]
Melyik állítás hamis?
\tcblower
a) Kisajátítás csak közérdekû célra lehetséges.\\
b) Kisajátítani csak ingatlant lehet.\\
\uline {c) A kisajátítást csak az állam kérheti.}\\
d) A kisajátítási eljárás során a kisajátítási hatóság tárgyalást tart.
\end{tcolorbox}
 
%122.
\begin{tcolorbox}[title={122. Kérdés}]
Melyik állítás igaz?
\tcblower
a) Az ingatlan elbirtoklási ideje 10 év.\\
\uline {b) Bûncselekménnyel megszerzett dolgon nem következik be az elbirtoklás.}\\
c) Ingatlan elbirtoklásához a tulajdonjogot az ingatlan-nyilvántartásba be kell jegyezni.\\
d) Elbirtoklással csak ingó dolog tulajdonjogát lehet megszerezni.
\end{tcolorbox}

%123. 
\begin{tcolorbox}[title={123. Kérdés}]
Melyik állítás hamis?
\tcblower
a) Villamoson felejtett dolgon nem lehet találással tulajdont szerezni.\\
b) Gazdátlan dolgon nem lehet találással tulajdont szerezni.\\
c) A találó csak akkor válik tulajdonossá, ha a korábbi tulajdonos 1 éven belül nem jelentkezik.\\
\uline {d) A találás származékos tulajdonszerzési mód.}
\end{tcolorbox}

%124. 
\begin{tcolorbox}[title={124. Kérdés}]
Melyik állítás hamis a zálogjogviszony esetén?
\tcblower
\uline {a) A jogviszony alanya a személyes kötelezett és a zálogjogosult.}\\
b) A zálogtárgy tulajdonosa a zálogkötelezett.\\
c) A zálogjogosult kielégítési joga akkor nyílik meg, ha a személyes kötelezett nem teljesít. \\
d) A zálogkötelezett csak a zálogtárggyal felel a zálogjogosultnak. 
\end{tcolorbox}

\end{frame}


\begin{frame}

%125. 
\begin{tcolorbox}[title={125. Kérdés}]
Melyik állítás hamis a zálogjogviszony esetén?
\tcblower
a) A jogviszony alanya a zálogkötelezett és a zálogjogosult.\\
\uline {b) A zálogtárgy tulajdonosa a zálogjogosult.}\\
c) A zálogjogosult kielégítési joga akkor nyílik meg, ha a személyes kötelezett nem teljesít. \\
d) A zálogkötelezett csak a zálogtárggyal felel a zálogjogosultnak. 
\end{tcolorbox}

%126.
\begin{tcolorbox}[title={126. Kérdés}]
Melyik állítás hamis a zálogjogviszony esetén?
\tcblower
a) A jogviszony alanya a zálogkötelezett és a zálogjogosult.\\
b) A zálogtárgy tulajdonosa a zálogkötelezett.\\
c) A zálogjogosult kielégítési joga akkor nyílik meg, ha a személyes kötelezett nem teljesít. \\
\uline {d) A zálogkötelezett teljes vagyonával felel a zálogjogosultnak. }
\end{tcolorbox}

%127. 
\begin{tcolorbox}[title={127. Kérdés}]
Melyik állítás hamis a zálogjogviszony esetén?
\tcblower
a) Az óvadék is a zálog egy fajtája.\\
\uline {b) Jelzálogjog csak ingatlanon alapítható.}\\
c) Kézizálogjog tárgya csak ingó dolog lehet.\\
d) A személyes kötelezett és a zálogkötelezett lehet ugyanaz a személy.
\end{tcolorbox}

\end{frame}


\begin{frame}

%128.
\begin{tcolorbox}[title={128. Kérdés}]
Melyiket nem teheti meg a haszonélvezeti jog jogosultja?
\tcblower
a) A dolgot nem hasznosíthatja.\\
\uline {b) A haszonélvezeti jogot nem ruházhatja át.}\\
c) A dolgot nem tarthatja a birtokában.\\
d) A fenti 3 mindegyikét megteheti.
\end{tcolorbox}

%129.
\begin{tcolorbox}[title={129. Kérdés}]
Az alábbi esetek közül melyikben nem használhatja a birtokos a dolgot?
\tcblower
a) Haszonkölcsön esetén.\\
\uline {b) Kézizálog esetén.}\\
c) Haszonélvezeti jog esetén.\\
d) A fenti esetek mindegyikében használhatja.
\end{tcolorbox}

\end{frame}

\begin{frame}[plain]
\begin{tcolorbox}[center, colback={myyellow}, coltext={black}, colframe={myyellow}]
    {\RHuge 08 Kotelmi jog}\\
\end{tcolorbox}
\end{frame}

\begin{frame}

%130.
\begin{tcolorbox}[title={130. Kérdés}]
Az alábbiak közül melyik magatartásra nem irányulhat a kötelem?
\tcblower
a) Valaminek az adására.\\
b) Valamilyen tevékenység elvégzésére.\\
c) Valamitõl való tartózkodásra.\\
\uline {d) Mindegyik fenti magatartásra irányulhat.}
\end{tcolorbox}

%131. 
\begin{tcolorbox}[title={131. Kérdés}]
Az alábbiak közül melyikbõl nem keletkezhet kötelem?
\tcblower
a) Jogalap nélküli gazdagodásból.\\
b) Személyiségi jog megsértésébõl.\\
c) Utaló magatartásból.\\
\uline {d) A fentiek mindegyikébõl keletkezhet.}
\end{tcolorbox}

%132. 
\begin{tcolorbox}[title={132. Kérdés}]
Az alábbiak közül melyik károkozás jogellenes?
\tcblower
a) A károsult beleegyezésével okozott kár.\\
b) Jogtalan támadás elhárítása során a támadónak okozott kár.\\
c) Szükséghelyzetben okozott kár.\\
\uline {d) A fentiek egyike sem jogellenes.}
\end{tcolorbox}

%133. 
\begin{tcolorbox}[title={133. Kérdés}]
Az alábbiak közül melyik esetben keletkezik kötelem (kötelezettség)?
\tcblower
a) Jogtalan támadás elhárítása során a támadónak okozott kár esetén.\\
b) A károsult beleegyezésével okozott kár esetén.\\
\uline {c) Szükséghelyzetben okozott kár esetén.}\\
d) Egyik esetben sem keletkezik.
\end{tcolorbox}

\end{frame}


\begin{frame}

%134. 
\begin{tcolorbox}[title={134. Kérdés}]
Melyik állítás igaz?
\tcblower
a) Kártérítési kötelezettséghez nem szükséges a kár bekövetkezése.\\
\uline {b) A kár bekövetkezését a károsultnak kell bizonyítania.}\\
c) Akit személyiségi jogában megsértenek, kártérítést követelhet.\\
d) Akit személyiségi jogában megsértenek, kártalanítást követelhet. 
\end{tcolorbox}

%135. 
\begin{tcolorbox}[title={135. Kérdés}]
Melyik állítás igaz?
\tcblower
\uline {a) A sérelemdíj megállapításához nem szükséges kár bekövetkezése.}\\
b) A becsületsértésért büntetõeljárásban kérhetõ sérelemdíj.\\
c) A becsületsértésért megítélt pénzbüntetés a sértettet illeti.\\
d) A személyiségi jog megsértése bûncselekmény.
\end{tcolorbox}

%136. 
\begin{tcolorbox}[title={136. Kérdés}]
Melyik állítás igaz?
\tcblower
a) Az ajándékozás egyoldalú nyilatkozat.\\
b) Ha valaki nyilvánosan díjat ígér egy teljesítményért, nem köteles azt teljesíteni.\\
c) Ha valaki közérdekû célra vállal kötelezettséget, az egy szerzõdés az állammal.\\
\uline {d) A szerzõdés soha nem lehet egyoldalú nyilatkozat.}
\end{tcolorbox}

\end{frame}


\begin{frame}

%137.
\begin{tcolorbox}[title={137. Kérdés}]
Melyik állítás igaz?
\tcblower
a) Ha valaki tévedésbõl többet utal át a vételárnál, nem követelheti azt vissza.\\
b) Ha valaki tévedésbõl többet utal át a vételárnál, az ajándékozásnak minõsül.\\
\uline {c) Ha valaki tévedésbõl többet utal át, a többletet a másik köteles visszatéríteni.}\\
d) Ha valaki tévedésbõl többet utal át, a többlet felét visszakövetelheti.
\end{tcolorbox}

%138. 
\begin{tcolorbox}[title={138. Kérdés}]
Melyik állítás igaz?
\tcblower
a) Értékpapírt csak írásban, okirati formában lehet kibocsátani.\\
b) Az értékpapír egy szerzõdés a kibocsátó és a jogosult között.\\
c) Az értékpapírban foglalt követelés nem átruházható.\\
\uline {d) A befektetési jegy az értékpapír egyik fajtája.}
\end{tcolorbox}

%139. 
\begin{tcolorbox}[title={139. Kérdés}]
Melyik állítás hamis?
\tcblower
a) A váltó hitelviszonyt megtestesítõ értékpapír.\\
\uline {b) A kötvény tagsági jogokat megtestesítõ értékpapír.}\\
c) A részvény tagsági jogokat megtestesítõ értékpapír.\\
d) A közraktári jegy dologi jogot megtestesítõ értékpapír.
\end{tcolorbox}

%140. 
\begin{tcolorbox}[title={140. Kérdés}]
Az alábbi esetek közül melyikben nem szûnik meg a kötelem (fõ szabály szerint)?
\tcblower
a) A szolgáltatás teljesítésével.\\
b) A kötelezett halálával.\\
\uline {c) Tartozásátvállalás esetén.}\\
d) A jogosult jogutód nélküli megszûnése esetén.
\end{tcolorbox}

\end{frame}


\begin{frame}

%141. 
\begin{tcolorbox}[title={141. Kérdés}]
Hányféle formában lehet jognyilatkozatot tenni? Sorolja is fel õket.
\tcblower
a) 2, amelyek a következõk:\\
\uline {b) 3, amelyek a következõk: szóban, írásban, ráutaló magatartással}\\
c) 4, amelyek a következõk:\\
d) 5, amelyek a következõk:
\end{tcolorbox}

%142. 
\begin{tcolorbox}[title={142. Kérdés}]
Mikor minõsül a jognyilatkozat írásba foglaltnak?
\tcblower
a) Ha a nyilatkozó saját kezûleg írta.\\
b) Ha két tanú aláírta.\\
\uline {c) Ha a nyilatkozó aláírta.}\\
d) Ha ügyvéd aláírta.
\end{tcolorbox}

%143. 
\begin{tcolorbox}[title={143. Kérdés}]
Az alábbiak közül ki nem készíthet közokiratot?
\tcblower
\uline {a) Ügyvéd.}\\
b) Bíróság.\\
c) Közjegyzõ.\\
d) Közigazgatási szerv.
\end{tcolorbox}

%144.
\begin{tcolorbox}[title={144. Kérdés}]
Melyik állítás hamis?
\tcblower
\uline {a) A képviseleti jog alapulhat vállalkozási szerzõdésen.}\\
b) A képviseleti jog alapulhat létesítõ okiraton.\\
c) A képviseleti jog alapulhat meghatalmazáson.\\
d) A képviseleti jog alapulhat jogszabályon. 
\end{tcolorbox}

\end{frame}


\begin{frame}

%145. 
\begin{tcolorbox}[title={145. Kérdés}]
Melyik állítás igaz?
\tcblower
a) Jognyilatkozatot csak személyesen lehet tenni.\\
b) A képviselõ jognyilatkozata a képviselõt jogosítja és kötelezi.\\
\uline {c) A meghatalmazás egyoldalú jognyilatkozat.}\\
d) A meghatalmazás szóban nem érvényes.
\end{tcolorbox}

%146. 
\begin{tcolorbox}[title={146. Kérdés}]
Hány év alatt évülnek el a követelések (fõ szabály szerint)?
\tcblower
a) 3 év alatt.\\
\uline {b) 5 év alatt. }\\
c) 1 év alatt.\\
d) A követelések nem évülnek el.
\end{tcolorbox}

%147.
\begin{tcolorbox}[title={147. Kérdés}]
Melyik állítás igaz?
\tcblower
a) Elévült követelés bírósági eljárásban csak jogi képviselettel érvényesíthetõ.\\
b) Ha valaki elévült követelést teljesít, utóbb visszakövetelheti azt.\\
c) A fogadásból eredõ követelés bíróság elõtt 1 évig érvényesíthetõ.\\
\uline {d) Mindhárom fenti állítás hamis.}
\end{tcolorbox}

\end{frame}

\begin{frame}[plain]
\begin{tcolorbox}[center, colback={myyellow}, coltext={black}, colframe={myyellow}]
    {\RHuge 09 Szerzodesek}\\
\end{tcolorbox}
\end{frame}

\begin{frame}

%148. 
\begin{tcolorbox}[title={148. Kérdés}]
Melyik állítás a legpontosabb a szerzõdésre vonatkozóan?
\tcblower
a) A szerzõdés a felek egyidejû és egybehangzó jognyilatkozata.\\
b) A szerzõdés a felek kölcsönös és egyidejû jognyilatkozata.\\
\uline {c) A szerzõdés a felek kölcsönös és egybehangzó jognyilatkozata.}\\
d) A szerzõdés a felek kölcsönös, egybehangzó és egyidejû jognyilatkozata.
\end{tcolorbox}

%149.
\begin{tcolorbox}[title={149. Kérdés}]
Az alábbiak közül melyik nem tartozik a szerzõdési jog legfontosabb alapelvei közé?
\tcblower
a) A szerzõdési szabadság elve.\\
b) A visszterhesség vélelme.\\
c) Az együttmûködési és tájékoztatási kötelezettség.\\
\uline {d) A gyengébb fél érdekei védelmének az elve.}
\end{tcolorbox}

%150. 
\begin{tcolorbox}[title={150. Kérdés}]
Mi történik ha a felek szerzõdési nyilatkozatai lényeges kérdésben eltérnek?
\tcblower
a) A szerzõdés megtámadható lesz.\\
b) A szerzõdés érvénytelen lesz.\\
\uline {c) A szerzõdés nem jön létre.}\\
d) A bíróság ítéletével kiküszöböli az eltérést.
\end{tcolorbox}

%151.
\begin{tcolorbox}[title={151. Kérdés}]
Melyik állítás hamis az ajánlatra vonatkozóan?
\tcblower
a) Aki ajánlatot tesz, meghatározott ideig kötve van az ajánlathoz.\\
\uline {b) Az ajánlati kötöttség idejét a felek közösen határozzák meg.}\\
c) Ha a másik fél határidõben elfogadja az ajánlatot, a szerzõdés létrejön.\\
d) Ha a másik fél eltérõ tartalommal fogadja el, az új ajánlatnak minõsül.
\end{tcolorbox}

\end{frame}


\begin{frame}

%152. 
\begin{tcolorbox}[title={152. Kérdés}]
Az alábbiak közül melyikbõl nem származhat szerzõdéskötési kötelezettség?
\tcblower
a) Jogszabályból.\\
b) Versenyeztetési eljárásból.\\
c) Elõszerzõdésbõl.\\
\uline {d) Mindháromból fentibõl származhat.}
\end{tcolorbox}

%153.
\begin{tcolorbox}[title={153. Kérdés}]
Melyik állítás igaz?
\tcblower
a) Az Általános Szerzõdési Feltételeket a felek elõre, közösen határozzák meg.\\
\uline {b) A fogyasztó csak természetes személy lehet.}\\
c) Távollévõk között kötött szerzõdés létrejöhet két vállalkozás között is.\\
d) Fogyasztói szerzõdés esetén a fogyasztót 14 napon belül elállási jog illeti meg.
\end{tcolorbox}

%154. 
\begin{tcolorbox}[title={154. Kérdés}]
Az érvénytelenségnek hányféle formája van a polgári jogban, és melyek ezek?
\tcblower
a) 3, éspedig a következõk:\\
\uline {b) 2, éspedig a következõk: semmisség, megtámadhatóság}\\
c) 1, éspedig a következõ:\\
d) A polgári jogban nem létezik az érvénytelenség fogalma.
\end{tcolorbox}

%155. 
\begin{tcolorbox}[title={155. Kérdés}]
Az alábbiak közül melyik a célzott joghatás hibája?
\tcblower
a) A tévedés.\\
b) A megtévesztés.\\
\uline {c) A jó erkölcsbe ütközés.}\\
d) A jogellenes fenyegetés.
\end{tcolorbox}

\end{frame}


\begin{frame}

%156.
\begin{tcolorbox}[title={156. Kérdés}]
Az alábbi esetek közül melyikben merül fel a szerzõdési akarat hibája?
\tcblower
\uline {a) A megtévesztéssel kötött szerzõdés.}\\
b) Az uzsorás szerzõdés.\\
c) A jogszabályba ütközõ szerzõdés.\\
d) A nem megfelelõ formában megkötött szerzõdés.
\end{tcolorbox}

%157. 
\begin{tcolorbox}[title={157. Kérdés}]
Melyik állítás igaz?
\tcblower
a) A jogszabályba ütközõ szerzõdés megtámadható.\\
b) A jogellenes fenyegetéssel kötött szerzõdés semmis.\\
\uline {c) Az uzsorás szerzõdés semmis.}\\
d) Mindhárom fenti állítás hamis.
\end{tcolorbox}

%158.
\begin{tcolorbox}[title={158. Kérdés}]
Az alábbiak közül melyiknek a jogkövetkezménye a szerzõdés semmissége?
\tcblower
a) A feltûnõ értékaránytalansággal kötött szerzõdés.\\
b) A megtévesztéssel kötött szerzõdés.\\
\uline {c) A jó erkölcsbe ütközõ szerzõdés.}\\
d) A jogellenes fenyegetéssel kötött szerzõdés.
\end{tcolorbox}

%159.
\begin{tcolorbox}[title={159. Kérdés}]
Melyik állítás hamis?
\tcblower
a) Érvénytelen szerzõdés alapján a teljesítés nem követelhetõ.\\
b) Hatálytalan szerzõdés alapján a teljesítés nem követelhetõ.\\
\uline {c) Érvénytelen szerzõdés alapján teljesített szolgáltatást nem lehet visszakövetelni.}\\
d) Érvénytelen szerzõdés utóbb érvényessé válhat a felek akaratából.
\end{tcolorbox}

\end{frame}


\begin{frame}

%160. 
\begin{tcolorbox}[title={160. Kérdés}]
Szerzõdésszegés esetén melyiket nem teheti meg a sérelmet szenvedett fél?
\tcblower
a) Továbbra is követelheti a teljesítést.\\
b) Fedezeti szerzõdést köthet és követelheti annak költségeit.\\
\uline {c) A nála levõ, a másik fél tulajdonát képezõ dologból kielégítést kereshet.}\\
d) A saját szolgáltatását visszatarthatja.
\end{tcolorbox}

%161.
\begin{tcolorbox}[title={161. Kérdés}]
Az alábbiak közül melyik nem tartozik a szerzõdésszegés nevesített esetei közé?
\tcblower
a) A késedelem.\\
\uline {b) Az elállás.}\\
c) A lehetetlenülés.\\
d) A teljesítés megtagadása.
\end{tcolorbox}

%162.
\begin{tcolorbox}[title={162. Kérdés}]
Melyik állítás hamis?
\tcblower
a) Ha jogosult késedelembe esik a kárveszély átszáll rá.\\
b) Ha a hiba a teljesítés után keletkezett, szavatossági igény nem érvényesíthetõ.\\
\uline {c) Ha a teljesítés lehetetlenné válik, a szerzõdés megtámadható.}\\
d) A kellékszavatossági igények 1 év alatt elévülnek.
\end{tcolorbox}

%163.
\begin{tcolorbox}[title={163. Kérdés}]
Mennyi a kellékszavatossági igények elévülési ideje fogyasztói szerzõdés esetén?
\tcblower
a) 1 év\\
\uline {b) 2 év}\\
c) 5 év\\
d) Fogyasztói szerzõdés esetén ezen igények nem évülnek el.
\end{tcolorbox}

\end{frame}


\begin{frame}

%164.
\begin{tcolorbox}[title={164. Kérdés}]
Az alábbiak közül melyiket nem választhatja a fogyasztó termékszavatossági igényként?
\tcblower
a) A kijavítást.\\
b) A kicserélést.\\
\uline {c) Az elállást.}\\
d) Mindhármat választhatja.
\end{tcolorbox}

%165.
\begin{tcolorbox}[title={165. Kérdés}]
Melyik állítás igaz a jótállásra vonatkozóan?
\tcblower
a) A jótállás minden szerzõdés esetén megilleti a szolgáltatás jogosultját.\\
b) A szavatosság és a jótállás ugyanazt jelenti.\\
c) A jótállási idõ nem lehet hosszabb 5 évnél.\\
\uline {d) A jótállás lehet önkéntes vagy kötelezõ.}
\end{tcolorbox}

%166. 
\begin{tcolorbox}[title={166. Kérdés}]
Az alábbi alanyváltozások közül melyik háromoldalú megállapodás?
\tcblower
a) A jogátruházás.\\
b) Az engedményezés.\\
\uline {c) A tartozásátvállalás.}\\
d) Egyik sem.
\end{tcolorbox}

%167. 
\begin{tcolorbox}[title={167. Kérdés}]
Melyik állítás hamis?
\tcblower
a) A szerzõdés felmondása egyoldalú jognyilatkozat.\\
\uline {b) Az elállás a jövõre nézve szünteti meg a szerzõdést.}\\
c) A szerzõdés felbontása a felek közös nyilatkozata.\\
d) A felek a szerzõdést bármikor megszüntethetik.
\end{tcolorbox}

\end{frame}

\begin{frame}[plain]
\begin{tcolorbox}[center, colback={myyellow}, coltext={black}, colframe={myyellow}]
    {\RHuge 10 Egyes szerzodesek}\\
\end{tcolorbox}
\end{frame}

\begin{frame}


%168. 
\begin{tcolorbox}[title={168. Kérdés}]
Az alábbiak közül melyik jog nem alapítható (szerezhetõ meg) bármikor?
\tcblower
a) Az eladási jog.\\
b) A vételi jog.\\
\uline {c) A visszavásárlási jog.}\\
d) Az elõvásárlási jog.
\end{tcolorbox}

%169. 
\begin{tcolorbox}[title={169. Kérdés}]
Melyik állítás hamis?
\tcblower
a) Az eladási jog jogosultja egyoldalú nyilatkozattal eladhatja a dolgot.\\
\uline {b) Az elõvásárlási jog jogosultja egyoldalú nyilatkozattal megveheti a dolgot.}\\
c) A vételi jog jogosultja egyoldalú nyilatkozattal megveheti a dolgot.\\
d) A visszavásárlási jog jogosultja egyoldalú nyilatkozattal megveheti a dolgot.
\end{tcolorbox}

%170.
\begin{tcolorbox}[title={170. Kérdés}]
Az alábbiak közül melyiket nem kell az ingatlan-nyilvántartásba bejegyeztetni?
\tcblower
a) A tulajdonjog fenntartást.\\
b) A vételi jogot.\\
c) Az eladási jogot.\\
\uline {d) Mindhárom fentit be kell jegyeztetni.}
\end{tcolorbox}

%171. 
\begin{tcolorbox}[title={171. Kérdés}]
Melyik állítás igaz?
\tcblower
a) Az ajándékozás egyoldalú jognyilatkozat.\\
b) Ajándékozás történhet ellenérték fejében is.\\
\uline {c) Az ajándék egyes esetekben visszakövetelhetõ.}\\
d) Az ajándékozásra a jogalap nélküli gazdagodás szabályait kell alkalmazni.
\end{tcolorbox}

\end{frame}


\begin{frame}

%172. 
\begin{tcolorbox}[title={172. Kérdés}]
Melyik állítás hamis a vállalkozási szerzõdésre vonatkozóan?
\tcblower
a) A vállalkozási szerzõdés eredménykötelem.\\
b) A megrendelõ a teljesítés elõtt bármikor elállhat.\\
\uline {c) A vállalkozó a teljesítés elõtt bármikor elállhat.}\\
d) A vállalkozót törvényes zálogjog illeti meg.
\end{tcolorbox}

%173.
\begin{tcolorbox}[title={173. Kérdés}]
Az alábbiak közül melyik nem a vállalkozási szerzõdés altípusa?
\tcblower
a) A tervezési szerzõdés.\\
b) A fuvarozási szerzõdés.\\
\uline {c) A szállítmányozási szerzõdés.}\\
d) A kivitelezési szerzõdés.
\end{tcolorbox}

%174. 
\begin{tcolorbox}[title={174. Kérdés}]
Melyik állítás hamis?
\tcblower
\uline {a) A megbízási szerzõdés eredménykötelem.}\\
b) A megbízó a szerzõdést felmondhatja.\\
c) A megbízott a szerzõdést felmondhatja.\\
d) A megbízottat törvényes zálogjog illeti meg.
\end{tcolorbox}

%175. 
\begin{tcolorbox}[title={175. Kérdés}]
Az alábbiak közül melyik nem a megbízási szerzõdés altípusa?
\tcblower
a) A bizományi szerzõdés.\\
b) A szállítmányozási szerzõdés.\\
c) A közvetítõi szerzõdés.\\
\uline {d) A fuvarozási szerzõdés.}
\end{tcolorbox}

\end{frame}

\begin{frame}[plain]
\begin{tcolorbox}[center, colback={myyellow}, coltext={black}, colframe={myyellow}]
    {\RHuge 11 Adatvédelem}\\
\end{tcolorbox}
\end{frame}

\begin{frame}


%176. 
\begin{tcolorbox}[title={176. Kérdés}]
Az alábbiak közül melyik nem minõsül különleges személyes adatnak?
\tcblower
a) Az érintett egészségi állapota.\\
\uline {b) Az érintett vagyoni helyzete.}\\
c) Az érintett vallási meggyõzõdése.\\
d) Az érintett politikai véleménye.
\end{tcolorbox}

%177.
\begin{tcolorbox}[title={177. Kérdés}]
Az alábbi mûveletek közül melyik nem minõsül adatkezelésnek?
\tcblower
a) Az adatok gyûjtése.\\
b) Az adatok tárolása.\\
c) Az adatok lekérdezése.\\
\uline {d) Mind a három fenti mûvelet adatkezelésnek minõsül.}
\end{tcolorbox}

%177/A.
\begin{tcolorbox}[title={177/A. Kérdés}]
Az alábbi mûveletek közül melyik nem minõsül adatkezelésnek?
\tcblower
a) Az adatok gyûjtése.\\
\uline {b) Az adatkezelési mûveletekhez kapcsolódó technikai feladatok elvégzése.}\\
c) Az adatok lekérdezése.\\
d) Mind a három fenti mûvelet adatkezelésnek minõsül.
\end{tcolorbox}

%178. 
\begin{tcolorbox}[title={178. Kérdés}]
Az alábbi 3 állítás (a,b,c) közül melyik hamis?
\tcblower
a) Személyes adatok kezeléséhez elég ha az érintett szóban hozzájárult.\\
b) Különleges adat csak törvényi elõírás alapján kezelhetõ.\\
c) Bûnügyi személyes adatot csak állami vagy önkormányzati szerv kezelhet.\\
\uline {d) Mind a három fenti állítás igaz.}
\end{tcolorbox}

\end{frame}


\begin{frame}

%179.
\begin{tcolorbox}[title={179. Kérdés}]
Az alábbiak közül melyik esetben nem tiltakozhat az érintett személyes adatának kezelése ellen?
\tcblower
a) Közvetlen üzletszerzés céljára történõ felhasználás esetén.\\
b) Közvélemény kutatási célú felhasználás esetén.\\
c) Tudományos célú felhasználás esetén.\\
\uline {d) Mindhárom fenti esetben tiltakozhat.}
\end{tcolorbox}

%179/A.
\begin{tcolorbox}[title={179/A. Kérdés}]
Az alábbiak közül melyik esetben nem tiltakozhat az érintett személyes adatának kezelése ellen?
\tcblower
a) Közvetlen üzletszerzés céljára történõ felhasználás esetén.\\
b) Közvélemény kutatási célú felhasználás esetén.\\
\uline {c) Ha az adatkezelést közérdekû célból törvény rendeli el.}\\
d) Mindhárom fenti esetben tiltakozhat.
\end{tcolorbox}

\end{frame}

\begin{frame}[plain]
\begin{tcolorbox}[center, colback={myyellow}, coltext={black}, colframe={myyellow}]
    {\RHuge 12 Szellemi alkotas}\\
\end{tcolorbox}
\end{frame}

\begin{frame}

%180. 
\begin{tcolorbox}[title={180. Kérdés}]
Az alábbiak közül melyik alkotás nem áll a szerzõi jog védelme alatt?
\tcblower
\uline {a) Ötlet, eljárás.}\\
b) Szoftver\\
c) Adatbázis\\
d) Térkép
\end{tcolorbox}

%181. 
\begin{tcolorbox}[title={181. Kérdés}]
Mennyi ideig védi a szerzõi jog a mûvet?
\tcblower
a) 50 évig.\\
b) A szerzõ élete végéig.\\
\uline {c) A szerzõ életében és halála után 70 évig.}\\
d) 20 évig.
\end{tcolorbox}

%182.
\begin{tcolorbox}[title={182. Kérdés}]
Melyik állítás hamis?
\tcblower
a) A felhasználási engedély korlátozható valamely területre.\\
b) A felhasználási engedély korlátozható megadott idõtartamra.\\
c) A felhasználási engedély korlátozható valamely felhasználási módra.\\
\uline {d) A felhasználási engedély szóban is érvényes.}
\end{tcolorbox}

%183.
\begin{tcolorbox}[title={183. Kérdés}]
Mennyi ideig védi a szabadalmi oltalom a találmányt?
\tcblower
a) 10 évig.\\
\uline {b) 20 évig.}\\
c) A szabadalmas élete végéig.\\
d) A szabadalmas életében és halála után 20 évig.
\end{tcolorbox}

\end{frame}

\end{document}
