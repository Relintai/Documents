% Compile twice!

\documentclass{beamer}
\usepackage{tikz}

\usepackage[T1]{fontenc}
\usepackage{amsfonts}
\usepackage{amsmath}
\usepackage[utf8]{inputenc}

\usetheme{boxes}


\begin{document}

\begin{frame}[plain]
\begin{tikzpicture}[overlay, remember picture]
\node[anchor=center] at (current page.center) {
\begin{beamercolorbox}[center]{title}
    {\Huge Diszkrét Matematika}\\
    {\Large Vizsgatételek}
\end{beamercolorbox}};
\end{tikzpicture}
\end{frame}

\begin{frame}[plain]
\begin{tikzpicture}[overlay, remember picture]
\node[anchor=center] at (current page.center) {
\begin{beamercolorbox}[center]{title}
    {\Huge Halmazok, Relációk}
\end{beamercolorbox}};
\end{tikzpicture}
\end{frame}


\begin{frame}

\begin{block}{Tétel: Minden dolog halmaza}
Nincs olyan halmaz, amelynek minden dolog eleme.
\end{block}

\begin{block}{Biz}
asasdad
\end{block}

\end{frame}

\begin{frame}

\begin{block}{Definíció: Unió}
Ha A és B halmazok, akkor A és B unióján a következő halmazt értjük:\\
$$A \cup B = \{x | x \in A \vee x \in B\}$$
\end{block}

\begin{block}{Tétel: Az unió tulajdonságai}
Legyenek A, B, C tetszőleges halmazok. Ekkor:

\begin{enumerate}
\item $A \cup \emptyset = A$
\item $A \cup B = B \cup A$ (Kommutativitás)
\item $A \cup (B \cup C) = (A \cup B) \cup )$ (Asszociativitás)
\item $A \cup A = A$ (Idempotencia)
\item $A \subseteq B$ akkor, és csak akkor, ha $A \cup B = B$
\end{enumerate}

\end{block}

\end{frame}

\begin{frame}

\begin{block}{Definíció: Metszet}
Ha A és B halmazok, akkor A és B metszetén a következő halmazt értjük:\\
$$A \cap B = \{x \in A \wedge x \in B\}$$
\end{block}

\begin{block}{Tétel: A metszet tulajdonságai}
Legyenek A, B, C tetszőleges halmazok. Ekkor:

\begin{enumerate}
\item $A \cap \emptyset = \emptyset$
\item $A \cap B = B \cap A$ (Kommutativitás)
\item $A \cap (B \cap C) = (A \cap B) \cap C$ (Asszociativitás)
\item $A \cap A = A$ (Idempotencia)
\item $A \subseteq B$ akkor, és csak akkor, ha $A \cap B = A$
\end{enumerate}
\end{block}

\end{frame}

\begin{frame}

\begin{block}{Tétel: Unió és metszet disztributivitása}
Legyenek A, B, C tetszőleges halmazok. Ekkor:

\begin{enumerate}
\item $A \cap (B \cup C) = (A \cap B) \cup (B \cap C)$ (A metszet disztributivitása az unióra nézve)

\item $A \cup (B \cap C) = (A \cup B) \cap (B \cup C)$ (Az unió disztributivitása a metszetre nézve)
\end{enumerate}
\end{block}

\end{frame}

\begin{frame}

\begin{block}{Definíció: Komplementer}
Ha X halmaz, A $\wedge$ X, akkor A halmaz X-re vonatkoztatott komplementere:\\
$$A' = X \setminus A$$
\end{block}

\begin{block}{Tétel: A komplementer tulajdonságai}
Legyenek A, B $\wedge$ X halmazok. Ekkor:

\begin{enumerate}
\item $(A')' = A$
\item $\emptyset' = X$
\item $A \cap A' = \emptyset$
\item $A \cup A' = X$
\item $A \subseteq B$ akkor, és csak akkor, ha $B' \subseteq A'$
\item $(A \cap B)' = A' \cup B'$
\item $(A \cup B)' = A' \cap B'$
\end{enumerate}
\end{block}

\end{frame}

\begin{frame}

\begin{block}{Definíció: Halmaz osztályfelbontása}
A tetszőleges X halmazt \textbf{osztályozzuk (osztályokra bontjuk)}, ha páronként diszjunkt, nemüres részhalmazainak uniójaként állítjuk elő.
\end{block}

\begin{block}{Az X $\in$ X elem \textbf{ekvivalencia osztálya}:}
$$\overline{x} = \{y \in X : y \sim x\}$$
\end{block}

\begin{block}{Tétel: Ekvivalenciareláció és osztályfelbontás kapcsolata}
Valamely X halmazon értelmezett $\sim$ ekvivalenciareláció X-nek egy osztályfelbontását adja. Megfordítva, az X halmaz minden osztályfelbontása egy $\sim$ ekvivalenciarelációt hoz létre.
\end{block}

\begin{block}{Biz}
asasdad
\end{block}

\end{frame}


\begin{frame}[plain]
\begin{tikzpicture}[overlay, remember picture]
\node[anchor=center] at (current page.center) {
\begin{beamercolorbox}[center]{title}
    {\Huge Algebrai struktúrák, számhalmazok}
\end{beamercolorbox}};
\end{tikzpicture}
\end{frame}

\begin{frame}

\begin{block}{Tétel: Egységelem és inverz félcsoportban
Félcsoportban legfeljebb egy egységelem létezik, és minden elemnek legfeljebb egy, az egységelemre vonatkozó inverze létezik.}
\end{block}

\begin{block}{Bizonyítás}
\end{block}

\end{frame}

\begin{frame}

\begin{block}{Tétel: Észrevételek gyűrűkben}
\end{block}

\end{frame}

\begin{frame}

\begin{block}{Tétel: Nullosztó és regularitás}

\end{block}

\begin{block}{Bizonyítás}
\end{block}

\end{frame}

\begin{frame}

\begin{block}{Természetes számok}
Halmaz, egy nullér, és egy injektív unér művelettel (rákövetkezés)


\end{block}

\end{frame}

\begin{frame}

\begin{block}{Tétel: N rendezése}
\end{block}

\end{frame}

\begin{frame}

\begin{block}{Tétel: Felső határ és arkhimédészi tulajdonság}
\end{block}

\begin{block}{Bizonyítás}
\end{block}

\end{frame}

\begin{frame}

\begin{block}{Tétel: Q nem felső határ tulajdonságú}
(12. dia lap alja)
\end{block}

\end{frame}

\begin{frame}

\begin{block}{Tétel: $\sqrt{2}$ nem racionális}
Nincs Q-ban olyan szám, amelynek négyzete 2.
\end{block}

\begin{block}{Bizonyítás}
\end{block}

\end{frame}


\begin{frame}

\begin{block}{Tétel: Az algebra alaptétele}
\end{block}

\end{frame}

\begin{frame}[plain]
\begin{tikzpicture}[overlay, remember picture]
\node[anchor=center] at (current page.center) {
\begin{beamercolorbox}[center]{title}
    {\Huge Számelmélet}
\end{beamercolorbox}};
\end{tikzpicture}
\end{frame}

\begin{frame}

\begin{block}{Tétel: Az oszthatóság tulajdonságai EIT-ban}
\end{block}

\end{frame}

\begin{frame}

\begin{block}{Tétel: Prím és irreducibilis elem EIT-ban}
\end{block}

\begin{block}{Bizonyítás}
\end{block}

\end{frame}

\begin{frame}

\begin{block}{Tétel: Maradékos osztás Z-ben}
\end{block}

\end{frame}

\begin{frame}

\begin{block}{Tétel: Prím és irreducibilis elem Z-ben}
\end{block}

\begin{block}{Bizonyítás}
\end{block}

\end{frame}

\begin{frame}

\begin{block}{Tétel: A számelmélet alaptétele}
\end{block}

\begin{block}{Bizonyítás}
\end{block}

\end{frame}

\begin{frame}

\begin{block}{Tétel: Eukleidész tétele}
\end{block}

\begin{block}{Bizonyítás}
\end{block}

\end{frame}

\begin{frame}

\begin{block}{Tétel: Kongruencia tulajdonságai}
\end{block}

\end{frame}

\begin{frame}

\begin{block}{Tétel: Omnibusz tétel}
\end{block}

\begin{block}{Bizonyítás}
\end{block}

\end{frame}

\begin{frame}

\begin{block}{Tétel: Euler-Fermat tétel}
\end{block}

\begin{block}{Bizonyítás}
\end{block}

\end{frame}

\begin{frame}

\begin{block}{Tétel: (Kis) Fermat tétel}
\end{block}

\begin{block}{Bizonyítás}
\end{block}

\end{frame}

\begin{frame}

\begin{block}{Tétel: A diofantikus egyenlet megoldása}
\end{block}

\begin{block}{Bizonyítás}
\end{block}

\end{frame}

\begin{frame}

\begin{block}{Tétel: Kínai maradéktétel}
\end{block}

\end{frame}

\begin{frame}

\begin{block}{Tétel: Számelméleti függvények}
\end{block}

\end{frame}

\begin{frame}

\begin{block}{Tétel: fi multiplikativitása}
\end{block}

\begin{block}{Bizonyítás}
\end{block}

\end{frame}

\begin{frame}

\begin{block}{Tétel: fi(n) kiszámolása}
\end{block}

\begin{block}{Bizonyítás}
\end{block}

\end{frame}

\begin{frame}[plain]
\begin{tikzpicture}[overlay, remember picture]
\node[anchor=center] at (current page.center) {
\begin{beamercolorbox}[center]{title}
    {\Huge Kombinatorika}
\end{beamercolorbox}};
\end{tikzpicture}
\end{frame}

\begin{frame}

\begin{block}{Tétel: Véges halmaz valódi részhalmaza}
\end{block}

\end{frame}

\begin{frame}

\begin{block}{Tétel: Skatulya-elv}
\end{block}

\begin{block}{Bizonyítás}
\end{block}

\end{frame}

\begin{frame}

\begin{block}{Tétel: Permutációk száma}
\end{block}

\begin{block}{Bizonyítás}
\end{block}

\end{frame}

\begin{frame}

\begin{block}{Tétel: Variációk száma}
\end{block}

\begin{block}{Bizonyítás}
\end{block}

\end{frame}

\begin{frame}

\begin{block}{Tétel: Ismétléses variációk száma}
\end{block}

\begin{block}{Bizonyítás}
\end{block}

\end{frame}

\begin{frame}

\begin{block}{Tétel: Kombinációk száma}
\end{block}

\begin{block}{Bizonyítás}
\end{block}

\end{frame}

\begin{frame}

\begin{block}{Tétel: Ismétléses kombinációk száma}
\end{block}

\begin{block}{Bizonyítás}
\end{block}

\end{frame}

\begin{frame}

\begin{block}{Tétel: Ismétléses permutációk száma}
\end{block}

\begin{block}{Bizonyítás}
\end{block}

\end{frame}

\begin{frame}

\begin{block}{Tétel: Binomiális tétel}
\end{block}

\begin{block}{Bizonyítás}
\end{block}

\end{frame}

\begin{frame}

\begin{block}{Tétel: Logikai szita formula}
\end{block}

\begin{block}{Bizonyítás}
\end{block}

\end{frame}








\begin{frame}

\begin{block}{Tétel: Egységelem és inverz félcsoportban}
\end{block}

\begin{block}{Bizonyítás}
\end{block}

\end{frame}


\end{document}