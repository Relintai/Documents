% Compile twice!
% With the current MiKTeX, you need to install the beamer, and the translator packages directly form the package manager!

% !TEX root = ./PrezA4Page.tex

% Uncomment these to get the presentation form
%\documentclass{beamer}
%\geometry{paperwidth=200mm,paperheight=200mm, top=0in, bottom=0.2in, left=0.2in, right=0.2in}

% Uncomment these, and comment the 2 lines above, to get a paper-type article
%\documentclass[10pt]{article}
%\usepackage{geometry}
%\geometry{top=0.2in, bottom=0.2in, left=0.2in, right=0.2in}
%\usepackage{beamerarticle}
%\renewcommand{\\}{\par\noindent}
%\setbeamertemplate{note page}[plain]

% Half A4 geometry
%\geometry{paperwidth=105mm,paperheight=297mm,top=0.2in, bottom=0.2in, left=0.2in, right=0.2in}

% "1/3" A4 geometry
%\geometry{paperwidth=105mm,paperheight=455mm,top=0.1in, bottom=0.1in, left=0.1in, right=0.1in}

% "1/6" A4 geometry
%\geometry{paperwidth=105mm,paperheight=891mm,top=0.1in, bottom=0.1in, left=0.1in, right=0.1in}

% "1/5" A4 geometry
%\geometry{paperwidth=105mm,paperheight=740mm,top=0.1in, bottom=0.1in, left=0.1in, right=0.1in}

% "1/4" A4 geometry
%\geometry{paperwidth=105mm,paperheight=594mm,top=0.1in, bottom=0.1in, left=0.1in, right=0.1in}

% Uncomment these, to put more than one slide / page into a generated page.
%\usepackage{pgfpages}
% Choose one
%\pgfpagesuselayout{2 on 1}[a_4paper]
%\pgfpagesuselayout{4 on 1}[a_4paper]
%\pgfpagesuselayout{8 on 1}[a_4paper]

% Includes
\usepackage{tikz}
\usepackage{tkz-graph}
\usetikzlibrary{shapes,arrows,automata}
\usepackage[T1]{fontenc}
\usepackage{amsfonts}
\usepackage{amsmath}
\usepackage[utf8]{inputenc}
\usepackage{booktabs}
\usepackage{array}
\usepackage{arydshln}
\usepackage{enumerate}
\usepackage[many, poster]{tcolorbox}
\usepackage{pgf}
\usepackage[makeroom]{cancel}
\usepackage{verbatim}

\providecommand{\includecolors}{
% Colors
\definecolor{myred}{rgb}{0.87,0.18,0}
\definecolor{myorange}{rgb}{1,0.4,0}
\definecolor{myyellowdarker}{rgb}{1,0.69,0}
\definecolor{myyellowlighter}{rgb}{0.91,0.73,0}
\definecolor{myyellow}{rgb}{0.97,0.78,0.36}
\definecolor{myblue}{rgb}{0,0.38,0.47}
\definecolor{mygreen}{rgb}{0,0.52,0.37}
\colorlet{mybg}{myyellow!5!white}
\colorlet{mybluebg}{myyellowlighter!3!white}
\colorlet{mygreenbg}{myyellowlighter!3!white}

\setbeamertemplate{itemize item}{\color{black}$-$}
\setbeamertemplate{itemize subitem}{\color{black}$-$}
\setbeamercolor*{enumerate item}{fg=black}
\setbeamercolor*{enumerate subitem}{fg=black}
\setbeamercolor*{enumerate subsubitem}{fg=black}

% These are different themes, only uncomment one at a time
\tcbset{enhanced,fonttitle=\mdseries,boxsep=7pt,arc=0pt,colframe={myyellowlighter},colbacktitle={myyellow},colback={mybg},coltitle={black}, coltext={black},attach boxed title to top left={xshift=-2mm,yshift=-2mm},boxed title style={size=small,arc=0mm}}

%\tcbset{colback=yellow!5!white,colframe=yellow!84!black}
%\tcbset{enhanced,colback=red!10!white,colframe=red!75!black,colbacktitle=red!50!yellow,fonttitle=
%\tcbset{enhanced,attach boxed title to top left}
%\tcbset{enhanced,fonttitle=\bfseries,boxsep=5pt,arc=8pt,borderline={0.5pt}{0pt}{red},borderline={0.5pt}{5pt}{blue,dotted},borderline={0.5pt}{-5pt}{green}}
}% fallback definition
\includecolors

\setbeamertemplate{itemize item}{\color{black}$-$}
\setbeamertemplate{itemize subitem}{\color{black}$-$}
\setbeamercolor*{enumerate item}{fg=black}
\setbeamercolor*{enumerate subitem}{fg=black}
\setbeamercolor*{enumerate subsubitem}{fg=black}

 \renewcommand{\familydefault}{\sfdefault}
%\renewcommand{\familydefault}{\rmdefault}

\renewcommand{\footnotesize}{\fontsize{1.2em}{0.2em}}
\renewcommand{\normalsize}{\fontsize{1.2em}{0.2em}}
\renewcommand{\large}{\footnotesize}
\renewcommand{\Large}{\footnotesize}


\renewcommand{\scriptsize}{\footnotesize}
\renewcommand{\LARGE}{\footnotesize}
\renewcommand{\Huge}{\footnotesize}

\renewcommand{\tiny}{\footnotesize}
\renewcommand{\small}{\footnotesize}

\fontsize{1.2em}{0.2em}
\selectfont

\newcommand{\RHuge}{\fontsize{1.8em}{0.3em}\selectfont}

\newsavebox\CBox
%\newcommand<>*\textBF[1]{\sbox\CBox{#1}\resizebox{\wd\CBox}{\ht\CBox}{\textbf#2{#1}}}
\newcommand<>*\textBF[1]{\only#2{\sbox\CBox{#1}\resizebox{\wd\CBox}{\ht\CBox}{\textbf{#1}}}}

% Beamer theme
\usetheme{boxes}

% tikz settings for the flowchart(s)
\tikzstyle{decision} = [diamond, minimum width=3cm, minimum height=1cm, text centered, draw=black, fill=green!15]
\tikzstyle{tcolorbox} = [rectangle, draw, fill=blue!15, text width=20em, text centered, minimum height=1em]

\tikzstyle{line} = [draw, -latex']
\tikzstyle{cloud} = [draw, ellipse,fill=red!20, node distance=3cm,
    minimum height=2em]
\tikzstyle{arrow} = [thick,->,>=stealth]

\newcolumntype{C}[1]{>{\centering\let\newline\\\arraybackslash\hspace{0pt}}m{#1}}
\renewcommand{\arraystretch}{1.2}

\setlength\dashlinedash{0.2pt}
\setlength\dashlinegap{1.5pt}
\setlength\arrayrulewidth{0.3pt}

\newcommand{\mtinyskip}{\vspace{0.2em}}
\newcommand{\msmallskip}{\vspace{0.3em}}
\newcommand{\mmedskip}{\vspace{0.5em}}
\newcommand{\mbigskip}{\vspace{1em}}
\renewcommand{\u}[1]{\underline{#1}}

\begin{document}

\begin{frame}[plain]
\begin{tcolorbox}[center, colback={myyellow}, coltext={black}, colframe={myyellow}]
    {\RHuge Lineáris Algebra és Geometria}\\
\end{tcolorbox}
\end{frame}


%\begin{tcolorbox}[title={Def.: }]
%\end{tcolorbox}

% --------------------   Vektorok koordinátavektora  --------------------

\begin{frame}[plain]
\begin{tcolorbox}[center, colback={myyellow}, coltext={black}, colframe={myyellow}]
    {\RHuge  A rész}
    \mmedskip
\end{tcolorbox}
\end{frame}

\begin{frame}
  \begin{tcolorbox}
Az alábbiakban – tájékoztató jelleggel – fölsoroljuk a vizsgadolgozat A részében szereplő főbb témaköröket néhány mintafeladattal, azok megoldásával és esetenként részletesebb magyarázattal. Ez a rész a szemléltetést szolgálja: a dolgozatban általában nem pont ezek a kérdések fognak szerepelni. A feladatok nehézségét szimbólumok jelzik. Ahol egy definíciót vagy tételt kell közvetlenül alkalmazni, ott -N- a jel. Ha a fogalom már nehezebb, vagy több lépést kell tenni, több fogalmat összekapcsolni, nemtriviális átalakítást kell végezni, ott -R- szerepel. A -Q- azt jelzi, hogy a feladat témája kifejezetten nehéz, vagy pedig ötlet kell a megoldáshoz.
  \end{tcolorbox}
\end{frame}

\begin{frame}[plain]
\begin{tcolorbox}[center, colback={myyellow}, coltext={black}, colframe={myyellow}]
    {\RHuge  (1) Vektorok koordinátavektora}
    \mmedskip
\end{tcolorbox}
\end{frame}

\begin{frame}
  \begin{tcolorbox}[title={1/1. -N-}]
      A $\{b_1,b_2,b_3\}$ vektorhalmaz bázis a $V \leq \mathbb{R}^n$ altérben. Határozzuk meg a $v = 2b_1 + 3b_2 -b_3$ vektor koordinátavektorát a $\{b_1,b_2,b_3\}$ bázisban.
  \tcblower
    A vektor bázisban vett koordinátavektorának fogalmát kéri számon.\\
    
    A vektorból kell kiszámítani a koordinátáit.\\
    \mmedskip 
  
   $[v]_{b_1,b_2,b_3} =$ $\begin{bmatrix} 
  				2  \\
  				3 \\
  				-1
			\end{bmatrix}$
  \end{tcolorbox}
\end{frame}

\begin{frame}
  \begin{tcolorbox}[title={1/2. -R-}]
      Legyen $\mathbb{B} = \{b_1,b_2,b_3\}$ bázis a $V \leq \mathbb{R}^n$ altérben. Egy $v \in V$ vektornak a $\{b_1 + b_2,b_2 + b_3,b_3\}$ bázisban fölírt koordinátavektora $[v]_{b_1+b_2,b_2+b_3,b_3} = [1 1 0]^T$.\\
      \mmedskip
      
      Mi $v$ koordinátavektora a $\mathbb{B}$ bázisban?
  \tcblower
    A feladat a vektor bázisban vett koordinátavektorának fogalmát kéri számon.\\
    \mmedskip
   
    Kétlépéses, először a feladat szövegében megadott koordinátavektort írjuk át a bázisvektorok lineáris kombinációjává: $v = 1 \cdot (b_1 + b_2) + 1 \cdot (b_2 + b_3) + 0 \cdot (b_3) = 1 \cdot b_1 + 2 \cdot b_2 + 1 \cdot b_3$, majd az így kapott lineáris kombinációt koordinátavektorrá.\\
     \mmedskip
     
   $[v]_{b_1,b_2,b_3} =$ $\begin{bmatrix} 
  				1  \\
  				2 \\
  				1
			\end{bmatrix}$
  \end{tcolorbox}
\end{frame}

\begin{frame}
  \begin{tcolorbox}[title={1/3. -R-}]
      A $\{b_1,b_2,b_3\}$ vektorhalmaz bázis a $V \leq \mathbb{R}^n$ altérben. Ebben a bázisban a $v$, illetve $w$ vektorok koordinátorvektora $[v]_{b_1,b_2,b_3} = $ $\begin{bmatrix} 
  				1  \\
  				0 \\
  				2
			\end{bmatrix}$ és $[w]{b_1,b_2,b_3} = $ $\begin{bmatrix} 
  				3  \\
  				1 \\
  				0
			\end{bmatrix}$. Írjuk föl a $v+2w$ vektort a $\{b_1,b_2,b_3\}$ vektorok lineáris kombinációjaként.
  \tcblower
    A feladat két fogalmat kérdez:\\
      hogyan kell a koordinátákból kiszámítani a vektorokat, de még azt is tudni kell, hogy hogyan kell koordinátavektorokkal műveleteket végezni.\\
      
      Tehát először a két koordinátavektor megfelelő lineáris kombinációját számoljuk ki: $| | 1 0 2  |  |+   2 | | 3 1 0 | | 7 2 2 |$, majd az így kapott oszlopvektort átírjuk a bázisvektorok lineáris kombinációjává.\\
      
      De megtehetjük azt is, hogy először a v és w vektorokat írjuk fel a bázisvektorok lineáris kombinációjaként, majd ebből számítjuk ki a $v+2w$-t.
      
   $v + 2w = 7b_1 + 2b_2 + 2b_3$
  \end{tcolorbox}
\end{frame}
  
\begin{frame}
  \begin{tcolorbox}[title={1/4. -Q-}]
      Tegyük föl, hogy a $\{b_1,b_2,b_3\}$ vektorhalmaz bázis a $V \leq \mathbb{R}^n$ altérben. Mi lesz ezen $b_i$ bázisvektorok $v$-vel jelölt összegénekakoordinátavektora a $\{b_1+b_2,b_2,b_3\}$ bázisban?
  \tcblower
    Az a kérdés, hogy a $b_1+b_2+b_3 = {\lambda}1(b_1+b_2)+{\lambda}2b_2+{\lambda}3b_3$ felírásban mennyi a ${\lambda}i$ ismeretlenek értéke.\\|
      Ezekre úgy kapunk lineáris egyenletrendszert, hogy a $b_i$ bázisvektor együtthatóját az egyenlet két oldalán összehasonlítjuk.\\
      
      Azonnal látszik, hogy $b_1 + b_2 + b_3 = 1 \cdot (b_1 + b_2) + 0 \cdot b_2 + 1 \cdot b_3$ megfelelő felírás, és ezért ezek az együtthatók adják a keresett koordinátavektort.\\
      
      Ennek a problémának a megoldását az általános esetben az elemi bázistranszformációról szóló tétel szolgáltatja.
      
   $[v]_{b_1,b_2,b_3} =$ 
  \end{tcolorbox}
\end{frame}  

\begin{frame}[plain]
\begin{tcolorbox}[center, colback={myyellow}, coltext={black}, colframe={myyellow}]
    {\RHuge  (2) Alterek alaptulajdonságai}
    \mmedskip
\end{tcolorbox}
\end{frame}

\begin{frame}
  \begin{tcolorbox}[title={2/1. -R-}]
      Konkrét vektorokat megadva mutassuk meg, miért nem alkot alteret $\mathbb{R}^3$-ben azon vektorok halmaza, ahol az első két komponens szorzata nulla. 
  \tcblower

    \mmedskip 
  
   Pl. benne vannak az altérben, de az összegük,nincs.
  \end{tcolorbox}
\end{frame}


\begin{frame}
  \begin{tcolorbox}[title={2/2. -Q-}]
      Konkrét mátrixokat megadva mutassuk meg, miért nem alkot alteret $\mathbb{R}^{2 x 2}$-ben a nulla determinánsú mátrixok halmaza. 
  \tcblower
    A nulla determinánsú mátrixok összege könnyen lehet nem nulla determinánsú, s hasonlóképpen nem nulla determinánsú mátrixok összege lehet nulla determinánsú.\\
    \mmedskip 
  
    Pl. det= det= 0, ugyanakkor det
  \end{tcolorbox}
\end{frame}

\begin{frame}
  \begin{tcolorbox}[title={2/3. -R-}]
      Tekintsük azoknak az $\mathbb{R}^3$-beli  vektoroknak a halmazát, amelyek az alábbi feltételnek tesznek eleget.\\ű
      
      Mely eset(ek)ben kapunk alteret?\\
      (A) $x1 = x2 + 2x3$\\
      (B) $x1 = x2 + 2$\\
      (C) $x1x2 = 0$\\
      (D) $x2 1 = 0$
  \tcblower

    \mmedskip
    
    Alter(ek): (A), (D)
  \end{tcolorbox}
\end{frame}

\begin{frame}
  \begin{tcolorbox}[title={2/4. -Q-}]
      Mely alábbi halmaz(ok) alkot(nak) alteret a $2x2$-es valós mátrixok halmazában mint $R$ fölötti vektortérben:\\
      \mmedskip
      
      (A) $\{A \in \mathbb{R}^{2 x 2}|detA = 0\}$\\
      (B) $\{A \in \mathbb{R}^{2 x 2}|detA 6= 0\}$\\
      (C) $\{A \in \mathbb{R}^{2 x 2}|A = AT\}$\\
      (D) $\{A \in \mathbb{R}^{2 x 2}|A2 = I2\}$
  \tcblower
    Azt kell megvizsgálnunk, hogy a megadott halmazok zártak-e az összeadásra, ill. a skalárral való szorzásra:\\

    Az (A) kérdésben nulla determinánsú mátrixok összege könnyen lehet nem nulla determinánsú, s hasonlóképpen nem nulla determinánsú mátrixok összege lehet nulla determinánsú.\\
    
    A (B)-ben (de érvelhetünk azzal is, hogy a nullmátrix nem ilyen);\\

    A szimmetria megőrződik mátrixok összeadásánál, ill. skalárral való szorzásánál (ez mutatja, hogy a (C) feladatban alterünk van).\\

    Az pedig hogy a (D) feladatban nem kapunk alteret, kiderül pl. abból is, hogy a nullmátrix nem teljesíti az adott feltételt. 
    \mmedskip 
    
    Alter(ek): (C)
  \end{tcolorbox}
\end{frame}

\begin{frame}
  \begin{tcolorbox}[title={2/5. -Q-}]
      Mutassunk egy olyan $H$ részhalmazt $\mathbb{R}^2$-ben (azaz a síkon), mely zárt a vektorok szokásos összeadására, mégsem alkot alteret $\mathbb{R}^2$-ben.
  \tcblower
    A feladat arra mutat rá, hogy altereknél mindkét műveletre való zártságot meg kell követelnünk:\\
    
    erre érdemes már a fölkészülés során is példát keresnünk, hogy jobban megérthessük a fogalmat.
    \mmedskip 
    
    Pl. $\{[a b]T \in \mathbb{R}^2|a,b > 0\}$
  \end{tcolorbox}
\end{frame}


\begin{frame}[plain]
\begin{tcolorbox}[center, colback={myyellow}, coltext={black}, colframe={myyellow}]
    {\RHuge  (3) Bázis, lineáris függetlenség, generálás, dimenzió, rang kapcsolata}
    \mmedskip
\end{tcolorbox}
\end{frame}

\begin{frame}
  \begin{tcolorbox}[title={3/1. -N-}]
      Alkalmas együtthatók megadásával mutassuk meg, hogy a $\{v,0,w\}$ vektorok rendszere lineárisan összefüggő.
  \tcblower

    \mmedskip 
  
    $0 \cdot v + 1 \cdot 0 + 0 \cdot w = 0$.
  \end{tcolorbox}
\end{frame}


\begin{frame}
  \begin{tcolorbox}[title={3/2. -N-}]
       Alkalmas együtthatók megadásával mutassuk meg, hogy az $\{u,v,w,v+2w\}$ vektorok rendszere lineárisan összefüggő.
  \tcblower

    \mmedskip 
  
    $0 \cdot u + 1 \cdot v + 2 \cdot w + (-1)(v + 2w) = 0.$.
  \end{tcolorbox}
\end{frame}


\begin{frame}
  \begin{tcolorbox}[title={3/3. -N-}]
       Legyen $v1 =  \in \mathbb{R}^2$ és $v2 =  \in \mathbb{R}^2$. Mely valós $c$ szám(ok)ra lesz $v1$ és $v2$ lineárisan összefüggő?
  \tcblower

    \mmedskip 
  
    $c = 3$.
  \end{tcolorbox}
\end{frame}


\begin{frame}
  \begin{tcolorbox}[title={3/4. -R-}]
       Mely $c \in R$ valós számokra lesznek lineárisan függetlenek az $[1 0 c]T, a [2 2 2]T$ és a $[4 2 3]T$ vektorok?
  \tcblower

    \mmedskip 
  
    $c 6= 1/2$.
  \end{tcolorbox}
\end{frame}


\begin{frame}
  \begin{tcolorbox}[title={3/5. -R-}]
       Legyen $v1 = 1 1  \in \mathbb{R}^3$. Adjunk meg olyan $v2$ és $v3$ vektorokat $\mathbb{R}^3$-ban, melyekre igaz, hogy a $v1,v2,v3$ vektorrendszer lineárisan összefüggő, de közülük bármely két vektor lineárisan független.
  \tcblower

    \mmedskip 
  

  \end{tcolorbox}
\end{frame}


\begin{frame}
  \begin{tcolorbox}[title={3/6. -N-}]
        Egy $U \leq \mathbb{R}^2$ altérben $[2 3]T$ generátorrendszert alkot. Adjunk meg $U$-ban egy háromelemű generátorrendszert.
  \tcblower

    \mmedskip 
  
     Pl. $[2 3]T, [4 6]T, [0 0]T$
  \end{tcolorbox}
\end{frame}


\begin{frame}
  \begin{tcolorbox}[title={3/7. -R-}]
    Egy $U \leq R5$ altérben van olyan $\{v1,v2,v3\}$ háromelemű lineárisan összefüggő vektorhalmaz, mely generátorrendszer U-ban. Hány eleme lehet $U$ egy bázisának?
  \tcblower

    \mmedskip 
  
     Bázis elemszáma lehet: 1,2
  \end{tcolorbox}
\end{frame}


\begin{frame}
  \begin{tcolorbox}[title={3/8. -R-}]
    Ha egy $U \leq R9$ altérben van $5$ elemű lineárisan összefüggő generátorrendszer is, meg $3$ elemű lineárisan független vektorrendszer is, akkor mik $dimU$ lehetséges értékei?

  \tcblower

    \mmedskip 
  
     dimU lehet: 3 vagy 4
  \end{tcolorbox}
\end{frame}


\begin{frame}
  \begin{tcolorbox}[title={3/9. -N-}]
     Legyenek $\{v1,v2,v3\}$ lineárisan független vektorok. Adjuk meg a $\{v1 + v2,v2,2v1 + v2\}$ által generált altér egy bázisát.

  \tcblower

    \mmedskip 
  
     Pl. : $\{v1 + v2,v2\}$
  \end{tcolorbox}
\end{frame}


\begin{frame}
  \begin{tcolorbox}[title={3/10. -Q-}]
     Adott egy ötelemű vektorrendszer, melynek a rangja $3.$ Eltávolítunk a rendszerből két vektort. Mik az új rendszer rangjának lehetséges értékei?
  \tcblower

    \mmedskip 
  
     A rang lehet: 1, 2 vagy 3

  \end{tcolorbox}
\end{frame}


\begin{frame}
  \begin{tcolorbox}[title={3/11. -Q-}]
     Legyen $U = \{[x1,x2,x3]T \in \mathbb{R}^3|x1 = 2x2 = 3x3\}$ az $\mathbb{R}^3$ altere. Hány olyan bázisa van az $U$ altérnek, mely tartalmazza a $v = [6,3,2]T$ vektort?
  \tcblower

    \mmedskip 
    Bázisok száma: 1
  \end{tcolorbox}
\end{frame}



\begin{frame}
  \begin{tcolorbox}[title={3/11. -Q-}]
Az első két feladatban 

A lineáris összefüggőséget kell bizonyítanunk.\\
Ehhez olyan együtthatókat kell megadnunk, amelyek nem mindegyike nulla (és nem olyanokat, hogy egyik sem nulla). Ha tehát a rendszerben látunk néhány összefüggő vektort, akkor a többinek nyugodtan adhatunk nulla együtthatót.

A harmadik feladatban arra érdemes emlékeznünk, hogy két nem nulla vektor akkor és csak akkor lineárisan összefüggő, ha egymás skalárszorosai.

A negyedik feladatot megoldhatjuk a szokásos homogén lineáris egyenletrendszer vizsgálatával (ha a megoldások száma nagyobb, mint 1, akkor a vektorrendszer összefüggő). Másik lehetőség annak a tételnek az alkalmazása, hogy a megadott vektorok pontosan akkor lineárisan függetlenek, ha a belőlük mint oszlopvektorokból képzett mátrix determinánsa nem 0. Mivel

, ezért a válasz $c 6= 1/2$. 

Az ötödik feladatban segít, ha tudjuk, hogy mit jelent az összefüggőség a lineáris függés nyelvén: a feltételek azt jelentik, hogy a vektorok egyike sem skalárszorosa a másiknak, de az egyik vektor a másik kettőnek a lineáris kombinációja. 

A hatodik feladatban azt az állítást használjuk, hogy generátorrendszer kibővítve is generátorrendszer marad, de vigyáznunk kell, hogy az új vektorokat is az altérből vegyük. 

A hetedik kérdés megválaszolásához az a tény segít, hogy minden generátorrendszerből kiválasztható bázis, az azonban lineárisan független. Így a háromelemű halmazból legalább egyet el kell távolítanunk, hogy bázist kaphassunk. Mivel az altérben van legalább három vektor, ezért az nem lehet 0 dimenziós. 

A nyolcadik feladat gondolata ugyanez, de hozzá kell tennünk, hogy független rendszer elemszáma kisebb vagy egyenlő, mint az altér dimenziója. 

A kilencedik feladathoz jó arra emlékezni, hogy ha egy U altér egy generátorrendszeréből kiválasztunk olyan vektorokat, amelyek egyrészt függetlenek, másrészt U adott generátorai már kifejezhetők velük (vagyis ez egy maximális független rendszer U adott generátorrendszerében), akkor ez a kiválasztott független rendszer bázis lesz U-ban.

A tizedik feladathoz tudni kell, hogy a rang a generált altér dimenziója. Ha tehát egy háromdimenziós altér ötelemű generátorrendszeréből eltávolítunk két vektort, akkor a maradék vektorrendszer által generált altér még mindig legalább egydimenziós (hiszen még egy háromelemű független rendszerből is maximum kettőt távolíthattunk el), de maradhatott is háromdimenziós (ha pl. a két „fölösleges” vektort dobtuk ki.) 

Végezetül a 11. feladatban egy egydimenziós alterünk van, s ebben bármely nem nulla vektor egyúttal bázis is. Ha a tér dimenziója nagyobb lenne, akkor – hivatkozva arra a tételre, mely szerint független rendszer kiegészíthető bázissá – máris megváltozna a helyzet: ilyenkor bármely, a dimenziószámnál kisebb elemszámú vektorhalmaz végtelen sok módon egészíthető ki bázissá.

  \end{tcolorbox}
\end{frame}



\begin{frame}[plain]
\begin{tcolorbox}[center, colback={myyellow}, coltext={black}, colframe={myyellow}]
    {\RHuge  (4) Bázis megadása, dimenzió}
    \mmedskip
\end{tcolorbox}
\end{frame}

\begin{frame}
  \begin{tcolorbox}[title={4/1. -N-}]
      Álljon az $U$ altér az $\mathbb{R}^3$ azon $v$ vektoraiból, ahol a második és a harmadik komponens egyenlő. Adjunk meg egy bázist $U$-ban.
  \tcblower

    \mmedskip 
    
    Pl. $[1 0 0]T$, $[0 1 1]T$
  \end{tcolorbox}
\end{frame}


\begin{frame}
  \begin{tcolorbox}[title={4/2. -N-}]
      Legyen $\{b_1,b_2,b_3\}$ bázis egy $U$ altérben. Hány dimenziós alteret generál $\{b_1 + b_2,b_2,b_1 -b_2\}$?
  \tcblower

    \mmedskip 
    
    Az altér dimenziója: 2
  \end{tcolorbox}
\end{frame}


\begin{frame}
  \begin{tcolorbox}[title={4/3. -R-}]
      Hány dimenziós azon $\mathbb{R}^n$-beli vektoroknak az altere, melyekben a komponensek összege $0$?
  \tcblower

    \mmedskip 
    
    A dimenzió: $n-1$
  \end{tcolorbox}
\end{frame}


\begin{frame}
  \begin{tcolorbox}[title={4/4. -R-}]
      Hány dimenziós azon $\mathbb{R}^n$-beli vektoroknak az altere, melyekben a komponensek négyzetösszege $0$?
  \tcblower

    \mmedskip 
    
    A dimenzió: $0$
  \end{tcolorbox}
\end{frame}


\begin{frame}
  \begin{tcolorbox}[title={4/5. -Q-}]
      Hány dimenziós alteret alkotnak $\mathbb{R}^3x3$-ban azok az A mátrixok, amelyekre teljesül, hogy $AT = -A$?
  \tcblower

    \mmedskip 
    
    A dimenzió: $3$
  \end{tcolorbox}
\end{frame}


\begin{frame}
  \begin{tcolorbox}[title={4/6. -R-}]
      Hány négydimenziós altér van $\mathbb{R}^4$-ben mint $R$ fölötti vektortérben?
  \tcblower

    \mmedskip 
    
    Alterek száma: $1$
  \end{tcolorbox}
\end{frame}


\begin{frame}
  \begin{tcolorbox}[title={4/7. -Q-}]
      Hány olyan altér van $\mathbb{R}^{2 x 2}$-ben mint $R$ fölötti vektortérben, mely tartalmazza az összes $2 x 2$-es szimmetrikus mátrixot?
  \tcblower

    \mmedskip 
    
    Alterek száma: $2$
  \end{tcolorbox}
\end{frame}


\begin{frame}
  \begin{tcolorbox}[title={4/7. -Q-}]
Az első feladatban föl kell írni az altér egy általános elemét és lineáris kombinációra bontani. A második feladat kapcsolódik a rang fogalmához is: a generált altér dimenziójának kiszámításához maximális független rendszert kell keresni a generátorok között (azaz olyat, amiből a többi generátor már kifejezhető). Az adott példában $\{b_1 + b_2,b_2\}$ ilyen. De bázist alkot ebben az altérben $\{b_1,b_2\}$ is, hiszen $b_1 = (b_1 + b_2) - b_2$ is eleme az altérnek. A harmadik feladatra (ez a 4. gyakorlat 7/g feladatának $y = 0$ speciális esete) gondolhatunk úgy, hogy egy homogén lineáris összefüggésünk van: az, hogy az elemek összege nulla, és ez eggyel csökkenti $\mathbb{R}^n$ dimenzióját. (Valójában egy homogén lineáris egyenletrendszer szabad változóit számoltuk meg. Bázist alkotnak azok a vektorok, melyekben az első $n-1$ komponens egyike 1, az utolsó komponens-1, a többi komponens pedig 0, de ezt megadni Q szintű feladat lenne az általános n miatt. Próbálkozhatunk a triviális bázis elemeinek „utánzásával”, de ellenőrizni kell a függetlenséget.) A negyedik feladat mutatja, mennyire gondosnak kell lennünk az előző feladat megoldási elvének alkalmazásakor. Ha ugyanis a komponensek közötti összefüggés nem lineáris, akkor egész más lehet az eredmény: itt csak a nullvektor van benne az altérben. Sőt, nem lináris összefüggéssel megadott vektorok általában nem is alkotnak alteret. Az ötödik feladat is hasonló jellegű, csak itt sokkal több homogén lineáris összefüggés van a komponensek között. Az ilyen mátrixokban a főátló minden eleme 0, a főátlóra szimmetrikus elemek pedig egymás ellentettjei. Ezért pl. a főátló fölötti elemeket szabad változónak tekintve a mátrix elemei már egyértelműen meg vannak határozva, ezért a dimenzió a szabadon választható mátrixelemek száma, azaz 3. (Itt is érdemes gyakorlásul egy bázist fölírni.) A hatodik feladatban azt kell tudni, hogy $\mathbb{R}^n$ egy valódi alterének dimenziója határozottan kisebb, mint $\mathbb{R}^n$-é. A hetedik feladat is hasonló jellegű, először azt kell észrevenni, hogy a $2x2$-es szimmetrikus mátrixok terének a dimenziója mindössze 1-gyel kisebb, mint a $\mathbb{R}^{2 x 2}$ téré, így a szimmetrikus mátrixok altere és az egész tér között nincs más altér.
  \end{tcolorbox}
\end{frame}



\begin{frame}[plain]
\begin{tcolorbox}[center, colback={myyellow}, coltext={black}, colframe={myyellow}]
    {\RHuge  (5) Lineáris egyenletrendszerek, megoldásszámuk}
    \mmedskip
\end{tcolorbox}
\end{frame}

\begin{frame}
  \begin{tcolorbox}[title={5/1. -N-}]
      Adjunk meg egy olyan lineáris egyenletrendszert, melyben két egyenlet van, három ismeretlen, és az egyenletrendszernek nincs megoldása.
  \tcblower

    \mmedskip 
    
    Pl.:
$x + y + z = 2$
$x + y + z = 3$
  \end{tcolorbox}
\end{frame}


\begin{frame}
  \begin{tcolorbox}[title={5/2. -N-}]
      Adjunk meg egy olyan inhomogén lineáris egyenletrendszert, melyben három egyenlet van, két ismeretlen, és az egyenletrendszernek végtelen sok megoldása van.
  \tcblower

    \mmedskip 
    
    Pl.:
$x + y = 1$
$2x + 2y = 2$
$3x + 3y = 3$

  \end{tcolorbox}
\end{frame}


\begin{frame}
  \begin{tcolorbox}[title={5/3. -N-}]
      Mi lehet a megoldások száma egy olyan valós együtthatós lineáris egyenletrendszernél, melyben az egyenletek száma 3, az ismeretlenek száma pedig 5?
  \tcblower

    \mmedskip 
    
    0 vagy végtelen
  \end{tcolorbox}
\end{frame}


\begin{frame}
  \begin{tcolorbox}[title={5/3. -N-}]
Lineáris egyenletrendszerek megoldásszámára egy nagyon fontos összefüggés van: ha több ismeretlen van, mint egyenlet, akkor nem lehet egyértelmű a megoldás (erre kérdez rá a harmadik feladat). Vagyis ha az egyenletrendszerben nincs elég egyenlet – nem tudunk eleget az ismeretlenekről –, akkor ugyan lehet ellentmondásos (ehhez már két egyenlet is elég, mint az első feladatban, sőt az egyetlen $0 \cdot x + 0 \cdot y = 1$ egyenlet is), de ha nem ellentmondásos, azaz van megoldás, akkor biztosan egynél több megoldás van. Valós és komplex együtthatós egyenletrendszerek esetében ilyenkor a megoldások száma végtelen. Az egyenletek számát szaporíthatjuk úgy, hogy a megoldások halmaza ne változzon: ehhez a rendszerbe már bent lévő egyenletek lineáris kombinációját kell bevenni (mint a második feladatban). Fontos, hogy homogén lineáris egyenletrendszernek – amikor az egyenletek jobb oldalán mindenütt 0 áll – mindig van megoldása: a triviális megoldás, amikor minden ismeretlen értéke nulla.
  \end{tcolorbox}
\end{frame}



\begin{frame}[plain]
\begin{tcolorbox}[center, colback={myyellow}, coltext={black}, colframe={myyellow}]
    {\RHuge  (6) Mátrixműveletek, inverz }
    \mmedskip
\end{tcolorbox}
\end{frame}

\begin{frame}
  \begin{tcolorbox}[title={6/1. -N-}]
    Legyen $A \in Rkxl$, $B \in Rpxr$ és $C \in Rtxu$. Adjunk szükséges és elégséges feltételt arra, hogy értelmes legyen az $ABT + C$ mátrixkifejezés.
  \tcblower

    \mmedskip 
    
    $l = r, k = t, p = u$
  \end{tcolorbox}
\end{frame}


\begin{frame}
  \begin{tcolorbox}[title={6/2. -N-}]
    Legyen A =. Adjunk meg egy olyan $2x2$-es, egész elemű, a nullmátrixtól különböző $B$ mátrixot, amelyre igaz, hogy $AB = 0$.

  \tcblower

    \mmedskip 
    
    Pl. B =
  \end{tcolorbox}
\end{frame}


\begin{frame}
  \begin{tcolorbox}[title={6/3. -N-}]
     Legyen $A = .$ Mely $c \in R$ számokra van olyan $2x2$-es, valós elemű, a nullmátrixtól különböző $B$ mátrix, amelyre igaz, hogy $AB = 0$?
  \tcblower

    \mmedskip 
    
    $c = 9$
  \end{tcolorbox}
\end{frame}


\begin{frame}
  \begin{tcolorbox}[title={6/4. -N-}]
     Adjunk meg egy olyan $2x2$-es egész elemű, a nullmátrixtól különböző A mátrixot, amelyre igaz, hogy van olyan 0 6= $B \in \mathbb{R}^{2 x 2}$, melyre $AB = 0$.
  \tcblower

    \mmedskip 
    
    Pl. A =
  \end{tcolorbox}
\end{frame}


\begin{frame}
  \begin{tcolorbox}[title={6/5. -N-}]
     Adjuk meg az A = mátrix inverzét.
  \tcblower

    \mmedskip 
    
    
  \end{tcolorbox}
\end{frame}


\begin{frame}
  \begin{tcolorbox}[title={6/6. -N-}]
     Adjuk meg az $A = [1 2 3]$ sormátrix egy jobb oldali inverzét.
  \tcblower

    \mmedskip 
    
     Pl.: $[1 0 0]T$
  \end{tcolorbox}
\end{frame}


\begin{frame}
  \begin{tcolorbox}[title={6/7. -N-}]
     Hány olyan jobb oldali inverze van az $A = [1 1 1]$ sormátrixnak, melyben minden elem egyenlő?
  \tcblower

    \mmedskip 
    
     Számuk: 1
  \end{tcolorbox}
\end{frame}


\begin{frame}
  \begin{tcolorbox}[title={6/8. -Q-}]
     Az alábbi mátrixok közül melyeknek nem lehet jobb oldali inverze a valós paraméterek semmilyen választására sem (azonos betűk azonos számokat jelölnek)? 
     

  \tcblower

    \mmedskip 
    
     (A), (C)
  \end{tcolorbox}
\end{frame}


\begin{frame}
  \begin{tcolorbox}[title={6/8. -Q-}]
Az első feladat arra kérdez rá, hogy mikor végezhető el a mátrixok összeadása, szorzása, illetve mik a transzponált mátrix méretei. Két mátrix szorzatának oszlopaiban a bal oldali mátrix oszlopainak lineáris kombinációi állnak. Ezért az $AB = 0$ akkor tud teljesülni nem nulla B mátrixszal, ha A oszlopai lineárisan öszefüggenek. Speciálisan ha A négyzetes mátrix, akkor ez azzal ekvivalens, hogy A determinánsa nulla. Ezzel a tétellel könnyű megválaszolni a harmadik feladatot, illetve keresni megfelelő mátrixot a negyedik feladatban. Kétszer kettes mátrix esetében érdemes megjegyezni az inverz képletét: inverz akkor létezik, ha a mátrix d determinánsa nem nulla, és ekkor az inverz mátrixot úgy kapjuk, hogy az eredeti mátrixban a főátló két elemét kicseréljük, a mellékátló elemeit ellentettjükre változtatjuk, végül a mátrix minden elemét elosztjuk d-vel (így kaptuk az ötödik feladat eredményét). Nem feltétlenül négyzetes mátrixra is érvényes, hogy pontosan akkor van jobb oldali inverze, ha rangja megegyezik a sorainak a számával, azaz ha a sorai lineárisan függetlenek. Erre a tételre a hatodik és a hetedik feladat megoldásában nincs szükség, mert azok megoldása közvetlen számolás (a hetedik feladatban az egyetlen megfelelő vektor $[1/3 1/3 1/3]T$). A nyolcadik feladatban viszont ezt a tételt alkalmazzuk. Az (A) rész két sora nyilván összefügg, a (C) esetében ez azért igaz, mert $\mathbb{R}^2$ dimenziója 2, és így bármely három vektor összefüggő. A (B)-ben például az a = d = 1 és $b = c = 0$ két független vektort ad, a (D) esetben meg a három sor független lesz ha a $6= 9$, hiszen ekkor a determináns értéke nem nulla.
  \end{tcolorbox}
\end{frame}


\begin{frame}[plain]
\begin{tcolorbox}[center, colback={myyellow}, coltext={black}, colframe={myyellow}]
    {\RHuge  (7) Permutációk, inverziók, determinánsok}
    \mmedskip
\end{tcolorbox}
\end{frame}

\begin{frame}
  \begin{tcolorbox}[title={7/1. -N-}]
    Az A = $[aij] \in R5x5$ mátrix determinánsának kiszámolásakor mennyi lesz az $a_31a_24a_53a_15a_42$ szorzathoz tartozó permutációban az inverziók száma?

  \tcblower

    \mmedskip 
    
    7
  \end{tcolorbox}
\end{frame}


\begin{frame}
  \begin{tcolorbox}[title={7/2. -R-}]
    Az $A = [aij] \in R5x5$ mátrix determinánsának kiszámolásakor az $a_11a_2ia_33a_45a_5j$ szorzatot $(-1)$-gyel szorozva kellett figyelembe venni. Határozzuk meg i-t és j-t.
  \tcblower

    \mmedskip 
    
    $i = j =
2 4$
  \end{tcolorbox}
\end{frame}


\begin{frame}
  \begin{tcolorbox}[title={7/3. -Q-}]
    Az $\{1,2,3,4,5\}$ számoknak hány olyan sorbarendezése (permutációja) van, melyben az inverziók száma 1?
  \tcblower

    \mmedskip 
    
     Ilyenek száma: 4
  \end{tcolorbox}
\end{frame}


\begin{frame}
  \begin{tcolorbox}[title={7/4. -Q-}]
    Az $\{1,2,3,4,5\}$ számok $i2jkl$ típusú sorbarendezései (azaz permutációi) között mennyi lehet az inverziók maximális száma?
  \tcblower

    \mmedskip 
    Max. inverziószám: 8
  \end{tcolorbox}
\end{frame}

\begin{frame}
  \begin{tcolorbox}[title={7/5. -N-}]
    Legyenek $A,B,C \in \mathbb{R}^3x3$ olyan valós mátrixok, melyekre $detA = 2, detB = 3$ és $detC = 4$. Mennyi lesz $2A2C-1B$ determinánsa?
  \tcblower

    \mmedskip 
    
     $23  \cdot 22  \cdot (1/4) \cdot 3 = 24$
  \end{tcolorbox}
\end{frame}

\begin{frame}
  \begin{tcolorbox}[title={7/6. -R-}]
    Legyen A =
, és
tegyük föl, hogy detA = 5. Mennyi lesz detB?

  \tcblower

    \mmedskip 
    
     detB = -5
  \end{tcolorbox}
\end{frame}


\begin{frame}
  \begin{tcolorbox}[title={7/6. -R-}]
    A determináns definíciójában a tagok előjelezését a következőképpen kell kiszámolni. A szorzatot rendezzük át úgy, hogy az aij-k első indexei növekedjenek 1-től n-ig. Ezek után a második indexek által alkotott permutációban számoljuk meg az inverziókat (vagyis azokat a párokat, amelyekben az elöl szereplő szám a nagyobb). Ha ezek száma páros, akkor az előjel +, különben -. Az első feladatban a kapott sorrend 54123. Ha n elemet permutálunk, akkor az inverziók maximális száman 2; ezt a számot az elemek monoton fogyó sorbarendezése produkálja, 0 inverziója pedig csak a természetes, monoton növő sorbarendezésnek van. Két elemet megcserélve, a permutáció paritása (előjele) mindig megváltozik. A második példában i és j csak 2 és 4 lehet valamelyik sorrendben, ezt a két permutációt kell kipróbálni. A harmadik feladatban csak két szomszédos elemet cserélhetünk meg az alapsorrendhez képest. A negyedik feladatban az az ötlet, hogy a 2 után legalább két, 2-nél nagyobb számnak kell lennie, ez két nem-inverziót elront, és így maximum 8 inverzió lehet. Az 52431 sorrend ezt elő is állítja. Az ötödik feladat a determinánsok szorzástételének egyszerű alkalmazása (amiből következik, hogy egy mátrix inverzének a determinánsa az eredeti mátrix determinánsának a reciproka). A hatodik feladatban az eredeti mátrixot transzponáltuk, megcseréltük az első két oszlopot, majd az új első oszlopot adtuk a harmadikhoz. A középső lépésnél a determináns előjelet vált, a másik kettőnél nem változik.

  \end{tcolorbox}
\end{frame}


\begin{frame}[plain]
\begin{tcolorbox}[center, colback={myyellow}, coltext={black}, colframe={myyellow}]
    {\RHuge  (8) Jobb oldali sajátértékek és sajátvektorok, diagonalizálhatóság}
    \mmedskip
\end{tcolorbox}
\end{frame}

\begin{frame}
  \begin{tcolorbox}[title={8/1. -N-}]
    Mik az A = $\mathbb{R}^3x3$ mátrix sajátértékei? 

  \tcblower

    \mmedskip 
    
    1, 4, 6
  \end{tcolorbox}
\end{frame}


\begin{frame}
  \begin{tcolorbox}[title={8/2. -N-}]
     Írjunk föl egy olyan mátrixot, amelynek karakterisztikus polinomja $(1-{\lambda})(2-{\lambda})(3-{\lambda})$.

  \tcblower

    \mmedskip 
    
     Pl. A =
  \end{tcolorbox}
\end{frame}


\begin{frame}
  \begin{tcolorbox}[title={8/3. -R-}]
    Írjunk föl egy olyan nem diagonalizálható mátrixot, melynek karakterisztikus polinomja ${\lambda}2$.

  \tcblower

    \mmedskip 
    
    Pl. $A =$
  \end{tcolorbox}
\end{frame}


\begin{frame}
  \begin{tcolorbox}[title={8/4. -Q-}]
    Az $A =  \in \mathbb{R}^{2 x 2}$ mátrix nem diagonalizálható $\mathbb{R}$ fölött. Mik $c$ lehetséges értékei?


  \tcblower

    \mmedskip 
    
    $c \leq 0$
  \end{tcolorbox}
\end{frame}


\begin{frame}
  \begin{tcolorbox}[title={8/5. -Q-}]
    Milyen $c \in \mathbb{R}$ valós értékekre lesz az 1 2 0 $c \in \mathbb{R}^{2 x 2}$ mátrix diagonalizálható?

  \tcblower

    \mmedskip 
    
     $c \neq 1$
  \end{tcolorbox}
\end{frame}


\begin{frame}
  \begin{tcolorbox}[title={8/6. -Q-}]
     Az $A =a b c d \in \mathbb{R}^{2 x 2}$ mátrixra . Mennyi az $a + 1 b c d + 1$ mátrix determinánsa? $det a + 1 b c$ 

  \tcblower

    \mmedskip 
    
     det
  \end{tcolorbox}
\end{frame}


\begin{frame}
  \begin{tcolorbox}[title={8/6. -Q-}]
     Egy mátrix valós sajátértékeit a karakterisztikus polinom valós gyökei adják: a főátló minden eleméből levonunk ${\lambda}$-t és kiszámítjuk a determinánst. Felső háromszögmátrix, speciálisan diagonális mátrix esetén a sajátértékek a főátlóból leolvashatók. A sajátvektorokat minden egyes sajátértékhez egy-egy lineáris egyenletrendszer megoldásával kaphatjuk meg. A diagonalizálhatóság feltétele az, hogy legyen „elegendő” sajátvektor, azaz a tér dimenziószámával megegyező mennyiségű lineárisan független sajátvektor. A harmadik feladat mátrixa esetében ez nem teljesül, mert a sajátvektorokr 0T alakúak, ahol r $6= 0$ (azaz a 0-hoz tartozó sajátaltér egydimenziós). Gyakran használt elégséges feltétel, hogy egy nxn-es mátrix diagonalizálható $\mathbb{R}$ fölött, ha n különböző valós sajátértéke van. A negyedik feladat mátrixának karakterisztikus polinomja ${\lambda}2$-c. Ennek c > 0-ra két különböző valós gyöke van, tehát ilyenkor a mátrix diagonalizálható; $c < 0$-ra nincs valós gyök, ezért $\mathbb{R}$ fölött nem diagonalizálhatjuk a mátrixot; végül c = 0-ra a megoldást a harmadik feladat adja. Hasonló érvelés adható az ötödik feladatnál is. A hatodik feladat feltétele azt mutatja, hogy a -1 sajátértéke a mátrixnak, ha tehát -1-et levonunk a főátló minden eleméből, épp az $A-(-1)I2$ mátrixot kapjuk, aminek a determinánsa 0, hiszen -1 gyöke a karakterisztikus polinomnak.

  \end{tcolorbox}
\end{frame}



\begin{frame}[plain]
\begin{tcolorbox}[center, colback={myyellow}, coltext={black}, colframe={myyellow}]
    {\RHuge  (9) Lineáris transzformációk mátrixa, sajátértékei, kép- és magtere; báziscsere }
    \mmedskip
\end{tcolorbox}
\end{frame}

\begin{frame}
  \begin{tcolorbox}[title={9/1. -N-}]
    Mi a kétdimenziós valós sík, azaz $\mathbb{R}^2$ origó körüli $+90$ fokos forgatásának mátrixa az i és j bázisban?

  \tcblower

    \mmedskip 
    
    $[{\varphi}]i,j;i,j = 0 -1 1 0$
  \end{tcolorbox}
\end{frame}


\begin{frame}
  \begin{tcolorbox}[title={9/2. -R-}]
    Mi a háromdimenziós valós tér, azaz $\mathbb{R}^3$ x-y-síkra való tükrözésének a mátrixa az $j,k,i$ bázisban? (Itt a bázisvektorok sorrendje a megszokottól eltérő.)

  \tcblower

    \mmedskip 
    
     $[{\varphi}]j,k,i;j,k,i =$
  \end{tcolorbox}
\end{frame}


\begin{frame}
  \begin{tcolorbox}[title={9/3. -N-}]
    Mik a kétdimenziós valós sík, azaz $\mathbb{R}^2$ origóra való tükrözésének sajátértékei?

  \tcblower

    \mmedskip 
    
    -1
  \end{tcolorbox}
\end{frame}


\begin{frame}
  \begin{tcolorbox}[title={9/4. -R-}]
    Az $\mathbb{R}^2 egy {\varphi}$ lineáris transzformációjára ${\varphi}(i) = 2i és {\varphi}(j) = i + 3j$. Mik ${\varphi}$ sajátértékei? 
    
  \tcblower

    \mmedskip 
    
     2 és 3
  \end{tcolorbox}
\end{frame}


\begin{frame}
  \begin{tcolorbox}[title={9/4. -R-}]
    Minden lineáris transzformációhoz rögzített bázis esetén egyértelműen tartozik egy négyzetes mátrix. Ezt úgy írhatjuk föl, hogy a bázisvektorokat leképezzük a lineáris transzformációval, e képvektorokat felírjuk a megadott bázisban, és az így kapott koordinátavektorokat írjuk a mátrix oszlopaiba. Ha változtatunk a bázison, akkor változhat a mátrix is; ez kiszámítható a bázistranszformáció képletéből $(A' = S-1AS)$. Az ötödik feladatban a képletre nincs szükség: az első bázisvektor, $b_1$ kétszeresének a képe szintén önmaga, azaz $2b_1$, s így az új mátrix első oszlopában az első koordináta $1$, a többi meg $0$. Ugyanakkor a $b_2$ és $b_3$ képében fele annyit kell „vennünk” $2b_1$-ből, mint $b_1$-ből, tehát az első sor második és harmadik eleme a felére csökken. Egy lináris transzformáció sajátértékeit kiszámolhatjuk úgy, hogy felírjuk a mátrixát egy alkalmas bázisban, és ennek vesszük a sajátértékeit. Ez a negyedik (kétlépcsős) feladat: a mátrix a triviális bázisban  $2 1 0 3$ . Néha lehetséges közvetlenül a definíció segítségével is boldogulni: a harmadik feladatban minden vektor az ellentettjébe megy, ezért az egyetlen sajátérték a $-1$. A hatodik és a hetedik feladatban felírhatjuk, hogy ${\varphi}({\lambda}e1 + {\mu}e2) = {\mu}(e1 + e2)$. Ez akkor nulla, azaz ${\lambda}e1 + {\mu}e2$ akkor van a magtérben, ha ${\mu} = 0$. A képtér elemei pedig $e1 + e2$ többszörösei. A nyolcadik feladatban azt is lehet használni, hogy $[{\varphi}(v)]B = [{\varphi}]B[v]B$. Az $M[x y z]T = 0$ egyenlet megoldása: $x$ és $y$ tetszőleges, $z = 0$. A magtér tehát kétdimenziós. Megoldható a feladat a dimenzióösszefüggés segítségével is, amely szerint a magtér dimenziója a mátrix oszlopainak száma mínusz a mátrix rangja. Harmadik megoldásként láthatjuk, hogy két bázisvektor a nullába megy, ezért a magtér legalább kétdimenziós, de háromdimenziós nem lehet, mert a mátrix nem nulla.

  \end{tcolorbox}
\end{frame}


\begin{frame}[plain]
\begin{tcolorbox}[center, colback={myyellow}, coltext={black}, colframe={myyellow}]
    {\RHuge  (10) Skaláris szorzat, vektorok szöge }
    \mmedskip
\end{tcolorbox}
\end{frame}

\begin{frame}
  \begin{tcolorbox}[title={10/1. -N-}]
    Mennyi az$[1 1 1 1]T$ és az $[1 -1 -1 -1]T$ vektorok szöge $\mathbb{R}^4$-ben, a szokásos ha, $bi = aTb$ skaláris szorzatra nézve?

  \tcblower

    \mmedskip 
    
    Szögük: $120^{\circ}$

  \end{tcolorbox}
\end{frame}


\begin{frame}
  \begin{tcolorbox}[title={10/2. -R-}]
    Az $[1 1 + i i]T \in C3$ vektor ha, $bi = b*a$ mellett merőleges az $[1 1 + i c]T$ vektorra. Határozzuk meg $ \in C$ értékét.


  \tcblower

    \mmedskip 
    
    c = -3i
  \end{tcolorbox}
\end{frame}


\begin{frame}
  \begin{tcolorbox}[title={10/3. -R-}]
    Álljon $W az \mathbb{R}^3$ azon vektoraiból, amelyek merőlegesek $a[0 1 1]T$ és $[1 1 0]T$ vektorok mindegyikére. Adjunk meg egy bázist $W$-ben.

  \tcblower

    \mmedskip 
    
    $\{[1 -1 1]T\}$
  \end{tcolorbox}
\end{frame}


\begin{frame}
  \begin{tcolorbox}[title={10/4. -Q-}]
    Mely $a,b \in \mathbb{R}$ értékekre lesz $|3a + 4b| = 5a_2 + b_2$

  \tcblower

    \mmedskip 
    
    $[a,b] = {\lambda}[3,4], {\lambda} \in \mathbb{R}$
  \end{tcolorbox}
\end{frame}


\begin{frame}
  \begin{tcolorbox}[title={10/4. -Q-}]
    A szokásos skaláris szorzatot $\mathbb{R}^n$-ben az ha, $bi = aTb$ összefüggés adja meg; két vektor merőlegessége azt jelenti, hogy a skaláris szorzatuk nulla. $\mathbb{R}^n$-ben a vektorok szögét a $cos{\gamma} =$ ha,$bi kak \cdot kbk$ összefüggéssel definiáljuk, erről szól az első feladat. (A megoldáshoz érdemes átismételni a szögfüggvények értékét a nevezetes szögeken.) A második feladat esetében a skaláris szorzat kiszámításánál vigyázzunk arra, hogy a második tényező komponenseit meg kell konjugálni. A harmadik feladatnál lineáris egyenletrendszert kapunk W elemeinek komponenseire. A negyedik feladat esetében arra kell ráismerni, hogy a feladatbeli egyenlet az u = [3 4]T és a $v = [a b]T$ vektorokra írja föl az $|hu,vi| = kuk \cdot kvk$ összefüggést, ami a Cauchy-egyenlőtlenség alapján akkor és csak akkor teljesülhet, ha $u$ és $v$ párhuzamosak.
  \end{tcolorbox}
\end{frame}


\begin{frame}[plain]
\begin{tcolorbox}[center, colback={myyellow}, coltext={black}, colframe={myyellow}]
    {\RHuge  (11) Geometriai vektorok: vektoriális és vegyes szorzat }
    \mmedskip
\end{tcolorbox}
\end{frame}

\begin{frame}
  \begin{tcolorbox}[title={11/1. -N-}]
    Ha [a]i,j,k = [0 1 1]T és a [b]i,j,k = [1 1 0]T, akkor számítsuk ki az a és b vektorok vektoriális szorzatát (koordinátákkal).

  \tcblower

    \mmedskip 
    
    [-1 1 -1]T
  \end{tcolorbox}
\end{frame}


\begin{frame}
  \begin{tcolorbox}[title={11/1. -N-}]
    A vektoriális szorzat koordinátáinak kiszámításához érdemes a „determinásos” képletet használni: egy 3 x 3-as determináns első sorába az i, j, k vektorokat írjuk, a másik két sorba pedig a megadott vektorok koordinátáit, majd a determinánst (formálisan) kifejtjük az első sora szerint. Ekkor az i, j, k vektorok egy lineáris kombinációját kapjuk: ez lesz a vektoriális szorzat és így a leggyorsabb megoldani az első feladatot. Ebből a képletből láthatjuk azt is, hogy a vegyes szorzat is egy olyan determináns értéke, amelynek soraiban a három megadott vektor van alkalmas sorrendben. Speciálisan három vektor akkor és csak akkor lineárisan összefüggő, azaz akkor és csak akkor vannak egy síkban, ha a vegyes szorzatuk nulla. Ez kapcsolódik a második és a harmadik feladat (B) részéhez is. Az utolsó három feladatban érdemes a derékszögű koordinátarendszerben a három tengelyirányú egységvektorra, i-re, j-re és k-ra nézni a vektoriális szorzat hatását, ezekkel érdemes kísérletezni az utolsó három feladatban. A második feladat (A) részében könnyű látni, hogy ha a és b nem esnek egy irányba, akkor a x b merőleges lesz az általuk meghatározott síkra, s ekkor c lehet akár a-val is egyenlő, az (a x b) x c vektoriális szorzat nem lesz a nullvektor. A (C) részben ha a és b vagy b és c párhuzamosak, akkor az ő vektoriális szorzatuk 0, és így 0 lesz a teljes vektoriális szorzat is. Ha viszont egyik vektorpár sem párhuzamos, akkor mindkét esetben ugyanazt a síkot feszítik ki (hiszen a,b és c lineárisan összefüggők), és így mindkét vektorpár vektoriális szorzata a síkra merőleges lesz, vagyis ezek párhuzamos vektorok, vektoriális szorzatuk tehát biztosan 0. A (D) esetben pedig pl. $a = c = i$, $b = j$ esetén a szorzat értéke k és -k skaláris szorzata lesz, tehát nem 0. Hasonló okoskodással lehet kielemezni az utolsó két feladatban szereplő állítások igazságértékét is.
  \end{tcolorbox}
\end{frame}


\begin{frame}[plain]
\begin{tcolorbox}[center, colback={myyellow}, coltext={black}, colframe={myyellow}]
    {\RHuge  (12) Kvadratikus alakok jellege (definitsége)}
    \mmedskip
\end{tcolorbox}
\end{frame}

\begin{frame}
  \begin{tcolorbox}[title={12/1. -N-}]
     Számítsuk ki az 1 2 2 1 mátrix által meghatározott kvadratikus alak értékét az[1 2]T vektoron.

  \tcblower

    \mmedskip 
    
    13
  \end{tcolorbox}
\end{frame}


\begin{frame}
  \begin{tcolorbox}[title={12/2. -R-}]
     Adjunk meg egy $v \in \mathbb{R}^2$ vektort, melyen az $A =  1 2 2 1$  mátrix által meghatározott kvadratikus alak negatív értéket vesz föl.


  \tcblower

    \mmedskip 
    
    Pl. $v =   1 -1 $
  \end{tcolorbox}
\end{frame}


\begin{frame}
  \begin{tcolorbox}[title={12/3. -R-}]
     Adjuk meg a   $in \mathbb{R}^3x$3 mátrix által meghatározott bilineáris függvény kvadratikus karakterét (definitségét).


  \tcblower

    \mmedskip 
    
    Indefinit.
  \end{tcolorbox}
\end{frame}


\begin{frame}
  \begin{tcolorbox}[title={12/4. -N-}]
     Milyen $c \in \mathbb{R}$ valós értékekre lesz az $1 c c c \in \mathbb{R}^{2 x 2}$ mátrix által meghatározott kvadratikus alak pozitív definit?

  \tcblower

    \mmedskip 
    
    $0 < c < 1$
  \end{tcolorbox}
\end{frame}


\begin{frame}
  \begin{tcolorbox}[title={12/4. -N-}]
     Egy szimmetrikus, valós A mátrixhoz tartozó kvadratikus alak $Q(u) = uTAu$. Abban az esetben ha $A =  a b b d$  és $u = [x y]T$, akkor $Q(u) = ax2 + 2bxy + dy2$. A második feladatban tehátolyan $x$ és $y$ számokat kell keresnünk, amelyekre $x2 + 4xy + y2$ negatív. Ha észrevesszük, hogy $x2 + 4xy + y2 = x2 + 4xy + 4y2 -3y2 = (x + 2y)2 -3y2$, akkor már könnyen találunk ilyen értékeket (de az alábbiak szerint a mátrix egyik sajátvektora is megfelelő). A kvadratikus alak jellege (karaktere, definitsége) azt mondja meg, milyen valós értékeket vesz föl a nem nulla vektorokon. Ezt a tulajdonságot le tudjuk olvasni a mátrix sajátértékeiből (amik szimmetrikus mátrix esetén mindig valósak), pontosabban azok előjeléből: $Q(u), u 6= 0$ sajátértékek előjele: jelleg: mindig pozitív mind pozitív pozitív definit mindig negatív mind negatív negatív definit mindig nemnegatív mind nemnegatív pozitív szemidefinit mindig nempozitív mind nempozitív negatív szemidefinit van pozitív is, negatív is van pozitív is, negatív is indefinit Ha $u = ei$ (a triviális bázis i-edik vektora), akkor $Q(u)$ az A mátrix főátlójának i-edik eleme. A harmadik feladatban a második és a harmadik diagonális elem pozitív, illetve negatív, s így kvadratikus alak pozitív és negatív értékeket is fölvesz, tehát biztosan indefinit. Bizonyos esetekben a következő kritérium segítségével is leolvasható a kvadratikus alak jellege. Legyen ${\Delta}0 = 1$ és ${\Delta}k$ a mátrix bal fölső sarkában lévő $k xk$-as részmátrix determinánsa (ez az ún. karakterisztikus sorozat). Pl. a karakterisztikus sorozat pontosan akkor áll csupa pozitív értékből, ha a kvadratikus alakunk pozitív definit. A negyedik feladatnál ezt a feltételt használjuk: az 1x1-es részmátrixhoz tartozó tag pozitív, s a pozitív definitség azzal lesz ekvivalens, hogy a determináns, $c-c2 = c(1-c)$ is pozitív. Ez csak a jelzett intervallumban valósulhat meg.
  \end{tcolorbox}
\end{frame}


\begin{frame}[plain]
\begin{tcolorbox}[center, colback={myyellow}, coltext={black}, colframe={myyellow}]
    {\RHuge  B rész}
    \mmedskip
\end{tcolorbox}
\end{frame}


\begin{frame}
  \begin{tcolorbox}[title={12/4. -N-}]
     Az alábbi listában azok a definíciók és állítások, tételek szerepelnek, melyeket a vizsgadolgozat B részében kérdezhetünk. A válaszoknál zárójelben néhol magyarázó megjegyzések is vannak, ezeket nem kell leírni a teljes pontszám eléréséhez.
  \end{tcolorbox}
\end{frame}

\begin{frame}
  \begin{tcolorbox}[title={1}]
      Mit jelent az, hogy egy $W {\subseteq} \mathbb{R}^n$ részhalmaz altér?

  \tcblower
$W {\subseteq} \mathbb{R}^n altér \mathbb{R}^n-ben$, ha: 1) $W$ nem üres; 2) $a,b \in W$ esetén $a + b \in W$ (azaz W zárt az összeadásra); 3) $a \in W és {\lambda} \in \mathbb{R}$ esetén ${\lambda}a \in W$ (azaz $W$ zárt a skalárral szorzásra). (Ahelyett, hogy $W$ nem üres, azt is írhatjuk, hogy $0 \in W$.)
  \end{tcolorbox}
\end{frame}


\begin{frame}
  \begin{tcolorbox}[title={2}]
      Definiáljuk, mit jelent az, hogy a $v1,...,vk \in \mathbb{R}^n$ vektorrendszer lineárisan független.

  \tcblower
A $v1,...,vk$ vektorrendszer akkor és csak akkor lineárisan független, ha csak a triviális lineáris kombinációja adja a nullvektort. Képletben: tetszőleges ${\lambda}1,{\lambda}2,...,{\lambda}k \in \mathbb{R}$ esetén, ha ${\lambda}1v1 + {\lambda}2v2 +  \cdot  \cdot  \cdot  + {\lambda}kvk = 0$, akkor minden $i$-re ${\lambda}i = 0$.
  \end{tcolorbox}
\end{frame}


\begin{frame}
  \begin{tcolorbox}[title={3}]
      Mit jelent az, hogy a $v1,...,vk \in \mathbb{R}^n$ vektorrendszer lineárisan összefüggő? A válaszban ne hivatkozzunk a lineáris függetlenség fogalmára.


  \tcblower
A $v1,...,vk$ vektorrendszer akkor és csak akkor lineárisan összefüggő (azaz nem lineárisan független), ha léteznek olyan ${\lambda}1,{\lambda}2,...,{\lambda}k \in \mathbb{R}$ nem mind nulla számok, melyekre ${\lambda}1v1 + {\lambda}2v2 +  \cdot  \cdot  \cdot  + {\lambda}kvk = 0$.)

  \end{tcolorbox}
\end{frame}

\begin{frame}
  \begin{tcolorbox}[title={4}]
     Mit jelent az, hogy egy v vektor lineárisan függ az $a_1,...,ak \in \mathbb{R}^n$ vektoroktól?

  \tcblower
Azt jelenti, hogy v felírható $a_1,...,ak$ lineáris kombinációjaként, azaz léteznek olyan ${\lambda}1,{\lambda}2,...,{\lambda}k \in \mathbb{R}$ skalárok, melyekre $v = {\lambda}1a_1 + {\lambda}2a_2 +  \cdot  \cdot  \cdot  + {\lambda}kak$.
  \end{tcolorbox}
\end{frame}


\begin{frame}
  \begin{tcolorbox}[title={5}]
    Jellemezzük egy $a_1,...,ak \in \mathbb{R}^n$ vektorrendszer lineáris összefüggőségét a lineáris függés fogalmával.


  \tcblower
Egy $a_1,...,ak \in \mathbb{R}^n$ vektorrendszer ($k \geq 2$ esetén) pontosan akkor lineárisan összefüggő, ha valamelyik i-re $ai$ lineárisan függ az $a_1,...,ai-1,ai+1,...,ak$ vektoroktól.
  \end{tcolorbox}
\end{frame}


\begin{frame}
  \begin{tcolorbox}[title={6}]
      Definiáljuk egy $V \leq \mathbb{R}^n$ altér bázisának fogalmát.


  \tcblower
Egy $b_1,...,bk$ vektorrendszert akkor mondunk a $V \leq \mathbb{R}^n$ altér bázisának, ha lineárisan független, és V minden vektorát előállíthatjuk a bi vektorok lineáris kombinációjaként. (A lineáris függetlenség helyettesíthető azzal a feltétellel, hogy ez a felírás egyértelmű, az előállíthatóság pedig azzal, hogy generátorrendszerről van szó.)
  \end{tcolorbox}
\end{frame}


\begin{frame}
  \begin{tcolorbox}[title={7}]
      Definiáljuk egy $a \in V \leq \mathbb{R}^n$ vektor koordinátavektorát a $V$ egy $b_1,...,bk$ bázisában fölírva.


  \tcblower
Az a vektor koordinátavektora pontosan akkor $[a]b_1,...,bk =$
${\lambda}1 . . . {\lambda}k$
, ha $a = {\lambda}1b_1+ \cdot  \cdot  \cdot +{\lambda}kbk$.
  \end{tcolorbox}
\end{frame}


\begin{frame}
  \begin{tcolorbox}[title={8}]
       Egy $A {\subseteq} \mathbb{R}^n$ vektorhalmaz esetén adjuk meg az $A$ által generált altér egy jellemzését.


  \tcblower
Az A által generált altér azokból az $\mathbb{R}^n$-beli vektorokból áll, amelyek előállnak $A$-beli vektorok lineáris kombinációiként, azaz amelyek lineárisan függnek az $A$-beli vektoroktól. (Ez a halmaz megegyezik az $A$-t tartalmazó $\mathbb{R}^n$-beli alterek metszetével.)

  \end{tcolorbox}
\end{frame}



\begin{frame}
  \begin{tcolorbox}[title={9}]
     Mondjuk ki a lineárisan független rendszerek és a generátorrendszerek elemszámát összehasonlító tételt (ez a kicserélési tétel egyik része).



  \tcblower
Minden lineárisan független rendszer elemszáma legfeljebb akkora, mint bármelyik generátorrendszer elemszáma.

  \end{tcolorbox}
\end{frame}


\begin{frame}
  \begin{tcolorbox}[title={10}]
     Definiáljuk egy $V \leq \mathbb{R}^n$ altér dimenzióját, $dimV$ -t.

  \tcblower
$dimV$ a $V$ egy bázisának elemszáma, illetve 0, ha $V = \{0\}$. (Ez a definíció azért értelmes, mert bármely két bázis elemszáma egyenlő.)

  \end{tcolorbox}
\end{frame}



\begin{frame}
  \begin{tcolorbox}[title={11}]
   Definiáljuk egy $v1,...,vk \in \mathbb{R}^n$ vektorrendszer rangját, $r(v1,...,vk)$-t.


  \tcblower
$r(v1,...,vk) = dimSpan(v1,...,vk)$, azaz egy vektorrendszer rangja megegyezik az általa generált (kifeszített) altér dimenziójával.

  \end{tcolorbox}
\end{frame}


\begin{frame}
  \begin{tcolorbox}[title={12}]
    Adjuk meg képlettel két mátrix, $A \in Rkxl és B \in Rlxn$ szorzatában, $AB$-ben az $i$-edik sor $j$-edik elemét, $i[AB]j$-t. Azt is mondjuk meg, $i$ és $j$ milyen értékére létezik ez az elem.



  \tcblower
Ha $1 \leq i \leq k és 1 \leq j \leq n, akkor i[AB]j =$
  \end{tcolorbox}
\end{frame}



\begin{frame}
  \begin{tcolorbox}[title={13}]
    Mondjunk ki két, a mátrixok transzponálását a többi szokásos mátrixművelettel összekapcsoló összefüggést.

  \tcblower
$A,B \in \mathbb{R}^{k x n} {\Rightarrow} (A + B)T = AT + BT {\lambda} \in R,A \in \mathbb{R}^{k x n} {\Rightarrow} ({\lambda}A)T = {\lambda}AT A \in Rkxl, B \in Rlxn {\Rightarrow} (AB)T = BTAT$

  \end{tcolorbox}
\end{frame}



\begin{frame}
  \begin{tcolorbox}[title={14}]
    Definiáljuk egy $A \in \mathbb{R}^{k x n}$ mátrix oszloprangját, illetve sorrangját, ${\varrho}O(A)-t és {\varrho}S(A)$-t.

  \tcblower
Ha $a_1,...,an \in Rk$ a mátrix oszlopai, akkor ${\varrho}O(A) = r(a_1,...,ak)$, azaz az oszloprang az oszlopok rendszerének rangja (vagyis az oszlopok által generált altér dimenziója.) Analóg módon, a mátrix sorrangja a sorok által generált altér dimenziója, vagy másképpen: ${\varrho}S(A) = {\varrho}O(AT)$.

  \end{tcolorbox}
\end{frame}

\begin{frame}
  \begin{tcolorbox}[title={15}]
   Mondjuk ki a mátrixok szorzatának oszloprangjára vonatkozó becslést.


  \tcblower
Ha létezik az $AB$ mátrixszorzat, akkor ${\varrho}O(AB) \leq {\varrho}O(A)$. (Igaz a ${\varrho}O(AB) \leq {\varrho}O(B)$ becslés is.)

  \end{tcolorbox}
\end{frame}


\begin{frame}
  \begin{tcolorbox}[title={16}]
     Mit nevezünk egy $A \in \mathbb{R}^{n x m}$ mátrix jobb, illetve kétoldali inverzének?

  \tcblower
Az $A(j) \in \mathbb{R}^{m x n}$ mátrix jobb oldali inverze $A$-nak, ha $AA(j) = In$, ahol $In$ az $n x n$-es egységmátrix. $A-1 \in \mathbb{R}^{m x n}$ kétoldali inverze A-nak, ha jobb oldali és bal oldali inverze is A-nak, azaz AA-1 = In, és A-1A = Im. (Ez utóbbi létezése esetén m = n.)

  \end{tcolorbox}
\end{frame}


\begin{frame}
  \begin{tcolorbox}[title={17}]
    A mátrixrang fogalmának fölhasználásával mondjuk ki annak szükséges és elégséges feltételét, hogy az $A \in \mathbb{R}^{n x m}$ mátrixnak létezzen jobb oldali inverze.

  \tcblower
Az $A \in \mathbb{R}^{n x m}$ mátrixnak pontosan akkor létezik jobb oldali inverze, ha ${\varrho}(A) = n$, azaz a mátrix rangja megegyezik a sorainak a számával.

  \end{tcolorbox}
\end{frame}


\begin{frame}
  \begin{tcolorbox}[title={18}]
   Definiáljuk a geometriai vektorok skaláris szorzatának fogalmát a vektorok hosszának és szögének segítségével.

  \tcblower
Jelölje $|a|$ az a geometriai vektor hosszát, ${\gamma}(a,b)$ pedig az $a$ és $b$ geometriai vektorok hajlásszögét. Ekkor az $a$ és $b$ skaláris szorzata $ab = |a||b|cos{\gamma}(a,b)$.

  \end{tcolorbox}
\end{frame}


\begin{frame}
  \begin{tcolorbox}[title={19}]
   Definiáljuk a geometriai vektorok vektoriális szorzatának fogalmát.
  \tcblower
Jelölje $|a|$ az a geometriai vektor hosszát, ${\gamma}(a,b)$ pedig az $a$ és $b$ geometriai vektorok hajlásszögét. Ekkor az $a$ és $b$ vektoriális szorzata az az $a x b$-vel jelölt vektor, melyre:\\

1) $|a x b| = |a||b|sin{\gamma}(ab)$;\\
2) $a x b {\perp} a,b$;\\
3) ha $|a x b| 6= 0$, akkor $a,b,a x b$ jobbrendszert alkot.

  \end{tcolorbox}
\end{frame}


\begin{frame}
  \begin{tcolorbox}[title={20}]
    Mondjuk ki a geometriai vektorokra vonatkozó kifejtési tételt.

  \tcblower
Ha $a,b,c$ geometriai vektorok, akkor $(a x b) x c = (ac)b - (bc)a$.

  \end{tcolorbox}
\end{frame}


\begin{frame}
  \begin{tcolorbox}[title={21}]
    Mondjuk ki a geometriai vektorokra vonatkozó felcserélési tételt.

  \tcblower
    Ha $a,b,c$ geometriai vektorok, akkor $(a x b)c = a(b x c)$. 
  \end{tcolorbox}
\end{frame}


\begin{frame}
  \begin{tcolorbox}[title={22}]
    Definiáljuk a geometriai vektorok vegyesszorzatának fogalmát.

  \tcblower
 Definiáljuk a geometriai vektorok vegyesszorzatának fogalmát.

  \end{tcolorbox}
\end{frame}


\begin{frame}
  \begin{tcolorbox}[title={23}]
   Adjuk meg az A mátrix determinánsát definiáló képletet. Mit jelent ebben az I(i1,...,in) kifejezés? 
  \tcblower
Legyen A =| |
$a_11 ... a_1n . . . . . . an1 ... ann \in \mathbb{R}^{n x n}$. Ekkor $detA = X i1,...,in (1,...,n)
(-1)I(i1,...,in)a_1i1a_2i2  \cdot  \cdot  \cdot anin$.
Itt az összegezés az $\{1,2,...,n\}$ számok minden permutációjára történik, $I(i1,...,in)$ pedig az adott permutáció inverzióinak a számát jelöli.

  \end{tcolorbox}
\end{frame}


\begin{frame}
  \begin{tcolorbox}[title={24}]
   Mondjunk ki egy olyan feltételt, mely ekvivalens azzal, hogy az $A \in \mathbb{R}^{n x n}$ mátrix determinánsa nem nulla.

  \tcblower
Mondjunk ki egy olyan feltételt, mely ekvivalens azzal, hogy az $A \in \mathbb{R}^{n x n}$ mátrix determinánsa nem nulla.

  \end{tcolorbox}
\end{frame}


\begin{frame}
  \begin{tcolorbox}[title={25}]
   Mit értünk az $A \in \mathbb{R}^{n x n}$ mátrix i-edik sorának j-edik eleméhez tartozó előjelezett aldeterminánson, $Aij$-n?

  \tcblower
Hagyjuk el az $A$ mátrix $i$-edik sorát és a $j$-edik oszlopát; az így kapott $(n-1)x(n-1)$-es mátrixot jelölje $Bij$. Ekkor a keresett előjelezett aldetermináns: $Aij = (-1)i+j detBij$.

  \end{tcolorbox}
\end{frame}

\begin{frame}
  \begin{tcolorbox}[title={26}]
   Adjuk meg képlettel az $A \in \mathbb{R}^{n x n}$ mátrix determinánsának $i$-edik sora szerinti kifejtését. A mátrix elemeit, illetve előjelezett aldeterminánsait $aij, ill$. $Aij$ jelöli.

  \tcblower
$detA =$
  \end{tcolorbox}
\end{frame}


\begin{frame}
  \begin{tcolorbox}[title={28}]
    DefiniáljukaVandermonde-determinánsfogalmát, ésmondjukkiazértékérevonatkozóállítást.

  \tcblower
$a_1,...,an \in R, n \geq 2 esetén V (a_1,...,an) =$



$(aj - ai)$.

  \end{tcolorbox}
\end{frame}


\begin{frame}
  \begin{tcolorbox}[title={30}]
    Mondjuk ki a determinánsokra vonatkozó szorzástételt.
  \tcblower
    Ha $A,B \in \mathbb{R}^{n x n}$ tetszőleges mátrixok, akkor $det(AB) = detA  \cdot  detB$.
  \end{tcolorbox}
\end{frame}


\begin{frame}
  \begin{tcolorbox}[title={31}]
   Mit jelent, hogy két négyzetes mátrix hasonló $\mathbb{R}$ felett?

  \tcblower
    $A,B \in \mathbb{R}^{n x n}$ hasonlók $\mathbb{R}$ felett, ha létezik olyan invertálható $S \in \mathbb{R}^{n x n}$ mátrix, melyre $B = S-1AS$. Ezt általában $A {\sim}_{\mathbb{R}} B$ jelöli.

  \end{tcolorbox}
\end{frame}


\begin{frame}
  \begin{tcolorbox}[title={32}]
    Mikor mondjuk egy mátrixra, hogy diagonalizálható $\mathbb{R}$ felett?

  \tcblower
    Egy $A \in \mathbb{R}^{n x n}$ négyzetes mátrix diagonalizálható $\mathbb{R}$ felett, ha hasonló $\mathbb{R}$ felett egy diagonális mátrixhoz, azaz létezik olyan S invertálható, ill. D diagonális mátrix $\mathbb{R}^{n x n}$-ben, hogy $D = S-1AS$.

  \end{tcolorbox}
\end{frame}


\begin{frame}
  \begin{tcolorbox}[title={33}]
    Definiáljuk egy mátrix jobb oldali sajátvektorának a fogalmát.

  \tcblower
    Legyen $A \in \mathbb{R}^{n x n}$ tetszőleges négyzetes mátrix. Egy $x \in \mathbb{R}^n$ vektort az $A$ mátrix jobb oldali sajátvektorának nevezünk, ha: 1) $x 6= 0$; 2) létezik ${\lambda}0 \in \mathbb{R}$ szám, melyre $Ax = {\lambda}0x$.

  \end{tcolorbox}
\end{frame}


\begin{frame}
  \begin{tcolorbox}[title={34}]
   Definiáljuk egy mátrix jobb oldali sajátértékének a fogalmát.
  \tcblower
     Legyen $A \in \mathbb{R}^{n x n}$ tetszőleges négyzetes mátrix. Egy ${\lambda}0 \in \mathbb{R}$ számot az $A$ mátrix jobb oldali sajátértékének nevezünk, ha van olyan $x \in \mathbb{R}^n$ vektor, melyre 1) $x 6= 0$; 2) $Ax = {\lambda}0x$.

  \end{tcolorbox}
\end{frame}


\begin{frame}
  \begin{tcolorbox}[title={35}]
    Mondjuk ki egy valós elemű mátrix $\mathbb{R}$ feletti diagonalizálhatóságának szükséges és elégséges feltételét a sajátvektorok fogalmának fölhasználásával.

  \tcblower
Egy $A \in \mathbb{R}^{n x n}$ mátrix pontosan akkor diagonalizálható $\mathbb{R}$ felett, ha létezik A sajátvektoraiból álló bázis $\mathbb{R}^n$-ben. 
  \end{tcolorbox}
\end{frame}



\begin{frame}
  \begin{tcolorbox}[title={36}]
     Definiáljuk az $A \in \mathbb{R}^{n x n}$ mátrix ${\lambda}0$ (jobb oldali) sajátértékéhez tartozó sajátalterének fogalmát.

  \tcblower
$W{\lambda}0 = {x \in \mathbb{R}^n |Ax = {\lambda}0x}$ a ${\lambda}0$-hoz tartozó sajátaltér. (Vagyis $W{\lambda}0$ a ${\lambda}0$ sajátértékű sajátvektorok halmaza, kiegészítve a nullvektorral.)

  \end{tcolorbox}
\end{frame}

\begin{frame}
  \begin{tcolorbox}[title={37}]
    Definiáljuk egy négyzetes mátrix karakterisztikus polinomjának fogalmát.
  \tcblower
Egy $A \in \mathbb{R}^{n x n}$ mátrix karakterisztikus polinomja $kA({\lambda}) = det(A-In{\lambda})$, ahol $In$ az $nxn$-es egységmátrix.

  \end{tcolorbox}
\end{frame}


\begin{frame}
  \begin{tcolorbox}[title={38}]
    Mondjuk ki a hasonló mátrixok karakterisztikus polinomjára vonatkozó állítást.

  \tcblower
Ha $A,B \in \mathbb{R}^{n x n}$ hasonlók $\mathbb{R}$ felett, akkor $kA({\lambda}) = kB({\lambda})$.

  \end{tcolorbox}
\end{frame}


\begin{frame}
  \begin{tcolorbox}[title={39}]
   Definiáljuk a komplex euklideszi tér fogalmát.
  \tcblower
Legyen V vektortér C felett. V -t komplex euklideszi térnek nevezzük, ha adva van egy $hx,yi : V x V {\rightarrow} C$ leképezés, melyre minden $x,y,z \in V$ és ${\lambda} \in C$ esetén: (1) $hy,xi = hx,yi$; (2) $h{\lambda}x,yi = {\lambda}hx,yi$ (és $hx,{\lambda}yi = {\lambda}hx,yi$); (3) $hx + y,zi = hx,zi + hy,zi$ (és $hx,y + zi = hx,yi + hx,zi$); (4) $hx,xi$ mindig valós és nemnegatív, továbbá csak akkor 0, ha $x = 0$.

  \end{tcolorbox}
\end{frame}


\begin{frame}
  \begin{tcolorbox}[title={40}]
   Definiáljuk az x vektor normáját egy euklideszi térben.

  \tcblower
Ha V valós vagy komplex euklideszi tér, akkor tetszőleges $x \in V$ vektorra $kxk =phx,xi$.

  \end{tcolorbox}
\end{frame}


\begin{frame}
  \begin{tcolorbox}[title={41}]
     Mondjuk ki az euklideszi terek vektoraira vonatkozó háromszög-egyenlőtlenséget.
  \tcblower
Ha V valós vagy komplex euklideszi tér, akkor tetszőleges $x,y \in V$ vektorokra: $kx + yk \leq kxk + kyk$ .

  \end{tcolorbox}
\end{frame}


\begin{frame}
  \begin{tcolorbox}[title={42}]
    Mondjuk ki a valós vagy komplex euklideszi terekre vonatkozó Cauchy-egyelőtlenséget, valamint azt, hogy mikor áll ebben egyenlőség.
  \tcblower
Ha V valós vagy komplex euklideszi tér, akkor tetszőleges $x,y \in V$ vektorokra teljesül, hogy $|hx,yi| \leq kxk \cdot kyk$, és egyenlőség akkor és csak akkor áll fönn, ha az $x$ és $y$ vektorok lineárisan összefüggőek (azaz párhuzamosak).

  \end{tcolorbox}
\end{frame}


\begin{frame}
  \begin{tcolorbox}[title={43}]
    Definiáljuk egy $V$ euklideszi tér ortonormált bázisának a fogalmát.

  \tcblower
Az $e1,...,en \in V$ vektorokbólállórendszertortonormáltbázisnaknevezzüka $V$ euklideszi térben, ha: 1) bázist alkotnak $V$ -ben; 2) az $ei$ vektorok páronként merőlegesek, azaz $i 6= j$ esetén $hei,eji = 0$; és 3) a vektorok normáltak, azaz $keik = 1 minden 1 \leq i \leq n$-re. 
  \end{tcolorbox}
\end{frame}


\begin{frame}
  \begin{tcolorbox}[title={44}]
    Mondjuk ki a valós szimmetrikus mátrixokra vonatkozó spektráltételt (azaz főtengelytételt).

  \tcblower
Egy $A \in \mathbb{R}^{n x n}$ mátrix esetén pontosan akkor létezik A sajátvektoraiból álló ortonormált bázis $\mathbb{R}^n$-ben (azaz $A$ pontosan akkor diagonalizálható $\mathbb{R}$ felett ortonormált bázisban), ha az $A$ mátrix szimmetrikus (azaz $AT = A$). (Ilyenkor az $A$ sajátértékei mind valósak.)

  \end{tcolorbox}
\end{frame}


\begin{frame}
  \begin{tcolorbox}[title={45}]
    Definiáljuk egy $A \in \mathbb{R}^{n x n}$ valós szimmetrikus mátrixhoz tartozó kvadratikus alakot.
  \tcblower
Az $A$-hoz tartozó kvadratikus alak az a $Q : \mathbb{R}^n {\rightarrow} \mathbb{R}$ függvény, melyre Q(x) = xTAx.
  \end{tcolorbox}
\end{frame}

\begin{frame}
  \begin{tcolorbox}[title={46}]
     Mondjuk meg, mit jelent az, hogy az $A \in \mathbb{R}^{n x n}$ szimmetrikus mátrixhoz tartozó Q kvadratikus alak pozitív definit, és jellemezzük ezt az esetet az $A$ sajátértékei segítségével.

  \tcblower
Q-t akkor nevezzük pozitív definitnek, ha minden $0 6= x \in \mathbb{R}^n$ vektorra $Q(x) > 0$. Ez pontosan akkor teljesül, ha az $A$ mátrix minden sajátértéke pozitív.
  \end{tcolorbox}
\end{frame}


\begin{frame}
  \begin{tcolorbox}[title={46}]
     Mondjuk meg, mit jelent az, hogy az $A \in \mathbb{R}^{n x n}$ szimmetrikus mátrixhoz tartozó Q kvadratikus alak negatív definit, és jellemezzük ezt az esetet az A karakterisztikus sorozata segítségével.

  \tcblower
Q-t akkor nevezzük negatív definitnek, ha minden $0 6= x \in \mathbb{R}^n$ vektorra $Q(x) < 0$. Ez pontosan akkor teljesül, ha az A karakterisztikus sorozata jelváltó. (Az A mátrix ${\Delta}o,...,{\Delta}n$ karakterisztikus sorozatának k-adik tagja az A bal fölső sarkában lévő $k x k$-as részmátrix determinánsa, illetve ${\Delta}o = 1$.)

  \end{tcolorbox}
\end{frame}




\begin{frame}[plain]
\begin{tcolorbox}[center, colback={myyellow}, coltext={black}, colframe={myyellow}]
    {\RHuge C rész}
    \mmedskip
\end{tcolorbox}
\end{frame}

\begin{frame}
  \begin{tcolorbox}[title={1. (4p)}]
    Mondjuk ki a $b_1,...,bk$ vektorok lineáris függetlenségének definícióját, majd bizonyítsuk be, hogy ha $B = \{b_1,...,bk\}$ lineárisan független vektorrendszer, akkor minden lineáris kombinációjukként felírható vektor egyértelműen írható föl B-beli vektorok lineáris kombinációjaként.
  \tcblower
  
    A $v1,...,vk$ vektorrendszer akkor és csak akkor lineárisan független, ha csak a triviális lineáris kombinációja adja a nullvektort.\\

    Képletben: tetszőleges ${\lambda}1,{\lambda}2,...,{\lambda}k \in R$ esetén, ha ${\lambda}1v1 + {\lambda}2v2 +  \cdot  \cdot  \cdot  + {\lambda}kvk = 0$, akkor minden $i$-re ${\lambda}i = 0$.\\

    Ha lenne egy vektor, melynek kétféle fölírása is létezne: $u = Pk i=1 {\lambda}ibi = Pk i=1 {\mu}ibi$, akkor a kétféle előállítást egymásból kivonva azt kapjuk, hogy\\

    $0 = Pk i=1({\lambda}i -{\mu}i)bi$.\\

    Ha a két előállítás különbözik, akkor valamelyik $({\lambda}i-{\mu}i)$ együttható nem nulla, s ez ellentmond a $B$ lineáris függetlenségének.
  \end{tcolorbox}
\end{frame}


\begin{frame}
  \begin{tcolorbox}[title={2. (4p)}]
     Tekintsük az alábbi két fogalmat:\\
     a) egy vektorrendszer lineárisan összefüggő;\\
     b) egy vektor lineárisan függ egy vektorrendszertől.\\
     \mmedskip
     
     Mondjuk ki azt az állítást, mely ezeket a fogalmakat összekapcsolja, és bizonyítsuk is be az állítást.
  \tcblower
  
    Ha $k \geq 2$, akkor a $v1,...,vk$ vektorrendszer akkor és csak akkor lineárisan összefüggő, ha létezik olyan $i$, hogy $vi$ lineárisan függ a többi $vj$ vektortól (azaz előáll mint a $v1,...,vi-1,vi+1,...,vk$ vektorok lineáris kombinációja).\\
    \mmedskip
    
    Tegyük föl először, hogy a megadott vektorrendszer lineárisan összefüggő.\\
    Ez azt jelenti, hogy a nullvektornak van egy olyan $Pk i=1 {\lambda}ivi = 0$ előállítása, melynél valamelyik ${\lambda}i$ együttható (pl. a ${\lambda}i0$) nem $0$.\\
    
    Ekkor a $vi0$ kifejezhető a többi vektor lineáris kombinációjaként, hiszen $vi0 = -(1/{\lambda}i0)Pj6=io {\lambda}jvj$.\\
    \mmedskip
    
    A fordított irányhoz tegyük föl, hogy $vi0$ lineárisan függ a többi $vj$ vektortól, azaz $vi0 = Pj6=i0 {\mu}jvj$ valamilyen ${\mu}j$ együtthatókra.\\
    
    De ekkor átrendezhetjük a fönti egyenlőséget úgy, hogy minden vektor az egyenlőség azonos oldalára kerüljön, s ekkor azt kapjuk, hogy a ${\mu}i0 = -1$ választással: $Pk i=1 {\mu}ivi = 0$. Ez nem triviális lineáris kombinációja a $vi$ vektoroknak, mert ${\mu}i0 = -1 6= 0$, tehát a vektorrendszer lineárisan összefüggő.
  \end{tcolorbox}
\end{frame}


\begin{frame}
  \begin{tcolorbox}[title={3. (4p)}]
      Mondjuk ki és igazoljuk azt az állítást, mely arról szól, mi történik, ha egy lineárisan független $a_1,...,ak$ vektorrendszerhez hozzávéve a $b$ vektort, az új, bővebb $a_1,...,ak,b$ vektorrendszer már összefüggővé válik.
  \tcblower
    Ha egy lineárisan független $a_1,...,ak$ vektorrendszerhez hozzávéve a $b$ vektort, a kapott $a_1,...,ak,b$ vektorrendszer már összefüggő, akkor $b$ lineárisan függ az $a_1,...,ak$ vektoroktól (azaz előállítható lineáris kombinációjukként).\\
    
    A feltétel szerint ugyanis léteznek olyan ${\lambda}1,...,{\lambda}k,{\mu}$ együtthatók, melyek közül legalább az egyik nem nulla, és melyekre $(Pk i=1 {\lambda}iai) + {\mu}b = 0$. Itt azonban ${\mu}$ nem lehet nulla, ellenkező esetben az $ai$ vektorok már önmagukban is előálítanák a nullvektort nem triviális módon, ez pedig ellentmond a lineáris függetlenségüknek.\\
    
    Ha átrendezzük az előbbi egyenlőséget úgy, hogy a $b$ vektort hagyjuk az egyik oldalon, akkor ${\mu}$-vel osztva $b = Pk i=1(-{\lambda}i/{\mu})ai$, ami azt mutatja, hogy $b$ lineárisan függ az $ai$ vektoroktól.
  \end{tcolorbox}
\end{frame}

\begin{frame}
  \begin{tcolorbox}[title={4. (4p)}]
      Mondjuk ki az $\mathbb{R}^n$-beli alterek fogalmának a műveletekre való zártsággal való definícióját, majd bizonyítsuk be, hogy $\mathbb{R}^n$ két alterének metszete is altér.
  \tcblower
    $W {\subseteq} \mathbb{R}^n$ pontosan akkor altér, ha nem üres (ekvivalens módon itt azt is megkövetelhetjük, hogy a nullvektor benne van $W$-ben), továbbá $w1,w2 \in W$ esetén $w1 +w2 \in W$, valamint $w \in W és {\lambda} \in R$ esetén ${\lambda}w \in W$. Tegyük föl, hogy $W1,W2$ altér $\mathbb{R}^n$-ben.\\
    
    Ekkor a nullvektor mindkét altérnek eleme, így $W1 {\cap}W2$ nem üres. Ha $w1,w2 \in W1 {\cap}W2$, akkor mindkét $i$ indexre $w1 +w2 \in Wi$ (hiszen $Wi$ altér), s ezért az összegvektor benne van a metszetben.\\
    
    Hasonlóan, ha $w \in W1 {\cap} W2$ és ${\lambda} \in \mathbb{R}$, akkor $Wi$ altér volta miatt ${\lambda}w \in Wi$ mindkét lehetséges $i$ indexre, így ${\lambda}w$ benne van a metszetükben is.
  \end{tcolorbox}
\end{frame}


\begin{frame}
  \begin{tcolorbox}[title={7. (4p)}]
      Definiáljuk az $n x n$-es egységmátrix fogalmát, és mondjuk meg, mi lesz az eredménye az egységmátrixszal való szorzásnak.\\
      
      Igazoljuk az egységmátrixszal jobbról való szorzásra vonatkozó összefüggést. 
  \tcblower
    Az $In \in \mathbb{R}^{n x n}$ egységmátrix $i$-edik sorának $j$-edik eleme $1$, ha $i = j$, és $0$ egyébként (azaz a főátlóban $1$-esek, a főátlón kívül $0$-k szerepelnek).\\
    
    Úgy is írhatjuk, hogy a mátrix általános eleme ${\delta}ij$, ahol ${\delta}ij$ a szokásos Kronecker-szimbólum.\\
    
    Tetszőleges $A \in \mathbb{R}^{k x n}$ és $B \in \mathbb{R}^{n x m}$ mátrixokra $AIn = A$ és $InB = B$.\\
    
    Az $AIn = A$ igazolásához jelölje $apq$ az $A$ mátrix általános elemét.\\
    
    A szorzatmátrixban az $i$-edik sor $j$-edik eleme a szorzás definíciója szerint $i[AIn]j = Pn t=1(ait{\delta}tj)$.\\
    
    Ebben az összegben ${\delta}tj = 0$ ha $t 6= j$, ezért csak az egyetlen $aij{\delta}jj = aij$ tag marad meg.
  \end{tcolorbox}
\end{frame}


\begin{frame}
  \begin{tcolorbox}[title={8. (4p)}]
       Mondjuk ki és bizonyítsuk be a mátrixok szorzatának transzponáltjára kimondott összefüggést. 
  \tcblower
    Ha $A \in Rkxl$ és $B \in Rlxn$, akkor $(AB)T = BTAT$. A szorzat transzponáltjának általános eleme ugyanis: $i[(AB)T]j = j[AB]i = Pl t=1 j[A]t  \cdot  t[B]i = Pl t=1 i[BT]t  \cdot  t[AT]j = i[BTAT]j$.
  \end{tcolorbox}
\end{frame}

\begin{frame}
  \begin{tcolorbox}[title={15. (4p)}]
       Definiáljuk egy $A \in \mathbb{R}^{n x n}$ mátrix jobb oldali sajátértékének fogalmát, majd igazoljuk, hogy az $A$ mátrix ${\lambda}$ sajátértékű (jobb oldali) sajátvektorai a nullvektorral kiegészítve alteret alkotnak $\mathbb{R}^n$-ben.
  \tcblower
    ${\lambda} \in R$ jobb oldali sajátértéke $A$-nak, ha van olyan $0 6= v \in \mathbb{R}^n$ vektor, melyre $Av = {\lambda}v$. (Ilyenkor $v$-t a ${\lambda}$-hoz tartozó (egyik) sajátvektornak nevezhetjük.)\\
    
    Azt kell igazolnunk, hogy $W{\lambda} = {v \in \mathbb{R}^n |Av = {\lambda}v} \leq \mathbb{R}^n$. Nyilván $0 \in W{\lambda}$, így $W{\lambda}$ nem üres. Ha $v1,v2 \in W{\lambda}$, akkor $A(v1 +v2) = Av1 +Av2 = {\lambda}v1 +{\lambda}v2 = {\lambda}(v1 +v2)$, azaz $W{\lambda}$ zárt az összeadásra. Végül, ha $v \in W{\lambda}$ és ${\mu} \in R$, akkor $A({\mu}v) = {\mu}Av = {\mu}({\lambda}v) = {\lambda}({\mu}v)$, vagyis $W{\lambda}$ zárt a skalárral való szorzásra is.
  \end{tcolorbox}
\end{frame}


\end{document}
