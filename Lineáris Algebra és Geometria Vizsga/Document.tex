% Compile twice!
% With the current MiKTeX, you need to install the beamer, and the translator packages directly form the package manager!

% !TEX root = ./PrezA4Page.tex

% Uncomment these to get the presentation form
%\documentclass{beamer}
%\geometry{paperwidth=200mm,paperheight=200mm, top=0in, bottom=0.2in, left=0.2in, right=0.2in}

% Uncomment these, and comment the 2 lines above, to get a paper-type article
%\documentclass[10pt]{article}
%\usepackage{geometry}
%\geometry{top=0.2in, bottom=0.2in, left=0.2in, right=0.2in}
%\usepackage{beamerarticle}
%\renewcommand{\\}{\par\noindent}
%\setbeamertemplate{note page}[plain]

% Half A4 geometry
%\geometry{paperwidth=105mm,paperheight=297mm,top=0.2in, bottom=0.2in, left=0.2in, right=0.2in}

% "1/3" A4 geometry
%\geometry{paperwidth=105mm,paperheight=455mm,top=0.1in, bottom=0.1in, left=0.1in, right=0.1in}

% "1/6" A4 geometry
%\geometry{paperwidth=105mm,paperheight=891mm,top=0.1in, bottom=0.1in, left=0.1in, right=0.1in}

% "1/5" A4 geometry
%\geometry{paperwidth=105mm,paperheight=740mm,top=0.1in, bottom=0.1in, left=0.1in, right=0.1in}

% "1/4" A4 geometry
%\geometry{paperwidth=105mm,paperheight=594mm,top=0.1in, bottom=0.1in, left=0.1in, right=0.1in}

% Uncomment these, to put more than one slide / page into a generated page.
%\usepackage{pgfpages}
% Choose one
%\pgfpagesuselayout{2 on 1}[a_4paper]
%\pgfpagesuselayout{4 on 1}[a_4paper]
%\pgfpagesuselayout{8 on 1}[a_4paper]

% Includes
\usepackage{tikz}
\usepackage{tkz-graph}
\usetikzlibrary{shapes,arrows,automata}
\usepackage[T1]{fontenc}
\usepackage{amsfonts}
\usepackage{amsmath}
\usepackage[utf8]{inputenc}
\usepackage{booktabs}
\usepackage{array}
\usepackage{arydshln}
\usepackage{enumerate}
\usepackage[many, poster]{tcolorbox}
\usepackage{pgf}
\usepackage[makeroom]{cancel}
\usepackage{verbatim}
\usepackage{skak}

\providecommand{\includecolors}{
% Colors
\definecolor{myred}{rgb}{0.87,0.18,0}
\definecolor{myorange}{rgb}{1,0.4,0}
\definecolor{myyellowdarker}{rgb}{1,0.69,0}
\definecolor{myyellowlighter}{rgb}{0.91,0.73,0}
\definecolor{myyellow}{rgb}{0.97,0.78,0.36}
\definecolor{myblue}{rgb}{0,0.38,0.47}
\definecolor{mygreen}{rgb}{0,0.52,0.37}
\colorlet{mybg}{myyellow!5!white}
\colorlet{mybluebg}{myyellowlighter!3!white}
\colorlet{mygreenbg}{myyellowlighter!3!white}

\setbeamertemplate{itemize item}{\color{black}$-$}
\setbeamertemplate{itemize subitem}{\color{black}$-$}
\setbeamercolor*{enumerate item}{fg=black}
\setbeamercolor*{enumerate subitem}{fg=black}
\setbeamercolor*{enumerate subsubitem}{fg=black}

% These are different themes, only uncomment one at a time
\tcbset{enhanced,fonttitle=\mdseries,boxsep=7pt,arc=0pt,colframe={myyellowlighter},colbacktitle={myyellow},colback={mybg},coltitle={black}, coltext={black},attach boxed title to top left={xshift=-2mm,yshift=-2mm},boxed title style={size=small,arc=0mm}}

%\tcbset{colback=yellow!5!white,colframe=yellow!84!black}
%\tcbset{enhanced,colback=red!10!white,colframe=red!75!black,colbacktitle=red!50!yellow,fonttitle=
%\tcbset{enhanced,attach boxed title to top left}
%\tcbset{enhanced,fonttitle=\bfseries,boxsep=5pt,arc=8pt,borderline={0.5pt}{0pt}{red},borderline={0.5pt}{5pt}{blue,dotted},borderline={0.5pt}{-5pt}{green}}
}% fallback definition
\includecolors

\setbeamertemplate{itemize item}{\color{black}$-$}
\setbeamertemplate{itemize subitem}{\color{black}$-$}
\setbeamercolor*{enumerate item}{fg=black}
\setbeamercolor*{enumerate subitem}{fg=black}
\setbeamercolor*{enumerate subsubitem}{fg=black}

 \renewcommand{\familydefault}{\sfdefault}
%\renewcommand{\familydefault}{\rmdefault}

\renewcommand{\footnotesize}{\fontsize{1.2em}{0.2em}}
\renewcommand{\normalsize}{\fontsize{1.2em}{0.2em}}
\renewcommand{\large}{\footnotesize}
\renewcommand{\Large}{\footnotesize}


\renewcommand{\scriptsize}{\footnotesize}
\renewcommand{\LARGE}{\footnotesize}
\renewcommand{\Huge}{\footnotesize}

\renewcommand{\tiny}{\footnotesize}
\renewcommand{\small}{\footnotesize}

\fontsize{1.2em}{0.2em}
\selectfont

\newcommand{\RHuge}{\fontsize{1.8em}{0.3em}\selectfont}

\newsavebox\CBox
%\newcommand<>*\textBF[1]{\sbox\CBox{#1}\resizebox{\wd\CBox}{\ht\CBox}{\textbf#2{#1}}}
\newcommand<>*\textBF[1]{\only#2{\sbox\CBox{#1}\resizebox{\wd\CBox}{\ht\CBox}{\textbf{#1}}}}

% Beamer theme
\usetheme{boxes}

% tikz settings for the flowchart(s)
\tikzstyle{decision} = [diamond, minimum width=3cm, minimum height=1cm, text centered, draw=black, fill=green!15]
\tikzstyle{tcolorbox} = [rectangle, draw, fill=blue!15, text width=20em, text centered, minimum height=1em]

\tikzstyle{line} = [draw, -latex']
\tikzstyle{cloud} = [draw, ellipse,fill=red!20, node distance=3cm,
    minimum height=2em]
\tikzstyle{arrow} = [thick,->,>=stealth]

\newcolumntype{C}[1]{>{\centering\let\newline\\\arraybackslash\hspace{0pt}}m{#1}}
\renewcommand{\arraystretch}{1.2}

\setlength\dashlinedash{0.2pt}
\setlength\dashlinegap{1.5pt}
\setlength\arrayrulewidth{0.3pt}

\newcommand{\mtinyskip}{\vspace{0.2em}}
\newcommand{\msmallskip}{\vspace{0.3em}}
\newcommand{\mmedskip}{\vspace{0.5em}}
\newcommand{\mbigskip}{\vspace{1em}}
\renewcommand{\u}[1]{\underline{#1}}

\begin{document}

\begin{frame}[plain]
\begin{tcolorbox}[center, colback={myyellow}, coltext={black}, colframe={myyellow}]
    {\RHuge Lineáris Algebra és Geometria}\\
\end{tcolorbox}
\end{frame}


%\begin{tcolorbox}[title={Def.: }]
%\end{tcolorbox}

% --------------------   Vektorok koordinátavektora  --------------------

\begin{frame}[plain]
\begin{tcolorbox}[center, colback={myyellow}, coltext={black}, colframe={myyellow}]
    {\RHuge  A rész}
    \mmedskip
\end{tcolorbox}
\end{frame}

\begin{frame}
  \begin{tcolorbox}
Az alábbiakban – tájékoztató jelleggel – fölsoroljuk a vizsgadolgozat A részében szereplő főbb témaköröket néhány mintafeladattal, azok megoldásával és esetenként részletesebb magyarázattal. Ez a rész a szemléltetést szolgálja: a dolgozatban általában nem pont ezek a kérdések fognak szerepelni. A feladatok nehézségét szimbólumok jelzik. Ahol egy definíciót vagy tételt kell közvetlenül alkalmazni, ott {\symknight} a jel. Ha a fogalom már nehezebb, vagy több lépést kell tenni, több fogalmat összekapcsolni, nemtriviális átalakítást kell végezni, ott {\symrook} szerepel. A {\symqueen} azt jelzi, hogy a feladat témája kifejezetten nehéz, vagy pedig ötlet kell a megoldáshoz.
  \end{tcolorbox}
\end{frame}

\begin{frame}[plain]
\begin{tcolorbox}[center, colback={myyellow}, coltext={black}, colframe={myyellow}]
    {\RHuge  (1) Vektorok koordinátavektora}
    \mmedskip
\end{tcolorbox}
\end{frame}

\begin{frame}
  \begin{tcolorbox}[title={1/1. {\symknight}}]
      A $\{b_1,b_2,b_3\}$ vektorhalmaz bázis a $V \leq \mathbb{R}^n$ altérben. Határozzuk meg a $v = 2b_1 + 3b_2 -b_3$ vektor koordinátavektorát a $\{b_1,b_2,b_3\}$ bázisban.
  \tcblower
    A feladat a vektor bázisban vett koordinátavektorának fogalmát kéri számon.\\
    \mmedskip
    
    A vektorból kell kiszámítani a koordinátáit.\\
    \mmedskip 
  
   $[v]_{b_1,b_2,b_3} =$ $\begin{bmatrix} 
  				2  \\
  				3 \\
  				-1
			\end{bmatrix}$
  \end{tcolorbox}
\end{frame}

\begin{frame}
  \begin{tcolorbox}[title={1/2. {\symrook}}]
      Legyen $\mathbb{B} = \{b_1,b_2,b_3\}$ bázis a $V \leq \mathbb{R}^n$ altérben. Egy $v \in V$ vektornak a $\{b_1 + b_2,b_2 + b_3,b_3\}$ bázisban fölírt koordinátavektora $[v]_{b_1+b_2,b_2+b_3,b_3} = [1 1 0]^T$.\\
      \mmedskip
      
      Mi $v$ koordinátavektora a $\mathbb{B}$ bázisban?
  \tcblower
    A feladat a vektor bázisban vett koordinátavektorának fogalmát kéri számon.\\
    \mmedskip
   
    Kétlépéses, először a feladat szövegében megadott koordinátavektort írjuk át a bázisvektorok lineáris kombinációjává: $v = 1 \cdot (b_1 + b_2) + 1 \cdot (b_2 + b_3) + 0 \cdot (b_3) = 1 \cdot b_1 + 2 \cdot b_2 + 1 \cdot b_3$, majd az így kapott lineáris kombinációt koordinátavektorrá.\\
     \mmedskip
     
   $[v]_{b_1,b_2,b_3} =$ $\begin{bmatrix} 
  				1  \\
  				2 \\
  				1
			\end{bmatrix}$
  \end{tcolorbox}
\end{frame}

\begin{frame}
  \begin{tcolorbox}[title={1/3. {\symrook}}]
      A $\{b_1,b_2,b_3\}$ vektorhalmaz bázis a $V \leq \mathbb{R}^n$ altérben. Ebben a bázisban a $v$, illetve $w$ vektorok koordinátorvektora $[v]_{b_1,b_2,b_3} = $ $\begin{bmatrix} 
  				1  \\
  				0 \\
  				2
			\end{bmatrix}$ és $[w]{b_1,b_2,b_3} = $ $\begin{bmatrix} 
  				3  \\
  				1 \\
  				0
			\end{bmatrix}$. Írjuk föl a $v+2w$ vektort a $\{b_1,b_2,b_3\}$ vektorok lineáris kombinációjaként.
  \tcblower
    A feladat két fogalmat kérdez:\\
    \mmedskip
    
    Hogyan kell a koordinátákból kiszámítani a vektorokat, de még azt is tudni kell, hogy hogyan kell koordinátavektorokkal műveleteket végezni.\\
    \mmedskip
      
    Tehát először a két koordinátavektor megfelelő lineáris kombinációját számoljuk ki: $\begin{bmatrix} 
  				1  \\
  				0 \\
  				2
			\end{bmatrix}$ $+ 2$ $\begin{bmatrix} 
  				3  \\
  				1 \\
  				0
			\end{bmatrix}$ $=$ $\begin{bmatrix} 
  				7  \\
  				2 \\
  				2
			\end{bmatrix}$, majd az így kapott oszlopvektort átírjuk a bázisvektorok lineáris kombinációjává.\\
			\mmedskip
      
      De megtehetjük azt is, hogy először a v és w vektorokat írjuk fel a bázisvektorok lineáris kombinációjaként, majd ebből számítjuk ki a $v+2w$-t.\\
      \mmedskip
      
      $v + 2w = 7b_1 + 2b_2 + 2b_3$
  \end{tcolorbox}
\end{frame}
  
\begin{frame}
  \begin{tcolorbox}[title={1/4. {\symqueen}}]
      Tegyük föl, hogy a $\{b_1,b_2,b_3\}$ vektorhalmaz bázis a $V \leq \mathbb{R}^n$ altérben. Mi lesz ezen $b_i$ bázisvektorok $v$-vel jelölt összegénekakoordinátavektora a $\{b_1+b_2,b_2,b_3\}$ bázisban?
  \tcblower
     Az a kérdés, hogy a $b_1+b_2+b_3 = {\lambda}_1(b_1+b_2)+{\lambda}_2b_2+{\lambda}_3b_3$ felírásban mennyi a ${\lambda}_i$ ismeretlenek értéke.\\
    \mmedskip
    
      Ezekre úgy kapunk lineáris egyenletrendszert, hogy a $b_i$ bázisvektor együtthatóját az egyenlet két oldalán összehasonlítjuk.\\
      \mmedskip
      
      Azonnal látszik, hogy $b_1 + b_2 + b_3 = 1 \cdot (b_1 + b_2) + 0 \cdot b_2 + 1 \cdot b_3$ megfelelő felírás, és ezért ezek az együtthatók adják a keresett koordinátavektort.\\
      \mmedskip
      
      Ennek a problémának a megoldását az általános esetben az elemi bázistranszformációról szóló tétel szolgáltatja.
      \mmedskip
      
     $[v]_{b_1 + b_2,b_2,b_3} =$ $\begin{bmatrix} 
  				1  \\
  				0 \\
  				1
			\end{bmatrix}$
  \end{tcolorbox}
\end{frame}  

\begin{frame}[plain]
\begin{tcolorbox}[center, colback={myyellow}, coltext={black}, colframe={myyellow}]
    {\RHuge  (2) Alterek alaptulajdonságai}
    \mmedskip
\end{tcolorbox}
\end{frame}

\begin{frame}
  \begin{tcolorbox}[title={2/1. {\symrook}}]
      Konkrét vektorokat megadva mutassuk meg, miért nem alkot alteret $\mathbb{R}^3$-ben azon vektorok halmaza, ahol az első két komponens szorzata nulla. 
  \tcblower

    \mmedskip 
  
   Pl. $\begin{bmatrix} 
  				1  \\
  				0 \\
  				0
			\end{bmatrix}$ és $\begin{bmatrix} 
  				0  \\
  				1 \\
  				0
			\end{bmatrix}$ benne vannak az altérben, de az összegük, $\begin{bmatrix} 
  				1  \\
  				1 \\
  				0
			\end{bmatrix}$ nincs.
  \end{tcolorbox}
\end{frame}


\begin{frame}
  \begin{tcolorbox}[title={2/2. {\symqueen}}]
      Konkrét mátrixokat megadva mutassuk meg, miért nem alkot alteret $\mathbb{R}^{2 x 2}$-ben a nulla determinánsú mátrixok halmaza. 
  \tcblower
    A nulla determinánsú mátrixok összege könnyen lehet nem nulla determinánsú, s hasonlóképpen nem nulla determinánsú mátrixok összege lehet nulla determinánsú.\\
    \mmedskip 
  
    Pl. det $\begin{bmatrix} 
  				1 & 0  \\
  				0 & 0\\
			\end{bmatrix}$ = det $\begin{bmatrix} 
  				0 & 0 \\
  				0 & 1 \\
			\end{bmatrix}$ = 0, ugyanakkor det($\begin{bmatrix} 
  				1 & 0  \\
  				0 & 0\\
			\end{bmatrix}$ $+$ $\begin{bmatrix} 
  				0 & 0 \\
  				0 & 1 \\
			\end{bmatrix}$) $=1 \neq 0$
  \end{tcolorbox}
\end{frame}

\begin{frame}
  \begin{tcolorbox}[title={2/3. {\symrook}}]
      Tekintsük azoknak az $\mathbb{R}^3$-beli $\begin{bmatrix} 
  				x_1  \\
  				x_2 \\
  				x_3
			\end{bmatrix}$ vektoroknak a halmazát, amelyek az alábbi feltételnek tesznek eleget.\\
	  \mmedskip
      
      Mely eset(ek)ben kapunk alteret?\\
      \mmedskip
      
      (A) $x_1 = x_2 + 2x_3$\\
      (B) $x_1 = x_2 + 2$\\
      (C) $x_1x_2 = 0$\\
      (D) $x^2_1 = 0$
  \tcblower

    \mmedskip
    
    Alter(ek): (A), (D)
  \end{tcolorbox}
\end{frame}

\begin{frame}
  \begin{tcolorbox}[title={2/4. {\symqueen}}]
      Mely alábbi halmaz(ok) alkot(nak) alteret a $2x2$-es valós mátrixok halmazában mint $\mathbb{R}$ fölötti vektortérben:\\
      \mmedskip
      
      (A) $\{A \in \mathbb{R}^{2 x 2}|detA = 0\}$\\
      (B) $\{A \in \mathbb{R}^{2 x 2}|detA \neq 0\}$\\
      (C) $\{A \in \mathbb{R}^{2 x 2}|A = A^T\}$\\
      (D) $\{A \in \mathbb{R}^{2 x 2}|A^2 = I_2\}$
  \tcblower
    Azt kell megvizsgálnunk, hogy a megadott halmazok zártak-e az összeadásra, ill. a skalárral való szorzásra:\\
    \mmedskip

    Az (A) kérdésben nulla determinánsú mátrixok összege könnyen lehet nem nulla determinánsú, s hasonlóképpen nem nulla determinánsú mátrixok összege lehet nulla determinánsú.\\
    \mmedskip
    
    A (B)-ben (de érvelhetünk azzal is, hogy a nullmátrix nem ilyen);\\
    \mmedskip

    A szimmetria megőrződik mátrixok összeadásánál, ill. skalárral való szorzásánál (ez mutatja, hogy a (C) feladatban alterünk van).\\
    \mmedskip

    Az pedig hogy a (D) feladatban nem kapunk alteret, kiderül pl. abból is, hogy a nullmátrix nem teljesíti az adott feltételt.\\
    \mmedskip 
    
    Alter(ek): (C)
  \end{tcolorbox}
\end{frame}

\begin{frame}
  \begin{tcolorbox}[title={2/5. {\symqueen}}]
      Mutassunk egy olyan $H$ részhalmazt $\mathbb{R}^2$-ben (azaz a síkon), mely zárt a vektorok szokásos összeadására, mégsem alkot alteret $\mathbb{R}^2$-ben.
  \tcblower
    A feladat arra mutat rá, hogy altereknél mindkét műveletre való zártságot meg kell követelnünk:\\
    \mmedskip
    
    erre érdemes már a fölkészülés során is példát keresnünk, hogy jobban megérthessük a fogalmat.\\
    \mmedskip 
    
    Pl. $\{[a b]^T \in \mathbb{R}^2|a,b > 0\}$
  \end{tcolorbox}
\end{frame}


\begin{frame}[plain]
\begin{tcolorbox}[center, colback={myyellow}, coltext={black}, colframe={myyellow}]
    {\RHuge  (3) Bázis, lineáris függetlenség, generálás, dimenzió, rang kapcsolata}
    \mmedskip
\end{tcolorbox}
\end{frame}

\begin{frame}
  \begin{tcolorbox}[title={3/1. {\symknight}}]
      Alkalmas együtthatók megadásával mutassuk meg, hogy a $\{v,0,w\}$ vektorok rendszere lineárisan összefüggő.
  \tcblower
    A lineáris összefüggőséget kell bizonyítanunk.\\
    \mmedskip

    Ehhez olyan együtthatókat kell megadnunk, amelyek nem mindegyike nulla (és nem olyanokat, hogy egyik sem nulla). Ha tehát a rendszerben látunk néhány összefüggő vektort, akkor a többinek nyugodtan adhatunk nulla együtthatót.\\
    \mmedskip 
  
    $0 \cdot v + 1 \cdot 0 + 0 \cdot w = 0$.
  \end{tcolorbox}
\end{frame}


\begin{frame}
  \begin{tcolorbox}[title={3/2. {\symknight}}]
       Alkalmas együtthatók megadásával mutassuk meg, hogy az $\{u,v,w,v+2w\}$ vektorok rendszere lineárisan összefüggő.
  \tcblower
    A lineáris összefüggőséget kell bizonyítanunk.\\
    \mmedskip

    Ehhez olyan együtthatókat kell megadnunk, amelyek nem mindegyike nulla (és nem olyanokat, hogy egyik sem nulla). Ha tehát a rendszerben látunk néhány összefüggő vektort, akkor a többinek nyugodtan adhatunk nulla együtthatót.\\
    \mmedskip 
  
    $0 \cdot u + 1 \cdot v + 2 \cdot w + (-1)(v + 2w) = 0$.
  \end{tcolorbox}
\end{frame}


\begin{frame}
  \begin{tcolorbox}[title={3/3. {\symknight}}]
       Legyen $v_1 = [1 \; 1]^T \in \mathbb{R}^2$ és $v_2 = [3 \; c]^T \in \mathbb{R}^2$. Mely valós $c$ szám(ok)ra lesz $v_1$ és $v_2$ lineárisan összefüggő?
  \tcblower
    Arra érdemes emlékeznünk, hogy két nem nulla vektor akkor és csak akkor lineárisan összefüggő, ha egymás skalárszorosai.\\
    \mmedskip 
  
    $c = 3$.
  \end{tcolorbox}
\end{frame}


\begin{frame}
  \begin{tcolorbox}[title={3/4. {\symrook}}]
       Mely $c \in R$ valós számokra lesznek lineárisan függetlenek az $[1 \; 0 \; c]^T, a [2 \; 2 \; 2]^T$ és a $[4 \; 2 \; 3]^T$ vektorok?
  \tcblower
    Megoldhatjuk a szokásos homogén lineáris egyenletrendszer vizsgálatával (ha a megoldások száma nagyobb, mint 1, akkor a vektorrendszer összefüggő).\\
    \mmedskip
    
    Másik lehetőség annak a tételnek az alkalmazása, hogy a megadott vektorok pontosan akkor lineárisan függetlenek, ha a belőlük mint oszlopvektorokból képzett mátrix determinánsa nem 0.\\
    Mivel $\begin{vmatrix} 
  				1 & 2 & 4 \\
  				0 & 2 & 2 \\
  				c & 2 & 3 \\
			\end{vmatrix}$ $= -4c + 2$, ezért a válasz $c \neq \frac{1}{2}$.\\
    \mmedskip 
  
    $c \neq 1/2$.
  \end{tcolorbox}
\end{frame}


\begin{frame}
  \begin{tcolorbox}[title={3/5. {\symrook}}]
       Legyen $v_1 = [1 \; 1 \; 0]^T \in \mathbb{R}^3$. Adjunk meg olyan $v_2$ és $v_3$ vektorokat $\mathbb{R}^3$-ban, melyekre igaz, hogy a $v_1,v_2,v_3$ vektorrendszer lineárisan összefüggő, de közülük bármely két vektor lineárisan független.
  \tcblower
    Segít, ha tudjuk, hogy mit jelent az összefüggőség a lineáris függés nyelvén: a feltételek azt jelentik, hogy a vektorok egyike sem skalárszorosa a másiknak, de az egyik vektor a másik kettőnek a lineáris kombinációja. \\
    \mmedskip 
  
	Pl. $v_1 = [1  \; 0  \; 0]^T,\; v_2 = [0  \; 1  \; 0]^T$
  \end{tcolorbox}
\end{frame}


\begin{frame}
  \begin{tcolorbox}[title={3/6. {\symknight}}]
        Egy $U \leq \mathbb{R}^2$ altérben $[2 \; 3]^T$ generátorrendszert alkot. Adjunk meg $U$-ban egy háromelemű generátorrendszert.
  \tcblower
    Azt az állítást használjuk, hogy generátorrendszer kibővítve is generátorrendszer marad, de vigyáznunk kell, hogy az új vektorokat is az altérből vegyük. \\
    \mmedskip 
  
     Pl. $[2  \; 3]^T, [4 \; 6]^T, [0 \; 0]^T$
  \end{tcolorbox}
\end{frame}


\begin{frame}
  \begin{tcolorbox}[title={3/7. {\symrook}}]
    Egy $U \leq \mathbb{R}^5$ altérben van olyan $\{v_1,v_2,v_3\}$ háromelemű lineárisan összefüggő vektorhalmaz, mely generátorrendszer U-ban. Hány eleme lehet $U$ egy bázisának?
  \tcblower
    A kérdés megválaszolásához az a tény segít, hogy minden generátorrendszerből kiválasztható bázis, az azonban lineárisan független.\\
    \mmedskip
    
    Így a háromelemű halmazból legalább egyet el kell távolítanunk, hogy bázist kaphassunk. Mivel az altérben van legalább három vektor, ezért az nem lehet 0 dimenziós.\\
    \mmedskip 
  
     Bázis elemszáma lehet: 1,2
  \end{tcolorbox}
\end{frame}


\begin{frame}
  \begin{tcolorbox}[title={3/8. {\symrook}}]
    Ha egy $U \leq \mathbb{R}^9$ altérben van $5$ elemű lineárisan összefüggő generátorrendszer is, meg $3$ elemű lineárisan független vektorrendszer is, akkor mik $dimU$ lehetséges értékei?

  \tcblower
    A feladat gondolata ugyanez, de hozzá kell tennünk, hogy független rendszer elemszáma kisebb vagy egyenlő, mint az altér dimenziója.\\
    \mmedskip 
  
     dimU lehet: 3 vagy 4
  \end{tcolorbox}
\end{frame}


\begin{frame}
  \begin{tcolorbox}[title={3/9. {\symknight}}]
     Legyenek $\{v_1,v_2,v_3\}$ lineárisan független vektorok. Adjuk meg a $\{v_1 + v_2,v_2,2v_1 + v_2\}$ által generált altér egy bázisát.

  \tcblower
A feladathoz jó arra emlékezni, hogy ha egy U altér egy generátorrendszeréből kiválasztunk olyan vektorokat, amelyek egyrészt függetlenek, másrészt U adott generátorai már kifejezhetők velük (vagyis ez egy maximális független rendszer U adott generátorrendszerében), akkor ez a kiválasztott független rendszer bázis lesz U-ban.\\
    \mmedskip 
  
     Pl. : $\{v_1 + v_2,v_2\}$
  \end{tcolorbox}
\end{frame}


\begin{frame}
  \begin{tcolorbox}[title={3/10. {\symqueen}}]
     Adott egy ötelemű vektorrendszer, melynek a rangja $3.$ Eltávolítunk a rendszerből két vektort. Mik az új rendszer rangjának lehetséges értékei?
  \tcblower
    A feladathoz tudni kell, hogy a rang a generált altér dimenziója.\\
    \mmedskip
    
    Ha tehát egy háromdimenziós altér ötelemű generátorrendszeréből eltávolítunk két vektort, akkor a maradék vektorrendszer által generált altér még mindig legalább egydimenziós (hiszen még egy háromelemű független rendszerből is maximum kettőt távolíthattunk el), de maradhatott is háromdimenziós (ha pl. a két „fölösleges” vektort dobtuk ki.)\\
    \mmedskip 
  
     A rang lehet: 1, 2 vagy 3
  \end{tcolorbox}
\end{frame}


\begin{frame}
  \begin{tcolorbox}[title={3/11. {\symqueen}}]
     Legyen $U = \{[x_1,x_2,x_3]^T \in \mathbb{R}^3|x_1 = 2x_2 = 3x_3\}$ az $\mathbb{R}^3$ altere. Hány olyan bázisa van az $U$ altérnek, mely tartalmazza a $v = [6,3,2]^T$ vektort?
  \tcblower  
    A feladatban egy egydimenziós alterünk van, s ebben bármely nem nulla vektor egyúttal bázis is.\\
    \mmedskip
    
    Ha a tér dimenziója nagyobb lenne, akkor – hivatkozva arra a tételre, mely szerint független rendszer kiegészíthető bázissá – máris megváltozna a helyzet:\\
    \mmedskip
    
    ilyenkor bármely, a dimenziószámnál kisebb elemszámú vektorhalmaz végtelen sok módon egészíthető ki bázissá.\\
    \mmedskip 
    Bázisok száma: 1
  \end{tcolorbox}
\end{frame}


\begin{frame}[plain]
\begin{tcolorbox}[center, colback={myyellow}, coltext={black}, colframe={myyellow}]
    {\RHuge  (4) Bázis megadása, dimenzió}
    \mmedskip
\end{tcolorbox}
\end{frame}


\begin{frame}
  \begin{tcolorbox}[title={4/1. {\symknight}}]
      Álljon az $U$ altér az $\mathbb{R}^3$ azon $v$ vektoraiból, ahol a második és a harmadik komponens egyenlő. Adjunk meg egy bázist $U$-ban.
  \tcblower
    Az feladatban föl kell írni az altér egy általános elemét és lineáris kombinációra bontani.\\
    \mmedskip 
    
    Pl. $[1 \; 0 \; 0]^T$, $[0 \; 1 \; 1]^T$
  \end{tcolorbox}
\end{frame}


\begin{frame}
  \begin{tcolorbox}[title={4/2. {\symknight}}]
      Legyen $\{b_1,b_2,b_3\}$ bázis egy $U$ altérben. Hány dimenziós alteret generál $\{b_1 + b_2,b_2,b_1 -b_2\}$?
  \tcblower
    A feladat kapcsolódik a rang fogalmához is: a generált altér dimenziójának kiszámításához maximális független rendszert kell keresni a generátorok között (azaz olyat, amiből a többi generátor már kifejezhető). Az adott példában $\{b_1 + b_2,b_2\}$ ilyen. De bázist alkot ebben az altérben $\{b_1,b_2\}$ is, hiszen $b_1 = (b_1 + b_2) - b_2$ is eleme az altérnek.\\
    \mmedskip 
    
    Az altér dimenziója: 2
  \end{tcolorbox}
\end{frame}


\begin{frame}
  \begin{tcolorbox}[title={4/3. {\symrook}}]
      Hány dimenziós azon $\mathbb{R}^n$-beli vektoroknak az altere, melyekben a komponensek összege $0$?
  \tcblower
    A feladatra (ez a 4. gyakorlat 7/g feladatának $y = 0$ speciális esete) gondolhatunk úgy, hogy egy homogén lineáris összefüggésünk van: az, hogy az elemek összege nulla, és ez eggyel csökkenti $\mathbb{R}^n$ dimenzióját. (Valójában egy homogén lineáris egyenletrendszer szabad változóit számoltuk meg. Bázist alkotnak azok a vektorok, melyekben az első $n-1$ komponens egyike 1, az utolsó komponens-1, a többi komponens pedig 0, de ezt megadni Q szintű feladat lenne az általános n miatt. Próbálkozhatunk a triviális bázis elemeinek „utánzásával”, de ellenőrizni kell a függetlenséget.)\\
    \mmedskip 
    
    A dimenzió: $n-1$
  \end{tcolorbox}
\end{frame}


\begin{frame}
  \begin{tcolorbox}[title={4/4. {\symrook}}]
      Hány dimenziós azon $\mathbb{R}^n$-beli vektoroknak az altere, melyekben a komponensek négyzetösszege $0$?
  \tcblower
    A feladat mutatja, mennyire gondosnak kell lennünk az előző feladat megoldási elvének alkalmazásakor. Ha ugyanis a komponensek közötti összefüggés nem lineáris, akkor egész más lehet az eredmény: itt csak a nullvektor van benne az altérben. Sőt, nem lináris összefüggéssel megadott vektorok általában nem is alkotnak alteret.\\
    \mmedskip 
    
    A dimenzió: $0$
  \end{tcolorbox}
\end{frame}


\begin{frame}
  \begin{tcolorbox}[title={4/5. {\symqueen}}]
      Hány dimenziós alteret alkotnak $\mathbb{R}^{3 x 3}$-ban azok az A mátrixok, amelyekre teljesül, hogy $A^T = -A$?
  \tcblower
    Az feladat is hasonló jellegű, csak itt sokkal több homogén lineáris összefüggés van a komponensek között. Az ilyen mátrixokban a főátló minden eleme 0, a főátlóra szimmetrikus elemek pedig egymás ellentettjei. Ezért pl. a főátló fölötti elemeket szabad változónak tekintve a mátrix elemei már egyértelműen meg vannak határozva, ezért a dimenzió a szabadon választható mátrixelemek száma, azaz 3. (Itt is érdemes gyakorlásul egy bázist fölírni.)\\
    \mmedskip 
    
    A dimenzió: $3$
  \end{tcolorbox}
\end{frame}


\begin{frame}
  \begin{tcolorbox}[title={4/6. {\symrook}}]
      Hány négydimenziós altér van $\mathbb{R}^4$-ben mint $\mathbb{R}$ fölötti vektortérben?
  \tcblower
    A feladatban azt kell tudni, hogy $\mathbb{R}^n$ egy valódi alterének dimenziója határozottan kisebb, mint $\mathbb{R}^n$-é.\\
    \mmedskip 
    
    Alterek száma: $1$
  \end{tcolorbox}
\end{frame}


\begin{frame}
  \begin{tcolorbox}[title={4/7. {\symqueen}}]
      Hány olyan altér van $\mathbb{R}^{2 x 2}$-ben mint $\mathbb{R}$ fölötti vektortérben, mely tartalmazza az összes $2 x 2$-es szimmetrikus mátrixot?
  \tcblower
    A feladat hasonló jellegű, először azt kell észrevenni, hogy a $2x_2$-es szimmetrikus mátrixok terének a dimenziója mindössze 1-gyel kisebb, mint a $\mathbb{R}^{2 x 2}$ téré, így a szimmetrikus mátrixok altere és az egész tér között nincs más altér.\\
    \mmedskip 
    
    Alterek száma: $2$
  \end{tcolorbox}
\end{frame}


\begin{frame}[plain]
\begin{tcolorbox}[center, colback={myyellow}, coltext={black}, colframe={myyellow}]
    {\RHuge  (5) Lineáris egyenletrendszerek, megoldásszámuk}
    \mmedskip
\end{tcolorbox}
\end{frame}

\begin{frame}
  \begin{tcolorbox}[title={5/1. {\symknight}}]
      Adjunk meg egy olyan lineáris egyenletrendszert, melyben két egyenlet van, három ismeretlen, és az egyenletrendszernek nincs megoldása.
  \tcblower

    \mmedskip 
    
    Pl.:\\
$x + y + z = 2$\\
$x + y + z = 3$
  \end{tcolorbox}
\end{frame}


\begin{frame}
  \begin{tcolorbox}[title={5/2. {\symknight}}]
      Adjunk meg egy olyan inhomogén lineáris egyenletrendszert, melyben három egyenlet van, két ismeretlen, és az egyenletrendszernek végtelen sok megoldása van.
  \tcblower

    \mmedskip 
    
    Pl.:\\
$x + y = 1$\\
$2x + 2y = 2$\\
$3x + 3y = 3$

  \end{tcolorbox}
\end{frame}


\begin{frame}
  \begin{tcolorbox}[title={5/3. {\symknight}}]
      Mi lehet a megoldások száma egy olyan valós együtthatós lineáris egyenletrendszernél, melyben az egyenletek száma 3, az ismeretlenek száma pedig 5?
  \tcblower

    \mmedskip 
    
    0 vagy végtelen
  \end{tcolorbox}
\end{frame}


\begin{frame}
  \begin{tcolorbox}[title={Megoldás {\symking}}]
Lineáris egyenletrendszerek megoldásszámára egy nagyon fontos összefüggés van: ha több ismeretlen van, mint egyenlet, akkor nem lehet egyértelmű a megoldás (erre kérdez rá a harmadik feladat).\\
\mmedskip

Vagyis ha az egyenletrendszerben nincs elég egyenlet – nem tudunk eleget az ismeretlenekről –, akkor ugyan lehet ellentmondásos (ehhez már két egyenlet is elég, mint az első feladatban, sőt az egyetlen $0 \cdot x + 0 \cdot y = 1$ egyenlet is), de ha nem ellentmondásos, azaz van megoldás, akkor biztosan egynél több megoldás van.\\
\mmedskip

Valós és komplex együtthatós egyenletrendszerek esetében ilyenkor a megoldások száma végtelen.\\
\mmedskip

Az egyenletek számát szaporíthatjuk úgy, hogy a megoldások halmaza ne változzon: ehhez a rendszerbe már bent lévő egyenletek lineáris kombinációját kell bevenni (mint a második feladatban). Fontos, hogy homogén lineáris egyenletrendszernek – amikor az egyenletek jobb oldalán mindenütt 0 áll – mindig van megoldása: a triviális megoldás, amikor minden ismeretlen értéke nulla.
  \end{tcolorbox}
\end{frame}



\begin{frame}[plain]
\begin{tcolorbox}[center, colback={myyellow}, coltext={black}, colframe={myyellow}]
    {\RHuge  (6) Mátrixműveletek, inverz }
    \mmedskip
\end{tcolorbox}
\end{frame}

\begin{frame}
  \begin{tcolorbox}[title={6/1. {\symknight}}]
    Legyen $A \in \mathbb{R}^{k \; x \; l}$, $B \in \mathbb{R}^{p \; x \; r}$ és $C \in \mathbb{R}^{t \; x \; u}$. Adjunk szükséges és elégséges feltételt arra, hogy értelmes legyen az $AB^T + C$ mátrixkifejezés.
  \tcblower
    $l = r, \; k = t, \; p = u$
  \end{tcolorbox}
\end{frame}


\begin{frame}
  \begin{tcolorbox}[title={6/2. {\symknight}}]
    Legyen $A =$ $\begin{bmatrix} 
  				1 & 1 \\
  				1 & 1 \\
			\end{bmatrix}$. Adjunk meg egy olyan $2 x 2$-es, egész elemű, a nullmátrixtól különböző $B$ mátrixot, amelyre igaz, hogy $AB = 0$.

  \tcblower

    \mmedskip 
    
    Pl. $B =$ $\begin{bmatrix} 
  				1 & 1 \\
  				-1 & -1 \\
			\end{bmatrix}$
  \end{tcolorbox}
\end{frame}


\begin{frame}
  \begin{tcolorbox}[title={6/3. {\symknight}}]
     Legyen $A =$ $\begin{bmatrix} 
  				1 & 3 \\
  				3 & c \\
			\end{bmatrix}$ Mely $c \in \mathbb{R}$ számokra van olyan $2 x 2$-es, valós elemű, a nullmátrixtól különböző $B$ mátrix, amelyre igaz, hogy $AB = 0$?
  \tcblower

    \mmedskip 
    
    $c = 9$
  \end{tcolorbox}
\end{frame}


\begin{frame}
  \begin{tcolorbox}[title={6/4. {\symknight}}]
     Adjunk meg egy olyan $2 x 2$-es egész elemű, a nullmátrixtól különböző A mátrixot, amelyre igaz, hogy van olyan $0 \neq B \in \mathbb{R}^{2 x 2}$, melyre $AB = 0$.
  \tcblower

    \mmedskip 
    
    Pl. A = $\begin{bmatrix} 
  				1 & 1 \\
  				1 & 1 \\
			\end{bmatrix}$
  \end{tcolorbox}
\end{frame}


\begin{frame}
  \begin{tcolorbox}[title={6/5. {\symknight}}]
     Adjuk meg az $A =$ $\begin{bmatrix} 
  				1 & 2 \\
  				4 & 3 \\
			\end{bmatrix}$ mátrix inverzét.
  \tcblower

    \mmedskip 
    
    $\begin{bmatrix} 
  			-{\frac{3}{5}} & {\frac{2}{5}} \\
  			-{\frac{4}{5}} & -{\frac{1}{5}} \\
			\end{bmatrix}$
  \end{tcolorbox}
\end{frame}


\begin{frame}
  \begin{tcolorbox}[title={6/6. {\symknight}}]
     Adjuk meg az $A = [1 \; 2 \; 3]$ sormátrix egy jobb oldali inverzét.
  \tcblower

    \mmedskip 
    
     Pl.: $[1 \; 0 \; 0]^T$
  \end{tcolorbox}
\end{frame}


\begin{frame}
  \begin{tcolorbox}[title={6/7. {\symknight}}]
     Hány olyan jobb oldali inverze van az $A = [1 \; 1 \; 1]$ sormátrixnak, melyben minden elem egyenlő?
  \tcblower

    \mmedskip 
    
     Számuk: 1
  \end{tcolorbox}
\end{frame}


\begin{frame}
  \begin{tcolorbox}[title={6/8. {\symqueen}}]
     Az alábbi mátrixok közül melyeknek nem lehet jobb oldali inverze a valós paraméterek semmilyen választására sem (azonos betűk azonos számokat jelölnek)? 
     
    (A) $\begin{bmatrix} 
  				a & b & c \\
  				a & b & c \\
			\end{bmatrix}$\\
	(B) $\begin{bmatrix} 
  				a & b & b \\
  				c & d & d \\
			\end{bmatrix}$\\
	(C) $\begin{bmatrix} 
  				a & b \\
  				c & d \\
  				e & f \\
			\end{bmatrix}$\\
	(D) $\begin{bmatrix} 
  				1 & 2 & 3 \\
  				4 & 5 & 6 \\
  				7 & 8 & a
			\end{bmatrix}$
  \tcblower

    \mmedskip 
    
     (A), (C)
  \end{tcolorbox}
\end{frame}


\begin{frame}
  \begin{tcolorbox}[title={Megoldások {\symking}}]
Az első feladat arra kérdez rá, hogy mikor végezhető el a mátrixok összeadása, szorzása, illetve mik a transzponált mátrix méretei. Két mátrix szorzatának oszlopaiban a bal oldali mátrix oszlopainak lineáris kombinációi állnak. Ezért az $AB = 0$ akkor tud teljesülni nem nulla B mátrixszal, ha A oszlopai lineárisan öszefüggenek.\\
\mmedskip

Speciálisan ha A négyzetes mátrix, akkor ez azzal ekvivalens, hogy A determinánsa nulla.\\
\mmedskip

Ezzel a tétellel könnyű megválaszolni a harmadik feladatot, illetve keresni megfelelő mátrixot a negyedik feladatban. \\
\mmedskip

Kétszer kettes mátrix esetében érdemes megjegyezni az inverz képletét: inverz akkor létezik, ha a mátrix d determinánsa nem nulla, és ekkor az inverz mátrixot úgy kapjuk, hogy az eredeti mátrixban a főátló két elemét kicseréljük, a mellékátló elemeit ellentettjükre változtatjuk, végül a mátrix minden elemét elosztjuk d-vel (így kaptuk az ötödik feladat eredményét).\\
\mmedskip

Nem feltétlenül négyzetes mátrixra is érvényes, hogy pontosan akkor van jobb oldali inverze, ha rangja megegyezik a sorainak a számával, azaz ha a sorai lineárisan függetlenek.\\
\mmedskip

Erre a tételre a hatodik és a hetedik feladat megoldásában nincs szükség, mert azok megoldása közvetlen számolás (a hetedik feladatban az egyetlen megfelelő vektor $[\frac{1}{3} \; \frac{1}{3} \; \frac{1}{3}]^T$).\\
\mmedskip

A nyolcadik feladatban viszont ezt a tételt alkalmazzuk.\\
Az (A) rész két sora nyilván összefügg.\\
A (C) esetében ez azért igaz, mert $\mathbb{R}^2$ dimenziója 2, és így bármely három vektor összefüggő.\\
A (B)-ben például az $a = d = 1$ és $b = c = 0$ két független vektort $ad$.\\
A0 (D) esetben meg a három sor független lesz ha a $\neq 9$, hiszen ekkor a determináns értéke nem nulla.
  \end{tcolorbox}
\end{frame}


\begin{frame}[plain]
\begin{tcolorbox}[center, colback={myyellow}, coltext={black}, colframe={myyellow}]
    {\RHuge  (7) Permutációk, inverziók, determinánsok}
    \mmedskip
\end{tcolorbox}
\end{frame}

\begin{frame}
  \begin{tcolorbox}[title={7/1. {\symknight}}]
    Az A = $[a_{ij}] \in \mathbb{R}^{5 \; x \; 5}$ mátrix determinánsának kiszámolásakor mennyi lesz az $a_{31}a_{24}a_{53}a_{15}a_{42}$ szorzathoz tartozó permutációban az inverziók száma?

  \tcblower

    \mmedskip 
    
    7
  \end{tcolorbox}
\end{frame}


\begin{frame}
  \begin{tcolorbox}[title={7/2. {\symrook}}]
    Az $A = [a_{ij}] \in \mathbb{R}^{5 \; x \; 5}$ mátrix determinánsának kiszámolásakor az $a_{11}a_{2i}a_{33}a_{45}a_{5j}$ szorzatot $(-1)$-gyel szorozva kellett figyelembe venni. Határozzuk meg $i$-t és $j$-t.
  \tcblower

    \mmedskip 
    
    $i = j = 2 4$
  \end{tcolorbox}
\end{frame}


\begin{frame}
  \begin{tcolorbox}[title={7/3. {\symqueen}}]
    Az $\{1,2,3,4,5\}$ számoknak hány olyan sorbarendezése (permutációja) van, melyben az inverziók száma 1?
  \tcblower

    \mmedskip 
    
     Ilyenek száma: 4
  \end{tcolorbox}
\end{frame}


\begin{frame}
  \begin{tcolorbox}[title={7/4. {\symqueen}}]
    Az $\{1,2,3,4,5\}$ számok $i2jkl$ típusú sorbarendezései (azaz permutációi) között mennyi lehet az inverziók maximális száma?
  \tcblower

    \mmedskip 
    Max. inverziószám: 8
  \end{tcolorbox}
\end{frame}

\begin{frame}
  \begin{tcolorbox}[title={7/5. {\symknight}}]
    Legyenek $A,B,C \in \mathbb{R}^{3 \; x \; 3}$ olyan valós mátrixok, melyekre $det \; A = 2, det \; B = 3$ és $det \; C = 4$. Mennyi lesz $2A^2C^{-1}B$ determinánsa?
  \tcblower

    \mmedskip 
    
     $2^3  \cdot 2^2  \cdot (1/4) \cdot 3 = 24$
  \end{tcolorbox}
\end{frame}

\begin{frame}
  \begin{tcolorbox}[title={7/6. {\symrook}}]
    Legyen $A =$ $\begin{bmatrix} 
  				a & b & c \\
  				d & e & f \\
  				g & h & i
			\end{bmatrix}$ $\in \mathbb{R}^{3 \; x \; 3}$, $B = $ $\begin{bmatrix} 
  				d & a & g + d \\
  				e & b & h + e \\
  				f & c & h + f
			\end{bmatrix}$ és tegyük föl, hogy det A = 5. Mennyi lesz det B?
  \tcblower

    \mmedskip 
    
     detB = -5
  \end{tcolorbox}
\end{frame}


\begin{frame}
  \begin{tcolorbox}[title={Megoldások {\symking}}]
    A determináns definíciójában a tagok előjelezését a következőképpen kell kiszámolni.\\
    \mmedskip
    
    A szorzatot rendezzük át úgy, hogy az $a_{ij}$-k első indexei növekedjenek 1-től n-ig.\\
    Ezek után a második indexek által alkotott permutációban számoljuk meg az inverziókat (vagyis azokat a párokat, amelyekben az elöl szereplő szám a nagyobb).\\
    \mmedskip
    
    Ha ezek száma páros, akkor az előjel +, különben -.\\
    \mmedskip
    
    Az első feladatban a kapott sorrend 54123.\\
    Ha n elemet permutálunk, akkor az inverziók maximális száman 2; ezt a számot az elemek monoton fogyó sorbarendezése produkálja, 0 inverziója pedig csak a természetes, monoton növő sorbarendezésnek van.\\
    \mmedskip
    
    Két elemet megcserélve, a permutáció paritása (előjele) mindig megváltozik.\\
    \mmedskip
    
    A második példában i és j csak 2 és 4 lehet valamelyik sorrendben, ezt a két permutációt kell kipróbálni.\\
    \mmedskip
    
    A harmadik feladatban csak két szomszédos elemet cserélhetünk meg az alapsorrendhez képest.\\
    \mmedskip
    
    A negyedik feladatban az az ötlet, hogy a 2 után legalább két, 2-nél nagyobb számnak kell lennie, ez két nem-inverziót elront, és így maximum 8 inverzió lehet. Az 52431 sorrend ezt elő is állítja.\\
    \mmedskip
    
    Az ötödik feladat a determinánsok szorzástételének egyszerű alkalmazása (amiből következik, hogy egy mátrix inverzének a determinánsa az eredeti mátrix determinánsának a reciproka).\\
    \mmedskip
    
    A hatodik feladatban az eredeti mátrixot transzponáltuk, megcseréltük az első két oszlopot, majd az új első oszlopot adtuk a harmadikhoz. A középső lépésnél a determináns előjelet vált, a másik kettőnél nem változik.
  \end{tcolorbox}
\end{frame}


\begin{frame}[plain]
\begin{tcolorbox}[center, colback={myyellow}, coltext={black}, colframe={myyellow}]
    {\RHuge  (8) Jobb oldali sajátértékek és sajátvektorok, diagonalizálhatóság}
    \mmedskip
\end{tcolorbox}
\end{frame}

\begin{frame}
  \begin{tcolorbox}[title={8/1. {\symknight}}]
    Mik az A = $\begin{bmatrix} 
  				1 & 2 & 3 \\
  				0 & 4 & 5 \\
  				0 & 0 & 6
			\end{bmatrix}$ $\mathbb{R}^{3 \; x \; 3}$ mátrix sajátértékei? 

  \tcblower

    \mmedskip 
    
    1, 4, 6
  \end{tcolorbox}
\end{frame}


\begin{frame}
  \begin{tcolorbox}[title={8/2. {\symknight}}]
     Írjunk föl egy olyan mátrixot, amelynek karakterisztikus polinomja $(1-{\lambda})(2-{\lambda})(3-{\lambda})$.

  \tcblower

    \mmedskip 
    
     Pl. A = $\begin{bmatrix} 
  				1 & 0 & 0 \\
  				0 & 2 & 0 \\
  				0 & 0 & 3
			\end{bmatrix}$
  \end{tcolorbox}
\end{frame}


\begin{frame}
  \begin{tcolorbox}[title={8/3. {\symrook}}]
    Írjunk föl egy olyan nem diagonalizálható mátrixot, melynek karakterisztikus polinomja ${\lambda}^2$.

  \tcblower

    \mmedskip 
    
    Pl. $A =$ $\begin{bmatrix} 
  				0 & 1 \\
  				0 & 0 \\
			\end{bmatrix}$
  \end{tcolorbox}
\end{frame}


\begin{frame}
  \begin{tcolorbox}[title={8/4. {\symqueen}}]
    Az $A = $ $\begin{bmatrix} 
  				0 & 1 \\
  				c & 0 \\
			\end{bmatrix}$ $ \in \mathbb{R}^{2 x 2}$ mátrix nem diagonalizálható $\mathbb{R}$ fölött. Mik $c$ lehetséges értékei?


  \tcblower

    \mmedskip 
    
    $c \leq 0$
  \end{tcolorbox}
\end{frame}


\begin{frame}
  \begin{tcolorbox}[title={8/5. {\symqueen}}]
    Milyen $c \in \mathbb{R}$ valós értékekre lesz az $\begin{bmatrix} 
  				1 & 2 \\
  				0 & c \\
			\end{bmatrix}$ $\in \mathbb{R}^{2 x 2}$ mátrix diagonalizálható?

  \tcblower

    \mmedskip 
    
     $c \neq 1$
  \end{tcolorbox}
\end{frame}


\begin{frame}
  \begin{tcolorbox}[title={8/6. {\symqueen}}]
     Az $A =$ $\begin{bmatrix} 
  				a & b \\
  				c & d \\
			\end{bmatrix}$ $\in \mathbb{R}^{2 x 2}$ mátrixra A = $\begin{bmatrix} 
  				1 \\
  				2 \\
			\end{bmatrix}$ = $\begin{bmatrix} 
  				-1 \\
  				-2 \\
			\end{bmatrix}$. Mennyi az $\begin{bmatrix} 
  				a + 1 & b \\
  				c & d + 1 \\
			\end{bmatrix}$ mátrix determinánsa?

  \tcblower

    \mmedskip 
    
     det $\begin{vmatrix} 
  				a + 1 & b \\
  				c & d + 1 \\
			\end{vmatrix}$ = 0
  \end{tcolorbox}
\end{frame}


\begin{frame}
  \begin{tcolorbox}[title={Megoldások {\symking}}]
     Egy mátrix valós sajátértékeit a karakterisztikus polinom valós gyökei adják: a főátló minden eleméből levonunk ${\lambda}$-t és kiszámítjuk a determinánst.\\
     \mmedskip
     
     Felső háromszögmátrix, speciálisan diagonális mátrix esetén a sajátértékek a főátlóból leolvashatók.\\
     \mmedskip
     
     A sajátvektorokat minden egyes sajátértékhez egy-egy lineáris egyenletrendszer megoldásával kaphatjuk meg.\\
     A diagonalizálhatóság feltétele az, hogy legyen „elegendő” sajátvektor, azaz a tér dimenziószámával megegyező mennyiségű lineárisan független sajátvektor.\\
     \mmedskip
     
     A harmadik feladat mátrixa esetében ez nem teljesül, mert a sajátvektorokr $[r \; 0]^T$ alakúak, ahol r $\neq 0$ (azaz a 0-hoz tartozó sajátaltér egydimenziós).\\
     Gyakran használt elégséges feltétel, hogy egy nxn-es mátrix diagonalizálható $\mathbb{R}$ fölött, ha n különböző valós sajátértéke van.\\
     \mmedskip
     
     A negyedik feladat mátrixának karakterisztikus polinomja ${\lambda}^2$-c. Ennek c > 0-ra két különböző valós gyöke van, tehát ilyenkor a mátrix diagonalizálható; $c < 0$-ra nincs valós gyök, ezért $\mathbb{R}$ fölött nem diagonalizálhatjuk a mátrixot; végül c = 0-ra a megoldást a harmadik feladat adja.\\
     \mmedskip
     
     Hasonló érvelés adható az ötödik feladatnál is. A hatodik feladat feltétele azt mutatja, hogy a -1 sajátértéke a mátrixnak, ha tehát -1-et levonunk a főátló minden eleméből, épp az $A-(-1)I_2$ mátrixot kapjuk, aminek a determinánsa 0, hiszen -1 gyöke a karakterisztikus polinomnak.
  \end{tcolorbox}
\end{frame}



\begin{frame}[plain]
\begin{tcolorbox}[center, colback={myyellow}, coltext={black}, colframe={myyellow}]
    {\RHuge  (9) Lineáris transzformációk mátrixa, sajátértékei, kép- és magtere; báziscsere }
    \mmedskip
\end{tcolorbox}
\end{frame}

\begin{frame}
  \begin{tcolorbox}[title={9/1. {\symknight}}]
    Mi a kétdimenziós valós sík, azaz $\mathbb{R}^2$ origó körüli $+90$ fokos forgatásának mátrixa az i és j bázisban?

  \tcblower

    \mmedskip 
    
    $[{\varphi}]^{i,j;i,j} =$ $\begin{bmatrix} 
  				0 & -1 \\
  				1 & 0 \\
			\end{bmatrix}$
  \end{tcolorbox}
\end{frame}


\begin{frame}
  \begin{tcolorbox}[title={9/2. {\symrook}}]
    Mi a háromdimenziós valós tér, azaz $\mathbb{R}^3$ x-y-síkra való tükrözésének a mátrixa az $j,k,i$ bázisban? (Itt a bázisvektorok sorrendje a megszokottól eltérő.)

  \tcblower

    \mmedskip 
    
     $[{\varphi}]^{j,k,i;j,k,i} = $ $\begin{bmatrix} 
  				1 & 0 & 0 \\
  				0 & -1 & 0 \\
  				0 & 0 & 1 \\
			\end{bmatrix}$
  \end{tcolorbox}
\end{frame}


\begin{frame}
  \begin{tcolorbox}[title={9/3. {\symknight}}]
    Mik a kétdimenziós valós sík, azaz $\mathbb{R}^2$ origóra való tükrözésének sajátértékei?

  \tcblower

    \mmedskip 
    
    -1
  \end{tcolorbox}
\end{frame}


\begin{frame}
  \begin{tcolorbox}[title={9/4. {\symrook}}]
    Az $\mathbb{R}^2 egy {\varphi}$ lineáris transzformációjára ${\varphi}(i) = 2i$ és ${\varphi}(j) = i + 3j$. Mik ${\varphi}$ sajátértékei? 
    
  \tcblower

    \mmedskip 
    
     2 és 3
  \end{tcolorbox}
\end{frame}


\begin{frame}
  \begin{tcolorbox}[title={Megoldások {\symking}}]
    Minden lineáris transzformációhoz rögzített bázis esetén egyértelműen tartozik egy négyzetes mátrix.\\
    \mmedskip
    
    Ezt úgy írhatjuk föl, hogy a bázisvektorokat leképezzük a lineáris transzformációval, e képvektorokat felírjuk a megadott bázisban, és az így kapott koordinátavektorokat írjuk a mátrix oszlopaiba.\\
    Ha változtatunk a bázison, akkor változhat a mátrix is; ez kiszámítható a bázistranszformáció képletéből $(A' = S^{-1}AS)$.\\
    \mmedskip
    
    Egy lináris transzformáció sajátértékeit kiszámolhatjuk úgy, hogy felírjuk a mátrixát egy alkalmas bázisban, és ennek vesszük a sajátértékeit.\\
    Ez a negyedik (kétlépcsős) feladat: a mátrix a triviális bázisban  $\begin{bmatrix} 
  				2 & 1 \\
  				0 & 3 \\
			\end{bmatrix}$.\\
    \mmedskip
    
    Néha lehetséges közvetlenül a definíció segítségével is boldogulni: a harmadik feladatban minden vektor az ellentettjébe megy, ezért az egyetlen sajátérték a $-1$.\\
  \end{tcolorbox}
\end{frame}


\begin{frame}[plain]
\begin{tcolorbox}[center, colback={myyellow}, coltext={black}, colframe={myyellow}]
    {\RHuge  (10) Skaláris szorzat, vektorok szöge }
    \mmedskip
\end{tcolorbox}
\end{frame}

\begin{frame}
  \begin{tcolorbox}[title={10/1. {\symknight}}]
    Mennyi az $[1 \; 1 \; 1 \; 1]^T$ és az $[1 \; -1 \; -1 \; -1]^T$ vektorok szöge $\mathbb{R}^4$-ben, a szokásos ha, ${\langle}a, b{\rangle} = a^Tb$ skaláris szorzatra nézve?

  \tcblower

    \mmedskip 
    
    Szögük: $120^{\circ}$

  \end{tcolorbox}
\end{frame}


\begin{frame}
  \begin{tcolorbox}[title={10/2. {\symrook}}]
    Az $[1 \; 1 + i \; i]^T \in \mathbb{C}^3$ vektor ha, ${\langle}a, b{\rangle} = b*a$ mellett merőleges az $[1 \; 1 + i \; c]^T$ vektorra. Határozzuk meg $c \in\mathbb{C}$ értékét.


  \tcblower

    \mmedskip 
    
    $c = -3i$
  \end{tcolorbox}
\end{frame}


\begin{frame}
  \begin{tcolorbox}[title={10/3. {\symrook}}]
    Álljon $W$ az $\mathbb{R}^3$ azon vektoraiból, amelyek merőlegesek $a[0 \; 1 \; 1]^T$ és $[1 \; 1 \; 0]^T$ vektorok mindegyikére. Adjunk meg egy bázist $W$-ben.

  \tcblower

    \mmedskip 
    
    $\{[1 \; -1 \; 1]^T\}$
  \end{tcolorbox}
\end{frame}


\begin{frame}
  \begin{tcolorbox}[title={10/4. {\symqueen}}]
    Mely $a,b \in \mathbb{R}$ értékekre lesz $|3a + 4b| = 5\sqrt{a_2 + b_2}$?

  \tcblower

    \mmedskip 
    
    $[a,b] = {\lambda}[3,4], {\lambda} \in \mathbb{R}$
  \end{tcolorbox}
\end{frame}


\begin{frame}
  \begin{tcolorbox}[title={Megoldások {\symking}}]
    A szokásos skaláris szorzatot $\mathbb{R}^n$-ben az ha, ${\langle}a, b{\rangle} = a^Tb$ összefüggés adja meg; két vektor merőlegessége azt jelenti, hogy a skaláris szorzatuk nulla.\\
    \mmedskip
    
    $\mathbb{R}^n$-ben a vektorok szögét a $cos{\gamma} =$ $\frac{{\langle}a, b{\rangle}}{||a|| \cdot ||b||}$ összefüggéssel definiáljuk, erről szól az első feladat. (A megoldáshoz érdemes átismételni a szögfüggvények értékét a nevezetes szögeken.)\\
    \mmedskip
    
    A második feladat esetében a skaláris szorzat kiszámításánál vigyázzunk arra, hogy a második tényező komponenseit meg kell konjugálni.\\
    \mmedskip
    
    A harmadik feladatnál lineáris egyenletrendszert kapunk W elemeinek komponenseire.\\
    \mmedskip
    
    A negyedik feladat esetében arra kell ráismerni, hogy a feladatbeli egyenlet az $u = [3 \; 4]^T$ és a $v = [a \; b]^T$ vektorokra írja föl az  $|{\langle}u, v{\rangle}| = ||u|| \cdot ||v||$ összefüggést, ami a Cauchy-egyenlőtlenség alapján akkor és csak akkor teljesülhet, ha $u$ és $v$ párhuzamosak.
  \end{tcolorbox}
\end{frame}


\begin{frame}[plain]
\begin{tcolorbox}[center, colback={myyellow}, coltext={black}, colframe={myyellow}]
    {\RHuge  (11) Geometriai vektorok: vektoriális és vegyes szorzat }
    \mmedskip
\end{tcolorbox}
\end{frame}

\begin{frame}
  \begin{tcolorbox}[title={11/1. {\symknight}}]
    Ha $[a]_{i,j,k} = [0 \; 1 \; 1]^T$ és a $[b]_{i,j,k} = [1 \; 1 \; 0]^T$, akkor számítsuk ki az $a$ és $b$ vektorok vektoriális szorzatát (koordinátákkal).

  \tcblower

    \mmedskip 
    
    $[-1 \; 1 \; -1]^T$
  \end{tcolorbox}
\end{frame}


\begin{frame}
  \begin{tcolorbox}[title={Megoldások {\symking}}]
    A vektoriális szorzat koordinátáinak kiszámításához érdemes a „determinásos” képletet használni:\\
    \mmedskip
    
    egy 3 x 3-as determináns első sorába az i, j, k vektorokat írjuk, a másik két sorba pedig a megadott vektorok koordinátáit, majd a determinánst (formálisan) kifejtjük az első sora szerint.\\
    Ekkor az i, j, k vektorok egy lineáris kombinációját kapjuk: ez lesz a vektoriális szorzat és így a leggyorsabb megoldani az első feladatot.\\
    Ebből a képletből láthatjuk azt is, hogy a vegyes szorzat is egy olyan determináns értéke, amelynek soraiban a három megadott vektor van alkalmas sorrendben.\\
    \mmedskip
    
    Speciálisan három vektor akkor és csak akkor lineárisan összefüggő, azaz akkor és csak akkor vannak egy síkban, ha a vegyes szorzatuk nulla. 
  \end{tcolorbox}
\end{frame}


\begin{frame}[plain]
\begin{tcolorbox}[center, colback={myyellow}, coltext={black}, colframe={myyellow}]
    {\RHuge  (12) Kvadratikus alakok jellege (definitsége)}
    \mmedskip
\end{tcolorbox}
\end{frame}

\begin{frame}
  \begin{tcolorbox}[title={12/1. {\symknight}}]
     Számítsuk ki az $\begin{bmatrix} 
  				1 & 2 \\
  				2 & 1 \\
			\end{bmatrix}$ mátrix által meghatározott kvadratikus alak értékét az $[1 \; 2]^T$ vektoron.

  \tcblower

    \mmedskip 
    
    13
  \end{tcolorbox}
\end{frame}


\begin{frame}
  \begin{tcolorbox}[title={12/2. {\symrook}}]
     Adjunk meg egy $v \in \mathbb{R}^2$ vektort, melyen az $A =$ $\begin{bmatrix} 
  				1 & 2 \\
  				2 & 1 \\
			\end{bmatrix}$  mátrix által meghatározott kvadratikus alak negatív értéket vesz föl.


  \tcblower

    \mmedskip 
    
    Pl. $v =$ $\begin{bmatrix} 
  				1 \\
  				-1 \\
			\end{bmatrix}$
  \end{tcolorbox}
\end{frame}


\begin{frame}
  \begin{tcolorbox}[title={12/3. {\symrook}}]
     Adjuk meg a $\begin{bmatrix} 
  				a & b & c \\
  				b & 1 & -2 \\
  				c & -2 & -1 \\
			\end{bmatrix}$ $\in \mathbb{R}^{3 \; x \; 3}$ mátrix által meghatározott bilineáris függvény kvadratikus karakterét (definitségét).


  \tcblower

    \mmedskip 
    
    Indefinit.
  \end{tcolorbox}
\end{frame}


\begin{frame}
  \begin{tcolorbox}[title={12/4. {\symknight}}]
     Milyen $c \in \mathbb{R}$ valós értékekre lesz az $\begin{bmatrix} 
  				1 & c \\
  				c & c \\
			\end{bmatrix}$ $\in \mathbb{R}^{2 x 2}$ mátrix által meghatározott kvadratikus alak pozitív definit?

  \tcblower

    \mmedskip 
    
    $0 < c < 1$
  \end{tcolorbox}
\end{frame}


\begin{frame}
  \begin{tcolorbox}[title={Megoldások {\symking}}]
     Egy szimmetrikus, valós A mátrixhoz tartozó kvadratikus alak $Q(u) = u^TAu$. Abban az esetben ha $A =$ $\begin{bmatrix} 
  				a & b \\
  				b & d \\
			\end{bmatrix}$ és $u = [x \; y]^T$, akkor $Q(u) = ax^2 + 2bxy + dy^2$.\\
			\mmedskip
			
			A második feladatban tehát olyan $x$ és $y$ számokat kell keresnünk, amelyekre $x_2 + 4xy + y2$ negatív. Ha észrevesszük, hogy $x_2 + 4xy + y2 = x_2 + 4xy + 4y2 -3y2 = (x + 2y)2 -3y2$, akkor már könnyen találunk ilyen értékeket (de az alábbiak szerint a mátrix egyik sajátvektora is megfelelő).\\
			
			A kvadratikus alak jellege (karaktere, definitsége) azt mondja meg, milyen valós értékeket vesz föl a nem nulla vektorokon. Ezt a tulajdonságot le tudjuk olvasni a mátrix sajátértékeiből (amik szimmetrikus mátrix esetén mindig valósak), pontosabban azok előjeléből:\\
			\mmedskip
			
			$\begin{vmatrix} 
			Q(u), u \neq 0 & \text{sajátértékek előjele} & \text{jelleg}\\
			\text{mindig pozitív} & \text{mind pozitív} & \text{pozitív definit}\\
			\text{mindig negatív} & \text{mind negatív} & \text{negatív definit}\\
			\text{mindig nemnegatív} & \text{mind nemnegatív} & \text{pozitív szemidefinit}\\
			\text{mindig nempozitív} & \text{mind nempozitív} & \text{negatív szemidefinit}\\
			\text{van pozitív is, negatív is} & \text{van pozitív is, negatív is} & \text{indefinit}\\
			\end{vmatrix}$
			
			\mmedskip
			
			Ha $u = e_i$ (a triviális bázis i-edik vektora), akkor $Q(u)$ az A mátrix főátlójának i-edik eleme. A harmadik feladatban a második és a harmadik diagonális elem pozitív, illetve negatív, s így kvadratikus alak pozitív és negatív értékeket is fölvesz, tehát biztosan indefinit.\\
			Bizonyos esetekben a következő kritérium segítségével is leolvasható a kvadratikus alak jellege. Legyen ${\Delta}_0 = 1$ és ${\Delta}_k$ a mátrix bal fölső sarkában lévő $k xk$-as részmátrix determinánsa (ez az ún. karakterisztikus sorozat). Pl. a karakterisztikus sorozat pontosan akkor áll csupa pozitív értékből, ha a kvadratikus alakunk pozitív definit.\\
			A negyedik feladatnál ezt a feltételt használjuk: az 1x1-es részmátrixhoz tartozó tag pozitív, s a pozitív definitség azzal lesz ekvivalens, hogy a determináns, $c-c^2 = c(1-c)$ is pozitív. Ez csak a jelzett intervallumban valósulhat meg.
  \end{tcolorbox}
\end{frame}


\begin{frame}[plain]
\begin{tcolorbox}[center, colback={myyellow}, coltext={black}, colframe={myyellow}]
    {\RHuge  B rész}
    \mmedskip
\end{tcolorbox}
\end{frame}


\begin{frame}
  \begin{tcolorbox}
     Az alábbi listában azok a definíciók és állítások, tételek szerepelnek, melyeket a vizsgadolgozat B részében kérdezhetünk. A válaszoknál zárójelben néhol magyarázó megjegyzések is vannak, ezeket nem kell leírni a teljes pontszám eléréséhez.
  \end{tcolorbox}
\end{frame}

\begin{frame}
  \begin{tcolorbox}[title={1}]
      Mit jelent az, hogy egy $W {\subseteq} \mathbb{R}^n$ részhalmaz altér?

  \tcblower
$W {\subseteq} \mathbb{R}^n$ altér $\mathbb{R}^n-ben$, ha:\\
\mmedskip

1) $W$ nem üres;\\
2) $a,b \in W$ esetén $a + b \in W$ (azaz W zárt az összeadásra);\\
3) $a \in W$ és ${\lambda} \in \mathbb{R}$ esetén ${\lambda}a \in W$ (azaz $W$ zárt a skalárral szorzásra). (Ahelyett, hogy $W$ nem üres, azt is írhatjuk, hogy $0 \in W$.)
  \end{tcolorbox}
\end{frame}


\begin{frame}
  \begin{tcolorbox}[title={2}]
      Definiáljuk, mit jelent az, hogy a $v_1,...,v_k \in \mathbb{R}^n$ vektorrendszer lineárisan független.

  \tcblower
A $v_1,...,v_k$ vektorrendszer akkor és csak akkor lineárisan független, ha csak a triviális lineáris kombinációja adja a nullvektort.\\
\mmedskip

Képletben: tetszőleges ${\lambda}_1,{\lambda}_2,...,{\lambda}_k \in \mathbb{R}$ esetén, ha ${\lambda}_1v_1 + {\lambda}_2v_2 +  \cdot  \cdot  \cdot  + {\lambda}_kv_k = 0$, akkor minden $i$-re ${\lambda}_i = 0$.
  \end{tcolorbox}
\end{frame}


\begin{frame}
  \begin{tcolorbox}[title={3}]
      Mit jelent az, hogy a $v_1,...,v_k \in \mathbb{R}^n$ vektorrendszer lineárisan összefüggő? A válaszban ne hivatkozzunk a lineáris függetlenség fogalmára.


  \tcblower
A $v_1,...,v_k$ vektorrendszer akkor és csak akkor lineárisan összefüggő (azaz nem lineárisan független), ha léteznek olyan ${\lambda}_1,{\lambda}_2,...,{\lambda}_k \in \mathbb{R}$ nem mind nulla számok, melyekre ${\lambda}_1v_1 + {\lambda}_2v_2 +  \cdot  \cdot  \cdot  + {\lambda}_kv_k = 0$.)

  \end{tcolorbox}
\end{frame}

\begin{frame}
  \begin{tcolorbox}[title={4}]
     Mit jelent az, hogy egy v vektor lineárisan függ az $a_1,...,ak \in \mathbb{R}^n$ vektoroktól?

  \tcblower
Azt jelenti, hogy v felírható $a_1,...,a_k$ lineáris kombinációjaként, azaz léteznek olyan ${\lambda}_1,{\lambda}_2,...,{\lambda}_k \in \mathbb{R}$ skalárok, melyekre $v = {\lambda}_1a_1 + {\lambda}_2a_2 +  \cdot  \cdot  \cdot  + {\lambda}_ka_k$.
  \end{tcolorbox}
\end{frame}


\begin{frame}
  \begin{tcolorbox}[title={5}]
    Jellemezzük egy $a_1,...,a_k \in \mathbb{R}^n$ vektorrendszer lineáris összefüggőségét a lineáris függés fogalmával.


  \tcblower
Egy $a_1,...,a_k \in \mathbb{R}^n$ vektorrendszer ($k \geq 2$ esetén) pontosan akkor lineárisan összefüggő, ha valamelyik i-re $a_i$ lineárisan függ az $a_1,...,ai-1,ai+1,...,a_k$ vektoroktól.
  \end{tcolorbox}
\end{frame}


\begin{frame}
  \begin{tcolorbox}[title={6}]
      Definiáljuk egy $V \leq \mathbb{R}^n$ altér bázisának fogalmát.


  \tcblower
Egy $b_1,...,b_k$ vektorrendszert akkor mondunk a $V \leq \mathbb{R}^n$ altér bázisának, ha lineárisan független, és V minden vektorát előállíthatjuk a $b_i$ vektorok lineáris kombinációjaként. (A lineáris függetlenség helyettesíthető azzal a feltétellel, hogy ez a felírás egyértelmű, az előállíthatóság pedig azzal, hogy generátorrendszerről van szó.)
  \end{tcolorbox}
\end{frame}


\begin{frame}
  \begin{tcolorbox}[title={7}]
      Definiáljuk egy $a \in V \leq \mathbb{R}^n$ vektor koordinátavektorát a $V$ egy $b_1,...,b_k$ bázisában fölírva.


  \tcblower
Az a vektor koordinátavektora pontosan akkor $[a]b_1,...,bk =$
${\lambda}_1 . . . {\lambda}_k$
, ha $a = {\lambda}_1b_1 + \cdot  \cdot  \cdot + {\lambda}_kb_k$.
  \end{tcolorbox}
\end{frame}


\begin{frame}
  \begin{tcolorbox}[title={8}]
       Egy $A {\subseteq} \mathbb{R}^n$ vektorhalmaz esetén adjuk meg az $A$ által generált altér egy jellemzését.


  \tcblower
Az A által generált altér azokból az $\mathbb{R}^n$-beli vektorokból áll, amelyek előállnak $A$-beli vektorok lineáris kombinációiként, azaz amelyek lineárisan függnek az $A$-beli vektoroktól. (Ez a halmaz megegyezik az $A$-t tartalmazó $\mathbb{R}^n$-beli alterek metszetével.)

  \end{tcolorbox}
\end{frame}



\begin{frame}
  \begin{tcolorbox}[title={9}]
     Mondjuk ki a lineárisan független rendszerek és a generátorrendszerek elemszámát összehasonlító tételt (ez a kicserélési tétel egyik része).



  \tcblower
Minden lineárisan független rendszer elemszáma legfeljebb akkora, mint bármelyik generátorrendszer elemszáma.

  \end{tcolorbox}
\end{frame}


\begin{frame}
  \begin{tcolorbox}[title={10}]
     Definiáljuk egy $V \leq \mathbb{R}^n$ altér dimenzióját, $dimV$ -t.

  \tcblower
$dimV$ a $V$ egy bázisának elemszáma, illetve 0, ha $V = \{0\}$. (Ez a definíció azért értelmes, mert bármely két bázis elemszáma egyenlő.)

  \end{tcolorbox}
\end{frame}



\begin{frame}
  \begin{tcolorbox}[title={11}]
   Definiáljuk egy $v_1,...,v_k \in \mathbb{R}^n$ vektorrendszer rangját, $r(v_1,...,v_k)$-t.


  \tcblower
$r(v_1,...,v_k) = dimSpan(v_1,...,v_k)$, azaz egy vektorrendszer rangja megegyezik az általa generált (kifeszített) altér dimenziójával.

  \end{tcolorbox}
\end{frame}


\begin{frame}
  \begin{tcolorbox}[title={12}]
    Adjuk meg képlettel két mátrix, $A \in R^{k \; x \; l}$ és $B \in \mathbb{R}^{l \; x \; n}$ szorzatában, $AB$-ben az $i$-edik sor $j$-edik elemét, $_{i} [AB]_j$-t. Azt is mondjuk meg, $i$ és $j$ milyen értékére létezik ez az elem.



  \tcblower
Ha $1 \leq i \leq k$ és $1 \leq j \leq n$, akkor $_{i} [AB]_j =$ $\sum_{t = 1}^l$ $_{i} [A]_t$ $\cdot$ $_{t} [B]_j$
  \end{tcolorbox}
\end{frame}



\begin{frame}
  \begin{tcolorbox}[title={13}]
    Mondjunk ki két, a mátrixok transzponálását a többi szokásos mátrixművelettel összekapcsoló összefüggést.

  \tcblower
$A,B \in \mathbb{R}^{k \; x \; n} {\Rightarrow} (A + B)^T = A^T + B^T$\\
${\lambda} \in \mathbb{R},A \in \mathbb{R}^{k \; x \; n} {\Rightarrow} ({\lambda}A)^T = {\lambda}A^T$\\
$A \in \mathbb{R}^{k \; x \; l}, B \in \mathbb{R}^{l \; x \; n} {\Rightarrow} (AB)^T = B^TA^T$

  \end{tcolorbox}
\end{frame}



\begin{frame}
  \begin{tcolorbox}[title={14}]
    Definiáljuk egy $A \in \mathbb{R}^{k \; x \; n}$ mátrix oszloprangját, illetve sorrangját, ${\varrho}o(A)$-t és ${\varrho}s(A)$-t.

  \tcblower
Ha $a_1,...,a_n \in \mathbb{R}^k$ a mátrix oszlopai, akkor ${\varrho}o(A) = r(a_1,...,a_k)$, azaz az oszloprang az oszlopok rendszerének rangja (vagyis az oszlopok által generált altér dimenziója.) Analóg módon, a mátrix sorrangja a sorok által generált altér dimenziója, vagy másképpen: ${\varrho}s(A) = {\varrho}o(A^T)$.

  \end{tcolorbox}
\end{frame}

\begin{frame}
  \begin{tcolorbox}[title={15}]
   Mondjuk ki a mátrixok szorzatának oszloprangjára vonatkozó becslést.


  \tcblower
Ha létezik az $AB$ mátrixszorzat, akkor ${\varrho}o(AB) \leq {\varrho}o(A)$. (Igaz a ${\varrho}o(AB) \leq {\varrho}o(B)$ becslés is.)

  \end{tcolorbox}
\end{frame}


\begin{frame}
  \begin{tcolorbox}[title={16}]
     Mit nevezünk egy $A \in \mathbb{R}^{n \; x \; m}$ mátrix jobb, illetve kétoldali inverzének?

  \tcblower
Az $A^{(j)} \in \mathbb{R}^{m \; x \; n}$ mátrix jobb oldali inverze $A$-nak, ha $AA^{(j)} = I_n$, ahol $I_n$ az $n \; x \; n$-es egységmátrix. $A^{-1} \in \mathbb{R}^{m \; x \; n}$ kétoldali inverze A-nak, ha jobb oldali és bal oldali inverze is A-nak, azaz $AA^{-1}$ = $I_n$, és $A^{-1}A = I_m$. (Ez utóbbi létezése esetén $m = n$.)

  \end{tcolorbox}
\end{frame}


\begin{frame}
  \begin{tcolorbox}[title={17}]
    A mátrixrang fogalmának fölhasználásával mondjuk ki annak szükséges és elégséges feltételét, hogy az $A \in \mathbb{R}^{n \; x \; m}$ mátrixnak létezzen jobb oldali inverze.

  \tcblower
Az $A \in \mathbb{R}^{n \; x \; m}$ mátrixnak pontosan akkor létezik jobb oldali inverze, ha ${\varrho}(A) = n$, azaz a mátrix rangja megegyezik a sorainak a számával.

  \end{tcolorbox}
\end{frame}


\begin{frame}
  \begin{tcolorbox}[title={18}]
   Definiáljuk a geometriai vektorok skaláris szorzatának fogalmát a vektorok hosszának és szögének segítségével.

  \tcblower
Jelölje $|a|$ az a geometriai vektor hosszát, ${\gamma}(a,b)$ pedig az $a$ és $b$ geometriai vektorok hajlásszögét. Ekkor az $a$ és $b$ skaláris szorzata $ab = |a||b|cos{\gamma}(a,b)$.

  \end{tcolorbox}
\end{frame}


\begin{frame}
  \begin{tcolorbox}[title={19}]
   Definiáljuk a geometriai vektorok vektoriális szorzatának fogalmát.
  \tcblower
Jelölje $|a|$ az a geometriai vektor hosszát, ${\gamma}(a,b)$ pedig az $a$ és $b$ geometriai vektorok hajlásszögét. Ekkor az $a$ és $b$ vektoriális szorzata az az $a x b$-vel jelölt vektor, melyre:\\

1) $|a x b| = |a||b|sin{\gamma}(ab)$;\\
2) $a x b {\perp} a,b$;\\
3) ha $|a x b| \neq 0$, akkor $a,b,a x b$ jobbrendszert alkot.

  \end{tcolorbox}
\end{frame}


\begin{frame}
  \begin{tcolorbox}[title={20}]
    Mondjuk ki a geometriai vektorokra vonatkozó kifejtési tételt.

  \tcblower
Ha $a,b,c$ geometriai vektorok, akkor $(a x b) x c = (ac)b - (bc)a$.

  \end{tcolorbox}
\end{frame}


\begin{frame}
  \begin{tcolorbox}[title={21}]
    Mondjuk ki a geometriai vektorokra vonatkozó felcserélési tételt.

  \tcblower
    Ha $a,b,c$ geometriai vektorok, akkor $(a x b)c = a(b x c)$. 
  \end{tcolorbox}
\end{frame}


\begin{frame}
  \begin{tcolorbox}[title={22}]
    Definiáljuk a geometriai vektorok vegyesszorzatának fogalmát.

  \tcblower
    Ha $a,b,c$ geometriai vektorok, akkor a vegyesszorzatuk az $(a x b)c$ skalár.

  \end{tcolorbox}
\end{frame}


\begin{frame}
  \begin{tcolorbox}[title={23}]
   Adjuk meg az A mátrix determinánsát definiáló képletet. Mit jelent ebben az $I(i1,...,in)$ kifejezés? 
  \tcblower
Legyen A =$\begin{bmatrix} 
  				a_{11} & \cdots & a_{1n}  \\
  				\vdots &   & \vdots \\
  				a_{n1} & \cdots & a_{nn}
			\end{bmatrix}$ $\in$ $\mathbb{R}^{n x n}$. Ekkor $det A = $ $\sum_{i_1, ..., i_n\\ (1, ..., n)} (-1)^{I(i_1, i_2, ..., i_n)} a_{1i_1} \cdot a_{1i_2} \cdot a_{1i_3} \cdot ... \cdot a_{ni_n}$\\
    \mmedskip
			

Itt az összegezés az $\{1,2,...,n\}$ számok minden permutációjára történik, $I(i_1,...,i_n)$ pedig az adott permutáció inverzióinak a számát jelöli.
  \end{tcolorbox}
\end{frame}


\begin{frame}
  \begin{tcolorbox}[title={24}]
   Mondjunk ki egy olyan feltételt, mely ekvivalens azzal, hogy az $A \in \mathbb{R}^{n x n}$ mátrix determinánsa nem nulla.

  \tcblower
$det \; A \neq 0 \iff {\varrho}(A) = n \iff$ A oszlopai (sorai) lineárisan függetlenek $\iff {\exists}A^{-1}$

  \end{tcolorbox}
\end{frame}


\begin{frame}
  \begin{tcolorbox}[title={25}]
   Mit értünk az $A \in \mathbb{R}^{n x n}$ mátrix i-edik sorának j-edik eleméhez tartozó előjelezett aldeterminánson, $A*{ij}$-n?

  \tcblower
Hagyjuk el az $A$ mátrix $i$-edik sorát és a $j$-edik oszlopát; az így kapott $(n-1)x(n-1)$-es mátrixot jelölje $B_{ij}$. Ekkor a keresett előjelezett aldetermináns: $A_{ij} = (-1)^{i+j} det B_{ij}$.

  \end{tcolorbox}
\end{frame}

\begin{frame}
  \begin{tcolorbox}[title={26}]
   Adjuk meg képlettel az $A \in \mathbb{R}^{n x n}$ mátrix determinánsának $i$-edik sora szerinti kifejtését. A mátrix elemeit, illetve előjelezett aldeterminánsait $a_{ij}$, ill. $A_{ij}$ jelöli.

  \tcblower
$detA =$ $\sum_{j = 1}^n$ $a_{ij}A_{ij}$
  \end{tcolorbox}
\end{frame}


\begin{frame}
  \begin{tcolorbox}[title={28}]
    Definiáljuk a Vandermonde-determináns fogalmát, és mondjuk ki az értékére vonatkozó állítást.

  \tcblower
    $a_1, ..., a_n \in \mathbb{R}, (n \geq 2)$ esetén $V_n(a_1, ..., a_n) =$ $\begin{vmatrix} 
  				1 & a_1 & a_1^2 & \cdots & a_1^{n-1} \\
  				\vdots & \vdots  & \vdots & & \vdots\\
  				1 & a_1 & a_1^2 & \cdots & a_1^{n-1}
			\end{vmatrix}$ $=$ $\prod_{1 \leq i < j \leq n} (a_i - a_j)$
  \end{tcolorbox}
\end{frame}


\begin{frame}
  \begin{tcolorbox}[title={30}]
    Mondjuk ki a determinánsokra vonatkozó szorzástételt.
  \tcblower
    Ha $A,B \in \mathbb{R}^{n x n}$ tetszőleges mátrixok, akkor $det(AB) = detA  \cdot  detB$.
  \end{tcolorbox}
\end{frame}


\begin{frame}
  \begin{tcolorbox}[title={31}]
   Mit jelent, hogy két négyzetes mátrix hasonló $\mathbb{R}$ felett?

  \tcblower
    $A,B \in \mathbb{R}^{n x n}$ hasonlók $\mathbb{R}$ felett, ha létezik olyan invertálható $S \in \mathbb{R}^{n x n}$ mátrix, melyre $B = S^{-1}AS$. Ezt általában $A {\sim}_{\mathbb{R}} B$ jelöli.

  \end{tcolorbox}
\end{frame}


\begin{frame}
  \begin{tcolorbox}[title={32}]
    Mikor mondjuk egy mátrixra, hogy diagonalizálható $\mathbb{R}$ felett?

  \tcblower
    Egy $A \in \mathbb{R}^{n x n}$ négyzetes mátrix diagonalizálható $\mathbb{R}$ felett, ha hasonló $\mathbb{R}$ felett egy diagonális mátrixhoz, azaz létezik olyan S invertálható, ill. D diagonális mátrix $\mathbb{R}^{n x n}$-ben, hogy $D = S^{-1}AS$.

  \end{tcolorbox}
\end{frame}


\begin{frame}
  \begin{tcolorbox}[title={33}]
    Definiáljuk egy mátrix jobb oldali sajátvektorának a fogalmát.

  \tcblower
    Legyen $A \in \mathbb{R}^{n x n}$ tetszőleges négyzetes mátrix. Egy $x \in \mathbb{R}^n$ vektort az $A$ mátrix jobb oldali sajátvektorának nevezünk, ha:\\
    \mmedskip
    
    1) $x \neq 0$;\\
    2) létezik ${\lambda}_0 \in \mathbb{R}$ szám, melyre $Ax = {\lambda}_0x$.

  \end{tcolorbox}
\end{frame}


\begin{frame}
  \begin{tcolorbox}[title={34}]
   Definiáljuk egy mátrix jobb oldali sajátértékének a fogalmát.
  \tcblower
     Legyen $A \in \mathbb{R}^{n x n}$ tetszőleges négyzetes mátrix. Egy ${\lambda}_0 \in \mathbb{R}$ számot az $A$ mátrix jobb oldali sajátértékének nevezünk, ha van olyan $x \in \mathbb{R}^n$ vektor, melyre\\
     \mmedskip
     
     1) $x \neq 0$;\\
     2) $Ax = {\lambda}_0x$.

  \end{tcolorbox}
\end{frame}


\begin{frame}
  \begin{tcolorbox}[title={35}]
    Mondjuk ki egy valós elemű mátrix $\mathbb{R}$ feletti diagonalizálhatóságának szükséges és elégséges feltételét a sajátvektorok fogalmának fölhasználásával.

  \tcblower
Egy $A \in \mathbb{R}^{n x n}$ mátrix pontosan akkor diagonalizálható $\mathbb{R}$ felett, ha létezik A sajátvektoraiból álló bázis $\mathbb{R}^n$-ben. 
  \end{tcolorbox}
\end{frame}



\begin{frame}
  \begin{tcolorbox}[title={36}]
     Definiáljuk az $A \in \mathbb{R}^{n x n}$ mátrix ${\lambda}_0$ (jobb oldali) sajátértékéhez tartozó sajátalterének fogalmát.

  \tcblower
$W_{{\lambda}_0} = \{x \in \mathbb{R}^n |Ax = {\lambda}_0x\}$ a ${\lambda}_0$-hoz tartozó sajátaltér. (Vagyis $W_{{\lambda}_0}$ a ${\lambda}_0$ sajátértékű sajátvektorok halmaza, kiegészítve a nullvektorral.)

  \end{tcolorbox}
\end{frame}

\begin{frame}
  \begin{tcolorbox}[title={37}]
    Definiáljuk egy négyzetes mátrix karakterisztikus polinomjának fogalmát.
  \tcblower
Egy $A \in \mathbb{R}^{n x n}$ mátrix karakterisztikus polinomja $k_A({\lambda}) = det(A-I_n{\lambda})$, ahol $I_n$ az $n x n$-es egységmátrix.

  \end{tcolorbox}
\end{frame}


\begin{frame}
  \begin{tcolorbox}[title={38}]
    Mondjuk ki a hasonló mátrixok karakterisztikus polinomjára vonatkozó állítást.

  \tcblower
Ha $A,B \in \mathbb{R}^{n x n}$ hasonlók $\mathbb{R}$ felett, akkor $k_A({\lambda}) = k_B({\lambda})$.

  \end{tcolorbox}
\end{frame}


\begin{frame}
  \begin{tcolorbox}[title={39}]
   Definiáljuk a komplex euklideszi tér fogalmát.
  \tcblower
  Legyen $V$ vektortér a $\mathbb{C}$ felett.\\
 			Azt mondjuk, hogy a $V$ komplex euklideszi tér, ha adott benne egy skaláris szorzatnak nevezett ${\langle}x, y{\rangle}$ $:$ $V$ $x$ $V$ $\rightarrow$ $\mathbb{C}$ függvény, melyre a következők teljesülnek minden $x, y, z \in V$ és $\lambda \in \mathbb{C}$ esetén:

			\begin{enumerate}
			\item ${\langle}y, x{\rangle} = \overline{{\langle}x, y{\rangle}}$
			\item ${\langle}{\lambda}x, y{\rangle} = {\lambda}{\langle}x, y{\rangle}$ és ${\langle}x, {\lambda}y{\rangle} = \overline{{\lambda}}{\langle}x, y{\rangle}$
			\item ${\langle}x + y, z{\rangle} = {\langle}x, z{\rangle} + {\langle}y, z{\rangle}$ és ${\langle}x, y + z{\rangle} = {\langle}x, y{\rangle} + {\langle}x, z{\rangle}$
			\item ${\langle}x, x{\rangle}$ mindíg (valós és) nemnegatív
			\item ${\langle}x, x{\rangle} = 0  \iff  x = 0$
			\end{enumerate}
  \end{tcolorbox}
\end{frame}


\begin{frame}
  \begin{tcolorbox}[title={40}]
   Definiáljuk az x vektor normáját egy euklideszi térben.

  \tcblower
Ha V valós vagy komplex euklideszi tér, akkor tetszőleges $x \in V$ vektorra $||x|| = \sqrt{{\langle}x, x{\rangle}}$.

  \end{tcolorbox}
\end{frame}


\begin{frame}
  \begin{tcolorbox}[title={41}]
     Mondjuk ki az euklideszi terek vektoraira vonatkozó háromszög-egyenlőtlenséget.
  \tcblower
Ha V valós vagy komplex euklideszi tér, akkor tetszőleges $x,y \in V$ vektorokra: $||x + y|| \leq ||x|| + ||y||$ .

  \end{tcolorbox}
\end{frame}


\begin{frame}
  \begin{tcolorbox}[title={42}]
    Mondjuk ki a valós vagy komplex euklideszi terekre vonatkozó Cauchy-egyelőtlenséget, valamint azt, hogy mikor áll ebben egyenlőség.
  \tcblower
Ha V valós vagy komplex euklideszi tér, akkor tetszőleges $x,y \in V$ vektorokra teljesül, hogy $|{\langle}x, x{\rangle}| \leq ||x|| \cdot ||y||$, és egyenlőség akkor és csak akkor áll fönn, ha az $x$ és $y$ vektorok lineárisan összefüggőek (azaz párhuzamosak).

  \end{tcolorbox}
\end{frame}


\begin{frame}
  \begin{tcolorbox}[title={43}]
    Definiáljuk egy $V$ euklideszi tér ortonormált bázisának a fogalmát.

  \tcblower
Az $e_1,...,e_n \in V$ vektorokból álló rendszert ortonormáltb ázisnak nevezzük a $V$ euklideszi térben, ha:\\
\mmedskip

1) bázist alkotnak $V$ -ben;\\
2) az $e_i$ vektorok páronként merőlegesek, azaz $i \neq j$ esetén ${\langle}ei,e_j{\rangle} = 0$; és\\
3) a vektorok normáltak, azaz $||e_i|| = 1$ minden $1 \leq i \leq n$-re. 
  \end{tcolorbox}
\end{frame}


\begin{frame}
  \begin{tcolorbox}[title={44}]
    Mondjuk ki a valós szimmetrikus mátrixokra vonatkozó spektráltételt (azaz főtengelytételt).

  \tcblower
Egy $A \in \mathbb{R}^{n x n}$ mátrix esetén pontosan akkor létezik A sajátvektoraiból álló ortonormált bázis $\mathbb{R}^n$-ben (azaz $A$ pontosan akkor diagonalizálható $\mathbb{R}$ felett ortonormált bázisban), ha az $A$ mátrix szimmetrikus (azaz $A^T = A$). (Ilyenkor az $A$ sajátértékei mind valósak.)

  \end{tcolorbox}
\end{frame}


\begin{frame}
  \begin{tcolorbox}[title={45}]
    Definiáljuk egy $A \in \mathbb{R}^{n x n}$ valós szimmetrikus mátrixhoz tartozó kvadratikus alakot.
  \tcblower
Az $A$-hoz tartozó kvadratikus alak az a $Q : \mathbb{R}^n {\rightarrow} \mathbb{R}$ függvény, melyre $Q(x) = x^TAx$.
  \end{tcolorbox}
\end{frame}

\begin{frame}
  \begin{tcolorbox}[title={46}]
     Mondjuk meg, mit jelent az, hogy az $A \in \mathbb{R}^{n x n}$ szimmetrikus mátrixhoz tartozó Q kvadratikus alak pozitív definit, és jellemezzük ezt az esetet az $A$ sajátértékei segítségével.

  \tcblower
Q-t akkor nevezzük pozitív definitnek, ha minden $0 \neq x \in \mathbb{R}^n$ vektorra $Q(x) > 0$. Ez pontosan akkor teljesül, ha az $A$ mátrix minden sajátértéke pozitív.
  \end{tcolorbox}
\end{frame}


\begin{frame}
  \begin{tcolorbox}[title={46}]
     Mondjuk meg, mit jelent az, hogy az $A \in \mathbb{R}^{n x n}$ szimmetrikus mátrixhoz tartozó Q kvadratikus alak negatív definit, és jellemezzük ezt az esetet az A karakterisztikus sorozata segítségével.

  \tcblower
Q-t akkor nevezzük negatív definitnek, ha minden $0 \neq x \in \mathbb{R}^n$ vektorra $Q(x) < 0$. Ez pontosan akkor teljesül, ha az A karakterisztikus sorozata jelváltó. (Az A mátrix ${\Delta}_o,...,{\Delta}_n$ karakterisztikus sorozatának k-adik tagja az A bal fölső sarkában lévő $k x k$-as részmátrix determinánsa, illetve ${\Delta}_o = 1$.)

  \end{tcolorbox}
\end{frame}




\begin{frame}[plain]
\begin{tcolorbox}[center, colback={myyellow}, coltext={black}, colframe={myyellow}]
    {\RHuge C rész}
    \mmedskip
\end{tcolorbox}
\end{frame}

\begin{frame}
  \begin{tcolorbox}[title={1. (4p)}]
    Mondjuk ki a $b_1,...,b_k$ vektorok lineáris függetlenségének definícióját, majd bizonyítsuk be, hogy ha $B = \{b_1,...,b_k\}$ lineárisan független vektorrendszer, akkor minden lineáris kombinációjukként felírható vektor egyértelműen írható föl B-beli vektorok lineáris kombinációjaként.
  \tcblower
  
    A $v_1,...,v_k$ vektorrendszer akkor és csak akkor lineárisan független, ha csak a triviális lineáris kombinációja adja a nullvektort.\\

    Képletben: tetszőleges ${\lambda}_1,{\lambda}_2,...,{\lambda}_k \in \mathbb{R}$ esetén, ha ${\lambda}_1v_1 + {\lambda}_2v_2 +  \cdot  \cdot  \cdot  + {\lambda}_kv_k = 0$, akkor minden $i$-re ${\lambda}_i = 0$.\\
    \mmedskip

    Ha lenne egy vektor, melynek kétféle fölírása is létezne: $u = \sum_{i = 1}^k {\lambda}_ib_i = \sum_{i = 1}^k {\mu}_ib_i$, akkor a kétféle előállítást egymásból kivonva azt kapjuk, hogy\\
    \mmedskip
    
    $0 = \sum_{i = 1}^k ({\lambda}_i - {\mu}_i)b_i$.\\
    \mmedskip
    
    Ha a két előállítás különbözik, akkor valamelyik $({\lambda}_i-{\mu}i)$ együttható nem nulla, s ez ellentmond a $B$ lineáris függetlenségének.
  \end{tcolorbox}
\end{frame}


\begin{frame}
  \begin{tcolorbox}[title={2. (4p)}]
     Tekintsük az alábbi két fogalmat:\\
     a) egy vektorrendszer lineárisan összefüggő;\\
     b) egy vektor lineárisan függ egy vektorrendszertől.\\
     \mmedskip
     
     Mondjuk ki azt az állítást, mely ezeket a fogalmakat összekapcsolja, és bizonyítsuk is be az állítást.
  \tcblower
  
    Ha $k \geq 2$, akkor a $v_1,...,v_k$ vektorrendszer akkor és csak akkor lineárisan összefüggő, ha létezik olyan $i$, hogy $v_i$ lineárisan függ a többi $v_j$ vektortól (azaz előáll mint a $v_1,...,_vi-1,v_i+1,...,v_k$ vektorok lineáris kombinációja).\\
    \mmedskip
    
    Tegyük föl először, hogy a megadott vektorrendszer lineárisan összefüggő.\\
    Ez azt jelenti, hogy a nullvektornak van egy olyan $\sum_{i = 1}^k {\lambda}_iv_i = 0$ előállítása, melynél valamelyik ${\lambda}_i$ együttható (pl. a ${\lambda}_{i_0}$) nem $0$.\\
    \mmedskip
    
    Ekkor a $v_{i_0}$ kifejezhető a többi vektor lineáris kombinációjaként, hiszen $v_{i_0} = -(1/{\lambda}_{i_0})\sum_{j \neq i_0} {\lambda}_jv_j$.\\
    \mmedskip
    
    A fordított irányhoz tegyük föl, hogy $vi0$ lineárisan függ a többi $v_j$ vektortól, azaz $v_{i_0} = \sum_{j \neq i_0} {\mu}_jv_j$ valamilyen ${\mu}_j$ együtthatókra.\\
    \mmedskip
    
    De ekkor átrendezhetjük a fönti egyenlőséget úgy, hogy minden vektor az egyenlőség azonos oldalára kerüljön, s ekkor azt kapjuk, hogy a ${\mu}_{i_0} = -1$ választással: $\sum_{i = 1}^k {\mu}_iv_i = 0$. Ez nem triviális lineáris kombinációja a $v_i$ vektoroknak, mert ${\mu}_{i_0} = -1 \neq 0$, tehát a vektorrendszer lineárisan összefüggő.
  \end{tcolorbox}
\end{frame}


\begin{frame}
  \begin{tcolorbox}[title={3. (4p)}]
      Mondjuk ki és igazoljuk azt az állítást, mely arról szól, mi történik, ha egy lineárisan független $a_1,...,a_k$ vektorrendszerhez hozzávéve a $b$ vektort, az új, bővebb $a_1,...,a_k,b$ vektorrendszer már összefüggővé válik.
  \tcblower
    Ha egy lineárisan független $a_1,...,a_k$ vektorrendszerhez hozzávéve a $b$ vektort, a kapott $a_1,...,a_k,b$ vektorrendszer már összefüggő, akkor $b$ lineárisan függ az $a_1,...,a_k$ vektoroktól (azaz előállítható lineáris kombinációjukként).\\
    \mmedskip
    
    A feltétel szerint ugyanis léteznek olyan ${\lambda}_1,...,{\lambda}_k,{\mu}$ együtthatók, melyek közül legalább az egyik nem nulla, és melyekre $(\sum_{i = 1}^k {\lambda}_ia_i) + {\mu}b = 0$. Itt azonban ${\mu}$ nem lehet nulla, ellenkező esetben az $a_i$ vektorok már önmagukban is előálítanák a nullvektort nem triviális módon, ez pedig ellentmond a lineáris függetlenségüknek.\\
    \mmedskip
    
    Ha átrendezzük az előbbi egyenlőséget úgy, hogy a $b$ vektort hagyjuk az egyik oldalon, akkor ${\mu}$-vel osztva $b = \sum_{i = 1}^k (-{\lambda}_i/{\mu})a_i$, ami azt mutatja, hogy $b$ lineárisan függ az $ai$ vektoroktól.
  \end{tcolorbox}
\end{frame}

\begin{frame}
  \begin{tcolorbox}[title={4. (4p)}]
      Mondjuk ki az $\mathbb{R}^n$-beli alterek fogalmának a műveletekre való zártsággal való definícióját, majd bizonyítsuk be, hogy $\mathbb{R}^n$ két alterének metszete is altér.
  \tcblower
    $W \; {\subseteq} \; \mathbb{R}^n$ pontosan akkor altér, ha nem üres (ekvivalens módon itt azt is megkövetelhetjük, hogy a nullvektor benne van $W$-ben), továbbá $w_1,w_2 \in W$ esetén $w_1  + w_2 \in W$, valamint $w \in W$ és ${\lambda} \in \mathbb{R}$ esetén ${\lambda}w \in W$. Tegyük föl, hogy $W_1,W_2$ altér $\mathbb{R}^n$-ben.\\
    \mmedskip
    
    Ekkor a nullvektor mindkét altérnek eleme, így $W_1 {\cap}W_2$ nem üres. Ha $w_1,w_2 \in W_1 {\cap}W_2$, akkor mindkét $i$ indexre $w_1 + w_2 \in W_i$ (hiszen $W_i$ altér), s ezért az összegvektor benne van a metszetben.\\
    \mmedskip
    
    Hasonlóan, ha $w \in W_1 {\cap} W_2$ és ${\lambda} \in \mathbb{R}$, akkor $W_i$ altér volta miatt ${\lambda}w \in W_i$ mindkét lehetséges $i$ indexre, így ${\lambda}w$ benne van a metszetükben is.
  \end{tcolorbox}
\end{frame}


\begin{frame}
  \begin{tcolorbox}[title={7. (4p)}]
      Definiáljuk az $n x n$-es egységmátrix fogalmát, és mondjuk meg, mi lesz az eredménye az egységmátrixszal való szorzásnak.\\
      
      Igazoljuk az egységmátrixszal jobbról való szorzásra vonatkozó összefüggést. 
  \tcblower
    Az $I_n \in \mathbb{R}^{n x n}$ egységmátrix $i$-edik sorának $j$-edik eleme $1$, ha $i = j$, és $0$ egyébként (azaz a főátlóban $1$-esek, a főátlón kívül $0$-k szerepelnek).\\
    \mmedskip
    
    Úgy is írhatjuk, hogy a mátrix általános eleme ${\delta}_{ij}$, ahol ${\delta}_{ij}$ a szokásos Kronecker-szimbólum.\\
    \mmedskip
        
    Tetszőleges $A \in \mathbb{R}^{k \; x \; n}$ és $B \in \mathbb{R}^{n \; x \; m}$ mátrixokra $AI_n = A$ és $I_nB = B$.\\
    \mmedskip
    
    Az $AI_n = A$ igazolásához jelölje $a_{pq}$ az $A$ mátrix általános elemét.\\
    \mmedskip
    
    A szorzatmátrixban az $i$-edik sor $j$-edik eleme a szorzás definíciója szerint $_{i} [AI_n]_j = \sum_{t = 1}^n (a_{it}{\delta}_{tj})$.\\
    \mmedskip
    
    Ebben az összegben ${\delta}_{tj} = 0$ ha $t \neq j$, ezért csak az egyetlen $a_{ij}{\delta}_{jj} = a_{ij}$ tag marad meg.
  \end{tcolorbox}
\end{frame}


\begin{frame}
  \begin{tcolorbox}[title={8. (4p)}]
       Mondjuk ki és bizonyítsuk be a mátrixok szorzatának transzponáltjára kimondott összefüggést. 
  \tcblower
    Ha $A \in R^{k \; x \; l}$ és $B \in \mathbb{R}^{l \; x \; n}$, akkor $(AB)^T = B^TA^T$.\\
    \mmedskip
    
    A szorzat transzponáltjának általános eleme ugyanis:\\
    \mmedskip
     $_{i} [(AB)^T]_j =$ $_{j} [AB]_i = \sum_{t = 1}^l $ $_{j} [A]_t \cdot $ $_{t} [B]_i = \sum_{t = 1}^l $ $_{i} [B^T]_t  \cdot  $ $_{t} [A^T]_j = $ $_{i} [B^TA^T]_j$.
  \end{tcolorbox}
\end{frame}

\begin{frame}
  \begin{tcolorbox}[title={15. (4p)}]
       Definiáljuk egy $A \in \mathbb{R}^{n x n}$ mátrix jobb oldali sajátértékének fogalmát, majd igazoljuk, hogy az $A$ mátrix ${\lambda}$ sajátértékű (jobb oldali) sajátvektorai a nullvektorral kiegészítve alteret alkotnak $\mathbb{R}^n$-ben.
  \tcblower
    ${\lambda} \in \mathbb{R}$ jobb oldali sajátértéke $A$-nak, ha van olyan $0 \neq v \in \mathbb{R}^n$ vektor, melyre $Av = {\lambda}v$. (Ilyenkor $v$-t a ${\lambda}$-hoz tartozó (egyik) sajátvektornak nevezhetjük.)\\
    \mmedskip
    
    Azt kell igazolnunk, hogy $W_{\lambda} = \{v \in \mathbb{R}^n |Av = {\lambda}v\} \leq \mathbb{R}^n$.\\
    \mmedskip
    
    Nyilván $0 \in W_{\lambda}$, így $W_{\lambda}$ nem üres.\\
    \mmedskip
    
    Ha $v_1,v_2 \in W_{\lambda}$, akkor $A(v_1 +v_2) = Av_1 +Av_2 = {\lambda}v_1 +{\lambda}v_2 = {\lambda}(v_1 +v_2)$, azaz $W_{\lambda}$ zárt az összeadásra.\\
    \mmedskip
    
    Végül, ha $v \in W_{\lambda}$ és ${\mu} \in \mathbb{R}$, akkor $A({\mu}v) = {\mu}Av = {\mu}({\lambda}v) = {\lambda}({\mu}v)$, vagyis $W_{\lambda}$ zárt a skalárral való szorzásra is.
  \end{tcolorbox}
\end{frame}


\end{document}
